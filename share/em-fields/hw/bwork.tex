%%%%%
%%%%% This problem is used by: 0sn,11,bwork
%%%%%
        The purpose of this problem is to prove that the constant of proportionality $a$
        in the equation $\der U_m=aB^2 \der v$, for the energy density of the magnetic field,
        is given by $a=c^2/8\pi k$  as asserted
        on page \pageref{benergy}.
        The geometry we'll use consists of two sheets of current, like a sandwich with
        nothing in between but some vacuum in which there is a magnetic field.
        The currents are in opposite directions, and we can imagine them as being
        joined together at the ends to form a complete circuit, like a tube made of
        paper that has been squashed almost flat. The sheets have lengths $L$ in the
        direction parallel to the current, and widths $w$. They are separated by
        a distance $d$, which, for convenience, we assume is small compared to $L$
        and $w$. Thus each sheet's contribution to the
        field is uniform, and can be approximated by the expression $2\pi k\eta/c^2$.\hwendpart
        (a) Make a drawing similar to the one in figure \figref{ebsheet} on page
        \pageref{fig:ebsheet}, and show that in this opposite-current configuration,
        the magnetic fields of the two sheets reinforce in the region between them, 
        producing double the field, but cancel on the outside.\hwendpart
        (b) By analogy with the case of a single strand of wire,
        one sheet's force on the other is $ILB_1$, were $I=\eta w$ is the
        total current in one sheet, and $B_1=B/2$ is the field contributed by only one of the
        sheets, since the sheet can't make any net force on itself.
        Based on your drawing and the right-hand rule, show that this force is repulsive.\hwendpart
        For the rest of the problem, consider a process in which
        the sheets start out touching, and are then separated to a distance $d$. Since
        the force between the sheets is repulsive, they do mechanical work on the
        outside world as they are separated, in much the same way that the piston in an
        engine does work as the gases inside the cylinder expand. At the same time,
        however, there is an induced emf which would tend to extinguish the current,
        so in order to maintain a constant current, energy will have to be drained from
        a battery. There are three
        types of energy involved: the increase in the magnetic field energy, the increase
        in the energy of the outside world, and the decrease in energy as the battery is
        drained. (We assume the sheets have very little resistance, so there is no ohmic
        heating involved.)\answercheck\hwendpart
        (c) Find the mechanical work done by the sheets,
        which equals the increase in the energy of the outside world. Show that your
        result can be stated in terms of $\eta$, the final volume $v=wLd$, and nothing
        else but numerical and physical constants.\answercheck\hwendpart
        (d) The power supplied by the battery is $P=I\Gamma_E$ (like $P=I\Delta V$, but
        with an emf instead of a voltage difference), and the circulation is given
        by $\Gamma=-\der\Phi_B/\der t$. The negative sign indicates that the battery is
        being drained. Calculate the energy supplied by the battery, and, as in part c,
        show that the result can be stated in terms of $\eta$, $v$, and 
        universal constants.\answercheck\hwendpart
        (e) Find the increase in the magnetic-field energy, in terms of
        $\eta$, $v$, and the unknown constant $a$.\answercheck\hwendpart
        (f) Use conservation of energy to relate your answers from parts c, d, and e,
        and solve for $a$.\answercheck        
