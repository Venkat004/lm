The neuron in the figure has been drawn fairly short, but
some neurons in your spinal cord have tails (axons) up to a
meter long. The inner and outer surfaces of the membrane act
as the ``plates'' of a capacitor. (The fact that it has been
rolled up into a cylinder has very little effect.) In order
to function, the neuron must create a voltage difference $V$
between the inner and outer surfaces of the membrane. Let
the membrane's thickness, radius, and length be $t$, $r$, and $L$.
(a) Calculate the energy that must be stored in the electric
field for the neuron to do its job. (In real life, the
membrane is made out of a substance called a dielectric,
whose electrical properties increase the amount of energy
that must be stored. For the sake of this analysis, ignore
this fact.)
<% hw_hint("neuronenergy") %>\answercheck\hwendpart
(b) An organism's evolutionary fitness should be better if
it needs less energy to operate its nervous system. Based on
your answer to part a, what would you expect evolution to
do to the dimensions $t$ and $r?$ What other constraints
would keep these evolutionary trends from going too far?
