%%%%%
%%%%% This problem is used by: 4em,6,cyclic-vbf
%%%%%
Section \ref{sec:calculating-magnetism} states the following rule:

For a positively charged particle, the direction of the $F$
vector is the one such that if you sight along it, the $\vc{B}$
vector is clockwise from the $v$ vector.

Make a three-dimensional model of the three vectors using
pencils or rolled-up pieces of paper to represent the
vectors assembled with their tails together. Now write down
every possible way in which the rule could be rewritten by
scrambling up the three symbols $F$, $\vc{B}$, and $v$. Referring
to your model, which are correct and which are incorrect?
