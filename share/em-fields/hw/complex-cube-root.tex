This problem deals with the cubes and cube roots of complex numbers, but the principles
involved apply more generally to other exponents besides 3 and 1/3. These examples
are designed to be much easier to do using the magnitude-argument representation
of complex numbers than with the cartesian representation. If done by the easiest technique,
none of these requires more than two or three lines of \emph{simple} math.
In the following, the
symbols $\theta$, $a$, and $b$ represent real numbers, and all angles are to be expressed in radians.
As often happens with fractional exponents, the cube root of a complex number will typically have
more than one possible value. (Cf.~$4^{1/2}$, which can be $2$ or $-2$.) In parts c and d,
this ambiguity is resolved explicitly in the instructions, in a way that is meant to make
the calculation as easy as possible.\\
(a) Calculate $\arg\left[(e^{i\theta})^3\right]$.\answercheck\hwendpart
(b) Of the points $u$, $v$, $w$, and $x$ shown in the figure, which could be
a cube root of $z$?\hwendpart
(c) Calculate $\arg\left[\sqrt[3]{a+bi}\right]$. For simplicity, assume that
$a+bi$ is in the first quadrant of the complex plane, and compute the answer
for a root that also lies in the first quadrant.\answercheck\hwendpart
(d) Compute
\begin{equation*}
  \frac{1+i}{(-2+2i)^{1/3}}.
\end{equation*}
Because there is more than one possible root to use in the denominator, multiple
answers are possible in this problem. Use the root that results in the final answer
that lies closest to the real line. (This is
also the easiest one to find by using the magnitude-argument techniques
introduced in the text.)\answercheck

