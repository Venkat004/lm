%%%%%
%%%%% This problem is used by: 0sn,11,forcebetweentwowires
%%%%%
        Two parallel wires of length $L$ carry currents $I_1$ and $I_2$. They are separated
        by a distance $R$, and we assume $R$ is much less than $L$, so that our
        results for long, straight wires are accurate. The goal of this
        problem is to compute the magnetic forces acting between the wires.\hwendpart
        (a) Neither wire can make a force on \emph{itself}.
        Therefore, our first step in computing wire 1's force on wire 2
        is to find the magnetic
        field made only by wire 1, in the space \emph{occupied} by wire 2.
        Express this field in terms of the given quantities.\answercheck\hwendpart
        (b) Let's model the current in wire 2 by pretending that there is a
         line charge inside it, possessing density per unit length $\lambda_2$
         and moving at velocity $v_2$. 
        Relate $\lambda_2$ and $v_2$ to the current $I_2$, using the result of problem
        \ref{hw:linechargecurrent}a.
        Now find the magnetic force wire 1 makes on wire 2, in terms of
        $I_1$, $I_2$, $L$, and $R$. <% hw_answer %>\hwendpart
        (c) Show that the units of the answer to part b work out to be newtons.
