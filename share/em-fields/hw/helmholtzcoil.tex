A Helmholtz coil is defined as a pair of identical
circular coils lying in parallel planes and
separated by a distance, $h$, equal to their
radius, $b$. (Each coil may have more than one turn of
wire.) Current circulates in the same direction in each
coil, so the fields tend to reinforce each other in the
interior region. This configuration has the advantage of
being fairly open, so that other apparatus can be easily
placed inside and subjected to the field while remaining
visible from the outside. The choice of $h=b$ results in the
most uniform possible field near the center. A photograph of a Helmholtz coil
is shown in example \ref{eg:circular-orbit} on page \pageref{eg:circular-orbit}.\hwendpart
(a) Find the
percentage drop in the field at the center of one coil,
compared to the full strength at the center of the whole
apparatus. \answercheck\hwendpart
(b) What value of $h$ (not equal to $b)$ would
make this difference equal to zero?\answercheck
