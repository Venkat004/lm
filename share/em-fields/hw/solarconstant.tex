%%%%%
%%%%% This problem is used by: 0sn,11,solarconstant
%%%%%
        (a) A beam of light has cross-sectional area $A$ and power $P$, i.e., $P$ is
        the number of joules per second that enter a window through which the beam
        passes. Find the energy density $U/v$ in terms of $P$, $A$, and universal constants.\\
        (b) Find $\tilde{\vc{E}}$ and $\tilde{\vc{B}}$, the amplitudes of the
        electric and magnetic fields, in terms of $P$, $A$, and universal constants (i.e., your answer
        should \emph{not} include $U$ or $v$). You will need the result of problem
        \ref{hw:poynting}a.
        A real beam of light usually consists of
        many short wavetrains, not one big sine wave, but don't
        worry about that.\answercheck<% hw_hint("solarconstant") %>\hwendpart
        (c) A beam of sunlight has an intensity of $P/A=1.35\times10^3\ \zu{W}/\zu{m}^2$,
        assuming no clouds or atmospheric absorption. This is known as the
        solar constant.\index{solar constant} Compute  $\tilde{\vc{E}}$ and $\tilde{\vc{B}}$,
        and compare with the strengths of static fields you experience in everyday life:
        $E \sim 10^6\ \zu{V}/\zu{m}$ in a thunderstorm, and $B \sim 10^{-3}$ T for the Earth's
        magnetic field.\answercheck
