The wavefunction $\Psi$ of an electron is a complex number. Make up an
example of a value for the wavefunction that is not a real number, and
consider the following expressions: $\Psi^2$, $|\Psi|^2$, $|\Psi^2|$.
Which of these would it make sense to interpret as a probability density?
All of them? Some? Only one?
<% hw_solution %>
