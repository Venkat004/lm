Nearly all naturally occurring oxygen nuclei are the isotope ${}^{16}\zu{O}$.
The extremely neutron-rich isotope ${}^{22}\zu{O}$ has been produced in accelerator
experiments, but only with great difficulty, and little is known about its properties.
The only states that have been observed and assigned reliable spins are the ground state,
with spin 0, and an excited state with spin 2 and an excitation energy of 3.2 MeV.
The excited state was detected by observing gamma rays for the $2\rightarrow0$ transition.
On the hypothesis that the spin-2 excited state is a rotation, predict the gamma-ray
energy that experimentalists should expect from the $4\rightarrow2$ transition
in the same rotational band.
