This problem refers to the analysis of the ammonia molecule in sec.~\ref{subsec:ammonia}, p.~\pageref{subsec:ammonia}.
(a) The bond lengths in this molecule are on the order of $0.1\ \zu{nm}$. Use this fact to estimate the moment of
inertia for rotation about the symmetry axis, and verify that states with $L_z>0$ are likely to be populated at
room temperature.\\
(b) The original 1955 paper by Townes and Schawlow on the microwave spectroscopy of ammonia detected about 55
lines lying between 17 and 29 GHz. Each of these corresponds to a certain value of $L$ and $L_z$. Since there
are many lines crowded together in this region of the spectrum, the issue arises of whether the resolution
of the experiment will be sufficient to distinguish them. One of the factors limiting the resolution is that
the molecules of ammonia gas have velocities that are random and randomly oriented, and this causes random
Doppler shifts in the lines. Estimate the Doppler shifts at room temperature and determine whether or not they
are likely to cause problems.
