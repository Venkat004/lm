\index{hydrogen atom!energies of states in}
<% begin_sec("History") %>
The experimental technique for measuring the energy levels
of an atom accurately is spectroscopy: the study of the
spectrum of light emitted (or absorbed) by the atom. Only
photons with certain energies can be emitted or absorbed by
a hydrogen atom, for example, since the amount of energy
gained or lost by the atom must equal the difference in
energy between the atom's initial and final states.
Spectroscopy had become a highly developed art
several decades before Einstein even proposed the photon,
and the Swiss spectroscopist Johann \index{Balmer, Johann}Balmer
determined in 1885 that there was a simple equation that
gave all the wavelengths emitted by hydrogen. In modern
terms, we think of the photon wavelengths merely as indirect
evidence about the underlying energy levels of the atom, and
we rework Balmer's result into an equation for these
atomic energy levels:
\begin{equation*}
                E_n    =  -\frac{2.2\times10^{-18}\ \zu{J}}{n^2}\eqquad,  
\end{equation*}
This energy includes both
the kinetic energy of the electron and the electrical
energy. The zero-level of the electrical energy
scale is chosen to be the energy of an electron and a proton
that are infinitely far apart. With this choice, negative
energies correspond to bound states and positive energies to unbound ones.

Where does the mysterious numerical factor of $2.2\times10^{-18}\ \zu{J}$
come from? In 1913 the Danish theorist Niels \index{Bohr, Niels}Bohr
realized that it was exactly numerically equal to a certain
combination of fundamental physical constants:
\begin{equation*}
 E_n = -\frac{mk^2e^4}{2\hbar^2}\cdot\frac{1}{n^2}\eqquad,  
\end{equation*}
where $m$ is the mass of the electron,  and $k$ is the Coulomb
force constant for electric forces.

Bohr was able to cook up a derivation of this equation based
on the incomplete version of quantum physics that had been
developed by that time, but his derivation is today mainly
of historical interest. It assumes that the electron follows
a circular path, whereas the whole concept of a path for a
particle is considered meaningless in our more complete
modern version of quantum physics. Although Bohr was able to
produce the right equation for the energy levels, his model
also gave various wrong results, such as predicting that the
atom would be flat, and that the ground state would have
$\ell=1$ rather than the correct $\ell=0$.
<% end_sec() %>
<% begin_sec("Approximate treatment") %>\label{start-approx-hydrogen-energies}
m4_ifelse(__sn,1,
[:
  Rather than leaping straight into a full mathematical treatment, we'll start by looking for some physical insight,
  which will lead to an approximate argument that correctly reproduces the form of the Bohr equation.
:],[:
  A full and correct treatment is impossible at the mathematical
  level of this book, but we can provide a straightforward
  explanation for the form of the equation using approximate
  arguments. 
:])
A typical standing-wave pattern for the electron
consists of a central oscillating area surrounded by a
region in which the wavefunction tails off. As discussed in
__subsection_or_section(schrodinger), the oscillating type of pattern is typically
encountered in the classically allowed region, while the
tailing off occurs in the classically forbidden region where
the electron has insufficient kinetic energy to penetrate
according to classical physics. We use the symbol $r$ for
the radius of the spherical boundary between the classically
allowed and classically forbidden regions.
Classically, $r$ would be the distance from the proton at which
the electron would have to stop,
turn around, and head back in.

If $r$ had the same value for every standing-wave pattern, then we'd
essentially be solving the particle-in-a-box problem in three dimensions,
with the box being a spherical cavity. Consider the energy levels of the particle in
a box compared to those of the hydrogen atom, \figref{hydrogen-versus-box}.
They're qualitatively different. The energy levels of the particle in a box
get farther and farther apart as we go higher in energy, and this feature
doesn't even depend on the details of whether the box is two-dimensional or three-dimensional,
or its exact shape. The reason for the spreading is that the box is taken to be
completely impenetrable, so its size, $r$, is fixed. A wave pattern with $n$ humps
has a wavelength proportional to $r/n$, and therefore a momentum proportional to $n$,
and an energy proportional to $n^2$. In the hydrogen atom, however, the force keeping
the electron bound isn't an infinite force encountered when it bounces off of a wall, it's
the attractive electrical force from the nucleus. If we put more energy into the electron,
it's like throwing a ball upward with a higher energy --- it will get farther out before
coming back down. This means that in the hydrogen atom, we expect $r$ to increase as we
go to states of higher energy. This tends to keep the wavelengths of the high energy
states from getting too short, reducing their kinetic energy. The closer and closer
crowding of the energy levels in hydrogen also makes sense because we know that there
is a certain energy that would be enough to make the electron escape completely, and
therefore the sequence of bound states cannot extend above that energy.
<% marg(80) %>
<%
  fig(
    'hydrogen-versus-box',
    %q{%
      The energy levels of a particle in a box, contrasted with those of the hydrogen atom.
    }
  )
%>
<% end_marg %>


When the electron is at the maximum classically allowed distance $r$ from the proton, it
has zero kinetic energy.
Thus when the
electron is at distance $r$, its energy is purely electrical:
\begin{equation}
  E    = -\frac{ke^2}{r}           
\end{equation}
Now comes the approximation. In reality, the electron's
wavelength cannot be constant in the classically allowed
region, but we pretend that it is. Since $n$ is the number
of nodes in the wavefunction, we can interpret it approximately
as the number of wavelengths that fit across the diameter
$2r$. We are not even attempting a derivation that would
produce all the correct numerical factors like 2 and $\pi $
and so on, so we simply make the approximation
\begin{equation}
        \lambda \sim \frac{r}{n}\eqquad.        
\end{equation}
Finally we assume that the typical kinetic energy of the
electron is on the same order of magnitude as the absolute
value of its total energy. (This is true to within a factor
of two for a typical classical system like a planet in a
circular orbit around the sun.) We then have
\begin{subequations}
\renewcommand{\theequation}{\theparentequation}
\begin{align}
        \text{absolute}&\text{ value of total energy} \\
        &= \frac{ke^2}{r} \notag \\
        &\sim K \notag \\ 
        &= p^2/2m \notag \\
        &= (h/\lambda)^2/2m \notag \\
        &\sim h^2n^2/2mr^2 \notag
\end{align}
\end{subequations}
We now solve the equation $ke^2/r \sim h^2n^2 / 2mr^2$ for $r$
and throw away numerical factors we can't hope to have
gotten right, yielding
\begin{equation}
        r \sim \frac{h^2n^2}{mke^2}\eqquad.
\end{equation}

Plugging $n=1$ into this equation gives $r=2$ nm, which is
indeed on the right order of magnitude. Finally we combine
equations [4] and [1] to find
\begin{equation*}
        E \sim -\frac{mk^2e^4}{h^2n^2}\eqquad,
\end{equation*}
which is correct except for the numerical factors we
never aimed to find.\label{end-approx-hydrogen-energies}
<% end_sec() %>

m4_ifelse(__sn,1,[:
<% begin_sec("Exact treatment of the ground state") %>
The general proof of the Bohr equation for all values of $n$ is beyond the mathematical
scope of this book, but it's fairly straightforward to verify it for a particular $n$,
especially given a lucky guess as to what functional form to try for the wavefunction.
The form that works for the ground state is
\begin{equation*}
  \Psi = ue^{-r/a}\eqquad,
\end{equation*}
where $r=\sqrt{x^2+y^2+z^2}$ is the electron's distance from the proton, and $u$ provides for normalization.
In the following, the result $\partial r/\partial x=x/r$ comes in handy.
Computing the partial derivatives that occur in the Laplacian, we obtain for the
$x$ term
\begin{align*}
  \frac{\partial\Psi}{\partial x}     &= \frac{\partial \Psi}{\partial r} \frac{\partial r}{\partial x} \\
                                      &= -\frac{x}{ar} \Psi \\
  \frac{\partial^2\Psi}{\partial x^2} &= -\frac{1}{ar} \Psi -\frac{x}{a}\left(\frac{\partial}{\der x}\frac{1}{r}\right)\Psi+ \left( \frac{x}{ar}\right)^2 \Psi\\
                                      &= -\frac{1}{ar} \Psi +\frac{x^2}{ar^3}\Psi+ \left( \frac{x}{ar}\right)^2 \Psi\eqquad,
\intertext{so}
  \nabla^2\Psi &= \left( -\frac{2}{ar} + \frac{1}{a^2} \right) \Psi\eqquad.
\end{align*}
The Schr\"odinger equation gives
\begin{align*}
  E\cdot\Psi &= -\frac{\hbar^2}{2m}\nabla^2\Psi + U\cdot\Psi \\
             &= \frac{\hbar^2}{2m}\left( \frac{2}{ar} - \frac{1}{a^2} \right)\Psi -\frac{ke^2}{r}\cdot\Psi
\end{align*}
If we require this equation to hold for all $r$, then we must have equality for both the terms of the form
$(\text{constant})\times\Psi$ and for those of the form $(\text{constant}/r)\times\Psi$. That means
\begin{align*}
  E &= -\frac{\hbar^2}{2ma^2} \\
\intertext{and}
  0 &=  \frac{\hbar^2}{mar} -\frac{ke^2}{r}\eqquad.
\end{align*}
These two equations can be solved for the unknowns $a$ and $E$, giving
\begin{align*}
  a &= \frac{\hbar^2}{mke^2} \\
\intertext{and}
  E &= -\frac{mk^2e^4}{2\hbar^2}\eqquad,
\end{align*}
where the result for the energy agrees with the Bohr equation for $n=1$.
The calculation of the normalization constant $u$ is relegated to homework problem \ref{hw:h-atom-normalization}.

<% self_check('kinky-hydrogen',<<-'SELF_CHECK'
We've verified that the function $\Psi = he^{-r/a}$ is a solution to the Schr\"odinger equation,
and yet it has a kink in it at $r=0$. What's going on here? Didn't I argue before that kinks are unphysical?
  SELF_CHECK
  ) %>

\begin{eg}{Wave phases in the hydrogen molecule}\label{eg:h2-details}
In example \ref{h2-bond} on page \pageref{h2-bond}, I argued that the existence of the $\zu{H}_2$ molecule
could essentially be explained by a particle-in-a-box argument: the molecule is a bigger box than an
individual atom, so each electron's wavelength can be longer, its kinetic energy lower.
Now that we're in possession of a mathematical expression for the wavefunction of the hydrogen
atom in its ground state, we can make this argument a little more rigorous and detailed.
Suppose that two hydrogen atoms are in a relatively cool sample of monoatomic hydrogen gas.
Because the gas is cool, we can assume that the atoms are in their ground states. Now suppose
that the two atoms approach one another. Making use again of the assumption that the gas is cool, it
is reasonable to imagine that the atoms approach one another slowly. Now the atoms come a little
closer, but still far enough apart that the region between them is classically forbidden. Each
electron can tunnel through this classically forbidden region, but the tunneling probability is
small. Each one is now found with, say, 99\% probability in its original home, but with 1\%
probability in the other nucleus. Each electron is now in a state consisting of a superposition
of the ground state of its own atom with the ground state of the other atom. There are two peaks
in the superposed wavefunction, but one is a much bigger peak than the other.

An interesting question now arises. What are the relative phases of the two electrons?
As discussed on page \pageref{fig:electron-wave-phase}, the \emph{absolute} phase of an electron's
wavefunction is not really a meaningful concept. Suppose atom A contains electron Alice, and
B electron Bob. Just before the collision, Alice may have wondered, ``Is my phase
positive right now, or is it negative? But of course I shouldn't ask myself such silly questions,''
she adds sheepishly.

<%
  fig(
    'h2-phase',
    %q{%
      Example \ref{eg:h2-details}.
    },
    {
      'width'=>'wide',
      'sidecaption'=>true
    }
  )
%>

But \emph{relative} phases \emph{are} well defined. As the two atoms draw closer and closer together,
the tunneling probability rises, and eventually gets so high that each electron is spending essentially
50\% of its time in each atom. It's now reasonable to imagine that either one of two possibilities could
obtain. Alice's wavefunction could either look like \subfigref{h2-phase}{1}, with the two peaks in
phase with one another, or it could look like \subfigref{h2-phase}{2}, with opposite phases. Because
\emph{relative} phases of wavefunctions are well defined, states 1 and 2 are physically 
distinguishable.\footnote{The reader who has studied chemistry may find it helpful to make contact
with the terminology and notation used by chemists. The state represented by
pictures 1 and 4 is known as a
$\sigma$ orbital, which is a type of ``bonding orbital.'' The state in 2 and 3 is a $\sigma^*$, a kind of
``antibonding orbital.'' Note that although we will not discuss electron spin or the Pauli exclusion
principle until sec.~\ref{subsec:electron-spin}, p.~\pageref{subsec:electron-spin}, those considerations
have no effect on this example, since the two electrons can have opposite spins.}
In particular, the kinetic energy of state 2 is much higher; roughly speaking, it is like the two-hump
wave pattern of the particle in a box, as opposed to 1, which looks roughly like the one-hump pattern
with a much longer wavelength. Not only that, but an electron in state 1 has a large probability of being found in the
central region, where it has a large negative electrical energy due to its interaction with both protons.
State 2, on the other hand, has a low probability of existing in that region. Thus state 1 represents the
true ground-state wavefunction of the $\zu{H}_2$ molecule, and putting both Alice and Bob in that state
results in a lower energy than their total energy when separated, so the molecule is bound, and will not
fly apart spontaneously.

State \subfigref{h2-phase}{3}, on the other hand, is not physically distinguishable from \subfigref{h2-phase}{2},
nor is \subfigref{h2-phase}{4} from \subfigref{h2-phase}{1}. Alice may say to Bob, ``Isn't it wonderful that
we're in state 1 or 4? I love being stable like this.'' But she knows it's not meaningful to ask herself
at a given moment which state she's in, 1 or 4.
\end{eg}


<% end_sec %>
:],[::])

\startdqs

\begin{dq}
States of hydrogen with $n$ greater than about 10 are
never observed in the sun. Why might this be?
\end{dq}

\begin{dq}
Sketch graphs of $r$ and $E$ versus $n$ for the hydrogen,
and compare with analogous graphs for the one-dimensional particle in a box.
\end{dq}
