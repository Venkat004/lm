<% begin_sec("Nonlocality and entanglement",nil,'nonlocality-and-entanglement') %>
<% begin_sec("Nonlocality",nil,'nonlocality') %>
People sometimes say that quantum mechanics is the set of rules for
describing the world of the very small, but this is a false
generalization, like saying that terriers are untrainable. How do we
define our measure of how small is small? The only distance scales
we've discussed have been wavelengths, and there is no upper limit on
wavelengths. The wavelength of an FM radio photon is bigger than my
terrier, who is very obedient to Newton's laws.  The only scale built
in to the structure of quantum mechanics is Planck's constant, and
Planck's constant has units of joules per hertz, not meters, so it
can't be converted into a distance. Quantum mechanics is, as far as we
can tell, a valid tool for describing systems at scales from
quarks to galaxies. 

<% marg(-5) %>
<%
  fig(
    'beam-splitter',
    %q{%
      A photon hits a piece of glass that reflects half of the
      light and transmits the other half.
    }
  )
%>
<% end_marg %>

So quantum behavior can occur at any scale, even large ones. For an example
that may be a little disturbing, consider the arrangement shown in figure
\figref{beam-splitter}.  A single photon comes
in from the left and encounters a diagonal piece of glass. The glass reflects half
the light and transmits half of it. The photon is a wave, and this is expected wave
behavior. But the photon is also a particle, and we can't have half a particle.
Therefore either camera A will detect a whole photon and B will see none, or it
will be the other way around. If we repeat the experiment many times times, we might
come up with a list of results like this:

\begin{tabular}{ll}
A & B \\
\hline
no & yes \\
yes & no \\
yes & no \\
no & yes \\
no & yes \\
yes & no \\
no & yes \\
yes & no 
\end{tabular}

\noindent An instant before the moment of detection, the photon is a
wave pattern that just happens to consist of two widely separated
pieces, each carrying half the energy.  The situation seems perfectly
symmetric, but then a moment later we find that B has detected the
photon and A hasn't. If B's detection of the photon is random, then
how does the information get to A that it had better \emph{not} detect
it?  This seems as though there is some sort of conspiracy being
carried out over arbitrarily large distances and with no time delay.
It's as though the two parts of the wave are a pair of criminal
suspects who would like to line up their stories but are being kept in
separate jail cells so that they can't communicate.  If the part of
the wave at B is going to be detected (at full strength, carrying
100\% of the energy $E=hf$), how does the part at A get the message
that it should fade away like the Cheshire cat?
This coordination would have to occur over very large distances ---
real-world experiments of this type have been done over distances of a
thousand kilometers, with the photons traveling either through outer
space or through fiber-optic cables.  Einstein derisively referred to
this apparent coordination as ``spooky action at a distance.''

Niels Bohr and two collaborators proposed in 1924 the
seemingly reasonable solution that there \emph{can't} be any such
coordination. Then the random detection of the photon by camera A
and camera B would be independent. Independent probabilities multiply,
so there would be a probability of $(1/2)(1/2)=1/4$ that both cameras
would see photons. This would violate conservation of energy, since
the original energy $E=hf$ would have been detected twice, and the universe
would have gained $1hf$ worth of total energy. But Bohr
pointed out that there would also be the same probability that neither
camera would detect a photon, in which case the change in the universe's
energy would be $-1hf$. On the average, energy would be conserved. According
to Bohr's theory, conservation of energy and momentum would not be absolute
laws of physics but only rules that would be true on the average.

The experimentalists Geiger and Bothe immediately set out to test this
prediction. They performed an experiment analogous to the one in
figure \figref{beam-splitter}, but with x-rays rather than visible
light.  Their results, published in 1926, showed that if one detector
saw the x-ray photon, the other did not, so that energy was always
conserved at the microscopic level, not just on the average.
We \emph{never} observe an outcome in which both A and B
detect a photon, or one in which neither detects it. That is, the
occurrence of event A (camera A sees a photon) and event B (camera B
sees one) are both random, but they are not independent. 

<% end_sec('nonlocality') %>

<% begin_sec("Entanglement",nil,'entanglement') %>
At a 1927 conference in Brussels, Einstein protested that this was a
problem, because the two detectors could in principle make their
observations simultaneously, and it would then seem that some
influence or communication was being transmitted between them faster
than the speed of light.  ``It seems to me,'' he complained, ''that
this difficulty cannot be overcome unless the description of the
process in terms of the \ldots wave is supplemented by some detailed
specification of the [trajectory of the particle]. \ldots If one works
only with \ldots waves, the interpretation \ldots, I think,
contradicts the postulate of relativity.''

The experimental fact ends up being that the spooky action at a
distance exists, and it does go faster than light. In 2012, Guerreiro
\emph{et al.}\footnote{\url{arxiv.org/abs/1204.1712}. The paper is
very readable.} carried out a very direct and conceptually simple
enactment of exactly the experiment in figure \figref{beam-splitter},
with electronic timing precise enough to prove that the detection
events at A and B were separated from each other by too great a
distance to have been linked by any influence traveling at $\le c$.
These findings are summarized by saying that quantum mechanics is
\emph{nonlocal}.  A single wave-particle can be spread out over
an arbitrarily large region of space, but its interactions that
transfer energy and momentum are always correlated over these
distances in such a way that the conservation laws are maintained. 

What Einstein had not originally appreciated was that these
correlations do not violate relativity because they do not actually
transport any energy, or even any information, between A and B. For
example, if Alice is at detector A, and Bob is at B, a million
kilometers away, Alice can detect the photon and know immediately that
Bob did not detect it. She learns something seemingly instantaneously
about Bob --- Bob is probably sad and disappointed right now. But
because Bob does not have any control over the result, he cannot use
this fact to send a message to Alice, so there is no transmission of
information. Alice and Bob's states are said to be \emph{entangled}.\index{entanglement}\label{bob-alice-entangled}

<% 
  fig(
    'entanglement-gloves',
    %q{%
      Entanglement is like finding that you only have your left glove, so that you must
      have left your right glove at home. There is a gain in information, but no
      sudden transmission of information from the dog to you.
    },
    {
      'width'=>'wide',
      'sidecaption'=>true
    }
  )
%>


As a useful analogy, suppose that you head off to work on a winter day
in New York. As you step out of the subway station into the cold air,
you reach into your pockets for your gloves, but you find that
you only have your left glove. Oh, you must have dropped your right
glove on the floor while you were petting your adorable terrier on the way out the
door. The presence of your left glove tells you that your right glove
must be at home. But there has been no
spooky action at a distance. You have simply recovered some
information about a region of space that lies at some distance from
you.

Einstein and Bohr had strong physical intuitions that led them to
incorrect predictions about experiments, and these predictions were
the fruits of inappropriate mental pictures of what was going on. If we take
the principles of quantum mechanics seriously, then the correct picture
is the following. Before the photon in figure \figref{beam-splitter} hits the glass
diagonal, the state of things is the following.
\begin{equation*}
\fbox{\parbox{60mm}{A photon is headed to the right.}}
\end{equation*}

Our photon is then partially reflected and partially transmitted. Now we have
a superposition of two wave patterns:
\begin{equation*}
 c\fbox{\parbox{35mm}{The photon has been reflected upward.}}
+c'\fbox{\parbox{35mm}{The photon has continue to the right.}},
\end{equation*}
where the amplitudes $c$ and $c'$ are equal in absolute value.\footnote{
Conservation of energy requires $c^2=1/2$ and $c'^2=1/2$,
even in classical physics. We could have, for
example, $c=1/\sqrt{2}$ and $c'=-1/\sqrt{2}$. Such a possible difference in
signs wouldn't concern us in this example. It would only be relevant if there
were some later opportunity for the two parts of the wave to recombine and
superimpose on one another, producing interference effects.}

Let's say that the cameras are at equal distances from the glass diagonal,
so that their chances to detect the photon occur simultaneously.\footnote{
According to special relativity, this simultaneity holds only in one frame
of reference, say the lab frame. But if simultaneity does hold in one frame,
then we can also say that in \emph{all} frames, the distance between the two
events is ``spacelike,'' i.e., they are too far apart to have been connected
by any causal influence propagating at $\le c$.
}
After detection, we have this:
\begin{equation*}
 c\fbox{\parbox{35mm}{Camera A detected a photon and B didn't.}}
+c'\fbox{\parbox{35mm}{B detected a photon and A didn't.}},
\end{equation*}
Here we have made the nontrivial assumption that material objects like
cameras obey the same wave-superposition rules as photons. This turns out
to be true. Cameras are made out of things like electrons,
and as we'll see in __section_or_chapter(matter-as-a-wave), things like
electrons are also wave-particles, and they obey all the same wave-particle rules
as photons. The states of the two cameras are now entangled.

You can see where this is going.  Alice had been standing by camera A,
watching anxiously, while Bob, a million kilometers away, was
breathlessly observing camera B. 
\begin{equation*}
 c\fbox{\parbox{38mm}{Alice saw a photon and Bob didn't. They consider this result to have been random.}}
+c'\fbox{\parbox{38mm}{Bob saw a photon and Alice didn't. They consider this result to have been random.}},
\end{equation*}
It doesn't \emph{seem} to Alice and Bob as though their brains are in
a superposition of two states. They \emph{feel} as though they have only
experienced the one possibility that actually happened, not a mixture of
both at the same time. And yet this picture of the physics explains very nicely
how the deterministic laws of physics produces a result that \emph{seems}
to them to have been random.

If Alice and Bob have been split into two ghostlike halves of
themselves, then conceivably these half-selves could undergo
interference, as in the double-slit experiment. But there are
practical reasons why we cannot actually detect such interference
effects. For one thing, Alice and Bob are macroscopic objects, with
energies $E$ on the order of many joules. Because Planck's constant is
small, their wave frequencies $f=E/h$ are extremely high, and their
wavelengths incredibly short (on the order of $10^{-34}\ \munit$!). We
have seen that diffraction becomes undetectable when wavelengths are
too short.  Furthermore, there is a phenomenon called decoherence,\index{decoherence}\label{decoherence-brief}
m4_ifelse(__sn,1,[:discussed further in sec.~\ref{subsec:decoherence}, p.~\pageref{subsec:decoherence},:],[:%:]) 
in which interactions with the environment tend to rapidly randomize
the wave-phases of large objects. When phases are randomized,
interference and diffraction effects become undetectable. 

Historically, it seemed absurd to the originators of quantum mechanics
to imagine a macroscopic object in a superposition of states.
The most celebrated example is called the Schr\"odinger's cat\label{schrodingers-cat}
experiment. Luckily for the cat, there probably was no
actual experiment --- it was simply a ``thought experiment''
that the German theorist Schr\"odinger discussed
with his colleagues. Schr\"odinger wrote:

\begin{quote}
One can even construct quite burlesque cases. A cat is shut
up in a steel container, together with the following
diabolical apparatus (which one must keep out of the direct
clutches of the cat): In a Geiger tube [radiation detector]
there is a tiny mass of radioactive substance, so little
that in the course of an hour perhaps one atom of it
disintegrates, but also with equal probability not even one;
if it does happen, the counter [detector] responds and ...
activates a hammer that shatters a little flask of prussic
acid [filling the chamber with poison gas]. If one has left
this entire system to itself for an hour, then one will say
to himself that the cat is still living, if in that time no
atom has disintegrated. The first atomic disintegration
would have poisoned it.
\end{quote}

<% marg(50) %>
<%
  fig(
    'cat',
    %q{Schr\"odinger's cat.}
  )
%>
<% end_marg %>

\noindent It seemed ridiculous to Schr\"odinger that at the end of
the hour, ``The uncertainty originally restricted to the atomic domain
has been transformed into a macroscopic uncertainty...,'' and the cat
would be in a superposed state. 

In modern language, people like Einstein and Schr\"odinger didn't feel
comfortable with nonlocality, or with entanglement of subatomic
particles, and they felt even less comfortable with applying these
concepts to macroscopic objects. Today, entanglement has been
demonstrated using objects that clearly deserve to be called
macroscopic.\index{entanglement!of macroscopic objects}\label{macroscopic-entanglement}
For example, in 2012, K.C.~Lee \emph{et al.} created a
version of the experiment in figure \figref{beam-splitter} in which
the cameras were replaced by small diamonds, about 1 mm in size.  They
were separated by 15 cm, which is a macroscopic distance. When a
photon hit one of the diamonds, it produced a vibration in the crystal
lattice.  This vibration was localized to a relatively small region
within the diamond, but this region was still large enough that one
has to admit that it qualifies as macroscopic. Its atoms had a total
weight of about 0.1 nanograms, which is a quantity big enough to weigh
on a state-of-the-art balance, and the region was about 0.01 mm in
size, which would make it visible with a magnifying glass. 

The quantum states of the two diamonds became entangled: if one had
detected the photon, the other hadn't. This entangled state was
maintained for only about 7 picoseconds before decoherence destroyed
the phase relationship between one diamond and the other. But Lee was
able to use additional photons to ``read out'' the quantum states in
only 0.5 ps, before decoherence occurred, and verify that there were
wave interference effects in which one diamond's quantum-mechanical
wave had a definite phase relationship with the other's.  Although
these experiments are difficult, they suggest that there is no
obstruction in principle to observing quantum-mechanical effects such
as superposition in arbitrarily large objects. 

m4_ifelse(__sn,1,[:
Entanglement is discussed in more mathematical detail in sec.~\ref{sec:more-entanglement}, p.~\pageref{sec:more-entanglement}.
:],[:
%
:])
<% end_sec('entanglement') %>
<% begin_sec("The Copenhagen and many-worlds approximations",nil,'ci-and-mwi') %>
When we last saw Alice and Bob, they were in this superposition of states,
\begin{equation*}
 c\fbox{\parbox{38mm}{Alice saw a photon and Bob didn't. They consider this result to have been random.}}
+c'\fbox{\parbox{38mm}{Bob saw a photon and Alice didn't. They consider this result to have been random.}},
\end{equation*}
with
\begin{equation*}
  |c|=|c'|.
\end{equation*}
Let's focus on Bob number one --- the sad Bob --- who didn't see a photon. This is just one of the disappointments
that Bob has experienced in his life, which include breaking up with his college crush and failing
to summit Kilimanjaro due to altitude sickness. But Bob is a sane, normal person, and he's not going
to spend the rest of his life obsessing over how things might have been, in another world. 
Like a banker writing off a bad debt, Bob decides to stop maintaining all the bookkeeping that is,
to him, irrelevant going forward. He now rewrites history and says that
\begin{equation*}
  |c|=1 \qquad \text{and} \qquad |c'|=0.
\end{equation*}
Technically speaking, this is wrong, because it is in principle still possible to have wave interference
effects between sad Bob and happy Bob. But in reality such effects are impractical to observe with
macroscopic objects, due to effects like short wavelengths and decoherence, so what Bob is doing
by ``clearing the books''
is an extremely good approximation. We will refer to this approximation by two different names,
the \emph{Copenhagen approximation} and the \emph{many-worlds approximation}, for the following historical and
psychological reasons.

In the early years of quantum mechanics, the school of
physicists\label{copenhagen-introduced}\index{Copenhagen approximation}\index{Copenhagen interpretation|see {Copenhagen approximation}} 
centering on Niels Bohr in Copenhagen were horribly confused about how
to interpret quantum mechanics. They had all kinds of wrong ideas,
such as the idea that quantum mechanics applied to individual atoms
but not to light or to macroscopic objects. They didn't know about
decoherence. They thought there was a clear dividing line between
microscopic things and macroscopic things (there isn't), and they
hypothesized that quantum mechanics only applied to microscopic ones
(it applies to both). They claimed that clearing the books was an
actual physical process, which they described as the ``collapse'' of
the wave. This was referred to as the Copenhagen interpretation, but
we can now see that it is an approximation. There are cases where it
is a bad approximation, e.g., at $t=3\ \zu{ps}$ during the experiment
by Lee \emph{et al.} (p.~\pageref{macroscopic-entanglement}), when
decoherence had started to happen but was only about half-way
complete.

The many-worlds approximation came along a little later.\footnote{It was
originally proposed in a 1957 PhD thesis by Hugh Everett, who called it
the relative state interpretation of quantum mechanics. Later it began
to be referred to as the many-worlds interpretation.} It consists of
making the same kind of ``clearing the books'' approximation, but recognizing
that there is no physical process of collapse.\index{many-worlds approximation}\index{many-worlds interpretation|see {many-worlds approximation}} 

Many physicists are philosophically attached to one or the other of these
approximations, and would object to my description of them as approximations.
My main purpose in writing this explanation is to immunize you against the
impression, which can be mistakenly picked up from many descriptions of quantum
mechanics, that a particular point of view on these topics (often the Copenhagen
approximation) is somehow ``standard.''
<% end_sec('ci-and-mwi') %>
<% end_sec('nonlocality-and-entanglement') %>
