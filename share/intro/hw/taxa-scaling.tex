%%%%%
%%%%% This problem is used by: 1np,1,taxa-scaling
%%%%%
% meta {"stars":1}
A taxon (plural taxa) is a group of living things. For example, \emph{Homo sapiens} and
\emph{Homo neanderthalensis} are both taxa --- specifically, they are two different species within
the genus \emph{Homo}. Surveys by botanists show that the number of plant taxa native to
a given contiguous land area $A$ is usually approximately proportional to $A^{1/3}$. (a) There are 70
different species of lupine native to Southern California, which has an area of about
$200,000\ \zu{km}^2$. The San Gabriel Mountains cover about $1,600\ \zu{km}^2$. Suppose
that you wanted to learn to identify all the species of lupine in the San Gabriels.
Approximately how many species would you have to familiarize yourself with?<% hw_answer %>\answercheck\hwendpart
(b) What is the interpretation of the fact that the exponent, $1/3$, is less than one?
