%%%%%
%%%%% This problem is used by: 1np,1,insect-x-ray
%%%%%
X-ray images aren't only used with human subjects but also, for example,
on insects and flowers. In 2003, a team of researchers at Argonne National Laboratory
used x-ray imagery to find for the first time that insects, although they do not have
lungs, do not necessarily breathe completely passively, as had been believed previously; many insects rapidly compress and
expand their trachea, head, and thorax in order to force air in and out of their bodies.
One difference between x-raying a human and an insect is that if a medical x-ray machine was
used on an insect, virtually 100\% of the x-rays would pass through its body, and there would
be no contrast in the image produced. Less penetrating x-rays of lower energies have to
be used. For comparison, a typical human body mass is about 70 kg, whereas
a typical ant is about 10 mg. Estimate the ratio of the thicknesses of tissue that must be
penetrated by x-rays in one case compared to the other.\answercheck
