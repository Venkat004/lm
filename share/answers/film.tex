1. No. To get the best possible interference, the thickness of the coating must be such
that the second reflected wave train lags behind the first by an integer number of
wavelengths. Optimal performance can therefore only be produced for one specific
color of light. The typical greenish color of the coatings shows that
they do the worst job for green light.

2. Light can be reflected either from the outer surface of the film or from the
inner surface, and there can be either constructive or destructive interference
between the two reflections. We see a pattern that varies across the surface because
its thickness isn't constant. We see rainbow colors because the condition for
destructive or constructive interference depends on wavelength. White light is
a mixture of all the colors of the rainbow, and at a particular place on the
soap bubble, part of that mixture, say red, may be reflected strongly, while
another part, blue for example, is almost entirely transmitted.



