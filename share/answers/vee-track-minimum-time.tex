(a) Other than $w$, the only thing with units that can occur in our answer is $g$.
If we want to combine a distance and an acceleration to produce a time, the
only way to do so is like $\sqrt{w/g}$, possibly multiplied by a unitless constant.\\
(b) It is convenient to introduce the notations $L$ for the length of one side of
the vee and $h$ for the height, so that $L^2=w^2+h^2$.
The acceleration is $a=g\sin\theta=gh/L$. To travel a distance $L$ with this
acceleration takes time
\begin{equation*}
  t=\sqrt{2L/a}=\sqrt{\left(\frac{2w}{g}\right)\left(\frac{h}{w}+\frac{w}{h}\right)}.
\end{equation*}
Let $x=h/w$. For a fixed value of $w$, this time is an increasing function of
$x+1/x$, so we want the value of $x$ that minimizes this expression.
Taking the derivative and setting it equal to zero gives $x=1$, or $h=w$.
In other words, the time is minimized if the angle is $45\degunit$.\\
(c) Plugging $x=1$ back in, we have $t^*=2t=4\sqrt{w/g}$, so the unitless factor
was 4.
