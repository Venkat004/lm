(a) See the figure below. The first refraction clearly
bends it inward. However, the back surface of the lens is
more slanted, so the ray makes a bigger angle with respect
to the normal at the back surface. The bending at the back
surface is therefore greater than the bending  at the front
surface, and the ray ends up being bent \emph{outward} more than inward.

\anonymousinlinefig{funkylenses}

(b) Lens 2 must act the same as lens 1. It's diverging. One
way of knowing this is time-reversal symmetry: if we flip
the original figure over and then reverse the direction of
the ray, it's still a valid diagram.

Lens 3 is diverging like lens 1 on top, and diverging
like lens 2 on the bottom. It's a diverging lens.

As for lens 4, any close-up diagram we draw of a particular
ray passing through it will look exactly like the corresponding
close-up diagram for some part of lens 1. Lens 4 behaves the same as lens 1.




