The foot of the rod is moving in a circle relative to the center of the rod,
with speed $v=\pi b/T$, and acceleration $v^2/(b/2)=(\pi^2/8)g$. This acceleration
is initially upward, and is greater in magnitude than $g$, so the foot of the rod
will lift off without dragging. We could also worry about whether the foot of the
rod would make contact with the floor again before the rod finishes up flat on its
back. This is a question that can be settled by graphing, or simply by inspection
of figure \figref{eg-toppling-rod} on page \pageref{fig:eg-toppling-rod}. The key here
is that the two parts of the acceleration are both independent of $m$ and $b$, so
the result is univeral, and it does suffice to check a graph in a single example.
In practical terms, this tells us something about how difficult the trick is to
do. Because $\pi^2/8=1.23$ isn't much greater than unity, a hit that is just a
little too weak (by a factor of $1.23^{1/2}=1.11$) will cause
a fairly obvious qualitative change in the results. This is easily observed
if you try it a few times with a pencil.
	


