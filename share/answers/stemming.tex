(a) There is no theoretical limit on how much normal force $F_N$ the climber can make on
the walls with each foot, so the frictional force can be made arbitrarily large. This means
that with any $\mu>0$, we can always get the vertical forces to cancel.
The theoretical minimum value of $\mu$ will be determined by the need for
the horizontal forces to cancel, so that the climber doesn't pop out of the
corner like a watermelon seed squeezed between two fingertips. The horizontal
component of the frictional force is always less than the magnitude of the
frictional force, which is turn is less than $\mu F_N$. To find the minimum
value of $\mu$, we set the static frictional force equal to $\mu F_N$.

Let the $x$ axis be along the plane that bisects the two walls, let $y$
be the horizontal direction perpendicular to $x$, and let $z$ be vertical.
Then cancellation of the forces in the $z$ direction is not the limiting factor,
for the reasons described above, and cancellation in $y$ is guaranteed by
symmetry, so the only issue is the cancellation of the $x$ forces.
We have $2F_s\cos(\theta/2)-2F_N\sin(\theta/2)=0$. Combining this with
$F_s=\mu F_N$ results in $\mu=\tan(\theta/2)$.

(b) For $\theta=0$, $\mu$ is very close to zero. That is,
we can always theoretically stay stuck between two parallel walls, simply by
pressing hard enough, even if the walls are made of ice or polished
marble with a coating of WD-40. As $\theta$ gets close to $180\degunit$,
$\mu$ blows up to infinity. We need at least some dihedral angle to do
this technique, because otherwise we're facing a flat wall, and there is nothing
to cancel the wall's normal force on our feet.

(c) The result is $99.0\degunit$, i.e., just a little wider than a right angle.
%
% calc -e "mu=1.17; theta=2atan(mu)/deg"
%    theta = 98.9589209047005
