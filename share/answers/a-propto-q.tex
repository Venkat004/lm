In the expression \begin{equation*} A = =
\frac{F_m}{m\sqrt{\left(\omega^2-\omega_\zu{o}^2\right)^2
+\omega_\zu{o}^2\omega^2Q^{-2}}} \end{equation*} from page
\pageref{resonance-amplitude}, substituting $\omega=\omega_\zu{o}$
makes the first term inside the square root vanish, which should make
the denominator pretty small, thereby producing a pretty big
amplitude. In the limit of $Q=\infty$, $Q^{-2}=0$, so the second term
vanishes, and $\omega=\omega_\zu{o}$ actually produces an infinite
amplitude. For values of $Q$ that are large but finite, we still
expect to get resonance pretty close to $\omega=\omega_\zu{o}$.
Setting $\omega=\omega_\zu{o}$ in the finite-$Q$ case, the first term
vanishes, we can simplify the square root, and the result ends up
being $A\propto 1/\sqrt{Q^{-2}}\propto Q$. This is only an
approximation, because we had to assume early on that $Q$ was large.
