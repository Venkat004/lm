(a) This occurs when the $d_i$ is infinite. Let's say
it's a converging mirror creating a virtual image, as in
problems 2 and 3. Then we'd get an infinite $d_i$ if we put
$d_o=f$, i.e., the object is at the focal point of the
mirror. The image is infinitely large, but it's also
infinitely far away, so its angular size isn't infinite; an angular
size can never be more than about $180\degunit$ since
you can't see in back of your head!.\\
(b) It's not possible to make the magnification infinite by
having $d_o=0$. The image location and object location are
related by $1/f=1/d_o-1/d_i$, so $1/d_i=1/d_o-1/f$. If $d_o$
is zero, then $1/d_o$ is infinite, $1/d_i$ is infinite, and
$d_i$ is zero as well. In other words, as $d_o$ approaches
zero, so does $d_i$, and $d_i/d_o$ doesn't blow up.
Physically, the mirror's curvature becomes irrelevant from the
point of view of a tiny flea sitting on its surface: the
mirror seems flat to the flea. So physically the magnification
would be 1, not infinity, for very small values of $d_o$.
