You'll need the result of problem \ref{hw:anganalogies} in order
to relate the energy and angular momentum of a rigidly rotating body.
Since this relationship involves a variable raised to a power,
you can't just graph the data and get the moment of inertia
directly. One way to get around this is to
manipulate one of the variables to make the graph linear.
Here is an example of this technique from
another context. Suppose you were given a table of the masses, $m$, of cubical
pieces of wood, whose sides had various lengths, $b$. You want
to find a best-fit value for the density of the wood.
The relationship is $m=\rho b^3$. The graph of $m$ versus $b$ would be a curve,
and you would not have any easy way to get the density from such
a graph. But by graphing $m$ versus $b^3$, you can produce a graph that
is linear, and whose slope equals the density.



