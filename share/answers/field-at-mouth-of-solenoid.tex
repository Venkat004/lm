The trick is to imagine putting together two identical
solenoids to make one double-length solenoid. The field of
the doubled solenoid is given by the vector sum of the two
solenoids' individual fields. At points on the axis,
symmetry guarantees that the individual fields lie along the
axis, and similarly for the total field. At the center of
one of the mouths, we thus have two parallel field vectors
of equal strength, whose sum equals the interior field. But
the interior field of the doubled solenoid is the same as
that of the individual ones, since the equation for the
field only depends on the number of turns per unit length.
Therefore the field at the center of a solenoid's mouth
equals exactly half the interior field.



