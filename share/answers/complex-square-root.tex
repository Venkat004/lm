Say we're looking for $u=\sqrt{z}$, i.e., we want a number $u$ that,
multiplied by itself, equals $z$. Multiplication multiplies the
magnitudes, so the magnitude of $u$ can be found by taking the square
root of the magnitude of $z$. Since multiplication also adds the
arguments of the numbers, squaring a number doubles its argument.
Therefore we can simply divide the argument of $z$ by two to find the
argument of $u$. This results in one of the square roots of $z$.
There is another one, which is $-u$, since $(-u)^2$ is the same as
$u^2$. This may seem a little odd: if $u$ was chosen so that doubling
its argument gave the argument of $z$, then how can the same be true
for $-u$? Well for example, suppose the argument of $z$ is
$4\degunit$. Then $\arg u=2\degunit$, and $\arg(-u)=182\degunit$.
Doubling 182 gives 364, which is actually a synonym for 4 degrees.
