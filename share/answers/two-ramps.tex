(a) We choose a coordinate system with positive pointing
to the right. Some people might expect that the ball would
slow down once it was on the more gentle ramp. This may be
true if there is significant friction, but Galileo's
experiments with inclined planes showed that when friction
is negligible, a ball rolling on a ramp has constant
acceleration, not constant speed. The speed stops increasing
as quickly once the ball is on the more gentle slope, but it
still keeps on increasing. The a-t graph can be drawn by
inspecting the slope of the v-t graph.

\anonymousinlinefig{soln-ramps-a}

(b) The ball will roll back down, so the second half of the
motion is the same as in part a. In the first (rising) half
of the motion, the velocity is negative, since the motion is
in the opposite direction compared to the positive $x$ axis.
The acceleration is again found by inspecting the slope of the v-t graph.

\anonymousinlinefig{soln-ramps-b}



