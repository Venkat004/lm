Proceeding as suggested in the hint, we form concentric
rings, each one extending from radius $b$ to radius
$b+\der b$.The area of such a ring equals its circumference
multipled by $\der b$, which is $(2\pi b)\der b$. Its charge is thus
$2\pi\sigma b\der b$.  Plugging this in to the expression from problem \ref{hw:ring-field-on-axis}
gives a contribution to the field $\der E=2\pi\sigma bka(a^2+b^2)^{-3/2}\der b$.
The total field is found by integrating this expression. The
relevant integral can be found in a table.
\begin{align*}
		E	 &=  \int_0^\infty \der E=2\pi\sigma bka(a^2+b^2)^{-3/2}\der b    \\
			 &=  2\pi\sigma ka  \int_0^\infty b (a^2+b^2)^{-3/2} \der b  \\
			 &=  2\pi\sigma ka\left[-\left(a^2+b^2\right)^{-1/2}\right]_{b=0}^\infty    \\
			 &=  2\pi\sigma k
\end{align*}



