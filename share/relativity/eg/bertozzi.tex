<% marg(-6) %>
<%
  fig(
    'electrons-limiting-speed',
    %q{Example \ref{eg:bertozzi-graph}.}
  )
%>
<% end_marg %>

\begin{eg}{Push as hard as you like \ldots}\label{eg:bertozzi-graph}
We don't have to depend on our imaginations to see what would happen if we kept on applying
a force to an object indefinitely and tried to accelerate it past $c$.
A nice experiment of this type was done by Bertozzi in 1964.
In this experiment, electrons were accelerated by an electric field $E$ through a distance
$\ell_1$.
Applying Newton's laws
gives Newtonian predictions $a_N$ for the acceleration and $t_N$ for the time 
required.\footnote{Newton's second law gives
$a_N=F/m=eE/m$. The constant-acceleration
equation $\Delta x=(1/2)at^2$ then gives $t_N=\sqrt{2m\ell_1/eE}$.}

The electrons were then allowed to fly down a pipe for a further distance $\ell_2=8.4\ \munit$
without being acted on by any force. The time of flight $t_2$ for this second distance
was used to find the final
velocity $v=\ell_2/t_2$ to which they had actually been accelerated.

Figure \figref{electrons-limiting-speed} shows the results.\footnote{To make the
low-energy portion of the graph legible, Bertozzi's highest-energy
data point is omitted.}
According to Newton, an acceleration $a_N$ acting for a time $t_N$ should produce a final velocity
$a_N t_N$. The solid line in the
graph shows the prediction of Newton's laws, which is that a constant force exerted steadily over
time will produce a velocity that rises linearly and without limit.

The experimental data, shown as black dots, clearly tell a different story. The velocity never goes
above a certain maximum value,
which we identify as $c$. The dashed line shows the predictions of special relativity, which
are in good agreement with the experimental results.
\end{eg}
