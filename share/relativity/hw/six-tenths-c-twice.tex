m4_ifelse(__problems,1,[:%
The figure illustrates a Lorentz transformation using the conventions
described in sec.~\ref{sec:relativity-basic-lorentz-transformation},
p.~\pageref{sec:relativity-basic-lorentz-transformation}.
:],[:%
The figure illustrates a Lorentz transformation using the conventions
employed in section \ref{sec:x-t-distortion}.
:])%
For simplicity, the
transformation chosen is one that lengthens one diagonal by a factor
of 2. Since Lorentz transformations preserve area, the other diagonal
is shortened by a factor of 2. Let the original frame of reference,
depicted with the square, be A, and the new one B. 
%
(a) By measuring with a ruler on the figure, show that the velocity of frame B relative to frame A is $0.6c$.
%
(b) Print out a copy of the page. With a ruler, draw a third parallelogram that represents a second successive Lorentz transformation, one
that lengthens the long diagonal by another factor of 2. Call this third frame C. Use measurements with a ruler to determine
frame C's velocity relative to frame A. Does it equal double the velocity found in part a? Explain why it should be expected to turn
out the way it does.\answercheck
% v=.6 exactly; combined v = tanh(2 atanh(.6))=.8824; verified graphically using inkscape that it's .8824
