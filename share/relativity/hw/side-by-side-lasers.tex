(a) In this chapter we've represented Lorentz transformations as distortions of a square into
various parallelograms, with the degree of distortion depending on the velocity of one
frame of reference relative to another. Suppose that one frame of reference was moving
at $c$ relative to another. Discuss what would happen in terms of distortion of a square,
and show that this is impossible by using an argument similar to the one used to rule out
transformations like the one in figure \figref{bowtie} on page \pageref{fig:bowtie}.\hwendpart
(b) Resolve the following paradox. Two pen-pointer lasers are placed side by side and aimed in parallel
directions. Their beams both travel at $c$ relative to the hardware, but each beam has a velocity of zero relative to the neighboring beam.
But the speed of light can't be zero; it's supposed to be the same in all frames of reference.
