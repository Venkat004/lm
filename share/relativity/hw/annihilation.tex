%%%%%
%%%%% This problem is used by: 0sn,7,annihilation
%%%%%
An antielectron collides with an electron that is at
rest. (An antielectron is a form of antimatter that is just
like an electron, but with the opposite charge.) The
antielectron and electron annihilate each other and produce
two gamma rays. (A gamma ray is a form of light. It has zero
mass.) Gamma ray 1 is moving in the same direction as the
antielectron was initially going, and gamma ray 2 is going
in the opposite direction. Throughout this problem, you
should work in natural units and use the notation $E$ to mean
the total mass-energy of a particle, i.e., its mass plus its
kinetic energy. Find the energies of the two gamma-rays, $E_1$
and $E_2$, in terms of $m$, the mass of an electron or
antielectron, and $E_\zu{o}$, the initial mass-energy of the
antielectron. You'll need the result of problem \ref{hw:rel-epm}a.
