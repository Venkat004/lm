The following is a list of common misconceptions about relativity. The class
will be split up into random groups, and each group will cooperate on developing
an explanation of the misconception, and then the groups will present their explanations
to the class. There may be multiple rounds, with students assigned to different
randomly chosen groups in successive rounds.

\begin{enumerate}

\item
  How can light have momentum if it has zero mass?

\item
  What does the world look like in a frame of reference moving at $c$?

\item
  Alice observes Betty coming toward her from the left at $c/2$, and  
  Carol from the right at $c/2$. Therefore Betty is moving at the speed
  of light relative to Carol.

\item
  Are relativistic effects such as length contraction and time
  dilation real, or do they just seem to be that way?

\item
  Special relativity only matters if you're moving close to the
  speed of light.

\item
  Special relativity says that everything is relative.

\item
  There is a common misconception that relativistic length contraction
  is what we would actually \emph{see}. Refute this by drawing a spacetime
  diagram for an object approaching an observer, and tracing rays of
  light emitted from the object's front and back that both reach the
  observer's eye at the same time.

\item
  When you travel close to the speed of light, your time slows down.

\item
  Is a light wave's wavelength relativistically length contracted by
  a factor of gamma?

\item
  Accelerate a baseball to ultrarelativistic speeds. Does it become a
  black hole?

\item
  Where did the Big Bang happen?

\item
  The universe can't be infinite in size, because it's only had a
  finite amount of time to expand from the point where the Big Bang
  happened.

\end{enumerate}
