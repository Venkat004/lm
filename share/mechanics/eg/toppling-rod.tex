<% marg(0) %>
<% fig(
    'eg-toppling-rod',
    'Example \ref{eg:toppling-rod}.'
  )
%>
<% end_marg %>
\begin{eg}{Toppling a rod}\label{eg:toppling-rod}
\egquestion A rod of length $b$ and mass $m$ stands upright. We want
to strike the rod at the bottom, causing it to fall and land flat.
Find the momentum, $p$, that should be delivered, in terms of $m$, $b$,
and $g$. Can this really be done without having the rod scrape on the floor?

\eganswer This is a nice example of a question that can very nearly be
answered based only on units. Since the three variables, $m$, $b$,
and $g$, all have different units, they can't be added or subtracted.
The only way to combine them mathematically is by multiplication or division.
Multiplying one of them by itself is exponentiation, so in general
we expect that the answer must be of the form
\begin{equation*}
  p = A m^j b^k g^l\eqquad,
\end{equation*}
where $A$, $j$, $k$, and $l$ are unitless constants. The result has
to have units of $\kgunit\unitdot\munit/\sunit$. To get kilograms to
the first power, we need
\begin{equation*}
  j=1\eqquad,
\end{equation*}
meters to the first power requires
\begin{equation*}
  k+l=1\eqquad,
\end{equation*}
and
seconds to the power $-1$ implies
\begin{equation*}
  l=1/2\eqquad.
\end{equation*}
We find $j=1$, $k=1/2$, and $l=1/2$, so the solution must be of the form
\begin{equation*}
  p = A m\sqrt{bg}\eqquad.
\end{equation*}
Note that no physics was required!

Consideration of units, however, won't help us to find the unitless constant
$A$. Let $t$ be the time the rod takes to fall, so that $(1/2)gt^2=b/2$.
If the rod is going to land exactly on its side, then the number of revolutions
it completes while in the air must be 1/4, or 3/4, or 5/4,  \ldots, but all the
possibilities greater than 1/4 would cause the head of the rod to collide with
the floor prematurely. The rod must therefore rotate at a rate that would
cause it to complete a full rotation in a time $T=4t$, and it has angular
momentum $L=(\pi/6)mb^2/T$.

The momentum lost by the object striking
the rod is $p$, and by conservation of momentum, this is the amount of
momentum, in the horizontal direction, that the rod acquires. In other words,
the rod will fly forward a little. However, this has no effect on the solution
to the problem. More importantly, the object striking the rod loses angular
momentum $bp/2$, which is also transferred to the rod. Equating this to the
expression above for $L$, we find $p=(\pi/12)m\sqrt{bg}$.

Finally, we need to know whether this can really be done without having the
foot of the rod scrape on the floor. The figure shows that the answer is no
for this rod of finite width, but it appears that the answer would be yes for
a sufficiently thin rod. This is analyzed further in homework problem \ref{hw:toppling-rod}
on page \pageref{hw:toppling-rod}.

\end{eg}
