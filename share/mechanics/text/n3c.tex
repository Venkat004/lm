In 1872, capitalist and former California governor Leland
\index{Stanford, Leland}Stanford asked photographer Eadweard
\index{Muybridge, Eadweard}Muybridge if he would work for
him on a project to settle a \$25,000 bet (a princely sum at
that time). Stanford's friends were convinced that a
trotting horse always had at least one foot on the ground,
but Stanford claimed that there was a moment during each
cycle of the motion when all four feet were in the air. The
human eye was simply not fast enough to settle the question.
In 1878, Muybridge finally succeeded in producing what
amounted to a motion picture of the horse, showing
conclusively that all four feet did leave the ground at one
point. (Muybridge was a colorful figure in San Francisco
history, and his acquittal for the murder of his wife's
lover was considered the trial of the century in California.)

The losers of the bet had probably been influenced by
Aristotelian reasoning, for instance the expectation that a
leaping horse would lose horizontal velocity while in the
air with no force to push it forward, so that it would be
more efficient for the horse to run without leaping. But
even for students who have converted wholeheartedly to
Newtonianism, the relationship between force and acceleration
leads to some conceptual difficulties, the main one being a
problem with the true but seemingly absurd statement that an
object can have an acceleration vector whose direction is
not the same as the direction of motion. The horse, for
instance, has nearly constant horizontal velocity, so its
$a_x$ is zero. But as anyone can tell you who has ridden a
galloping horse, the horse accelerates up and down. The
horse's acceleration vector therefore changes back and forth
between the up and down directions, but is never in the same
direction as the horse's motion. In this chapter, we will
examine more carefully the properties of the velocity,
acceleration, and force vectors. No new principles are
introduced, but an attempt is made to tie things together
and show examples of the power of the vector formulation of Newton's laws.

<% begin_sec("The Velocity Vector",0,'velocity-vector') %>\index{velocity!as a vector}\index{vector!velocity}

For motion with constant velocity, the velocity vector is
\begin{equation*}
        \vc{v} = \Delta \vc{r}/\Delta t\eqquad. \qquad \hfill \shoveright{\text{[only for constant velocity]}}
\end{equation*}
The $\Delta\vc{r}$ vector points in the direction of the motion,
and dividing it by the scalar $\Delta t$ only changes its
length, not its direction, so the velocity vector points in
the same direction as the motion.
m4_ifelse(__me,1,[:%
  When the velocity vector is not constant, we form it from the components
  $v_x=\der x/\der t$, $v_y=\der y/\der t$, and $v_z=\der z/\der t$.
  This set of three equations can be notated more compactly as
  \begin{equation*}
    \vc{v}=\der\vc{r}/\der t\eqquad.
  \end{equation*}
  This is an example of a more general rule about differentiating vectors: to differentiate a vector,
  take the derivative component by component.
  Even when the velocity vector
  is not constant, it still points along the direction of motion.

  \begin{eg}{A car bouncing on its shock absorbers}
  \egquestion
  A car bouncing on its shock absorbers has a position as a function of time given by
  \begin{equation*}
    \vc{r} = bt\hat{\vc{x}}+(c\sin\omega t)\hat{\vc{y}}\eqquad,
  \end{equation*}
  where $b$, $c$, and $\omega$ (Greek letter omega) are constants.
  Infer the units of the constants, find the velocity, and check the units of
  the result.

  \eganswer
  The components of the position vector are $bt$ and $c\sin\omega t$, and if these
  are both to have units of meters, then $b$ must have units of $\munit/\sunit$ and $c$ units of meters.
  The sine function requires a unitless input, so $\omega$ must have units of $\sunit^{-1}$ (interpreted
  as radians per second, e.g., if $c=2\pi\ \zu{rad}/\sunit$, then the car completes one cycle of vertical
  oscillation in one second).

  Differentiating component by component, we find
  \begin{equation*}
    \vc{v} = b\hat{\vc{x}}+(c\omega\cos\omega t)\hat{\vc{y}}\eqquad.
  \end{equation*}
  The units of the first component are simply the units of $b$, $\munit/\sunit$, which makes sense.
  The units of the second component are \\
  $\munit\cdot\sunit^{-1}$, which also checks out.
  \end{eg}
:],[:%
  When the velocity is not
  constant, i.e., when the $x-t$, $y-t$, and $z-t$ graphs are not
  all linear, we use the slope-of-the-tangent-line approach to
  define the components $v_x$, $v_y$, and $v_z$, from which we
  assemble the velocity vector.
  Even when the velocity vector
  is not constant, it still points along the direction of motion.
:])%

Vector addition is the correct way to generalize the
one-dimensional concept of adding velocities in relative
motion, as shown in the following example:

<% marg(70) %>
<%
  fig(
    'greyhound',
    %q{The racing greyhound's velocity vector is in the direction of its motion, i.e., tangent to
       its curved path.}
  )
%>
\spacebetweenfigs
<%
  fig(
    'eg-cross-river',
    %q{Example \ref{eg:cross-river}.}
  )
%>
<% end_marg %>
\begin{eg}{Velocity vectors in relative motion}\label{eg:cross-river}
\egquestion You wish to cross a river and arrive at a dock
that is directly across from you, but the river's current
will tend to carry you downstream. To compensate, you must
steer the boat at an angle. Find the angle $\theta $, given
the magnitude, $|\vc{v}_{WL}|$, of the water's velocity relative
to the land, and the maximum speed, $|\vc{v}_{BW}|$, of which the
boat is capable relative to the water.

\eganswer The boat's velocity relative to the land equals
the vector sum of its velocity with respect to the water and
the water's velocity with respect to the land,
\begin{equation*}
            \vc{v}_{BL} =  \vc{v}_{BW}+  \vc{v}_{WL}\eqquad.
\end{equation*}
If the boat is to travel straight across the river, i.e.,
along the $y$ axis, then we need to have $\vc{v}_{BL,x}=0$. This $x$
component equals the sum of the $x$ components of the other two vectors,
\begin{equation*}
            \vc{v}_{BL,x}  =  \vc{v}_{BW,x} +  \vc{v}_{WL,x}\eqquad,
\end{equation*}
or
\begin{equation*}
            0  =  -|\vc{v}_{BW}| \sin  \theta  + |\vc{v}_{WL}|\eqquad.
\end{equation*}
Solving for $\theta $, we find $\sin  \theta  =  |\vc{v}_{WL}|/|\vc{v}_{BW}|,$ so
\begin{equation*}
            \theta = \sin^{-1}\frac{|\vc{v}_{WL}|}{|\vc{v}_{BW}|}\eqquad.
\end{equation*}
\end{eg}

\worked{annie-oakley}{Annie Oakley}

\startdqs

\begin{dq}
Is it possible for an airplane to maintain a constant
velocity vector but not a constant $|\vc{v}|?$  How about the
opposite -- a constant $|\vc{v}|$ but not a constant velocity vector? Explain.
\end{dq}

\begin{dq}
New York and Rome are at about the same latitude, so the
earth's rotation carries them both around nearly the same
circle. Do the two cities have the same velocity vector
(relative to the center of the earth)? If not, is there any
way for two cities to have the same velocity vector?
\end{dq}

<% end_sec() %>

<% begin_sec("The Acceleration Vector",nil) %>\index{acceleration!as a vector}\index{vector!acceleration}

m4_ifelse(__me,1,[:%
When the acceleration is constant,
we have
\begin{equation*}
        \vc{a} = \Delta\vc{v}/\Delta t\eqquad, \qquad \shoveright{\text{[only for constant acceleration]}}
\end{equation*}
which can be written in terms of initial and final velocities as
\begin{equation*}
        \vc{a} = (\vc{v}_f-\vc{v}_i)/\Delta t\eqquad. \qquad \shoveright{\text{[only for constant acceleration]}}
\end{equation*}
In general, we define the acceleration vector as the derivative
  \begin{equation*}
    \vc{a}=\der\vc{v}/\der t\eqquad.
  \end{equation*}
:],[:%
When the acceleration is constant, we can define the acceleration vector as
\begin{equation*}
        \vc{a} = \Delta\vc{v}/\Delta t\eqquad, \qquad \shoveright{\text{[only for constant acceleration]}}
\end{equation*}
which can be written in terms of initial and final velocities as
\begin{equation*}
        \vc{a} = (\vc{v}_f-\vc{v}_i)/\Delta t\eqquad. \qquad \shoveright{\text{[only for constant acceleration]}}
\end{equation*}
Otherwise, we can use the type of graphical definition described in section
\ref{sec:velocity-vector} for the velocity vector.
:])

<% marg(0) %>
<%
  fig(
    'delta-v-arrows-1',
    %q{%
      A change in the magnitude of the
      velocity vector implies an acceleration.
    }
  )
%>
\spacebetweenfigs
<%
  fig(
    'delta-v-arrows-2',
    %q{%
      A change in the direction of the
      velocity vector also produces a nonzero 
      $\Delta\vc{v}$ vector, and thus a nonzero
      acceleration vector, $\Delta\vc{v}/\Delta t$.
    }
  )
%>
<% end_marg %>

Now there are two ways in which we could have a nonzero
acceleration. Either the magnitude or the direction of the
velocity vector could change. This can be visualized with
arrow diagrams as shown in figures 
\figref{delta-v-arrows-1} and \figref{delta-v-arrows-2}. Both the magnitude
and direction can change simultaneously, as when a car
accelerates while turning. Only when the magnitude of the
velocity changes while its direction stays constant do we
have a $\Delta v$ vector and an acceleration vector along
the same line as the motion.

<% self_check('acceleration-vector',<<-'SELF_CHECK'
(1) In figure \\figref{delta-v-arrows-1}, is the object speeding up, or slowing
down? (2) What would the diagram look like if $\\vc{v}_i$ was the
same as $\\vc{v}_f?$ (3) Describe how the $\\Delta \\vc{v}$ vector is
different depending on whether an object is speeding up or slowing down.
  SELF_CHECK
  ) %>

\pagebreak

The acceleration vector points in the direction that an accelerometer would point,
as in figure \ref{fig:accelerometer}.


<%
  fig(
    'accelerometer',
    %q{The car has just swerved to the right. The air freshener hanging from the
        rear-view mirror acts as an accelerometer, showing that the acceleration vector is to the right.},
    {
      'width'=>'wide',
      'sidecaption'=>true
    }
  )
%>


<% self_check('projectile-a-vector',<<-'SELF_CHECK'
In projectile motion, what direction does the acceleration vector have?
  SELF_CHECK
  ) %>

<%
  fig(
    'rappelling-reprise',
    %q{Example }+ref_workaround('eg:rappelling-reprise')+%q{.},
    {
      'width'=>'fullpage'
    }
  )
%>

\begin{eg}{Rappelling}\label{eg:rappelling-reprise}
In figure \figref{rappelling-reprise},
the rappeller's velocity has long
periods of gradual change interspersed with short periods of
rapid change. These correspond to periods of small
acceleration and force, and periods of large acceleration and force.
\end{eg}

<%
  fig(
    'horse-outlines',
    %q{Example }+ref_workaround('eg:horse-outlines')+%q{.},
    {
      'width'=>'fullpage'
    }
  )
%>

\begin{eg}{The galloping horse}\label{eg:horse-outlines}
Figure \figref{horse-outlines} on page \pageref{fig:horse-outlines}
shows outlines traced from the first, third,
fifth, seventh, and ninth frames in Muybridge's series of
photographs of the galloping horse. The estimated location
of the horse's center of mass is shown with a circle, which
bobs above and below the horizontal dashed line.

   If we don't care about calculating velocities and
accelerations in any particular system of units, then we can
pretend that the time between frames is one unit. The
horse's velocity vector as it moves from one point to the
next can then be found simply by drawing an arrow to connect
one position of the center of mass to the next. This
produces a series of velocity vectors which alternate
between pointing above and below horizontal.

   The $\Delta\vc{v}$ vector is the vector which we would have
to add onto one velocity vector in order to get the next
velocity vector in the series. The $\Delta\vc{v}$ vector
alternates between pointing down (around the time when the
horse is in the air, B) and up (around the time when the
horse has two feet on the ground, D).
\end{eg}

\startdqs

\begin{dq}
When a car accelerates, why does a bob hanging from the
rearview mirror swing toward the back of the car? Is it
because a force throws it backward? If so, what force?
Similarly, describe what happens in the other cases described above.
\end{dq}

\begin{dq}
Superman is guiding a crippled spaceship into port. The ship's
engines are not working. If Superman suddenly changes the direction
of his force on the ship, does the ship's velocity vector change
suddenly? Its acceleration vector? Its direction of motion?
\end{dq}


<% end_sec() %>

\vfill

<% begin_sec("The force vector and simple machines",nil,'force-vector') %>%
\index{force!as a vector}\index{vector!force}

Force is relatively easy to intuit as a vector. The force
vector points in the direction in which it is trying to
accelerate the object it is acting on.

<% marg(-40) %>
<%
  fig(
    'sled',
    %q{Example \ref{eg:sled}.}
  )
%>
\spacebetweenfigs
<%
  fig(
    'eg-ramp-1',
    %q{%
      The applied force
      $\vc{F}_A$ pushes the
      block up the frictionless ramp.
    }
  )
%>
\spacebetweenfigs
<%
  fig(
    'eg-ramp-3',
    %q{%
      If the block is to move at constant
      velocity, Newton's first law says that
      the three force vectors acting on it
      must add up to zero. To perform 
      vector addition, we put the vectors tip to
      tail, and in this case we are adding
      three vectors, so each one's tail goes
      against the tip of the previous one.
      Since they are supposed to add up to
      zero, the third vector's tip must come
      back to touch the tail of the first 
      vector. They form a triangle, and since the
      applied force is perpendicular to the
      normal force, it is a right triangle.
    }
  )
%>

<% end_marg %>
Since force vectors are so much easier to visualize than
acceleration vectors, it is often helpful to first find the
direction of the (total) force vector acting on an object,
and then use that to find the direction of
the acceleration vector. Newton's second law
tells us that the two must be in the same direction.

\begin{eg}{A component of a force vector}\label{eg:sled}
Figure \figref{sled}, redrawn from a classic 1920 textbook, shows
a boy pulling another child on a sled. His force has both a horizontal
component and a vertical one, but only the horizontal one accelerates
the sled. (The vertical component just partially cancels the force of gravity,
causing a decrease in the normal force between the runners and the snow.)
There are two triangles in the figure. One triangle's hypotenuse is the
rope, and the other's is the magnitude of the force. These triangles are
similar, so their internal angles are all the same, but they are not the
same triangle. One is a distance triangle, with sides measured in meters, the other
a force triangle, with sides in newtons. In both cases, the horizontal leg
is 93\% as long as the hypotenuse. It does not make sense, however, to
compare the sizes of the triangles --- the force triangle is not smaller
in any meaningful sense.
\end{eg}

\begin{eg}{Pushing a block up a ramp}\label{eg:ramp-mechanical-advantage}
\egquestion Figure \figref{eg-ramp-1} shows a block being pushed up a
frictionless ramp at constant speed by an externally applied force
$\vc{F}_A$. How much force is required, in terms of the block's
mass, $m$, and the angle of the ramp, $\theta $?

\eganswer We analyze the forces on the block and introduce notation for the other forces besides
$\vc{F}_A$:

\begin{tabular}{|p{50mm}|p{50mm}|}
\hline
\emph{force acting on block}  &   \emph{3rd-law partner} \\
\hline
ramp's normal force on block,   &  block's normal force on ramp, \\
$\vc{F}_N$, \hfill \anonymousinlinefig{../../../share/misc/arrows/1-oclock} & \hfill \anonymousinlinefig{../../../share/misc/arrows/7-oclock} \\ 
\hline
external object's force on block (type irrelevant), $\vc{F}_A$ \hfill \anonymousinlinefig{../../../share/misc/arrows/10-oclock-short}  &  block's force on external object (type irrelevant), \hfill \anonymousinlinefig{../../../share/misc/arrows/4-oclock-short}\\
\hline
planet earth's gravitational force on block, $\vc{F}_W$ \hfill $\downarrow$  &  block's gravitational force on earth, \hfill $\uparrow$\\
\hline
\end{tabular}

Because the block is being pushed up at constant speed, it
has zero acceleration, and the total force on it must be
zero. From figure \figref{eg-ramp-3}, we find
\begin{align*}
        |\vc{F}_A|      &=  |\vc{F}_W| \sin  \theta   \\
                        &=  mg \sin  \theta\eqquad.
\end{align*}

Since the sine is always less than one, the applied force is
always less than $mg$, i.e., pushing the block up the
ramp is easier than lifting it straight up. This is
presumably the principle on which the pyramids were
constructed: the ancient Egyptians would have had a hard
time applying the forces of enough slaves to equal the full
weight of the huge blocks of stone.

Essentially the same analysis applies to several other
simple machines, such as the wedge and the screw.
\end{eg}

\pagebreak

<%
  fig(
    'hw-layback',
    %q{Example \ref{eg:layback} and problem \ref{hw:layback} on p.~\pageref{hw:layback}.},
    {
      'width'=>'wide',
      'sidecaption'=>true
    }
  )
%>

\begin{eg}{A layback}\label{eg:layback}
The figure shows a rock climber 
using a technique called a layback. He can make the normal forces
$\vc{F}_{N1}$ and $\vc{F}_{N2}$ large, which has the side-effect of increasing
the frictional forces $\vc{F}_{F1}$ and $\vc{F}_{F2}$, so that he doesn't slip
down due to the gravitational (weight) force $\vc{F}_W$. The purpose of the
problem is not to analyze all of this in detail, but simply to practice
finding the components of the forces based on their magnitudes.
To keep the notation simple, let's write $F_{N1}$ for 
$|\vc{F}_{N1}|$, etc. The crack overhangs by a small, positive angle $\theta\approx9\degunit$.

In this example, we 
 determine the
$x$ component of $\vc{F}_{N1}$. The other nine components are left as an exercise to
the reader (problem \ref{hw:layback}, p.~\pageref{hw:layback}).

The easiest method is the one demonstrated in example \ref{eg:component-shortcut}
on p.~\pageref{eg:component-shortcut}. Casting vector $\vc{F}_{N1}$'s shadow on the ground,
we can tell that it would point to the left, so its $x$ component is
negative. The only two possibilities for its $x$ component are therefore
$-F_{N1}\cos\theta$ or $-F_{N1}\sin\theta$. We expect this force to have
a large $x$ component and a much smaller $y$. Since $\theta$ is small,
$\cos\theta\approx 1$, while $\sin\theta$ is small. Therefore
the $x$ component must be $-F_{N1}\cos\theta$.
\end{eg}

<% marg(-300) %>
<%
  fig(
    'eg-push-broom',
    %q{Example \ref{eg:push-broom}.}
  )
%>
<% end_marg %>

\begin{eg}{Pushing a broom}\label{eg:push-broom}
\egquestion Figure \figref{eg-push-broom} shows a man pushing a broom at an angle $\theta$ relative to the horizontal. The mass $m$
of the broom is concentrated at the brush. If the magnitude of the broom's acceleration is $a$, find the force $F_H$ that the
man must make on the handle.

\eganswer First we analyze all the forces on the brush.

\begin{tabular}{|p{50mm}|p{50mm}|}
\hline
\emph{force acting on brush}  &   \emph{3rd-law partner} \\
\hline
handle's normal force & brush's normal force \\
   on brush, $F_H$, \hfill \anonymousinlinefig{../../../share/misc/arrows/4-oclock}
    & on handle, \hfill \anonymousinlinefig{../../../share/misc/arrows/10-oclock} \\
\hline
earth's gravitational force & brush's gravitational force \\
  on brush, $mg$, \hfill $\downarrow$ & on earth, \hfill $\uparrow$ \\
\hline
floor's normal force & brush's normal force \\
  on brush, $F_N$, \hfill $\uparrow$ & on floor, \hfill $\downarrow$ \\
\hline
floor's kinetic friction force & brush's kinetic friction force \\
  on brush, $F_k$, \hfill $\leftarrow$ & on floor, \hfill $\rightarrow$ \\
\hline
\end{tabular}

Newton's second law is:
\begin{equation*}
  \vc{a} = \frac{\vc{F}_H+m\vc{g}+\vc{F}_N+\vc{F}_k}{m}\eqquad,
\end{equation*}
where the addition is vector addition.
If we actually want to carry out the vector addition of the forces, we have to do either graphical addition (as in example
\ref{eg:ramp-mechanical-advantage}) or analytic addition. Let's do analytic addition, which means finding all
the components of the forces, adding the $x$'s, and adding the $y$'s.

Most of the forces have components that are trivial to express in terms of their magnitudes,
the exception being $\vc{F}_H$, whose components we can determine
using the technique demonstrated in example \ref{eg:component-shortcut}
on p.~\pageref{eg:component-shortcut} and example \ref{eg:layback} on p.~\pageref{eg:layback}.
Using the coordinate system shown in the figure, the results are:

\begin{tabular}{ll}
F_{Hx}=F_H\cos\theta & F_{Hy}=-F_H\sin\theta \\
mg_x = 0 & mg_y = -mg \\
F_{Nx} = 0 & F_{Ny} = F_N \\
F_{kx} = -F_k & F_{ky} = 0
\end{tabular}

\noindent Note that we don't yet know the magnitudes $F_H$, $F_N$, and $F_k$. That's all right. First we need to set up
Newton's laws, and \emph{then} we can worry about solving the equations.

Newton's second law in the $x$ direction gives:
\begin{equation}\label{eqn:push-broom-x}
  a = \frac{F_H\cos\theta-F_k}{m}
\end{equation}

The acceleration in the vertical direction is zero, so Newton's second law in the $y$ direction tells us that
\begin{equation}\label{eqn:push-broom-y}
  0 = -F_H\sin\theta-mg+F_N\eqquad.
\end{equation}

Finally, we have the relationship between kinetic friction and the normal force,
\begin{equation}\label{eqn:push-broom-fk}
  F_k = \mu_k F_N\eqquad.
\end{equation}

Equations \eqref{eqn:push-broom-x}-\eqref{eqn:push-broom-fk} are three equations, which we can use to determine
the three unknowns, $F_H$, $F_N$, and $F_k$. Straightforward algebra gives
\begin{equation*}
  F_H = m \left(\frac{a+\mu_k g}{\cos\theta-\mu_k\sin\theta}\right)
\end{equation*}
\end{eg}

\worked{cargo-plane}{A cargo plane}
\worked{angle-of-repose}{The angle of repose}
\worked{wagon-uphill}{A wagon}

\startdqs

<% marg(50) %>
<%
  fig(
    'dq-pressblock',
    %q{Discussion question \ref{dq:pressblock}.},
    {
      'anonymous'=>true
    }
  )
%>
\spacebetweenfigs
<%
  fig(
    'dq-roller-coaster',
    %q{Discussion question \ref{dq:roller-coaster-mechanics}.},
    {
      'anonymous'=>true
    }
  )
%>
<% end_marg %>
\begin{dq}\label{dq:pressblock}
The figure shows a block being pressed diagonally upward
against a wall, causing it to slide up the wall. Analyze the
forces involved, including their directions.
\end{dq}

\begin{dq}\label{dq:roller-coaster-mechanics}
The figure shows a roller coaster car rolling down and
then up under the influence of gravity. Sketch the car's
velocity vectors and acceleration vectors. Pick an
interesting point in the motion and sketch a set of force
vectors acting on the car whose vector sum could have
resulted in the right acceleration vector.
\end{dq}


<% end_sec() %>
<% begin_sec("m4_ifelse(__me,1,[:More about :])Calculus With Vectors",0,'',{'calc'=>true}) %>\index{calculus!with vectors}
m4_ifelse(__me,1,[:
Our definition of the  derivative of a vector implies
:],[:
Using the unit vector notation introduced in section
\ref{sec:unit-vector-notation}, the definitions of the velocity and acceleration components
given in chapter \ref{ch:three-d} can be translated into calculus notation as
\begin{align*}
     \vc{v} &= \frac{\der x}{\der t} \hat{\vc{x}} + \frac{\der y}{\der t} \hat{\vc{y}} + \frac{\der z}{\der t} \hat{\vc{z}}\\
\intertext{and}
     \vc{a} &= \frac{\der v_x}{\der t} \hat{\vc{x}} + \frac{\der v_y}{\der t} \hat{\vc{y}} + \frac{\der v_z}{\der t} \hat{\vc{z}}\eqquad. 
\end{align*}
To make the notation less cumbersome, we generalize the
concept of the derivative to include derivatives of vectors,
so that we can abbreviate the above equations as
\begin{align*}
  \vc{v} &= \frac{\der \vc{r}}{\der t} \\
\intertext{and}
  \vc{a} &= \frac{\der \vc{v}}{\der t}\eqquad.
\end{align*}
In words, to take the derivative of a vector, you take the
derivatives of its components and make a new vector out of
those.
This definition means that the derivative of a vector
function has 
:])%
the familiar properties
\begin{align*}
  \frac{\der(c\vc{f})}{\der t} &= c \frac{\der\vc{f}}{\der t} \qquad \shoveright{\text{[$c$ is a constant]}} \\
\intertext{and}
  \frac{\der(\vc{f}+\vc{g})}{\der t} &=  \frac{\der\vc{f}}{\der t}  +  \frac{\der\vc{g}}{\der t}\eqquad.
\end{align*}
The integral of a vector is likewise defined as integrating
component by component.

\begin{eg}{The second derivative of a vector}
\egquestion Two objects have positions as functions of time
given by the equations
\begin{align*}
        \vc{r}_1  &= 3t^2\hat{\vc{x}} + t\hat{\vc{y}}
\intertext{and}
        \vc{r}_2  &= 3t^4\hat{\vc{x}} + t\hat{\vc{y}}\eqquad.
\end{align*}
Find both objects' accelerations using calculus. Could
either answer have been found without calculus?

\eganswer Taking the first derivative of each component, we find
\begin{align*}
        \vc{v}_1  &=   6t\hat{\vc{x}} + \hat{\vc{y}} \\
        \vc{v}_2  &=   12t^3\hat{\vc{x}} + \hat{\vc{y}}\eqquad,
\end{align*}
and taking the derivatives again gives acceleration,
\begin{align*}
        \vc{a}_1  &=    6\hat{\vc{x}}\\
        \vc{a}_2  &=    36t^2\hat{\vc{x}}\eqquad.
\end{align*}
The first object's acceleration could have been found
without calculus, simply by comparing the $x$ and $y$
coordinates with the constant-acceleration equation 
$\Delta x = v_\zu{o}\Delta t+\frac{1}{2}a\Delta t^2$. The
second equation, however, isn't just a second-order
polynomial in $t$, so the acceleration isn't constant, and
we really did need calculus to find the corresponding acceleration.
\end{eg}

\begin{eg}{The integral of a vector}
\egquestion Starting from rest, a flying saucer of mass $m$ 
is observed to vary its propulsion
with mathematical precision according to the equation
\begin{equation*}
  \vc{F} = bt^{42}\hat{\vc{x}} + ct^{137}\hat{\vc{y}}\eqquad. \\
\end{equation*}
(The aliens inform us that the numbers 42 and 137 have a special religious
significance for them.) Find the saucer's velocity as a function of time.

\eganswer From the given force, we can easily find the acceleration
\begin{align*}
  \vc{a} &= \frac{\vc{F}}{m} \\
         &= \frac{b}{m}t^{42}\hat{\vc{x}} + \frac{c}{m}t^{137}\hat{\vc{y}}\eqquad. 
\end{align*}
The velocity vector $\vc{v}$ is the integral with respect
to time of the acceleration,
\begin{align*}
  \vc{v} &= \int \vc{a} \der t \\
         &= \int \left( \frac{b}{m}t^{42}\hat{\vc{x}} + \frac{c}{m}t^{137}\hat{\vc{y}} \right) \der t\eqquad, \\
\intertext{and integrating component by component gives}
         &= \left( \int \frac{b}{m}t^{42} \der t \right) \hat{\vc{x}} + \left( \int \frac{c}{m}t^{137} \der t\right) \hat{\vc{y}}\\
         &= \frac{b}{43m}t^{43} \hat{\vc{x}} + \frac{c}{138m}t^{138}\hat{\vc{y}}\eqquad,
\end{align*}
where we have omitted the constants of integration, since the saucer was starting from rest.
\end{eg}

\begin{eg}{A fire-extinguisher stunt on ice}
\egquestion Prof. Puerile smuggles a fire extinguisher into a
skating rink. Climbing out onto the ice without any skates
on, he sits down and pushes off from the wall with his feet,
acquiring an initial velocity $v_\zu{o}\hat{\vc{y}}$. At $t=0$, he then
discharges the fire extinguisher at a 45-degree angle so
that it applies a force to him that is backward and to the
left, i.e., along the negative $y$ axis and the positive $x$
axis. The fire extinguisher's force is strong at first, but
then dies down according to the equation $|\vc{F}|=b-ct$,
where $b$ and $c$ are constants. Find the professor's
velocity as a function of time.

\eganswer Measured counterclockwise from the $x$ axis, the
angle of the force vector becomes 315\degunit. Breaking the
force down into $x$ and $y$ components, we have
\begin{align*}
        F_x     &=    |\vc{F}| \cos  315\degunit  \\
             &=     (b-ct)  \\
        F_y     &=    |\vc{F}| \sin  315\degunit  \\
             &=     (-b+ct)\eqquad.
\end{align*}
In unit vector notation, this is
\begin{equation*}
        F     =     (b-ct)\hat{\vc{x}} +  (-b+ct)\hat{\vc{y}}\eqquad.
\end{equation*}
Newton's second law gives
\begin{align*}
        \vc{a}     &=    \vc{F}/m \\
                   &=    \frac{b-ct}{\sqrt{2}m}\hat{\vc{x}} + \frac{-b+ct}{\sqrt{2}m}\hat{\vc{y}}\eqquad.
\end{align*}
To find the velocity vector as a function of time, we need
to integrate the acceleration vector with respect to time,
\begin{align*}
        \vc{v}     &=  \int \vc{a} \der t    \\
                   &=  \int \left(  \frac{b-ct}{\sqrt{2}m} \: \hat{\vc{x}} + \frac{-b+ct}{\sqrt{2}m} \: \hat{\vc{y}} \right) \der t    \\
                   &=  \frac{1}{\sqrt{2}m} \int \bigl[ (b-ct) \: \hat{\vc{x}} \: + \: (-b+ct) \: \hat{\vc{y}} \bigr] \der t    \\
\end{align*}
A vector function can be integrated component by component,
so this can be broken down into two integrals,

\begin{align*}
        \vc{v}     &=  \frac{\hat{\vc{x}}}{\sqrt{2}m} \int  (b-ct) \der t \: + \: \frac{\hat{\vc{y}}}{\sqrt{2}m}\int (-b+ct) \der t    \\
             &=   \left(\frac{bt-\frac{1}{2}ct^2}{\sqrt{2}m}+\text{constant \#1}\right)\hat{\vc{x}} 
                 + \left(\frac{-bt+\frac{1}{2}ct^2}{\sqrt{2}m}+\text{constant \#2}\right)\hat{\vc{y}}
\end{align*}
Here the physical significance of the two constants of
integration is that they give the initial velocity. Constant
\#1 is therefore zero, and constant \#2 must equal
$v_\zu{o}$. The final result is
\begin{equation*}
        \vc{v}     =      \left(\frac{bt-\frac{1}{2}ct^2}{\sqrt{2}m}\right)\hat{\vc{x}} 
                 + \left(\frac{-bt+\frac{1}{2}ct^2}{\sqrt{2}m}+v_\zu{o}\right)\hat{\vc{y}}\eqquad.
\end{equation*}
\end{eg}

<% end_sec() %>\begin{summary}

\begin{summarytext}[]%don't want two headings in a row that say summary

The velocity vector points in the direction of the object's
motion. Relative motion can be described by vector
addition of velocities.

The acceleration vector need not point in the same direction
as the object's motion. We use the word ``acceleration'' to
describe any change in an object's velocity vector, which
can be either a change in its magnitude or a change in its direction.

An important application of the vector addition of forces is
the use of Newton's first law to analyze mechanical systems.

\end{summarytext}

\end{summary}
