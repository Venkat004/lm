<% begin_sec("Work: The Transfer of Mechanical Energy",0,'work-concept') %>

  <% begin_sec("The concept of work") %>

The mass contained in a closed system is a conserved
quantity, but if the system is not closed, we also have ways
of measuring the amount of mass that goes in or out. The
water company does this with a meter that records your water use.

Likewise, we often
have a system that is not closed, and
would like to know how much energy comes in or out.
Energy, however, is not a physical substance
like water, so energy transfer cannot be measured with the
same kind of meter. How can we tell, for instance, how much
useful energy a tractor can ``put out'' on one tank of gas?

The law of conservation of energy guarantees that all the
chemical energy in the gasoline will reappear in some form,
but not necessarily in a form that is useful for doing farm
work. Tractors, like cars, are extremely inefficient, and
typically 90\% of the energy they consume is converted
directly into heat, which is carried away by the exhaust and
the air flowing over the radiator. We wish to distinguish
the energy that comes out directly as heat from the energy
that serves to accelerate a trailer or to plow a field, so
we define a technical meaning of the ordinary word
``\index{work!defined}work'' to express the distinction:

\begin{lessimportant}[definition of work]
Work is the amount of energy transferred into or out of a system,
not counting energy transferred by heat conduction.
\end{lessimportant}

<% self_check('work-scalar-joules',<<-'SELF_CHECK'
Based on this definition, is work a vector, or a scalar? What are its units?
  SELF_CHECK
  ) %>

<% marg(82) %>
<%
  fig(
    'work-concept-diagram',
    %q{Work is a transfer of energy.}
  )
%>
\spacebetweenfigs
<%
  fig(
    'tractor-at-cliff',
    %q{%
      The tractor raises the weight over
      the pulley, increasing its gravitational
      potential energy.
    }
  )
%>
\spacebetweenfigs
<%
  fig(
    'tractor-pulling-trailer',
    %q{%
      The tractor accelerates the trailer,
      increasing its kinetic energy.
    }
  )
%>
\spacebetweenfigs
<%
  fig(
    'tractor-pulling-plow',
    %q{%
      The tractor pulls a plow. Energy is
      expended in frictional heating of the
      plow and the dirt, and in breaking dirt
      clods and lifting dirt up to the sides of
      the furrow.
    }
  )
%>
<% end_marg %>
The \index{heat conduction!distinguished from work}\index{work!distinguished
from heat conduction}\index{conduction of heat!distinguished
from work}conduction of heat is to be distinguished from
heating by friction. When a hot potato heats up your hands
by conduction, the energy transfer occurs without any force,
but when friction heats your car's brake shoes, there is a
force involved. The transfer of energy with and without a
force are measured by completely different methods, so we
wish to include heat transfer by frictional heating under
the definition of work, but not heat transfer by conduction.
The definition of work could thus be restated as the amount
of energy transferred by forces.

  <% end_sec() %> % The concept of work

  <% begin_sec("Calculating work as force multiplied by distance") %>\index{work!calculated as $Fd$}

The examples in figures \figref{tractor-at-cliff}-\figref{tractor-pulling-plow}
show that there are
many different ways in which energy can be transferred. Even
so, all these examples have two things in common:

\begin{enumerate}
 \item A force is involved.

 \item The tractor travels some distance as it does the work.
\end{enumerate}

In \figref{tractor-at-cliff}, the increase in the height of the weight,
$\Delta y$, is the same as the distance the tractor travels,
which we'll call $d$. For simplicity, we discuss the case where the
tractor raises the weight at constant speed, so that there
is no change in the kinetic energy of the weight, and we
assume that there is negligible friction in the pulley, so
that the force the tractor applies to the rope is the same
as the rope's upward force on the weight. By Newton's first
law, these forces are also of the same magnitude as the
earth's gravitational force on the weight. The increase in
the weight's potential energy is given by $F\Delta y$, so
the work done by the tractor on the weight equals $Fd$,
the product of the force and the distance moved:
\begin{equation*}
                W  =  Fd\eqquad.
\end{equation*}
In example \figref{tractor-pulling-trailer}, the tractor's force on the trailer
accelerates it, increasing its kinetic energy. If frictional
forces on the trailer are negligible, then the increase in
the trailer's kinetic energy can be found using the same
algebra that was used on page \pageref{pe-derivation} to find the
potential energy due to gravity. Just as in example \figref{tractor-at-cliff}, we have
\begin{equation*}
                W  =  Fd\eqquad.
\end{equation*}
Does this equation always give the right answer? Well, sort
of. In example \figref{tractor-pulling-plow}, there are two quantities of work you
might want to calculate: the work done by the tractor on the
plow and the work done by the plow on the dirt.\label{plow-discussion} These two
quantities can't both equal $Fd$. Most of the energy transmitted
through the cable goes into frictional heating of the plow and
the dirt. The work done by the plow on the dirt is less than
the work done by the tractor on the plow, by an amount equal to
the heat absorbed by the plow. It turns out that the equation
$W=Fd$ gives the work done by the tractor, not the
work done by the plow. How are you supposed to know when the
equation will work and when it won't? The somewhat complex
answer is postponed until section \ref{sec:when-is-work-fd}. Until then, we will
restrict ourselves to examples in which $W=Fd$
gives the right answer; essentially the reason the ambiguities come
up is that when one surface is slipping past another, $d$ may be
hard to define, because the two surfaces move different distances.

<%
  fig(
    'baseball-pitch',
    %q{%
      The baseball pitcher put kinetic energy
      into the ball, so he did work on it. To do the greatest possible amount of
      work, he applied the greatest possible force over the greatest possible distance.
    },
    {
      'width'=>'wide',
      'sidecaption'=>true
    }
  )
%>

We have also been using examples in which the force is in
the same direction as the motion, and the force is constant.
(If the force was not constant, we would have to represent
it with a function, not a symbol that stands for a number.)
To summarize, we have:

\begin{important}[rule for calculating work (simplest version)]
The work done by a force can be calculated as
\begin{equation*}
                W  =  Fd\eqquad,
\end{equation*}
if the force is constant and in the same direction as the
motion. Some ambiguities are encountered in cases such
as kinetic friction.
\end{important}

\pagebreak[4]

<% marg(0) %>
<%
  fig(
    'earthquake-fault-slip',
    %q{\raggedright Example \ref{eg:earthquake-fault-slip}.}
  )
%>
<% end_marg %>



\begin{eg}{Mechanical work done in an earthquake}\label{eg:earthquake-fault-slip}
\egquestion In 1998, geologists discovered evidence for a big
prehistoric earthquake in Pasadena, between 10,000 and
15,000 years ago. They found that the two sides of the fault
moved 6.7 m relative to one another, and estimated that
the force between them was $1.3\times10^{17}\ \nunit$. How much
energy was released?

\eganswer Multiplying the force by the distance gives
$9\times10^{17}\ \junit$. For comparison, the Northridge earthquake
of 1994, which killed 57 people and did 40 billion dollars
of damage, released 22 times less energy.
\end{eg}

\begin{eg}{The fall factor}\label{eg:fall-factor}
Counterintuitively, the rock climber may be in more danger in figure \subfigref{fall-factor}{1}
than later when she gets up to position \subfigref{fall-factor}{2}.
<% marg(8) %>
<%
  fig(
    'fall-factor',
    %q{Example \ref{eg:fall-factor}. Surprisingly, the climber is in more danger at 1 than at 2.
       The distance $d$ is the amount by which the rope will stretch while work is done to transfer
       the kinetic energy of a fall out of her body.}
  )
%>
<% end_marg %>

Along her route, the climber
has placed removable rock anchors (not shown) and carabiners attached to the anchors.
She clips the rope into each carabiner so that it can travel but can't pop out. In both 1 and 2, she
has ascended a certain distance above her last anchor, so that if she falls, she will drop through a height
$h$ that is about twice this distance, and this fall height is about the same in both cases.
In fact, $h$ is somewhat larger than twice her height above the last anchor, because the rope is
intentionally designed to stretch under the big force of a falling climber who suddenly brings it taut.

To see why we want a stretchy rope, consider the equation $F=W/d$ in the case where $d$ is zero;
$F$ would theoretically become infinite. In a fall, the climber loses a fixed amount of gravitational
energy $mgh$. This is transformed into an equal amount of kinetic energy as she falls, and eventually
this kinetic energy has to be transferred out of her body when the rope comes up taut.
If the rope was not stretchy, then the distance traveled at the point where the rope attaches to
her harness would be zero, and the force exerted would theoretically be infinite. 
Before the rope reached the theoretically infinite tension $F$ it would break (or her back
would break, or her anchors would be pulled out of the rock).
We want the rope
to be stretchy enough to make $d$ fairly big, so that dividing $W$ by $d$ gives a small 
force.\footnote{Actually $F$ isn't constant, because the tension in the rope increases steadily as
it stretches, but this is irrelevant to the present analysis.}

In \subfigref{fall-factor}{1} and \subfigref{fall-factor}{2}, the fall $h$ is about the same.
What is different is the length $L$ of rope that has been paid out. A longer rope can stretch
more, so the distance $d$ traveled after the ``catch'' is proportional to $L$.
Combining $F=W/d$, $W\propto h$, and $d\propto L$, we have $F\propto h/L$.
For these reasons, rock climbers define a \emph{fall factor} $f=h/L$. The larger fall factor 
in \subfigref{fall-factor}{1} is more
dangerous.\index{fall factor}
\end{eg}

  <% end_sec() %> % Calculating work as force multiplied by distance

  <% begin_sec("Machines can increase force, but not work.") %>

Figure \ref{fig:pulling-stump} shows a pulley arrangement for doubling the
force supplied by the tractor (book 1, section 5.6). The
tension in the left-hand rope is equal throughout, assuming
negligible friction, so there are two forces pulling the
pulley to the left, each equal to the original force exerted
by the tractor on the rope. This doubled force is transmitted
through the right-hand rope to the stump.

<%
  fig(
    'pulling-stump',
    %q{%
      The pulley doubles the force the tractor
      can exert on the stump.
    },
    {
      'width'=>'wide',
      'sidecaption'=>true
    }
  )
%>

It might seem as though this arrangement would also double
the work done by the tractor, but look again. As the tractor
moves forward 2 meters, 1 meter of rope comes around the
pulley, and the pulley moves 1 m to the left. Although the
pulley exerts double the force on the stump, the pulley and
stump only move half as far, so the work done on the stump
is no greater that it would have been without the pulley.

The same is true for any mechanical arrangement that
increases or decreases force, such as the gears on a
ten-speed bike. You can't get out more work than you put in,
because that would violate conservation of energy. If you
shift gears so that your force on the pedals is amplified,
the result is that you just have to spin the pedals more times.

  <% end_sec() %> % Machines can increase force, but not work

  <% begin_sec("No work is done without motion.") %>

It strikes most students as nonsensical when they are told
that if they stand still and hold a heavy bag of cement,
they are doing no work on the bag. Even if it makes sense
mathematically that $W=Fd$ gives zero when $d$ is
zero, it seems to violate common sense. You would certainly
become tired! The solution is simple.
Physicists have taken over the common word ``work'' and
given it a new technical meaning, which is the transfer of
energy. The energy of the bag of cement is not changing, and
that is what the physicist means by saying no work is done on the bag.

There is a transformation of energy, but it is taking place
entirely within your own muscles, which are converting
chemical energy into heat. Physiologically, a human muscle
is not like a tree limb, which can support a weight
indefinitely without the expenditure of energy. Each muscle
cell's contraction is generated by zillions of little
molecular machines, which take turns supporting the tension.
When a particular molecule goes on or off duty, it moves,
and since it moves while exerting a force, it is doing work.
There is work, but it is work done by one molecule in a
muscle cell on another.


<% marg(0) %>
<%
  fig(
    'spring-pos-neg-work',
    %q{%
      %
      Whenever energy is transferred out of the spring, the same amount has to be transferred into the ball, and vice
      versa. As the spring compresses, the ball is doing positive work on the spring (giving up its KE and transferring
      energy into the spring as PE), and as it decompresses the ball is doing negative work (extracting energy).
    }
  )
%>
<% end_marg %>

  <% end_sec() %> % No work is done without motion

  <% begin_sec("Positive and negative work") %>
\index{work!positive and negative}

When object A transfers energy to object B, we say that A
does positive work on B. B is said to do negative work
on A. In other words, a machine like a tractor is defined as
doing positive work. This use of the plus and minus signs
relates in a logical and consistent way to their use in
indicating the directions of force and motion in one
dimension. In figure \figref{spring-pos-neg-work}, suppose we
choose a coordinate system with the $x$ axis pointing to the
right. Then the force the spring exerts on the ball is
always a positive number. The ball's motion, however,
changes directions. The symbol $d$ is really just a shorter
way of writing the familiar quantity $\Delta x$, whose
positive and negative signs indicate direction.

While the ball is moving to the left, we use $d<0$ to
represent its direction of motion, and the work done by the
spring, $Fd$, comes out negative. This indicates that
the spring is taking kinetic energy out of the ball, and
accepting it in the form of its own potential energy.

As the ball is reaccelerated to the right, it has $d>0$,
$Fd$ is positive, and the spring does positive work on
the ball. Potential energy is transferred out of the spring
and deposited in the ball as kinetic energy.

In summary:

\begin{lessimportant}[rule for calculating work (including cases of negative work)]
The work done by a force can be calculated as
\begin{equation*}
                W  =  Fd\eqquad,
\end{equation*}
if the force is constant and along the same line as the
motion. The quantity $d$ is to be interpreted as a synonym
for $\Delta x$, i.e., positive and negative signs are used
to indicate the direction of motion. Some ambiguities are
encountered in cases such as kinetic friction.
\end{lessimportant}

<% self_check('ball-does-work-on-spring',<<-'SELF_CHECK'
In figure
\\figref{spring-pos-neg-work}, what about the work done by the ball on the spring?\\linebreak[4]
  SELF_CHECK
  ) %>

There are many examples where the transfer of energy out of
an object cancels out the transfer of energy in. When the
tractor pulls the plow with a rope, the rope does negative
work on the tractor and positive work on the plow. The total
work done by the rope is zero, which makes sense, since it
is not changing its energy.

It may seem that when your arms do negative work by lowering
a bag of cement, the cement is not really transferring
energy into your body. If your body was storing potential
energy like a compressed spring, you would be able to raise
and lower a weight all day, recycling the same energy. The
bag of cement does transfer energy into your body, but your
body accepts it as heat, not as potential energy. The
tension in the muscles that control the speed of the motion
also results in the conversion of chemical energy to heat,
for the same physiological reasons discussed previously in
the case where you just hold the bag still.

<% marg(100) %>
<%
  fig(
    'muscle-contraction',
    %q{%
      \emph{Left:} No mechanical work occurs in the man's body while he holds himself motionless.
      There is a transformation of chemical energy into heat, but this happens at the microscopic level inside the tensed muscles.
      \emph{Right:} When the woman lifts herself, her arms do positive work on her body, transforming chemical energy into
      gravitational potential energy and heat. On the way back down, the arms' work is negative; gravitational potential energy is transformed
      into heat. (In exercise physiology, the man is said to be doing isometric exercise, while the woman's is concentric and then eccentric.)
    }
  )
%>
\spacebetweenfigs
<%
  fig(
    'brake-shoes',
    %q{%
      Because the force is in the opposite
      direction compared to the motion, the
      brake shoe does negative work on the
      drum, i.e., accepts energy from it in the
      form of heat.
    }
  )
%>
<% end_marg %>
One of the advantages of electric cars over gasoline-powered
cars is that it is just as easy to put energy back in a
battery as it is to take energy out. When you step on the
brakes in a gas car, the brake shoes do negative work on the
rest of the car. The kinetic energy of the car is transmitted
through the brakes and accepted by the brake shoes in the
form of heat. The energy cannot be recovered. Electric cars,
however, are designed to use regenerative braking. The
brakes don't use friction at all. They are electrical, and
when you step on the brake, the negative work done by the
brakes means they accept the energy and put it in the
battery for later use. This is one of the reasons why an
electric car is far better for the environment than a gas
car, even if the ultimate source of the electrical energy
happens to be the burning of oil in the electric company's
plant. The electric car recycles the same energy over and
over, and only dissipates heat due to air friction and
rolling resistance, not braking. (The electric company's
power plant can also be fitted with expensive pollution-reduction
equipment that would be prohibitively expensive or bulky
for a passenger car.)

\vfill\pagebreak[4]\startdqs

\begin{dq}
Besides the presence of a force, what other things
differentiate the processes of frictional heating and heat conduction?
\end{dq}

\begin{dq}
Criticize the following incorrect statement: ``A force
doesn't do any work unless it's causing the object to move.''
\end{dq}



\begin{dq}\label{dq:stopping-distance}
To stop your car, you must first have time to react, and
then it takes some time for the car to slow down. Both of
these times contribute to the distance you will travel
before you can stop. The figure shows how the average
stopping distance increases with speed. Because the stopping
distance increases more and more rapidly as you go faster,
the rule of one car length per 10 m.p.h. of speed is not
conservative enough at high
speeds. In terms of work and
kinetic energy, what is the reason for the more rapid
increase at high speeds?
\end{dq}

<%
  fig(
    'dq-stopping-distances',
    %q{Discussion question \ref{dq:stopping-distance}.},
    {
      'width'=>'wide',
      'sidecaption'=>true,
      'anonymous'=>true
    }
  )
%>

  <% end_sec() %> % Positive and negative work

<% end_sec() %> % Work: The Transfer of Mechanical Energy

<% begin_sec("Work in Three Dimensions",3) %>\index{work!in three dimensions}

<% marg(0) %>
<%
  fig(
    'force-at-points-of-compass',
    %q{%
      A force can do positive, negative, or zero
      work, depending on its direction relative to the direction of the motion.
    }
  )
%>
<% end_marg %>

  <% begin_sec("A force perpendicular to the motion does no work.") %>

Suppose work is being done to change an object's kinetic
energy. A force in the same direction as its motion will
speed it up, and a force in the opposite direction will slow
it down. As we have already seen, this is described as doing
positive work or doing negative work on the object. All the
examples discussed up until now have been of motion in one
dimension, but in three dimensions the force can be at any
angle $\theta $ with respect to the direction of motion.

What if the force is perpendicular to the direction of
motion? We have already seen that a force perpendicular to
the motion results in circular motion at constant speed. The
kinetic energy does not change, and we conclude that no work
is done when the force is perpendicular to the motion.

So far we have been reasoning about the case of a single
force acting on an object, and changing only its kinetic
energy. The result is more generally true, however. For
instance, imagine a hockey puck sliding across the ice. The
ice makes an upward normal force, but does not transfer
energy to or from the puck.

  <% end_sec() %> % A force perpendicular to the motion does no work

  <% begin_sec("Forces at other angles") %>\index{work!force not collinear with motion}

Suppose the force is at some other angle with respect to the
motion, say $\theta=45\degunit$. Such a force could be broken
down into two components, one along the direction of the
motion and the other perpendicular to it. The force vector
equals the vector sum of its two components, and the
principle of vector addition of forces thus tells us that
the work done by the total force cannot be any different
than the sum of the works that would be done by the two
forces by themselves. Since the component perpendicular to
the motion does no work, the work done by the force must be
\begin{equation*}
                W  =  F_{\parallel} |\vc{d}|\eqquad,                   \qquad  \text{[work done by
a constant force]}
\end{equation*}
where the vector $\vc{d}$ is simply a less cumbersome version of
the notation $\Delta\vc{r}$. This result can be rewritten via trigonometry as
\begin{equation*}
                W  =  |\vc{F}| |\vc{d}| \cos  \theta\eqquad.            \qquad
\text{[work done by a constant force]}
\end{equation*}
Even though this equation has vectors in it, it depends only
on their magnitudes, and the magnitude of a vector is a
scalar. Work is therefore still a scalar quantity, which
only makes sense if it is defined as the transfer of energy.
Ten gallons of gasoline have the ability to do a certain
amount of mechanical work, and when you pull in to a
full-service gas station you don't have to say ``Fill 'er up
with 10 gallons of south-going gas.''
<% marg(80) %>
<%
  fig(
    'force-at-another-angle',
    %q{%
      Work is only done by
      the component of the force parallel to the motion.
    }
  )
%>
<% end_marg %>

Students often wonder why this equation involves a cosine
rather than a sine, or ask if it would ever be a sine. In
vector addition, the treatment of sines and cosines seemed
more equal and democratic, so why is the cosine so special
now? The answer is that if we are going to describe, say, a
velocity vector, we must give both the component \emph{parallel}
to the $x$ axis and the component \emph{perpendicular} to
the $x$ axis (i.e., the $y$ component). In calculating work,
however, the force component perpendicular to the motion is
irrelevant --- it changes the direction of motion without
increasing or decreasing the energy of the object on which
it acts. In this context, it is \emph{only} the parallel
force component that matters, so only the cosine occurs.

<% marg(0) %>
<%
  fig(
    'breaking-trail',
    %q{Self-check. (Breaking Trail, by Walter E. Bohl.)}
  )
%>
<% end_marg %>

<% self_check('breaking-trail',<<-'SELF_CHECK'
(a) Work is the transfer of energy. According to this
definition, is the horse in the picture doing work on the
pack? (b) If you calculate work by the method described in
this section, is the horse in figure \\figref{breaking-trail} doing work on the pack?
  SELF_CHECK
  ) %>

\begin{eg}{Pushing a broom}
\egquestion If you exert a force of 21 N on a push broom, at
an angle 35 degrees below horizontal, and walk for 5.0 m,
how much work do you do? What is the physical significance
of this quantity of work?

\eganswer Using the second equation above, the work done equals
\begin{equation*}
                (21\ \nunit)(5.0\ \munit)(\cos  35\degunit)  =  86\ \junit\eqquad.
\end{equation*}
The form of energy being transferred is heat in the floor
and the broom's bristles. This comes from the chemical
energy stored in your body. (The majority of the calories
you burn are dissipated directly as heat inside your body
rather than doing any work on the broom. The 86 J is only
the amount of energy transferred through the broom's handle.)
\end{eg}

\begin{eg}{A violin}
As a violinist draws the bow across a string, the bow hairs exert
both a normal force and a kinetic frictional force on the string. The normal
force is perpendicular to the direction of motion, and does no work.
However, the frictional force is in the same direction as the motion
of the bow, so it does work: energy is transferred to the string,
causing it to vibrate.

One way of playing a violin more loudly is to use longer
strokes. Since $W=Fd$, the greater distance results in more work.

A second way of getting a louder sound is to press the bow more
firmly against the strings. This increases the normal force, and
although the normal force itself does no work, an increase in the
normal force has the side effect of increasing the frictional force,
thereby increasing $W=Fd$.

The violinist moves the bow back and forth, and sound is produced on both
the ``up-bow'' (the stroke toward the player's left) and the ``down-bow''
(to the right). One may, for example, play a series of notes in alternation
between up-bows and down-bows. However, if the notes are of unequal length,
the up and down motions tend to be unequal, and if the player is not
careful, she can run out of bow in the middle of a note! To keep this
from happening, one can move the bow more quickly on the shorter notes,
but the resulting increase in $d$ will make the shorter notes louder than
they should be. A skilled player compensates by reducing the force.
\end{eg}

  <% end_sec() %> % Forces at other angles
<% end_sec() %> % Work in Three Dimensions

% Dot product is here and not optional for Mechanics, later in chapter and optional for LM.
<% begin_if(__me) %>
  m4_include(__share/text/dot_product.tex)
<% end_if %>

<% begin_sec("Varying Force") %>\index{work!done by a varying force}

Up until now we have done no actual calculations of work in
cases where the force was not constant. 
m4_ifelse(__me,1,[:%
The question of how
to treat such cases is mathematically analogous to the issue
of how to generalize the equation 
$(\text{distance})=(\text{velocity})(\text{time})$
to cases where the velocity was not constant. We have to make
the equation into an integral:
\begin{equation*}
  W = \int F \der x
\end{equation*}
The examples in this section are ones in which the force is
varying, but is always along the same line as the motion.
:],[:%
The question of how
to treat such cases is mathematically analogous to the issue
of how to generalize the equation 
$(\text{distance})=(\text{velocity})(\text{time})$
to cases where the velocity was not constant. There, we
found that the correct generalization was to find the area
under the graph of velocity versus time. The equivalent
thing can be done with work:

\begin{lessimportant}[general rule for calculating work]
The work done by a force $F$ equals the area under the curve
on a graph of $F_{\parallel}$ versus $x$. (Some ambiguities are
encountered in cases such as kinetic friction.)
\end{lessimportant}
The examples in this section are ones in which the force is
varying, but is always along the same line as the motion, so
$F$ is the same as $F_{\parallel}$.
:])

<% self_check('work-constant-and-not-constant',<<-'SELF_CHECK'
In which of the following examples would it be OK to
calculate work using $Fd$, and in which ones would you
have to m4_ifelse(__me,1,[:integrate:],[:use the area under the $F-x$ graph:])?

(a) A fishing boat cruises with a net dragging behind it.

(b) A magnet leaps onto a refrigerator from a distance.

(c) Earth's gravity does work on an outward-bound space probe.
  SELF_CHECK
  ) %>

m4_ifelse(__me,1,[:%
<% marg(37) %>
<%
  fig(
    'cart-on-spring',
    %q{%
      The spring does work on the cart.
      (Unlike the ball in section \ref{sec:work-concept}, the cart
      is attached to the spring.)
    }
  )
%>
<% end_marg %>
\begin{eg}{Work done by a spring}
An important and straightforward example is the calculation
of the work done by a spring that obeys Hooke's law,
\begin{equation*}
 F \approx -k(x-x_\zu{o})\eqquad,
\end{equation*}
where $x_\zu{o}$ is the equilibrium position and
the minus sign is because this is the force being exerted by
the spring, not the force that would have to act on the
spring to keep it at this position. That is, if the position
of the cart in figure \figref{cart-on-spring}
is to the right of equilibrium, the spring pulls
back to the left, and vice-versa.
Integrating, we find that the work done between $x_1$ and $x_2$ is
\begin{equation*}
  W=\left.-\frac{1}{2}k(x-x_\zu{o})^2\right|_{x_1}^{x_2}\eqquad. 
\end{equation*}
\end{eg}

\begin{eg}{Work done by gravity}\label{eg:work-done-by-gravity}
Another important example
is the work done by gravity when the change in height is not
small enough to assume a constant force. Newton's law of gravity is
\begin{equation*}
 F = \frac{GMm}{r^2}\eqquad,
\end{equation*}
which can be integrated to give
\begin{align*}
 W &= \int_{r_1}^{r_2} \frac{GMm}{r^2} \der r\\
 &= -GMm\left(\frac{1}{r_2}-\frac{1}{r_1}\right)\eqquad.
\end{align*}
\end{eg}
:],[:%
%------------- LM version
<% marg(37) %>
<%
  fig(
    'cart-on-spring',
    %q{%
      The spring does work on the cart.
      (Unlike the ball in section \ref{sec:work-concept}, the cart
      is attached to the spring.)
    }
  )
%>
\spacebetweenfigs
<%
  fig(
    'triangular-work-graph',
    %q{%
      The area of the shaded triangle
      gives the work done by the spring as
      the cart moves from the equilibrium
      position to position $x$.
    }
  )
%>
<% end_marg %>

An important and straightforward example is the calculation
of the work done by a spring that obeys Hooke's law,
\begin{equation*}
 F \approx -k\left(x-x_\zu{o}\right)\eqquad.
\end{equation*}
The minus sign is because this is the force being exerted by
the spring, not the force that would have to act on the
spring to keep it at this position. That is, if the position
of the cart in figure \figref{cart-on-spring}
is to the right of equilibrium, the spring pulls
back to the left, and vice-versa.

We calculate the work done when the spring is initially at
equilibrium and then decelerates the car as the car moves to
the right. The work done by the spring on the cart equals
the minus area of the shaded triangle, because the triangle
hangs below the $x$ axis. The area of a triangle is half its
base multiplied by its height, so
\begin{equation*}
  W = -\frac{1}{2}k\left(x-x_\zu{o}\right)^2\eqquad.
\end{equation*}
This is the amount of kinetic energy lost by the cart as the
spring decelerates it.

It was straightforward to calculate the work done by the
spring in this case because the graph of $F$ versus $x$ was
a straight line, giving a triangular area. But if the curve
had not been so geometrically simple, it might not have been
possible to find a simple equation for the work done, or an
equation might have been derivable only using calculus.
Optional section \ref{sec:calculus-applied-to-work} gives an important example of such an
application of calculus.


<%
  fig(
    'cno-cycle',
    %q{Example \ref{eg:cno-cycle}.},
    {
      'width'=>'wide',
      'sidecaption'=>true
    }
  )
%>

\begin{eg}{Energy production in the sun}\label{eg:cno-cycle}
The sun produces energy through nuclear reactions in which
nuclei collide and stick together. The figure depicts one
such reaction, in which a single proton (hydrogen nucleus)
collides with a carbon nucleus, consisting of six protons
and six neutrons. Neutrons and protons attract other
neutrons and protons via the strong nuclear force, so as the
proton approaches the carbon nucleus it is accelerated. In
the language of energy, we say that it loses nuclear
potential energy and gains kinetic energy. Together, the
seven protons and six neutrons make a nitrogen nucleus.
Within the newly put-together nucleus, the neutrons and
protons are continually colliding, and the new proton's
extra kinetic energy is rapidly shared out among all the
neutrons and protons. Soon afterward, the nucleus calms down
by releasing some energy in the form of a gamma ray, which
helps to heat the sun.

The graph shows the force between the carbon nucleus and the
proton as the proton is on its way in, with the distance in
units of femtometers (1 fm=$10^{-15}$ m). Amusingly, the
force turns out to be a few newtons: on the same order of
magnitude as the forces we encounter ordinarily on the human
scale. Keep in mind, however, that a force this big exerted
on a single subatomic particle such as a proton will produce
a truly fantastic acceleration (on the order of $10^{27}\ \munit/\sunit^2$!).

Why does the force have a peak around $x=3$ fm, and become
smaller once the proton has actually merged with the
nucleus? At $x=3$ fm, the proton is at the edge of the crowd
of protons and neutrons. It feels many attractive forces
from the left, and none from the right. The forces add up to
a large value. However if it later finds itself at the
center of the nucleus, $x=0$, there are forces pulling it
from all directions, and these force vectors cancel out.

We can now calculate the energy released in this reaction by
using the area under the graph to determine the amount of
mechanical work done by the carbon nucleus on the proton.
(For simplicity, we assume that the proton came in ``aimed''
at the center of the nucleus, and we ignore the fact that it
has to shove some neutrons and protons out of the way in
order to get there.) The area under the curve is about 17
squares, and the work represented by each square is
\begin{equation*}
                (1\ \nunit)(10^{-15}\ \munit)  =  10^{-15}\ \junit\eqquad,
\end{equation*}
so the total energy released is about
\begin{equation*}
                (10^{-15}\ \junit/\text{square})(17\ \text{squares}) =  1.7\times10^{-14}\ \junit\eqquad.
\end{equation*}
This may not seem like much, but remember that this is only
a reaction between the nuclei of two out of the zillions of
atoms in the sun. For comparison, a typical \emph{chemical}
reaction between two atoms might transform on the order of
$10^{-19}$ J of electrical potential energy into heat ---
100,000 times less energy!

As a final note, you may wonder why reactions such as these
only occur in the sun. The reason is that there is a
repulsive electrical force between nuclei. When two nuclei
are close together, the electrical forces are typically
about a million times weaker than the nuclear forces, but
the nuclear forces fall off much more quickly with distance
than the electrical forces, so the electrical force is the
dominant one at longer ranges. The sun is a very hot gas, so
the random motion of its atoms is extremely rapid, and a
collision between two atoms is sometimes violent enough to
overcome this initial electrical repulsion.
\end{eg}

<% end_sec() %> % Varying Force

<% begin_sec("Applications of Calculus",nil,'calculus-applied-to-work',{'calc'=>true}) %>
\index{work!calculated with calculus}

The student who has studied integral calculus will recognize
that the graphical rule given in the previous section can be
reexpressed as an integral,
\begin{equation*}
 W = \int_{x_1}^{x_2}F \der x\eqquad.
\end{equation*}
We can then immediately find by the fundamental theorem of
calculus that force is the derivative of work with
respect to position,
\begin{equation*}
 F = \frac{\der W}{\der x}\eqquad.
\end{equation*}
For example, a crane raising a one-ton block on the moon
would be transferring potential energy into the block at
only one sixth the rate that would be required on Earth, and
this corresponds to one sixth the force.

Although the work done by the spring could be calculated
without calculus using the area of a triangle, there are
many cases where the methods of calculus are needed in order
to find an answer in closed form. The most important example
is the work done by gravity when the change in height is not
small enough to assume a constant force. Newton's law of gravity is
\begin{equation*}
 F = \frac{GMm}{r^2}\eqquad,
\end{equation*}
which can be integrated to give

\begin{align*}
 W &= \int_{r_1}^{r_2} \frac{GMm}{r^2} \der r\\
 &= -GMm\left(\frac{1}{r_2}-\frac{1}{r_1}\right)\eqquad.
\end{align*}

\vfill
%------------- end LM version
:])
<% end_sec() %> % Applications of Calculus --- FIXME -- shouldn't this be inside the conditional for LM?
<% begin_sec("Work and Potential Energy",3) %>
\index{work!related to potential energy}
\index{potential energy!related to work}

The techniques for calculating work can also be applied to
the calculation of potential energy. If a certain force
depends only on the distance between the two participating
objects, then the energy released by changing the distance
between them is defined as the potential energy, and the
amount of potential energy lost equals minus the work done by the force,
\begin{equation*}
                \Delta PE  =  -W\eqquad.
\end{equation*}
The minus sign occurs because positive work indicates that
the potential energy is being expended and converted to some other form.

It is sometimes convenient to pick some arbitrary position
as a reference position, and derive an equation for once and
for all that gives the potential energy relative to this position
\begin{equation*}
                PE_x  =  -W_{\text{ref}\rightarrow x}\eqquad.    
\qquad  \text{[potential energy at a point $x$]}
\end{equation*}
To find the energy transferred into or out of potential
energy, one then subtracts two different values of this equation.

These equations might almost make it look as though work and
energy were the same thing, but they are not. First,
potential energy measures the energy that a system
\emph{has} stored in it, while work measures how much energy
is \emph{transferred} in or out. Second, the techniques for
calculating work can be used to find the amount of energy
transferred in many situations where there is no potential
energy involved, as when we calculate the amount of kinetic
energy transformed into heat by a car's brake shoes.

\begin{eg}{A toy gun}
\egquestion A toy gun uses a \index{work!done by a spring}\index{potential
energy!of a spring}\index{spring!potential energy of}\index{spring!work
done by}spring with a spring constant of 10 N/m to shoot a
ping-pong ball of mass 5 g. The spring is compressed to 10
cm shorter than its equilibrium length when the gun is
loaded. At what speed is the ball released?

\eganswer The equilibrium point is the natural choice for a
reference point. Using the equation found previously
for the work, we have
\begin{equation*}
                PE_x  =   \frac{1}{2}k\left(x-x_\zu{o}\right)^2\eqquad.
\end{equation*}
The spring loses contact with the ball at the equilibrium
point, so the final potential energy is
\begin{equation*}
                PE_f          =  0\eqquad.
\end{equation*}
The initial potential energy is
\begin{align*}
 PE_i &= \frac{1}{2}(10\ \nunit/\munit)(0.10\ \munit)^2\eqquad. \\
 &= 0.05\ \junit.
\end{align*}
The loss in potential energy of 0.05 J means an increase
in kinetic energy of the same amount. The velocity of the
ball is found by solving the equation $KE=(1/2)mv^2$ for $v$,
\begin{align*}
 v &= \sqrt{\frac{2KE}{m}}\\
 &= \sqrt{\frac{(2)(0.05\ \junit)}{0.005\ \kgunit}}\\
 &= 4\ \munit/\sunit\eqquad.
\end{align*}
\end{eg}

<% marg(0) %>
<%
  fig(
    'pe-grav-versus-r',
    %q{Example \ref{eg:pe-grav}, gravitational potential energy as a function of distance.}
  )
%>
<% end_marg %>

\begin{eg}{Gravitational potential energy}\label{eg:pe-grav}\index{potential energy!gravitational}
\egquestion We have already found the equation
$\Delta PE = -F\Delta y$ for the gravitational potential
energy when the change in height is not enough to cause a
significant change in the gravitational force $F$. What if
the change in height is enough so that this assumption is no
longer valid? Use the equation $W=GMm(1/r_2-1/r_1)$ derived in
m4_ifelse(__me,1,example \ref{eg:work-done-by-gravity},section \ref{sec:calculus-applied-to-work})
to find the potential energy, using $r=\infty$
as a reference point.

\eganswer The potential energy equals minus the work that
would have to be done to bring the object from $r_1=\infty$
 to $r= r_2$, which is
\begin{equation*}
                   PE  =  -\frac{GMm}{r}\eqquad.
\end{equation*}
This is simpler than the equation for the work, which is an
example of why it is advantageous to record an equation for
potential energy relative to some reference point, rather
than an equation for work.
\end{eg}

Although the equations derived in the previous two examples
may seem arcane and not particularly useful except for toy
designers and rocket scientists, their usefulness is
actually greater than it appears. The equation for the
potential energy of a spring can be adapted to any other
case in which an object is compressed, stretched, twisted,
or bent. While you are not likely to use the equation for
gravitational potential energy for anything practical, it is
directly analogous to an equation that is extremely useful
in chemistry, which is the equation for the potential energy
of an electron at a distance $r$ from the nucleus of its
atom. As discussed in more detail later in the course, the
electrical force between the electron and the nucleus is
proportional to $1/r^2$, just like the gravitational force
between two masses. Since the equation for the force is of
the same form, so is the equation for the potential energy.

\vfill

<%
  fig(
    'voyager',
    %q{%
      The twin Voyager space 
      probes were perhaps the greatest scientific successes of the space program. Over a
      period of decades, they flew by all the planets of the
      outer solar system, probably accomplishing more of
      scientific interest than the entire space shuttle program
      at a tiny fraction of the cost. Both Voyager probes
      completed their final planetary flybys with speeds greater
      than the escape velocity at that distance from the sun,
      and so headed on out of the solar system on hyperbolic
      orbits, never to return. Radio contact has been lost, and
      they are now likely to travel interstellar space for billions
      of years without colliding with anything or being detected
      by any intelligent species.
    },
    {
      'width'=>'wide',
      'sidecaption'=>true
    }
  )
%>

\startdqs

\begin{dq}
What does the graph of $PE=(1/2)k\left(x-x_\zu{o}\right)^2$ look like as a function
of $x$? Discuss the physical significance of its features.
\end{dq}

\begin{dq}
What does the graph of $PE=-GMm/r$ look like as a
function of $r?$ Discuss the physical significance of its
features. How would the equation and graph change if some
other reference point was chosen rather than $r=\infty$?
\end{dq}

\begin{dq}
Starting at a distance $r$ from a planet of mass $M$, how
fast must an object be moving in order to have a hyperbolic
orbit, i.e., one that never comes back to the planet? This
velocity is called the escape velocity. Interpreting the
result, does it matter in what direction the velocity is?
Does it matter what mass the object has? Does the object
escape because it is moving too fast for gravity to act on it?
\end{dq}

\begin{dq}
Does a spring have an ``escape velocity?''
\end{dq}

\begin{dq}
m4_ifelse(__lm_series,1,[:Calculus-based question::]) If the form of energy being
transferred is potential energy, then the equations 
$F=\der W/\der x$ and $W=\int F \der x$
become
$F=-\der PE/\der x$ and $PE=-\int F \der x$.
How would you then apply the following
calculus concepts: zero derivative at minima and maxima, and
the second derivative test for concavity up or down.
\end{dq}

<% end_sec() %> % Work and Potential Energy

<% begin_sec("When Does Work Equal Force Times Distance?",nil,'when-is-work-fd',{'optional'=>true}) %>\index{work!not equal to $Fd$}

In the example of the tractor pulling the plow discussed on page
\pageref{plow-discussion},
the work did not equal $Fd$.
The purpose of this section is to explain more fully how the
quantity $Fd$ can and cannot be used. To simplify
things, I write $Fd$ throughout this section, but more
generally everything said here would be true for the area
under the graph of $F_{\parallel}$ versus $d$.

The following two \index{work-kinetic energy theorem}theorems
allow most of the ambiguity to be cleared up.

\begin{lessimportant}[the work-kinetic-energy theorem]
The change in kinetic energy associated with the motion of
an object's center of mass is related to the total force acting
on it and to the distance traveled by its center of mass
according to the equation $\Delta KE_{cm}=F_\text{total}d_{cm}$.
\end{lessimportant}

<% marg(300) %>
\formatlikecaption{%
\begin{flushleft}\textbf{The work-KE theorem}\end{flushleft}
\begin{flushleft}\textit{Proof}\end{flushleft}
      For simplicity, we have assumed $F_\text{total}$ to be constant, and therefore
      $a_\text{cm}=F_\text{total}/m$ is also constant, and the constant-acceleration equation
      \begin{equation*}
        v_\text{cm,f}^2=v_\text{cm,i}^2+2a_\text{cm}\Delta x_\text{cm}
      \end{equation*}
      applies. Multiplying by $m/2$ on both sides and applying Newton's second law gives
      \begin{equation*}
        KE_\text{cm,f}^2=KE_\text{cm,i}^2+F_\text{total}\Delta x_\text{cm},
      \end{equation*}
      which is the result that was to be proved.

\begin{flushleft}\textit{Further interpretation}\end{flushleft}
The logical structure of this book is that although Newton's laws are discussed
before conservation laws, the conservation laws are taken to be fundamental, since
they are true even in cases where Newton's laws fail.
Many treatments of this subject present the work-KE theorem as a proof that kinetic
energy behaves as $(1/2)mv^2$. This is a matter of taste, but
one can just as well rearrange the equations in the proof above
to solve for the unknown $a_\text{cm}$ and prove Newton's second law as a
consequence of conservation of energy. Ultimately we have a great deal of freedom
in choosing which equations to take as definitions, which to take as empirically
verified laws of nature, and which to take as theorems.

\noindent Regardless of how we slice things, we require both mathematical consistency and consistency
with experiment. As described on p.~\pageref{ke-logic}, the work-KE theorem is
an important part of this interlocking system of relationships.\label{work-ke-logic}
}%
<% end_marg %>

A proof is given in the sidebar, along with some interpretation of how this result relates to the
logical structure of our presentation. Note that despite the traditional
name, it does not necessarily tell the amount of work done,
since the forces acting on the object could be changing
other types of energy besides the KE associated with
its center of mass motion.

The second theorem does relate directly to work:

\begin{lessimportant}
When a contact force acts between two objects and the two surfaces
do not slip past each other, the work done equals $Fd$, where $d$ is
the distance traveled by the point of contact.
\end{lessimportant}

\noindent This one has no generally accepted name, so we refer to it
simply as the second theorem.

A great number of physical situations can be analyzed with
these two theorems, and often it is advantageous to apply
both of them to the same situation.

\begin{eg}{An ice skater pushing off from a wall}
The work-kinetic energy theorem tells us how to calculate
the skater's kinetic energy if we know the amount of force
and the distance her center of mass travels while she is pushing off.

The second theorem tells us that the wall does no work on
the skater. This makes sense, since the wall does not have
any source of energy.
\end{eg}

\begin{eg}{Absorbing an impact without recoiling?}
\egquestion Is it possible to absorb an impact without
recoiling? For instance, would a brick wall ``give'' at all
if hit by a ping-pong ball?

\eganswer There will always be a recoil. In the example
proposed, the wall will surely have some energy transferred
to it in the form of heat and vibration. The second theorem
tells us that we can only have nonzero work if the distance
traveled by the point of contact is nonzero.
\end{eg}

m4_ifelse(__lm_series,1,[:\pagebreak[4]:])

\begin{eg}{Dragging a refrigerator at constant velocity}
Newton's first law tells us that the total force on the
refrigerator must be zero: your force is canceling the
floor's kinetic frictional force. The work-kinetic energy
theorem is therefore true but useless. It tells us that
there is zero total force on the refrigerator, and that the
refrigerator's kinetic energy doesn't change.

The second theorem tells us that the work you do equals
your hand's force on the refrigerator multiplied by the
distance traveled. Since we know the floor has no source of
energy, the only way for the floor and refrigerator to gain
energy is from the work you do. We can thus calculate the
total heat dissipated by friction in the refrigerator and the floor.

Note that there is no way to find how much of the heat is
dissipated in the floor and how much in the refrigerator.
\end{eg}

\begin{eg}{Accelerating a cart}
If you push on a cart and accelerate it, there are two
forces acting on the cart: your hand's force, and the static
frictional force of the ground pushing on the wheels in
the opposite direction.

Applying the second theorem to your force tells us how to
calculate the work you do.

Applying the second theorem to the floor's force tells us
that the floor does no work on the cart. There is no motion
at the point of contact, because the atoms in the floor are
not moving. (The atoms in the surface of the wheel are also
momentarily at rest when they touch the floor.) This makes
sense, since the floor does not have any source of energy.

The work-kinetic energy theorem refers to the total force,
and because the floor's backward force cancels part of your
force, the total force is less than your force. This tells
us that only part of your work goes into the kinetic energy
associated with the forward motion of the cart's center of
mass. The rest goes into rotation of the wheels.
\end{eg}

<% end_sec() %> % When Does Work Equal Force Times Distance?

% Dot product is here and optional for LM, earlier in chapter and not optional for Mechanics.
<% begin_if(__me==0) %>
  m4_include(__share/text/dot_product.tex)
<% end_if %>

%--- Mechanics has proof of uniqueness of dot product:
m4_ifelse(__me,1,[:
<% begin_sec("Uniqueness of the dot product",nil,'dot-product-uniquness',{'optional'=>true}) %>\label{subsec:dotproduct}%label for both SN and Mechanics
        In this section I prove that the vector dot product is unique, in the sense that there is
        no other possible way to define it that is consistent with rotational invariance and that
        reduces appropriately to ordinary multiplication in one dimension.

        Suppose we want to find some way to multiply two vectors to get a scalar, and we don't know
        how this operation should be defined. Let's consider what we would get
        by performing this operation on various combinations of the unit vectors
        $\hat{\vc{x}}$, $\hat{\vc{y}}$, and $\hat{\vc{z}}$.
        Rotational invariance requires that we handle the three coordinate axes
        in the same way, without giving special treatment to any of them, so we must
        have $\hat{\vc{x}}\cdot\hat{\vc{x}}=\hat{\vc{y}}\cdot\hat{\vc{y}}=\hat{\vc{z}}\cdot\hat{\vc{z}}$
        and $\hat{\vc{x}}\cdot\hat{\vc{y}}=\hat{\vc{y}}\cdot\hat{\vc{z}}=\hat{\vc{z}}\cdot\hat{\vc{x}}$.
        This is supposed to be a way of generalizing ordinary multiplication, so for consistency
        with the property $1\times1=1$ of ordinary numbers, the result
        of multiplying a magnitude-one vector by itself had better be the scalar 1, so
        $\hat{\vc{x}}\cdot\hat{\vc{x}}=\hat{\vc{y}}\cdot\hat{\vc{y}}=\hat{\vc{z}}\cdot\hat{\vc{z}}=1$.
        Furthermore, there is no way to satisfy rotational invariance unless we define the mixed
        products to be zero,
        $\hat{\vc{x}}\cdot\hat{\vc{y}}=\hat{\vc{y}}\cdot\hat{\vc{z}}=\hat{\vc{z}}\cdot\hat{\vc{x}}=0$;
        for example, a 90-degree rotation of our frame of reference about the $z$ axis
        reverses the sign of $\hat{\vc{x}}\cdot\hat{\vc{y}}$, but rotational invariance requires
        that $\hat{\vc{x}}\cdot\hat{\vc{y}}$ produce the same result either way, and zero is the
        only number that stays the same when we reverse its sign. Establishing these six
        products of unit vectors suffices to define the operation in general, since any
        two vectors that we want to multiply can be broken down into components, e.g.,
        $(2\hat{\vc{x}}+3\hat{\vc{z}})\cdot\hat{\vc{z}}
        =2\hat{\vc{x}}\cdot\hat{\vc{z}}+3\hat{\vc{z}}\cdot\hat{\vc{z}}=0+3=3$. Thus by
        requiring rotational invariance and consistency with multiplication of ordinary
        numbers, we find that there is only one possible way to define a multiplication
        operation on two vectors that gives a scalar as the result. (There is, however,
        a different operation, discussed in chapter \ref{ch:angular-momentum}, which multiplies two vectors
        to give a vector.)
<% end_sec() %> % Uniqueness of the dot product
:])
