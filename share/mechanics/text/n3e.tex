Cruise your radio dial today and try to find any popular
song that would have been imaginable without Louis
Armstrong. By introducing solo improvisation into jazz,
Armstrong took apart the jigsaw puzzle of popular music and
fit the pieces back together in a different way. In the same
way, Newton reassembled our view of the universe. Consider
the titles of some recent physics books written for the
general reader: The God Particle, Dreams of a Final Theory.
Without Newton, such attempts at
universal understanding would not merely have seemed a
little pretentious, they simply would not have occurred to anyone.

<% marg(100) %>
<%
  fig(
    'kepler',
    %q{%
      Johannes Kepler found a mathematical
      description of the motion of the planets, which led to Newton's
      theory of gravity.
    }
  )
%>
<% end_marg %>
This chapter is about Newton's theory of gravity, which he
used to explain the motion of the planets as they orbited
the sun. Whereas this book has concentrated on Newton's laws
of motion, leaving gravity as a dessert, Newton tosses off
the laws of motion in the first 20 pages of the Principia
Mathematica and then spends the next 130 discussing the
motion of the planets. Clearly he saw this as the crucial
scientific focus of his work. Why? Because in it he showed
that the same laws of motion applied to the heavens as to
the earth, and that the gravitational force that made an
apple fall was the same as the force that kept the earth's
motion from carrying it away from the sun. What was radical
about Newton was not his laws of motion but his concept of a
universal science of physics.

<% begin_sec("Kepler's Laws",0) %>\index{Kepler's laws}

<% marg(65) %>
<%
  fig(
    'tycho-brahe',
    %q{%
      Tycho Brahe made his name as an
      astronomer by showing that the bright
      new star, today called a supernova,
      that appeared in the skies in 1572 was
      far beyond the Earth's atmosphere.
      This, along with Galileo's discovery of
      sunspots, showed that contrary to Aristotle, the heavens were not perfect
      and unchanging. Brahe's fame as an
      astronomer brought him patronage
      from King Frederick II, allowing him to
      carry out his historic high-precision
      measurements of the planets' motions.
      A contradictory character, Brahe enjoyed lecturing other nobles about the
      evils of dueling, but had lost his own
      nose in a youthful duel and had it replaced with a prosthesis made of an
      alloy of gold and silver. Willing to 
      endure scandal in order to marry a peasant, he nevertheless used the feudal
      powers given to him by the king to
      impose harsh forced labor on the inhabitants of his parishes. The result
      of their work, an Italian-style palace
      with an observatory on top, surely
      ranks as one of the most luxurious
      science labs ever built. Kepler described Brahe as dying
      of a ruptured bladder after falling from a wagon on the way
      home from a party, but other contemporary accounts and modern medical
      analysis suggest mercury poisoning, possibly as a result of court intrigue.
    }
  )
%>
<% end_marg %>
Newton wouldn't have been able to figure out \emph{why} the
planets move the way they do if it hadn't been for the
astronomer Tycho Brahe\index{Brahe, Tycho} (1546-1601) and his protege Johannes
Kepler\index{Kepler, Johannes} (1571-1630), who together came up with
the first simple and accurate description of \emph{how} the
planets actually do move. The difficulty of their task is
suggested by figure \figref{retrograde}, which shows how the
relatively simple orbital motions of the earth and Mars
combine so that as seen from earth Mars appears to be
staggering in loops like a drunken sailor.

<%
  fig(
    'retrograde',
    %q{%
      As the Earth and Mars revolve around the
      sun at different rates, the combined effect of their motions makes
      Mars appear to trace a strange, looped path across the background
      of the distant stars.
    },
    {
      'width'=>'wide'
    }
  )
%>

Brahe, the last of the great naked-eye astronomers,
collected extensive data on the motions of the planets over
a period of many years, taking the giant step from the
previous observations' accuracy of about 10 minutes of arc
(10/60 of a degree) to an unprecedented 1 minute. The
quality of his work is all the more remarkable considering
that his observatory consisted of four giant brass
protractors mounted upright in his castle in Denmark. Four
different observers would simultaneously measure the
position of a planet in order to check for mistakes and
reduce random errors.

With Brahe's death, it fell to his former assistant Kepler
to try to make some sense out of the volumes of data.
Kepler, in contradiction to his late boss, had formed a
prejudice, a correct one as it turned out, in favor of the
theory that the earth and planets revolved around the sun,
rather than the earth staying fixed and everything rotating
about it. Although motion is relative, it is not just a
matter of opinion what circles what. The earth's rotation
and revolution about the sun make it a noninertial reference
frame, which causes detectable violations of Newton's laws
when one attempts to describe sufficiently precise
experiments in the earth-fixed frame. Although such direct
experiments were not carried out until the 19th century,
what convinced everyone of the sun-centered system in the
17th century was that Kepler was able to come up with a
surprisingly simple set of mathematical and geometrical
rules for describing the planets' motion using the
sun-centered assumption. After 900 pages of calculations and
many false starts and dead-end ideas, Kepler finally
synthesized the data into the following \index{Kepler's laws}three laws:


\index{Kepler's laws!elliptical orbit law}
\begin{lessimportant}[Kepler's elliptical orbit law]
The planets orbit the sun in
elliptical orbits with the sun at one focus.
\end{lessimportant}

\index{Kepler's laws!equal-area law}
\begin{lessimportant}[Kepler's equal-area law]
The line connecting a planet to the sun sweeps out equal
areas in equal amounts of time.
\end{lessimportant}

\index{Kepler's laws!law of periods}
\begin{lessimportant}[Kepler's law of periods]
The time required for a planet to orbit the sun, called its
period, is proportional to the long axis of the ellipse
raised to the 3/2 power. The constant of proportionality is
the same for all the planets.
\end{lessimportant}

Although the planets' orbits are ellipses rather than
circles, most are very close to being circular.  The earth's
orbit, for instance, is only flattened by 1.7\% relative to
a circle.  In the special case of a planet in a circular
orbit, the two foci (plural of ``focus'') coincide at the
center of the circle, and Kepler's elliptical orbit law thus
says that the circle is centered on the sun. The equal-area
law implies that a planet in a circular orbit moves around
the sun with constant speed.  For a circular orbit, the law
of periods then amounts to a statement that the time for one
orbit is proportional to $r^{3/2}$, where $r$ is the radius. If
all the planets were moving in their orbits at the same
speed, then the time for one orbit would simply depend on
the circumference of the circle, so it would only be
proportional to $r$ to the first power. The more drastic
dependence on $r^{3/2}$ means that the outer planets must be
moving more slowly than the inner planets.

<% end_sec() %>
<% begin_sec("Newton's Law of Gravity",0,'newton-gravity') %>\index{Newton, Isaac!law of gravity}

<% begin_sec("The sun's force on the planets obeys an inverse square law.") %>

<% marg(55) %>
<%
  fig(
    'kepler-ellipse-stretch',
    %q{%
      An ellipse is a circle that has been
      distorted by shrinking and stretching
      along perpendicular axes.
    }
  )
%>
\spacebetweenfigs
<%
  fig(
    'kepler-ellipse-foci',
    %q{%
      An ellipse can be constructed by tying
      a string to two pins and drawing like
      this with the pencil stretching the string
      taut.  Each pin constitutes one focus
      of the ellipse.
    }
  )
%>
\spacebetweenfigs
<%
  fig(
    'kepler-equal-area',
    %q{%
      If the time interval taken by the planet to move from P to Q is equal to the time
      interval from R to S, then according to Kepler's equal-area law, the two shaded
      areas are equal. The planet is moving faster during interval RS than it did during
      PQ, which Newton later determined was due to the sun's gravitational force 
      accelerating it. The equal-area law predicts exactly how much it will speed up.
    }
  )
%>

<% end_marg %>
Kepler's laws were a beautifully simple explanation of what
the planets did, but they didn't address why they moved as
they did. Did the sun exert a force that pulled a planet
toward the center of its orbit, or, as suggested by
Descartes, were the planets circulating in a whirlpool of
some unknown liquid? Kepler, working in the Aristotelian
tradition, hypothesized not just an inward force exerted by
the sun on the planet, but also a second force in the
direction of motion to keep the planet from slowing down.
Some speculated that the sun attracted the planets magnetically.

Once Newton had formulated his laws of motion and taught
them to some of his friends, they began trying to connect
them to Kepler's laws. It was clear now that an inward force
would be needed to bend the planets' paths. This force was
presumably an attraction between the sun and each planet.
(Although the sun does accelerate in response to the
attractions of the planets, its mass is so great that the
effect had never been detected by the prenewtonian
astronomers.) Since the outer planets were moving slowly
along more gently curving paths than the inner planets,
their accelerations were apparently less. This could be
explained if the sun's force was determined by distance,
becoming weaker for the farther planets. Physicists were
also familiar with the noncontact forces of electricity and
magnetism, and knew that they fell off rapidly with
distance, so this made sense.

In the approximation of a circular orbit, the magnitude of
the sun's force on the planet would have to be
\begin{equation}
    F=   ma=      mv^2/r \qquad .
\end{equation}
Now although this equation has the magnitude, $v$, of the
velocity vector in it, what Newton expected was that there
would be a more fundamental underlying equation for the
force of the sun on a planet, and that that equation would
involve the distance, $r$, from the sun to the object, but
not the object's speed, $v$ --- motion doesn't make
objects lighter or heavier.

<% self_check('gravity-independent-of-v',<<-'SELF_CHECK'
If eq. [1] really was generally applicable, what would
happen to an object released at rest in some empty region
of the solar system?
  SELF_CHECK
  ) %>

Equation [1] was thus a useful piece of information which
could be related to the data on the planets simply because
the planets happened to be going in nearly circular orbits,
but Newton wanted to combine it with other equations and
eliminate $v$ algebraically in order to find a deeper truth.

To eliminate $v$, Newton used the equation
\begin{equation}
        v    =   \frac{\text{circumference}}{T} =    \frac{2\pi r}{T}   \qquad .
\end{equation}
Of course this equation would also only be valid for planets
in nearly circular orbits. Plugging this into eq. [1]
to eliminate $v$ gives
\begin{equation}
        F     =  \frac{4\pi^2mr}{T^2}     \qquad   .    
\end{equation}
This unfortunately has the side-effect of bringing in the
period, $T$, which we expect on similar physical grounds
will not occur in the final answer. That's where the
circular-orbit case, $T \propto r^{3/2}$,  of \index{Kepler's
laws!law of periods}Kepler's law of periods comes in. Using
it to eliminate $T$ gives a result that depends only on the
mass of the planet and its distance from the sun:
\begin{multline*}
    F\propto m/r^2   \qquad   . \hfill \shoveright{\text{[force of the sun on a planet of mass}}\\
                                       \shoveright{\text{$m$ at a distance $r$ from the sun; same}}\\
                                       \text{proportionality constant for all the planets]}
\end{multline*}
(Since Kepler's law of periods is only a proportionality,
the final result is a proportionality rather than an
equation, so there is no point in hanging on to the
factor of $4\pi ^2$.)

As an example, the ``twin planets'' Uranus and Neptune have
nearly the same mass, but Neptune is about twice as far from
the sun as Uranus, so the sun's gravitational force on
Neptune is about four times smaller.

<% self_check('fill-in-inverse-square',<<-'SELF_CHECK'
Fill in the steps leading from equation [3] to $F\\propto m/r^2$.
  SELF_CHECK
  ) %>

<% marg(150) %>
<%
  fig(
    'earth-moon-apple',
    %q{%
      The moon's acceleration is
      $60^2=3600$ times smaller than the apple's.
    }
  )
%>
<% end_marg %>

<% end_sec() %>
<% begin_sec("The forces between heavenly bodies are the same type of force as terrestrial gravity.") %>

OK, but what kind of force was it? It probably wasn't
magnetic, since magnetic forces have nothing to do with
mass. Then came Newton's great insight. Lying under an apple
tree and looking up at the moon in the sky,  he saw an apple
fall. Might not the earth also attract the moon with the
same kind of gravitational force? The moon orbits the earth
in the same way that the planets orbit the sun, so maybe the
earth's force on the falling apple, the earth's force on the
moon, and the sun's force on a planet were all the
same type of force.

There was an easy way to test this hypothesis numerically.
If it was true, then we would expect the gravitational
forces exerted by the earth to follow the same $F\propto m/r^2$
rule as the forces exerted by the sun, but with a different
constant of proportionality appropriate to the earth's
gravitational strength.  The issue arises now of how to
define the distance, $r$, between the earth and the apple.
An apple in England is closer to some parts of the earth
than to others, but suppose we take $r$ to be the distance
from the center of the earth to the apple, i.e., the radius
of the earth.  (The issue of how to measure $r$ did not
arise in the analysis of the planets' motions because the
sun and planets are so small compared to the distances
separating them.)  Calling the proportionality constant $k$, we have
\begin{align*}
        F_\text{earth on apple}    &=     k \: m_\text{apple} / r_\text{earth}^2  \\
        F_\text{earth on moon}     &=     k \: m_\text{moon}  / d_\text{earth-moon}^2   \qquad   .
\end{align*}
Newton's second law says $a=F/m$, so
\begin{align*}
        a_\text{apple}  &=  k \: / \: r_{earth}^2  \\
        a_\text{moon}   &=  k \: / \: d_\text{earth-moon}^2 \qquad .
\end{align*}
The Greek astronomer Hipparchus had already found 2000 years
before that the distance from the earth to the moon was
about 60 times the radius of the earth, so if Newton's
hypothesis was right, the acceleration of the moon would
have to be $60^2=3600$ times less than the acceleration
of the falling apple.

Applying $a=v^2/r$ to the acceleration of the moon yielded
an acceleration that was indeed 3600 times smaller than 
$9.8\ \munit/\sunit^2$, and Newton was convinced he had unlocked the secret
of the mysterious force that kept the moon and planets in their orbits.

<% end_sec() %>
<% begin_sec("Newton's law of gravity") %>

The proportionality $F\propto  m/r^2$ for the gravitational
force on an object of mass $m$ only has a consistent
proportionality constant for various objects if they are
being acted on by the gravity of the same object. Clearly
the sun's gravitational strength is far greater than the
earth's, since the planets all orbit the sun and do not
exhibit any very large accelerations caused by the earth (or
by one another). What property of the sun gives it its great
gravitational strength? Its great volume?  Its great mass?
Its great temperature? Newton reasoned that if the force was
proportional to the mass of the object being acted on, then
it would also make sense if the determining factor in the
gravitational strength of the object exerting the force was
its own mass. Assuming there were no other factors affecting
the gravitational force, then the only other thing needed to
make quantitative predictions of gravitational forces would
be a proportionality constant. Newton called that proportionality
constant $G$, so here is the complete form of the law of gravity he hypothesized.
\begin{important}[Newton's law of gravity]
\begin{multline*}
        F  =  \frac{Gm_1m_2}{r^2}    \hfill  \shoveright{\text{[gravitational force between objects of mass}}\\
                                                  \shoveright{\text{ $m_1$ and $m_2$, separated by a distance $r$; $r$ is not}}\\
                                                   \text{the radius of anything ]}
\end{multline*}
\end{important}

<% marg(50) %>
<%
  fig(
    'what-g-is',
    %q{%
      Students often have a hard time understanding
      the physical meaning of $G$. It's just a proportionality constant that tells you
      how strong gravitational forces are. If you could change it, all the gravitational
      forces all over the universe would get stronger or weaker.
      Numerically, the gravitational attraction between
      two 1-kg masses separated by a distance of 1 m is $6.67\times10^{-11}\ \nunit$,
      and this is what $G$ is in SI units.
    }
  )
%>
<% end_marg %>
Newton conceived of gravity as an attraction between any two
masses in the universe. The constant $G$ tells us how
many newtons the attractive force is for two 1-kg masses
separated by a distance of 1 m. The experimental
determination of $G$ in ordinary units (as opposed to the
special, nonmetric, units used in astronomy) is described in
section \ref{sec:weighing-the-earth}. This difficult measurement was not accomplished
until long after Newton's death.

\begin{eg}{The units of $G$}
\egquestion What are the units of $G$?

\eganswer Solving for $G$ in Newton's law of gravity gives
\begin{equation*}
     G = \frac{Fr^2}{m_1m_2}  \qquad   ,
\end{equation*}
so the units of $G$ must be $\nunit\unitdot\munit^2/\kgunit^2$. Fully
adorned with units, the value of $G$ is $6.67\times10^{-11}\ \nunit\unitdot\munit^2/\kgunit^2$.
\end{eg}

\begin{eg}{Newton's third law}
\egquestion Is Newton's law of gravity consistent with Newton's third law?

\eganswer The third law requires two things. First, $m_1$'s
force on $m_2$ should be the same as $m_2$'s force on $m_1$.
This works out, because the product $m_1m_2$ gives the
same result if we interchange the labels 1 and 2. Second,
the forces should be in opposite directions. This condition
is also satisfied, because Newton's law of gravity refers to
an attraction: each mass pulls the other toward itself.
\end{eg}

<% marg(10) %>
<%
  fig(
    'pluto',
    %q{%
      Example \ref{eg:pluto}.
      Computer-enhanced images of Pluto
      and Charon, taken by the Hubble
      Space Telescope.
    }
  )
%>
<% end_marg %>

\begin{eg}{Pluto and Charon}\label{eg:pluto}
\egquestion Pluto's moon Charon is unusually large considering
Pluto's size, giving them the character of a double planet.
Their masses are $1.25\times10^{22}$  and $1.9x10^{21}$ kg,
and their average distance from one another is $1.96\times10^4$
 km. What is the gravitational force between them?

\eganswer If we want to use the value of $G$ expressed in SI
(meter-kilogram-second) units, we first have to convert the
distance to $1.96\times10^7\ \munit$. The force is
\begin{multline*}
    \frac{
      \left(6.67\times10^{-11}\ \nunit\unitdot\munit^2/\kgunit^2\right)
      \left(1.25\times10^{22}\ \kgunit\right)
      \left(1.9 \times10^{21}\ \kgunit\right)
    }{\left(1.96\times10^7\ \munit\right)^2}    \\
    =  4.1\times10^{18}\  \nunit
\end{multline*}
\end{eg}


The proportionality to $1/r^2$ in Newton's law of gravity
was not entirely unexpected.  Proportionalities to $1/r^2$
are found in many other phenomena in which some effect
spreads out from a point. For instance, the intensity of the
light from a candle is proportional to $1/r^2$, because at a
distance $r$ from the candle, the light has to be spread out
over the surface of an imaginary sphere of area $4\pi r^2$.
The same is true for the intensity of sound from a
firecracker, or the intensity of gamma radiation emitted by
the Chernobyl reactor. It's important, however, to realize
that this is only an analogy. Force does not travel through
space as sound or light does, and force is not a substance
that can be spread thicker or thinner like butter on toast.

Although several of Newton's contemporaries had speculated
that the force of gravity might be proportional to $1/r^2$,
none of them, even the ones who had learned Newton's laws of
motion, had had any luck proving that the resulting orbits
would be ellipses, as Kepler had found empirically.  Newton
did succeed in proving that elliptical orbits would result
from a $1/r^2$ force, but we postpone the proof until the
chapter \ref{ch:angular-momentum} because it can be
accomplished much more easily using the concepts of energy
and angular momentum.

<% marg(60) %>
<%
  fig(
    'conic-sections',
    %q{%
      The conic sections are the curves
      made by cutting the surface of an infinite cone with a plane.
    }
  )
%>
\spacebetweenfigs
<%
  fig(
    'cannon',
    %q{%
      An imaginary cannon able to shoot
      cannonballs at very high speeds is
      placed on top of an imaginary, very
      tall mountain that reaches up above
      the atmosphere. Depending on the
      speed at which the ball is fired, it may
      end up in a tightly curved elliptical orbit, 1, a 
      circular orbit, 2, a bigger elliptical orbit,
      3, or a nearly straight hyperbolic orbit, 4.
    }
  )
%>

<% end_marg %>

Newton also predicted that orbits in the shape of hyperbolas\index{hyperbolic orbit}\index{orbit!hyperbolic}\index{elliptical orbit}\index{orbit!elliptical}\index{orbit!circular}
should be possible, and he was right.  Some comets, for
instance, orbit the sun in very elongated ellipses, but
others pass through the solar system on hyperbolic paths,
never to return. Just as the trajectory of a faster baseball
pitch is flatter than that of a more slowly thrown ball, so
the curvature of a planet's orbit depends on its speed. A
spacecraft can be launched at relatively low speed,
resulting in a circular orbit about the earth, or it can be
launched at a higher speed, giving a more gently curved
ellipse that reaches farther from the earth, or it can be
launched at a very high speed which puts it in an even less
curved hyperbolic orbit. As you go very far out on a
hyperbola, it approaches a straight line, i.e., its curvature
eventually becomes nearly zero.

Newton also was able to prove that Kepler's second law
(sweeping out equal areas in equal time intervals) was a
logical consequence of his law of gravity.  Newton's version
of the proof is moderately complicated, but the proof
becomes trivial once you understand the concept of angular
momentum, which will be covered later in the course. The
proof will therefore be deferred until section \ref{sec:elliptical-orbits}.

<% self_check('kepler-hyperbolic',<<-'SELF_CHECK'
Which of Kepler's laws would it make sense to apply
to hyperbolic orbits?
  SELF_CHECK
  ) %>

\worked{ceres}{Visiting Ceres}
m4_ifelse(__me,1,[:%:],[:\worked{geosynchronous}{Geosynchronous orbit}:])
\worked{prove-a-equals-g}{Why $a$ equals $g$}
\worked{ida}{Ida and Dactyl}
\worked{upsilon-andromedae}{Another solar system}
\worked{alien-gangster}{Weight loss}
\worked{receding-moon}{The receding moon}

\startdqs

\begin{dq}
How could Newton find the speed of the moon to plug in to $a=v^2/r?$
\end{dq}

\begin{dq}
Two projectiles of different mass shot out of guns on the
surface of the earth at the same speed and angle will follow
the same trajectories, assuming that air friction is
negligible.  (You can verify this by throwing two objects
together from your hand and seeing if they separate or stay
side by side.)  What corresponding fact would be true for
satellites of the earth having different masses?
\end{dq}

\begin{dq}
What is wrong with the following statement?  ``A comet in
an elliptical orbit speeds up as it approaches the sun,
because the sun's force on it is increasing.''
\end{dq}

\begin{dq}
Why would it not make sense to expect the earth's
gravitational force on a bowling ball to be inversely
proportional to the square of the distance between their
surfaces rather than their centers?
\end{dq}

\begin{dq}
Does the earth accelerate as a result of the moon's
gravitational force on it?  Suppose two planets were bound
to each other gravitationally the way the earth and moon
are, but the two planets had equal masses.  What would
their motion be like?
\end{dq}

\begin{dq}
Spacecraft normally operate by firing their engines only
for a few minutes at a time, and an interplanetary probe
will spend months or years on its way to its destination
without thrust. Suppose a spacecraft is in a circular orbit
around Mars, and it then briefly fires its engines in
reverse, causing a sudden decrease in speed. What will this
do to its orbit? What about a forward thrust?
\end{dq}



<% end_sec() %>
<% end_sec() %>
<% begin_sec("Apparent Weightlessness",4,'apparent-weightlessness') %>

If you ask somebody at the bus stop why astronauts are
weightless, you'll probably get one of the following
two incorrect answers:

(1) They're weightless because they're so far from the earth.

(2) They're weightless because they're moving so fast.

The first answer is wrong, because the vast majority of
astronauts never get more than a thousand miles from the
earth's surface. The reduction in gravity caused by their
altitude is significant, but not 100\%. The second answer is
wrong because Newton's law of gravity only depends on
distance, not speed.

The correct answer is that astronauts in orbit around the
earth are not really weightless at all. Their weightlessness
is only apparent. If there was no gravitational force on the
spaceship, it would obey Newton's first law and move off on
a straight line, rather than orbiting the earth. Likewise,
the astronauts inside the spaceship are in orbit just like
the spaceship itself, with the earth's gravitational force
continually twisting their velocity vectors around. The
reason they appear to be weightless is that they are in the
same orbit as the spaceship, so although the earth's gravity
curves their trajectory down toward the deck, the deck drops
out from under them at the same rate.

Apparent weightlessness can also be experienced on earth.
Any time you jump up in the air, you experience the same
kind of apparent weightlessness that the astronauts do.
While in the air, you can lift your arms more easily than
normal, because gravity does not make them fall any faster
than the rest of your body, which is falling out from under
them. The Russian air force now takes rich foreign tourists
up in a big cargo plane and gives them the feeling of
weightlessness for a short period of time while the plane is
nose-down and dropping like a rock.

<% end_sec() %>
<% begin_sec("Vector Addition of Gravitational Forces",0) %>

\epigraphlong{Pick a flower on earth and you move the farthest star.}{Paul Dirac}

When you stand on the ground, which part of the earth is
pulling down on you with its gravitational force? Most
people are tempted to say that the effect only comes from
the part directly under you, since gravity always pulls
straight down. Here are three observations that might help
to change your mind:

\begin{itemize}

\item  If you jump up in the air, gravity does not stop affecting
you just because you are not touching the earth: gravity is
a noncontact force.  That means you are not immune from the
gravity of distant parts of our planet just because you
are not touching them.

\item  Gravitational effects are not blocked by intervening
matter.  For instance, in an eclipse of the moon, the earth
is lined up directly between the sun and the moon, but only
the sun's light is blocked from reaching the moon, not its
gravitational force --- if the sun's gravitational force on
the moon was blocked in this situation, astronomers would be
able to tell because the moon's acceleration would change
suddenly.  A more subtle but more easily observable example
is that the tides are caused by the moon's gravity, and
tidal effects can occur on the side of the earth facing away
from the moon.  Thus, far-off parts of the earth are not
prevented from attracting you with their gravity just
because there is other stuff between you and them.

\item  Prospectors sometimes search for underground deposits of
dense minerals by measuring the direction of the local
gravitational forces, i.e., the direction things fall or the
direction a plumb bob hangs.  For instance, the gravitational
forces in the region to the west of such a deposit would
point along a line slightly to the east of the earth's
center.  Just because the total gravitational force on you
points down, that doesn't mean that only the parts of the
earth directly below you are attracting you.  It's just that
the sideways components of all the force vectors acting on
you come very close to canceling out.

\end{itemize}
<% marg(0) %>
<%
  fig(
    'mineral-deposit',
    %q{%
      Gravity only appears to pull straight
      down because the near perfect symmetry of the earth makes the sideways
      components of the total force on an
      object cancel almost exactly. If the
      symmetry is broken, e.g., by a dense
      mineral deposit, the total force is a little
      off to the side.
    }
  )
%>
<% end_marg %>

A cubic centimeter of lava in the earth's mantle, a grain of
silica inside Mt. Kilimanjaro, and a flea on a cat in Paris
are all attracting you with their gravity. What you feel is
the vector sum of all the gravitational forces exerted by
all the atoms of our planet, and for that matter by all the
atoms in the universe.

When Newton tested his theory of gravity by comparing the
orbital acceleration of the moon to the acceleration of a
falling apple on earth, he assumed he could compute the
earth's force on the apple using the distance from the apple
to the earth's center. Was he wrong? After all, it isn't
just the earth's center attracting the apple, it's the whole
earth. A kilogram of dirt a few feet under his backyard in
England would have a much greater force on the apple than a
kilogram of molten rock deep under Australia, thousands of
miles away. There's really no obvious reason why the force
should come out right if you just pretend that the earth's
whole mass is concentrated at its center. Also, we know that
the earth has some parts that are more dense, and some parts
that are less dense. The solid crust, on which we live, is
considerably less dense than the molten rock on which it
floats. By all rights, the computation of the vector sum of
all the forces exerted by all the earth's parts should
be a horrendous mess.

Actually, Newton had sound reasons for treating
the earth's mass as if it was concentrated at its center.
First, although Newton no doubt suspected the earth's
density was nonuniform, he knew that the direction of its
total gravitational force was very nearly toward the earth's
center. That was strong evidence that the distribution of
mass was very symmetric, so that we can think of the earth
as being made of layers, like an onion, with each layer
having constant density throughout. (Today there is further
evidence for symmetry based on measurements of how the
vibrations from earthquakes and nuclear explosions travel
through the earth.) He then considered the
gravitational forces exerted by a single such thin shell,
and proved the following theorem, known
as the shell theorem:

\begin{lessimportant}\index{shell theorem}
If an object lies outside a thin, spherical shell of mass,
then the vector sum of all the gravitational forces exerted
by all the parts of the shell is the same as if the shell's
mass had been concentrated at its center.  If the object
lies inside the shell, then all the gravitational forces cancel out exactly.
\end{lessimportant}

\noindent For terrestrial gravity, each shell acts as though its mass
was at the center, so the result
is the same as if the whole mass was
there. m4_ifelse(__lm_series,1,[::],[:The shell theorem is proved on p.~\pageref{sec:shell-theorem-proof}.:])
<% marg(110) %>
<%
  fig(
    'shell-theorem',
    %q{%
      Cut-away view of a spherical shell of
      mass. A, who is outside the shell, feels gravitational forces from
      every part of the shell --- stronger
      forces from the closer parts, and
      weaker ones from the parts farther
      away. The shell theorem states that
      the vector sum of all the forces is the
      same as if all the mass had been concentrated at the center of the shell.
      B, at the center, is clearly weightless, because the shell's gravitational forces cancel out.
      Surprisingly, C also feels exactly zero gravitational force.
    }
  )
%>
<% end_marg %>


The second part of the shell theorem, about the gravitational
forces canceling inside the shell, is a little surprising.
Obviously the forces would all cancel out if you were at the
exact center of a shell, but it's not at all obvious that they should still cancel
out perfectly if you are inside the shell but off-center.
The whole idea might seem academic, since we don't know of
any hollow planets in our solar system that astronauts could
hope to visit, but actually it's a useful result for
understanding gravity within the earth, which is an
important issue in geology.  It doesn't matter that the
earth is not actually hollow.  In a mine shaft at a depth
of, say, 2 km, we can use the shell theorem to tell us that
the outermost 2 km of the earth has no net gravitational
effect, and the gravitational force is the same as what
would be produced if the remaining, deeper, parts of the
earth were all concentrated at its center.
<% marg(-60) %>
<%
  fig(
    'toutatis',
    %q{%
      The asteroid Toutatis, imaged by the space probe Chang'e-2 in 2012, is shaped like a bowling pin.
    }
  )
%>
<% end_marg %>

      The shell theorem doesn't apply to things that aren't spherical. 
At the point marked with a dot in figure \figref{toutatis},
      we might imagine that gravity was in the direction shown by the dashed arrow, pointing
      toward the asteroid's center of mass, so that
      the surface would be a vertical cliff almost a kilometer tall. In reality, calculations based
      on the assumption of uniform density show that the direction of the gravitational field is approximately
      as shown by the solid arrow, making the slope only about $60\degunit$.\footnote{Hudson \emph{et al.},
      Icarus 161 (2003) 346} This happens because gravity
at this location is more strongly affected by the nearby ``neck'' than by the more distant ``belly.''
      This slope is still believed to be too steep to keep dirt and rocks from sliding off (see problem
      \ref{hw:angle-of-repose}, p.~\pageref{hw:angle-of-repose}).

\pagebreak

<% self_check('mineshaft',<<-'SELF_CHECK'
Suppose you're at the bottom of a deep mineshaft, which
means you're still quite far from the center of the earth.
The shell theorem says that the shell of mass you've gone
inside exerts zero total force on you. Discuss which parts
of the shell are attracting you in which directions, and how
strong these forces are. Explain why it's at least plausible
that they cancel.
  SELF_CHECK
  ) %>

\startdqs

\begin{dq}
If you hold an apple, does the apple exert a gravitational
force on the earth?  Is it much weaker than the earth's
gravitational force on the apple?  Why doesn't the earth
seem to accelerate upward when you drop the apple?
\end{dq}

\begin{dq}
When astronauts travel from the earth to the moon, how
does the gravitational force on them change as they progress?
\end{dq}

\begin{dq}
How would the gravity in the first-floor lobby of a
massive skyscraper compare with the gravity in an open field
outside of the city?
\end{dq}

\begin{dq}
In a few billion years, the sun will start undergoing
changes that will eventually result in its puffing up into a
red giant star. (Near the beginning of this process, the
earth's oceans will boil off, and by the end, the sun will
probably swallow the earth completely.) As the sun's surface
starts to get closer and closer to the earth, how will the
earth's orbit be affected?
\end{dq}

<% end_sec() %>
<% begin_sec("Weighing the Earth",0,'weighing-the-earth') %>

\enlargethispage{-4\baselineskip}

Let's look more closely at the application of Newton's law
of gravity to objects on the earth's surface.  Since the
earth's gravitational force is the same as if its mass was
all concentrated at its center,  the force on a falling
object of mass $m$ is given by
\begin{equation*}
            F    =    G \: M_\text{earth} \: m \: / \: r_\text{earth}^2    \qquad   .
\end{equation*}
The object's acceleration equals $F/m$, so the object's mass
cancels out and we get the same acceleration for all falling
objects, as we knew we should:
\begin{equation*}
            g  =  G \: M_\text{earth} \: / \: r_\text{earth}^2     \qquad   .
\end{equation*}
<%
  fig(
    'cavendish',
    %q{%
      Cavendish's apparatus. The two large balls
      are fixed in place, but the rod from
      which the two small balls hang is free
      to twist under the influence of the
      gravitational forces.
    },
    {
      'width'=>'wide'
    }
  )
%>

Newton  knew neither the mass of the earth nor a numerical
value for the constant $G$.  But if someone could measure
$G$, then it would be possible for the first time in history
to determine the mass of the earth!  The only way to measure
$G$ is to measure the gravitational force between two
objects of known mass, but that's an exceedingly difficult
task, because the force between any two objects of ordinary
size is extremely small.  The English physicist Henry
Cavendish was the first to succeed, using the apparatus
shown in figures \figref{cavendish} and \figref{cavendish-simplified}.
  The two larger balls were lead
spheres 8 inches in diameter, and each one attracted the
small ball near it.  The two small balls hung from the ends
of a horizontal rod, which itself hung by a thin thread.
The frame from which the larger balls hung could be rotated
by hand about a vertical axis, so that for instance the
large ball on the right would pull its neighboring small
ball toward us and while the small ball on the left would be
pulled away from us.  The thread from which the small balls
hung would thus be twisted through a small angle, and by
calibrating the twist of the thread with known forces, the
actual gravitational force could be determined.  Cavendish
set up the whole apparatus in a room of his house, nailing
all the doors shut to keep air currents from disturbing the
delicate apparatus.  The results had to be observed through
telescopes stuck through holes drilled in the walls.
Cavendish's experiment provided the first numerical values
for $G$ and for the mass of the earth.  The presently
accepted value of $G$ is
$6.67\times10^{-11}\ \nunit\unitdot\munit^2/\kgunit^2$.
<% marg(80) %>
<%
  fig(
    'cavendish-simplified',
    %q{%
      A simplified version of Cavendish's
      apparatus.
    }
  )
%>
<% end_marg %>

Knowing $G$ not only allowed the determination of the
earth's mass but also those of the sun and the other
planets. For instance, by observing the acceleration of one
of Jupiter's moons, we can infer the mass of Jupiter. The
following table gives the distances of the planets from the
sun and the masses of the sun and planets. (Other data are
given in the back of the book.)

\begin{tabular}{|l|p{40mm}|p{40mm}|}
\hline
  &
  average distance from the sun, in units of the earth's average distance from the sun &
  mass, in units of the earth's mass \\
\hline
sun & --- & 330,000 \\
\hline
Mercury & 0.38 & 0.056 \\
\hline
Venus & 0.72 & 0.82 \\
\hline
earth & 1 & 1 \\
\hline
Mars & 1.5 & 0.11 \\
\hline
Jupiter & 5.2 & 320 \\
\hline
Saturn & 9.5 & 95 \\
\hline
Uranus & 19 & 14 \\
\hline
Neptune & 30 & 17 \\
\hline
Pluto & 39 & 0.002 \\
\hline
\end{tabular}

m4_ifelse(__me,1,[:
        The following example applies the numerical techniques of section
        \ref{numsection}.

        \begin{eg}{From the earth to the moon}
        The Apollo 11 mission landed the first humans on the moon in 1969.
        In this example, we'll estimate the time it took to get to the moon, and compare
        our estimate with the actual time, which was 73.0708 hours from the engine
        burn that took the ship out of earth orbit to the engine burn that inserted it into
        lunar orbit. During this time, the ship was coasting with the engines off, except
        for a small course-correction burn, which we neglect. More importantly, we
        do the calculation for a straight-line trajectory rather than the real S-shaped one,
        so the result can only be expected to agree roughly with what really happened.
        The following data come from the original press kit, which NASA has scanned and
        posted on the Web:\\
        \begin{tabular}{ll}
                \qquad initial altitude & $ 3.363\times10^5\ \zu{m}$\\
                \qquad initial velocity & $ 1.083\times10^4\ \zu{m/s}$\\
        \end{tabular}\\
        The endpoint of the the straight-line trajectory is a free-fall impact on
        the lunar surface, which is also unrealistic (luckily for the astronauts).

        The force acting on the ship is
        \begin{align*}
                 F        &= -\frac{ GM_{e} m}{ r^2}
                                        +\frac{ GM_{m} m}{ (r_{m}- r)^2} \qquad ,
        \intertext{but since everything is proportional to the mass of the ship,  m,
                we can divide it out}
                 \frac{F}{m}        &= -\frac{ GM_{e} }{ r^2}
                                        +\frac{ GM_{m} }{ (r_{m}- r)^2} \qquad ,
        \end{align*}
        and the variables \verb-F- in the program
        is actually the force per unit mass $F/m$. The program is a straightforward
        modification of the function
        \verb-meteor- on page \pageref{meteorlisting}.
\begin{listing}{1}<%code_listing('apollo.py',%q{
import math
def apollo(vi,n):
  bigg=6.67e-11     # gravitational constant, SI
  me=5.97e24        # mass of earth, kg
  mm=7.35e22        # mass of moon, kg
  em=3.84e8         # earth-moon distance, m
  re=6.378e6        # radius of earth, m
  rm=1.74e6         # radius of moon, m
  v=vi
  x=re+3.363e5      # re+initial altitude
  xf=em-rm          # surface of moon
  dt = 360000./n    # split 100 hours into n parts
  t = 0.
  for i in range(n):
    dx = v*dt
    x = x+dx                # Change x.
    if x>xf:
      return t/3600.
    a = -bigg*me/x**2+bigg*mm/(em-x)**2
    t = t + dt
    dv = a*dt
    v = v+dv
})%>\end{listing}
\begin{verbatim}
>>> print apollo(1.083e4,1000000)
59.7488999991
>>> vi=1.083e4
\end{verbatim}
        This is pretty decent agreement with the real-world time of 73 hours, considering the wildly inaccurate trajectory
        assumed. It's interesting to see how much the duration of the trip changes if
        we increase the initial velocity by only ten percent:
\begin{verbatim}
>>> print apollo(1.2e4,1000000)
18.3682
\end{verbatim}
        \noindent{}The most important
        reason for using the lower speed was that if something had gone wrong,
        the ship would have been able to whip around the moon and take a ``free return''
        trajectory back to the earth, without having to do any further burns. At a higher speed,
        the ship would have had so much kinetic energy that in the absence of any further
        engine burns, it would have escaped from the earth-moon system. The Apollo 13
        mission had to take a free return trajectory after an explosion crippled the spacecraft.
        \end{eg}
:])

\startdqs

\begin{dq}
It would have been difficult for Cavendish to start
designing an experiment without at least some idea of the
order of magnitude of $G$.  How could he estimate it in
advance to within a factor of 10?
\end{dq}

\begin{dq}
Fill in the details of how one would determine Jupiter's
mass by observing the acceleration of one of its moons. Why
is it only necessary to know the acceleration of the moon,
not the actual force acting on it? Why don't we need to know
the mass of the moon? What about a planet that has no moons,
such as Venus --- how could its mass be found?
\end{dq}

<% end_sec() %>
<% begin_sec("Dark energy",nil,'dark-energy',{'optional'=>true}) %>

%%%%%%%%%%%%%%%%%%%%%%%%%%%%%%%%%%%%%%%%%%%%%%%%%%%%%
__incl(text/dark_energy)

\enlargethispage{\baselineskip}

m4_ifelse(__lm_series,1,[:
Dark energy is discussed in more detail on p.~\pageref{subsec:dark-stuff-gr}.
:])
%%%%%%%%%%%%%%%%%%%%%%%%%%%%%%%%%%%%%%%%%%%%%%%%%%%%%%%%%%%%
<% end_sec() %>

m4_ifelse(__me,0,[:%
<% begin_sec("A gravitational test of Newton's first law",nil,'battat',{'optional'=>true}) %>
% -------- ME has this earlier.

This section describes a high-precision test of Newton's first law.\index{Newton's laws of motion!first law!test of}
The left panel of figure \figref{battat} shows a mirror on the moon.
By reflecting laser pulses from the mirror, the distance from the earth to
the moon has been measured to the phenomenal precision of a few centimeters, or about one part in $10^{10}$. This distance
changes for a variety of known reasons. The biggest effect is that the moon's orbit is not a circle but an ellipse,
with its long axis about 11\% longer than its short one. A variety of other effects can also be accounted for, including such
exotic phenomena as the slightly nonspherical shape of the earth, and the gravitational forces of bodies as small and distant as Pluto.
Suppose for simplicity that all these effects had never existed, so that the moon was initially placed in a perfectly circular orbit around
the earth, and the earth in a perfectly circular orbit around the sun. 

<%
  fig(
    'battat',
    %q{%
      Left: The Apollo 11 mission left behind a mirror, which in this photo shows the reflection of the black sky. 
      Right: A highly exaggerated example of an observation that would disprove Newton's first law. The radius of the
      moon's orbit gets bigger and smaller over the course of a year.
    },
    {
      'width'=>'fullpage'
    }
  )                   
%>

If we then observed something like what is shown in the right panel of
figure \figref{battat}, Newton's first law would be disproved. If space itself is symmetrical in all directions, then there is no reason for
the moon's orbit to poof up near the top of the diagram and contract near the bottom. The only possible explanation would be that there was
some preferred frame of reference\index{frame of reference!preferred}
of the type envisioned by Aristotle, and that our solar system was moving relative to it.
Another test for a preferred frame was described in example \ref{eg:clock-comparison-inertia}
on p.~\pageref{eg:clock-comparison-inertia}.

One could then imagine that the gravitational force of the earth on the moon could be affected by the moon's motion relative to this frame.
The lunar laser ranging data\footnote{Battat, Chandler, and Stubbs, \url{http://arxiv.org/abs/0710.0702}} contain no measurable effect of the type shown
in figure \figref{battat}, so that if the moon's orbit is distorted in this way (or in a variety of other ways), the distortion must be less than
a few centimeters. This constitutes a very strict upper limit on violation of Newton's first law by gravitational forces.
If the first law is violated, and the violation causes a fractional change in gravity that is proportional to the velocity relative to the
hypothetical preferred frame, then the change is no more than about one part in $10^7$, even if the velocity is comparable to the speed of
light.
<% end_sec() %>
:])

m4_ifelse(__lm_series,1,[::],[:
<% begin_sec("Proof of the shell theorem",nil,'shell-theorem-proof',{'optional'=>true}) %>\index{shell theorem!proof}
<% marg(-10) %>
<%
  fig(
    'shell-theorem-proof-force',
    %q{%
      A spherical shell of mass $M$ interacts with a pointlike mass $m$.
    }
  )
%>
<% end_marg %>

Referring to figure \figref{shell-theorem-proof-force}, let $b$ be the radius of the shell, $h$ its thickness, and $\rho$
its density.
Its volume is then $V$=(area)(thickness)=$4\pi{}b^2h$, and
its mass is $M=\rho{}V=4\pi{}\rho{}b^2h$. The strategy is to divide the shell
up into rings as shown, with each ring
extending from $\theta$ to $\theta+\der{}\theta$. Since the ring is infinitesimally
skinny, its entire mass lies at the same distance, $r$, from mass $m$.
The width of such a ring is
found by the definition of radian measure to be $w=b\der{}\theta$, and its mass
is $\der{}M=(\rho)$(circumference)(thickness)(width)=
$(\rho)(2\pi{}b \sin{} \theta)(h)(b\der{}\theta)$=$2\pi\rho b^2h\sin{}\theta\der{}\theta$.
To save writing, we define $A=GMm/s^2$. For the case where $m$ is outside the shell, our goal is to prove that the force $F$ acting on
$m$ equals $A$. Let the axis of symmetry be $x$, and let the contribution of this ring to the total force be $\der F_x$.
\begin{align*}
  F &= \int \der F_x \\
    &= \int \frac{Gm \der M }{r^2}\cos\alpha \\
    &= \int \frac{Gm \cdot 2\pi\rho b^2h\sin\theta\der\theta}{r^2}\cos\alpha \\
    &= \left(\frac{s^2}{2}\right)A\int \frac{\sin\theta\der\theta}{r^2}\cos\alpha 
\end{align*}
From the law of cosines we find
\begin{align*}
  r^2 &= b^2 + s^2 - 2bs\cos\theta \qquad ,\\
  b^2 &= r^2 + s^2 -2rs\cos\alpha \qquad ,\\
\intertext{and differentiation of the former gives}  
  2r\der r & =        2bs\sin\theta\der\theta \qquad .
\end{align*}
We can now write the integrand entirely in terms of the single variable of integration $r$.
\begin{align*}
  F &= \left(\frac{s}{2b}\right)A \int_{s-b}^{s+b} \frac{r\der r}{r^2}\cos\alpha \\
    &= \left(\frac{1}{4b}\right)A \int_{s-b}^{s+b} \frac{\der r}{r}\left(r+\frac{s^2-b^2}{r}\right) \\
    &= \left(\frac{1}{4b}\right)A (2b+2b) \\
    &= A
\end{align*}
This is what we wanted to prove for the case where $m$ is on the outside. The inside case is problem
\ref{hw:shell-theorem-inside}. A more elegant method of proof is to use Gauss's theorem, which is usually
introduced in a class on electricity and magnetism or vector calculus; the concept is that the gravitational
field can be visualized in terms of lines of gravitational force spreading out from a mass, and the number
of lines coming out through a surface is independent of the exact geometry of the surface and the mass
distribution.
It is interesting to note that the result depends on both the fact that
the exponent of $r$ in Newton's law of gravity is $-2$ (problem \ref{hw:shell-theorem-exponent}) and on the fact that space has three dimensions.
<% end_sec() %>
:])

\begin{summary}

\begin{vocab}

\vocabitem{ellipse}{a flattened circle; one of the conic sections}

\vocabitem{conic section}{a curve formed by the intersection of a plane
and an infinite cone}

\vocabitem{hyperbola}{another conic section; it does not close back on itself}

\vocabitem{period}{the time required for a planet to complete one orbit;
more generally, the time for one repetition of some repeating motion}

\vocabitem{focus}{one of two special points inside an ellipse: the
ellipse consists of all points such that the sum of the
distances to the two foci equals a certain number; a
hyperbola also has a focus}

\end{vocab}

\begin{notation}

\notationitem{$G$}{the constant of proportionality in Newton's law of
gravity; the gravitational force of attraction between two
1-kg spheres at a center-to-center distance of 1 m}

\end{notation}

\begin{summarytext}

Kepler deduced three empirical laws from data on the
motion of the planets:
\begin{description}
  \item[Kepler's elliptical orbit law:] The planets orbit the sun in
  elliptical orbits with the sun at one focus.
  %
  \item[Kepler's equal-area law:] The line connecting a planet to the
  sun sweeps out equal areas in equal amounts of time.
  %
  \item[Kepler's law of periods:] The time required for a planet to
  orbit the sun is proportional to the long axis of the
  ellipse raised to the 3/2 power. The constant of proportionality
  is the same for all the planets.
\end{description}
Newton was able to find a more fundamental explanation for
these laws. Newton's law of gravity states that the
magnitude of the attractive force between any two objects in
the universe is given by
\begin{equation*}
F=Gm_1m_2/r^2   \qquad   .
\end{equation*}
Weightlessness of objects in orbit around the earth is only
apparent. An astronaut inside a spaceship is simply falling
along with the spaceship. Since the spaceship is falling out
from under the astronaut, it appears as though there was no
gravity accelerating the astronaut down toward the deck.

Gravitational forces, like all other forces, add like
vectors. A gravitational force such as we ordinarily feel is
the vector sum of all the forces exerted by all the parts of
the earth. As a consequence of this, Newton proved the
\emph{shell theorem} for gravitational forces:

If an object lies outside a thin, uniform shell of mass,
then the vector sum of all the gravitational forces exerted
by all the parts of the shell is the same as if all the
shell's mass was concentrated at its center.  If the object
lies inside the shell, then all the gravitational forces cancel out exactly.

\end{summarytext}

\end{summary}
