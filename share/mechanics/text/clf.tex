\optionalchapternote{This chapter is optional, and should probably
be omitted from a two-semester survey course. It can be covered
at any time after chapter \ref{ch:work}.}

In a developing country like China, a refrigerator is the
mark of a family that has arrived in the middle class, and a
car is the ultimate symbol of wealth. Both of these
are \emph{heat engines}: devices for 
converting between heat and other
forms of energy.\index{heat engine}\index{engine!heat}
Unfortunately for the Chinese, neither is a very efficient
device. Burning fossil fuels has made China's big cities the
most polluted on the planet, and the country's total energy
supply isn't sufficient to support American levels of energy
consumption by more than a small fraction of China's
population. Could we somehow manipulate
energy in a more efficient way?

Conservation of energy is a statement that the total amount of
energy is constant at all times, which encourages us to
believe that any energy transformation can be undone ---
indeed, the laws of physics you've learned so far don't even
distinguish the past from the future.
If you get in a car and drive around the block,
the net effect is to consume some of the energy you paid
for at the gas station,  using it to heat the neighborhood.
There would not seem to be any fundamental physical principle
to prevent you from recapturing all that heat and using it again
the next time you want to go for a drive. More modestly,
why don't engineers design a car engine so that
it recaptures the heat
energy that would otherwise be wasted via
the radiator and the exhaust?

Hard experience, however, has shown that designers of more
and more efficient engines run into a brick wall at a
certain point. The generators that the electric company uses
to produce energy at an oil-fueled plant are indeed much
more efficient than a car engine, but even if one is willing
to accept a device that is very large, expensive, and
complex, it turns out to be impossible to make a perfectly
efficient heat engine --- not just impossible with present-day
technology, but impossible due to a set of fundamental physical
principles known as the science of \emph{thermodynamics}.\index{thermodynamics}
And thermodynamics isn't just a pesky set of constraints on 
heat engines. Without thermodynamics, there is no way to explain
the direction of time's arrow --- why we can remember the past
but not the future, and why it's easier to break Humpty Dumpty than
to put him back together again.

<% begin_sec("Pressure and Temperature",0) %>
When we heat an object, we speed up the mind-bogglingly
complex random motion of its molecules. One method for
taming complexity is the conservation laws, since they tell
us that certain things must remain constant regardless of
what process is going on. Indeed, the law of conservation of
energy is also known as the first law of thermodynamics.\index{thermodynamics!first law of}

But as alluded to in the introduction to this chapter,
conservation of energy by itself is not powerful enough to
explain certain empirical facts about heat. A second way to
sidestep the complexity of heat is to ignore heat's
atomic nature and concentrate on quantities like
temperature and pressure that tell us about a system's
properties as a whole. This approach is called
macroscopic in contrast to the microscopic method of attack.
Pressure and temperature were fairly well understood in the
age of Newton and Galileo, hundreds of years before there
was any firm evidence that atoms and molecules even existed.

Unlike the conserved quantities such as mass, energy, momentum,
and angular momentum, neither pressure nor temperature is
additive. Two cups of coffee have twice the heat
energy of a single cup, but they do not have twice the
temperature. Likewise, the painful pressure on your eardrums
at the bottom of a pool is not affected if you insert or
remove a partition between the two halves of the pool.

<% begin_sec("Pressure") %>\index{pressure}

We restrict ourselves to a discussion of pressure in fluids
at rest and in equilibrium. In physics, the term ``fluid''
is used to mean either a gas or a liquid. The important
feature of a fluid can be demonstrated by comparing with a
cube of jello on a plate. The jello is a solid. If you shake
the plate from side to side, the jello will respond by
shearing, i.e., by slanting its sides, but it will tend to
spring back into its original shape. A solid can sustain
shear forces, but a fluid cannot. A fluid does not resist a
change in shape unless it involves a change in volume.

If you're at the bottom of a pool, you can't relieve the
pain in your ears by turning your head. The water's
force on your eardrum is always the same, and is always
perpendicular to the surface where the eardrum contacts the water. If your
ear is on the east side of your head, the water's force is
to the west. If you keep your head in the same spot while turning
around so your ear is on the north, the force will still be the same in
magnitude, and it will change its direction so that it is
still perpendicular to the eardrum: south. This shows that
pressure has no direction in space, i.e., it is a scalar. The direction
of the force is determined by the orientation of the surface on which
the pressure acts, not by the pressure itself. A fluid flowing over a surface can also
exert frictional forces, which are parallel to the surface, but the present
discussion is restricted to fluids at rest.

Experiments also show that a fluid's force on a surface is
proportional to the surface area. The vast force of the water
behind a dam, for example, in proportion to the
dam's great surface area. (The bottom of the dam experiences
a higher proportion of its force.)

Based on these experimental results, it appears that the
useful way to define pressure is as follows.
The pressure of a fluid at a given point is defined as $F_\perp/A$,
where $A$ is the area of a small surface inserted in the fluid
at that point, and $F_\perp$ is the component of the fluid's force on
the surface which is perpendicular to the surface.

<% marg(100) %>
<%
  fig(
    'pressuregauge',
    %q{%
      A simple pressure gauge consists of a cylinder open at one
      end, with a piston and a spring inside. The depth to which
      the spring is depressed is a measure of the pressure. To
      determine the absolute pressure, the air needs to be pumped
      out of the interior of the gauge, so that there is no air
      pressure acting outward on the piston. In many practical
      gauges, the back of the piston is open to the atmosphere, so
      the pressure the gauge registers equals the pressure of the
      fluid minus the pressure of the atmosphere.
    }
  )
%>
<% end_marg %>
This is essentially how a pressure gauge works. The reason
that the surface must be small is so that there will not be
any significant difference in pressure between one part of it
and another part.  The SI units
of pressure are evidently $\nunit/\munit^2$, and this combination
can be abbreviated as the pascal, 1 Pa=1 $\nunit/\munit^2$.\index{pascal!unit} The
pascal turns out to be an inconveniently small unit, so car
tires, for example, have recommended pressures imprinted on
them in units of kilopascals.

\begin{eg}{Pressure in U.S. units}
In U.S. units, the unit of force is the pound, and the unit
of distance is the inch. The unit of pressure is therefore
pounds per square inch, or p.s.i. (Note that the pound is
not a unit of mass.)
\end{eg}

\vfill\pagebreak[4]

\begin{eg}{Atmospheric pressure in U.S. and metric units}
\egquestion
A figure that many people in the U.S. remember is
that atmospheric pressure is about 15 pounds per square
inch. What is this in metric units?

\eganswer
\begin{align*}
        \frac{\text{15 lb}}{\text{1 in}^2}
                &= \frac{68\ \nunit}{(0.0254\ \munit)^2}\\
                 &=   1.0\times10^5\ \nunit/\munit^2 \\
                 &=  100\ \zu{kPa}
\end{align*}
\end{eg}

<% begin_sec("Only pressure differences are normally significant.") %>
If you spend enough time on an airplane,
the pain in your ears subsides. This is because your body
has gradually been able to admit more air into the cavity
behind the eardrum. Once the pressure inside is equalized
with the pressure outside, the inward and outward forces on
your eardrums cancel out, and there is no physical sensation
to tell you that anything unusual is going on. For this
reason, it is normally only pressure differences that have
any physical significance. Thus deep-sea fish are perfectly
healthy in their habitat because their bodies have enough
internal pressure to cancel the pressure from the water in
which they live; if they are caught in a net and brought to
the surface rapidly, they explode because their internal
pressure is so much greater than the low pressure outside.

\begin{eg}{Getting killed by a pool pump}
\egquestion
My house has a pool, which I maintain myself. A
pool always needs to have its water circulated through a
filter for several hours a day in order to keep it clean.
The filter is a large barrel with a strong clamp that holds
the top and bottom halves together. My filter has a
prominent warning label that warns me not to try to open the
clamps while the pump is on, and it shows a cartoon of a
person being struck by the top half of the pump. The
cross-sectional area of the filter barrel is 0.25 $\zu{m}^2$. Like
most pressure gauges, the one on my pool pump actually reads
the difference in pressure between the pressure inside the
pump and atmospheric pressure. The gauge reads 90 kPa. What
is the force that is trying to pop open the filter?

\eganswer
If the gauge told us the absolute pressure of the
water inside, we'd have to find the force of the water
pushing outward and the force of the air pushing inward, and
subtract in order to find the total force. Since air
surrounds us all the time, we would have to do such a
subtraction every time we wanted to calculate anything
useful based on the gauge's reading. The manufacturers of
the gauge decided to save us from all this work by making it
read the difference in pressure between inside and outside,
so all we have to do is multiply the gauge reading by the
cross-sectional area of the filter:
\begin{align*}
         F         &=  PA\\
                &=  (90\times10^3\ \nunit/\munit^2)( 0.25\ \munit^2)\\        
               &= 22000\ \nunit
\end{align*}
That's a lot of force!
\end{eg}

The word ``suction'' and other related words contain a
hidden misunderstanding related to this point about pressure
differences. When you suck
water up through a straw, there is nothing in your mouth
that is attracting the water upward. The force that lifts
the water is from the pressure of the water in the cup. By
creating a partial vacuum in your mouth, you decreased the
air's downward force on the water so that it no longer
exactly canceled the upward force.

<% marg(50) %>
<%
  fig(
    'sidetoside',
    %q{%
      This doesn't happen. If pressure could vary horizontally
      in equilibrium, the cube of water would accelerate
      horizontally. This is a contradiction, since we assumed the
      fluid was in equilibrium.
    }
  )
%>
\spacebetweenfigs
<%
  fig(
    'toptobottom',
    %q{%
      This does happen. The sum of the forces from the
      surrounding parts of the fluid is upward, canceling the
      downward force of gravity.
    }
  )
%>
\spacebetweenfigs
<%
  fig(
    'funkycontainer',
    %q{%
      The pressure is the same at all the points marked with
      dots.
    }
  )
%>

<% end_marg %>

<% end_sec() %>
<% begin_sec("Variation of pressure with depth") %>
The pressure within a fluid in equilibrium can only depend
on depth, due to gravity. If the pressure could vary from
side to side, then a piece of the fluid in between, \ref{fig:sidetoside},
would be subject to unequal forces from the parts of the
fluid on its two sides. But fluids do not exhibit shear
forces, so there would be no other force that could keep this
piece of fluid from accelerating. This contradicts the
assumption that the fluid was in equilibrium.

<% self_check('solidspressureside',<<-'SELF_CHECK'
How does this proof fail for solids?
  SELF_CHECK
  ) %>

To find the variation with depth, we consider the vertical
forces acting on a tiny, imaginary cube of the fluid having
height $\Delta y$ and areas $\der A$ on the top and bottom. Using positive
numbers for upward forces, we have
\begin{align*}
        P_{bottom}\Delta A - P_{top}\Delta A - F_g  =  0\eqquad.
\end{align*}
The weight of the fluid is $F_g = mg = \rho Vg = \rho\:\Delta A\Delta y\:g$, where $\rho$ is
the density of the fluid, so the difference in pressure is
\begin{multline*}
        \Delta P  =  -\rho g \Delta y\eqquad. 
\hfill\shoveright{\text{[variation in pressure with depth for}}\\
\hfill\shoveright{\text{a fluid of density $\rho$ in equilibrium;}}\\
\hfill\shoveright{\text{positive $y$ is up.]}}
\end{multline*}

The factor of $\rho$ explains why we notice the difference in
pressure when diving 3 m down in a pool, but not when going
down 3 m of stairs. Note also that the equation only tells us the
difference in pressure, not the absolute pressure. The
pressure at the surface of a swimming pool equals the
atmospheric pressure, not zero, even though the depth is
zero at the surface. The blood in your body does not even
have an upper surface.

\begin{eg}{Pressure of lava underneath a volcano}
\egquestion
A volcano has just finished erupting, and a pool
of molten lava is lying at rest in the crater. The lava has
come up through an opening inside the volcano that connects
to the earth's molten mantle. The density of the lava is 4.1
$\zu{g}/\zu{cm}^3$. What is the pressure in the lava underneath the base
of the volcano, 3000 m below the surface of the pool?

\eganswer
\begin{align*}
        \Delta P         &=  \rho g\Delta y\\
                        &= ( 4.1\ \zu{g}/\zu{cm}^3)( 9.8\ \munit/\sunit^2)(3000\ \zu{m})\\
                        &= ( 4.1\times10^6\ \zu{g}/\zu{m}^3)
                                        ( 9.8\ \munit/\sunit^2)(3000\ \zu{m})\\
                        &= ( 4.1\times10^3\ \zu{kg}/\zu{m}^3)
                                        ( 9.8\ \munit/\sunit^2)(3000\ \zu{m})\\
                        &= 1.2\times10^8\ \nunit/\munit^2 \\
                        &= 1.2\times10^8\ \zu{Pa}
\end{align*}
This is the difference between the pressure we want to find and
atmospheric pressure at the surface. The latter, however, is tiny compared
to the $\Delta P$ we just calculated, so what we've found is essentially the
pressure, $P$.
\end{eg}

\begin{eg}{Atmospheric pressure}\label{eg:pressure-at-altitude}
This example uses calculus.

Gases, unlike liquids, are quite compressible, and at a given temperature, the
density of a gas is approximately proportional to the pressure. The proportionality
constant is discussed in section \ref{sec:microscopicidealgas}, but for now let's just
call it $k$, $\rho= kP$. Using this fact, we can find the variation of
atmospheric pressure with altitude, assuming constant temperature:
\begin{align*}
        \der P &= -\rho g\der y\\
        \der P &= - kPg\der y\\
        \frac{\der P}{ P} &= - kg\der y\\
        \zu{ln}\: P &= - kgy+\text{constant} \qquad \text{[integrating both sides]}\\
         P &= (\text{constant}) e^{- kgy} \qquad \text{[exponentiating both sides]}
\end{align*}
Pressure falls off exponentially with height. There is no sharp cutoff to the
atmosphere, but the exponential gets extremely small by the time you're
ten or a hundred miles up.
\end{eg}
<% marg(24) %>
<%
  fig(
    'fever',
    %q{%
      We have to wait for the thermometer to equilibrate its
      temperature with the temperature of Irene's armpit.
    }
  )
%>
<% end_marg %>

<% end_sec() %>
<% end_sec() %>
<% begin_sec("Temperature") %>
<% begin_sec("Thermal equilibrium") %>
We use the term temperature casually, but what is it
exactly? Roughly speaking, temperature is a measure of how
concentrated the heat energy is in an object. A large,
massive object with very little heat energy in it has a low
temperature.

But physics deals with operational definitions, i.e.,
definitions of how to measure the thing in question. How do
we measure temperature? One common feature of all
temperature-measuring devices is that they must be left for
a while in contact with the thing whose temperature is being
measured. When you take your temperature with a fever
thermometer, you wait for the mercury inside to come
up to the same temperature as your body. The thermometer
actually tells you the temperature of its own working fluid
(in this case the mercury). In general, the idea of
temperature depends on the concept of thermal equilibrium.
When you mix cold eggs from the refrigerator with flour that
has been at room temperature, they rapidly reach a
compromise temperature. What determines this compromise
temperature is conservation of energy, and the amount of
energy required to heat or cool each substance by one
degree. But without even having constructed a temperature
scale, we can see that the important point is the phenomenon
of thermal equilibrium itself: two objects left in contact
will approach the same temperature. We also assume that if
object A is at the same temperature as object B, and B is at
the same temperature as C, then A is at the same temperature
as C. This statement is sometimes known as the zeroth law of
thermodynamics,\index{thermodynamics!zeroth law of}
so called because after the first, second, and third
laws had been developed, it was realized that there was
another law that was even more fundamental.

<% marg(50) %>
<%
  fig(
    'otters',
    %q{%
      Thermal equilibrium can be prevented. Otters
      have a coat of fur that traps air bubbles for insulation.
      If a swimming otter was in thermal equilibrium with cold water, it would be dead.
      Heat is still conducted from the otter's body to the water, but much more slowly than
      it would be in a warm-blooded animal that didn't have this special adaptation.
    }
  )
%>
\spacebetweenfigs
<%
  fig(
    'hot-air-balloon',
    %q{%
      A hot air balloon is inflated. Because of thermal
      expansion, the hot air is less dense than the surrounding cold air, and
      therefore floats as the cold air drops underneath it and pushes it up
      out of the way.
    }
  )
%>
<% end_marg %>
<% end_sec() %>
<% begin_sec("Thermal expansion") %>
The familiar mercury thermometer operates on the principle
that the mercury, its working fluid, expands when heated and
contracts when cooled. In general, all substances expand and
contract with changes in temperature. The zeroth law of
thermodynamics guarantees that we can construct a
comparative scale of temperatures that is independent of
what type of thermometer we use. If a thermometer gives a
certain reading when it's in thermal equilibrium with
object A, and also gives the same reading for object B, then
A and B must be the same temperature, regardless of the
details of how the thermometers works.

What about constructing a temperature scale in which every
degree represents an equal step in temperature? The Celsius
scale has 0 as the freezing point of water and 100 as its
boiling point. The hidden assumption behind all this is that
since two points define a line, any two thermometers that
agree at two points must agree at all other points. In
reality if we calibrate a mercury thermometer and an alcohol
thermometer in this way, we will find that a graph of one
thermometer's reading versus the other is not a perfectly
straight $y=x$ line. The subtle inconsistency becomes a
drastic one when we try to extend the temperature scale
through the points where mercury and alcohol boil or freeze.
Gases, however, are much more consistent among themselves in
their thermal expansion than solids or liquids, and the
noble gases like helium and neon are more consistent with
each other than gases in general. Continuing to search for
consistency, we find that noble gases are more consistent
with each other when their pressure is very low.

<% marg(50) %>
<%
  fig(
    'gasthermometer',
    %q{%
      A simplified version of an ideal gas thermometer. The
      whole instrument is allowed to come into thermal equilibrium
      with the substance whose temperature is to be measured, and
      the mouth of the cylinder is left open to standard pressure.
      The volume of the noble gas gives an indication of
      temperature.
    }
  )
%>
\spacebetweenfigs
<%
  fig(
    'abszeroextrap',
    %q{%
      The volume of 1 kg of neon gas as a function of
      temperature (at standard pressure). Although neon would
      actually condense into a liquid at some point, extrapolating
      the graph to zero volume gives the same temperature as
      for any other gas: absolute zero.
    }
  )
%>

<% end_marg %>
As an idealization, we imagine a gas in which the atoms interact only
with the sides of the container, not with each other. Such a gas is perfectly
nonreactive (as the noble gases very nearly are), and never
condenses to a liquid (as the noble gases do only at
extremely low temperatures). Its atoms take up a negligible
fraction of the available volume. Any gas can be made to behave very much
like this if the pressure is extremely low, so that the atoms hardly ever
encounter each other. Such a gas is called an ideal gas, and we define
the Celsius scale in terms of the volume of the gas in a thermometer
whose working substance is an ideal gas maintained at a
fixed (very low) pressure, and which is calibrated at 0 and
100 degrees according to the melting and boiling points of
water. The Celsius scale is not just a comparative scale but
an additive one as well: every step in temperature is equal,
and it makes sense to say that the difference in temperature
between 18 and $28\degcunit$ is the same as the difference between 48
and 58.

<% end_sec() %>
<% begin_sec("Absolute zero and the Kelvin scale") %>
We find that if we extrapolate a graph of volume versus
temperature, the volume becomes zero at nearly the same
temperature for all gases: -273\degcunit. Real gases will all
condense into liquids at some temperature above this, but an
ideal gas would achieve zero volume at this temperature,
known as absolute zero. The most useful temperature scale in
scientific work is one whose zero is defined by absolute
zero, rather than by some arbitrary standard like the
melting point of water. The ideal temperature scale
for scientific work, called the Kelvin scale, is
the same as the Celsius scale, but shifted by 273 degrees to
make its zero coincide with absolute zero. Scientists use
the Celsius scale only for comparisons or when a change in
temperature is all that is required for a calculation. Only
on the Kelvin scale does it make sense to discuss ratios of
temperatures, e.g., to say that one temperature is twice as
hot as another.\index{temperature!macroscopic definition}
\index{temperature!Celsius}\index{temperature!Kelvin}\index{temperature!absolute zero}
\index{thermometer}\index{Celsius (unit)}\index{kelvin (unit)}

\begin{eg}{Which temperature scale to use}
\egquestion
You open an astronomy book and encounter the
equation
\begin{equation*}
        (\text{light emitted}) = (\text{constant}) \times  T^ 4
\end{equation*}
for the light emitted by a star as a function of its surface
temperature. What temperature scale is implied?

\eganswer
The equation tells us that doubling the
temperature results in the emission of 16 times as much
light. Such a ratio only makes sense if the Kelvin scale is
used.
\end{eg}


 %
\vfill
<% end_sec() %>
<% end_sec() %>
<% end_sec %>
<% begin_sec("Microscopic Description of an Ideal Gas",3,'microscopicidealgas') %>
<% begin_sec("Evidence for the kinetic theory") %>
Why does matter have the thermal properties it does? The
basic answer must come from the fact that matter is made of
atoms.
How, then, do the atoms give rise to the bulk properties we
observe? Gases, whose thermal properties are so simple,
offer the best chance for us to construct a simple
connection between the microscopic and macroscopic worlds.

A crucial observation is that although solids and liquids
are nearly incompressible, gases can be compressed, as when
we increase the amount of air in a car's tire while hardly
increasing its volume at all. This makes us suspect that the
atoms in a solid are packed shoulder to shoulder, while a
gas is mostly vacuum, with large spaces between molecules.
Most liquids and solids have densities about 1000 times
greater than most gases, so evidently each molecule in a gas
is separated from its nearest neighbors by a space something
like 10 times the size of the molecules themselves.

If gas molecules have nothing but empty space between them,
why don't the molecules in the room around you just fall to
the floor? The only possible answer is that they are in
rapid motion, continually rebounding from the walls, floor
and ceiling. In chapter \ref{ch:energy-zoo},
we have already seen some of the evidence for
the kinetic theory of heat, which states that heat is the kinetic energy of
randomly moving molecules. This theory was proposed by
Daniel Bernoulli in 1738, and met with considerable opposition, because
there was no precedent for this kind of
perpetual motion. No rubber ball, however elastic, rebounds
from a wall with exactly as much energy as it originally
had, nor do we ever observe a collision between balls in
which none of the kinetic energy at all is converted to heat
and sound. The analogy is a false one, however. A rubber
ball consists of atoms, and when it is heated in a
collision, the heat is a form of motion of those atoms. An
individual molecule, however, cannot possess heat. Likewise
sound is a form of bulk motion of molecules, so colliding
molecules in a gas cannot convert their kinetic energy to
sound. Molecules can indeed induce vibrations such as sound
waves when they strike the walls of a container, but the
vibrations of the walls are just as likely to impart energy
to a gas molecule as to take energy from it. Indeed, this
kind of exchange of energy is the mechanism by which the
temperatures of the gas and its container become
equilibrated.

 %
<% end_sec() %>
<% begin_sec("Pressure, volume, and temperature",nil,'pvnrt') %>
A gas exerts pressure on the walls of its container, and in
the kinetic theory we interpret this apparently constant
pressure as the averaged-out result of vast numbers of
collisions occurring every second between the gas molecules
and the walls. The empirical facts about gases can be
summarized by the relation
\begin{equation*}
        PV \propto nT   , \qquad \text{[ideal gas]}
\end{equation*}
which really only holds exactly for an ideal gas. Here $n$ is
the number of molecules in the sample of gas.

\begin{eg}{Volume related to temperature}
The proportionality of volume to temperature at fixed
pressure was the basis for our definition of temperature.
\end{eg}

\begin{eg}{Pressure related to temperature}
Pressure is proportional to temperature when volume is held
constant. An example is the increase in pressure in a car's
tires when the car has been driven on the freeway for a
while and the tires and air have become hot.
\end{eg}

We now connect these empirical facts to the kinetic theory
of a classical ideal gas. For simplicity, we assume that the gas is
monoatomic (i.e., each molecule has only one atom), and that
it is confined to a cubical box of volume $V$, with $L$ being
the length of each edge and $A$ the area of any wall. An atom
whose velocity has an $x$ component $v_x$ will collide regularly
with the left-hand wall, traveling a distance $2L$ parallel to
the $x$ axis between collisions with that wall. The time between collisions
is $\Delta t=2L/v_x$, and in each collision the $x$ component of
the atom's momentum is reversed from $-mv_x$  to $mv_x$. The total
force on the wall is
\begin{equation*}
        F =   \frac{\Delta p_{x,1}}{\Delta t_1}+\frac{\Delta p_{x,2}}{\Delta t_2}+\ldots  \qquad \text{[monoatomic ideal gas]}\eqquad,
\end{equation*}
where the indices 1, 2, $\ldots$ refer to the individual atoms.
Substituting $\Delta p_{x,i}=2mv_{x,i}$ and $\Delta t_i=2L/v_{x,i}$, we
have
\begin{equation*}
                F        =  \frac{mv_{x,1}^2}{L}+\frac{mv_{x,2}^2}{L}+\ldots \qquad \text{[monoatomic ideal gas]}\eqquad.
\end{equation*}
The quantity $mv_{x,i}^2$ is twice the contribution to the
kinetic energy from the part of the atom's center of mass
motion that is parallel to the $x$ axis. Since we're assuming
a monoatomic gas, center of mass motion is the only type of
motion that gives rise to kinetic energy. (A more complex
molecule could rotate and vibrate as well.) If the quantity
inside the sum included the $y$ and $z$ components, it would be
twice the total kinetic energy of all the molecules. By
symmetry, it must therefore equal 2/3 of the total kinetic
energy, so
\begin{equation*}
        F        =   \frac{2KE_{total}}{3L} \qquad \text{[monoatomic ideal gas]}\eqquad.
\end{equation*}
Dividing by $A$ and using $AL=V$, we have
\begin{equation*}
        P        =   \frac{2KE_{total}}{3V} \qquad \text{[monoatomic ideal gas]}\eqquad.
\end{equation*}
This can be connected to the empirical relation $PV \propto nT$ if we
multiply by $V$ on both sides and rewrite $KE_{total}$ as $nKE_{av}$,
where $KE_{av}$ is the average kinetic energy per
molecule:
\begin{equation*}
        PV        =          \frac{2}{3}nKE_{av} \qquad \text{[monoatomic ideal gas]}\eqquad.
\end{equation*}
For the first time we have an interpretation for the
temperature based on a microscopic description of matter: in
a monoatomic ideal gas, the temperature is a measure of the
average kinetic energy per molecule. The proportionality
between the two is $KE_{av}=(3/2)kT$, where the constant of
proportionality $k$, known as Boltzmann's constant, has a
numerical value of $1.38\times10^{-23}\ \junit/\kunit$. In terms of
Boltzmann's constant, the relationship among the bulk
quantities for an ideal gas becomes
\begin{equation*}
        PV        =  nkT\eqquad, \qquad \text{[ideal gas]}
\end{equation*}
which is known as the ideal gas law. \index{ideal gas law}\index{Boltzmann's constant}
Although I won't prove it here, this equation applies to all ideal gases, even
though the derivation assumed a monoatomic ideal gas in a cubical box.
 (You may have seen it
written elsewhere as $PV=NRT$, where $N=n/N_A$ is the number of
moles of atoms, $R=kN_A$, and $N_A=6.0\times10^{23}$, called
Avogadro's number, is essentially the number of hydrogen
atoms in 1 g of hydrogen.)\index{Avogadro's number}

\begin{eg}{Pressure in a car tire}
\egquestion
After driving on the freeway for a while, the air
in your car's tires heats up from $10\degcunit$ to $35\degcunit$. How much
does the pressure increase?

\eganswer
The tires may expand a little, but we assume this
effect is small, so the volume is nearly constant. From the
ideal gas law, the ratio of the pressures is the same as the
ratio of the absolute temperatures,
\begin{align*}
          P_2/ P_1
                &= T_2/ T_1\\
                &=(308\ \kunit)/(283\ \kunit)\\
                &= 1.09\eqquad,\\
\end{align*}
or a 9\% increase.
\end{eg}

\begin{eg}{Earth's senescence}
Microbes were the only life on Earth up until the relatively recent
advent of multicellular life, and are arguably still the dominant form
of life on our planet. Furthermore, the sun has been gradually heating
up ever since it first formed, and this continuing process will soon
(``soon'' in the sense of geological time) eliminate multicellular life
again. Heat-induced decreases in the atmosphere's $\zu{CO}_2$ content will
kill off all complex plants within about 500 million years, and
although some animals may be able to live by eating algae, it will only
be another few hundred million years at most until the planet is
completely heat-sterilized.

Why is the sun getting brighter? The only thing that keeps a star
like our sun from collapsing due to its own gravity is the pressure
of its gases. The sun's energy comes from nuclear reactions at its core,
and the net result of these reactions is to fuse hydrogen atoms into
helium atoms. It takes four hydrogens to make one helium,
so the number of atoms in the sun is continuously decreasing. Since
$PV=nkT$, this causes a decrease in pressure, which makes the core
contract. As the core contracts, collisions between hydrogen atoms
become more frequent, and the rate of fusion reactions increases.
\end{eg}


<% marg(30) %>
<%
  fig(
    'space-suit',
    %q{A space suit (example \ref{eg:p-delta-v}).}
  )
%>
<% end_marg %>
\begin{eg}{A piston, a refrigerator, and a space suit}\label{eg:p-delta-v}
Both sides of the equation $PV=nkT$ have units of energy.
Suppose the pressure in a cylinder of gas pushes a piston out,
as in the power stroke of an automobile engine. Let the
cross-sectional area of the piston and cylinder be $A$,
and let the piston travel a small distance $\Delta x$. Then
the gas's force on the piston $F=PA$ does an amount of
mechanical work $W=F\Delta x=PA\Delta x=P\Delta V$, where
$\Delta V$ is the change in volume. This energy has to come
from somewhere; it comes from cooling the gas.
In a car, what this means is that we're harvesting
the energy released by burning the gasoline.

In a refrigerator, we use the same process to cool
the gas, which then cools the food.

In a space suit, the quantity $P\Delta V$ represents
the work the astronaut has to do because bending her
limbs changes the volume of the suit. The suit inflates
under pressure like a balloon, and doesn't want to bend.
This makes it very tiring to work for any significant
period of time.
\end{eg}

 %
\vfill
<% end_sec() %>
<% end_sec() %>
<% begin_sec("Entropy",3) %>
<% begin_sec("Efficiency and grades of energy") %>
Some forms of energy are more convenient than others in
certain situations. You can't run a spring-powered
mechanical clock on a battery, and you can't run a
battery-powered clock with mechanical energy. However, there
is no fundamental physical principle that prevents you from
converting 100\% of the electrical energy in a battery into
mechanical energy or vice-versa. More efficient motors and
generators are being designed every year. In general, the
laws of physics permit perfectly efficient conversion within
a broad class of forms of energy.

Heat is different. Friction tends to convert other forms of
energy into heat even in the best lubricated machines. When
we slide a book on a table, friction brings it to a stop and
converts all its kinetic energy into heat, but we never
observe the opposite process, in which a book spontaneously
converts heat energy into mechanical energy and starts
moving! Roughly speaking, heat is different because it is
disorganized. Scrambling an egg is easy. Unscrambling it is
harder.
<% marg(64) %>
<%
  fig(
    'turbinehotcold',
    %q{%
      The temperature difference between the hot and cold
      parts of the air can be used to extract mechanical energy,
      for example with a fan blade that spins because of the
      rising hot air currents.
    }
  )
%>
\spacebetweenfigs
<%
  fig(
    'turbineuniftemp',
    %q{%
      If the temperature of the air is first allowed to become
      uniform, then no mechanical energy can be extracted. The
      same amount of heat energy is present, but it is no longer
      accessible for doing mechanical work.
    }
  )
%>
<% end_marg %>

We summarize these observations by saying that heat is a
lower grade of energy than other forms such as mechanical
energy.

Of course it is possible to convert heat into other forms of
energy such as mechanical energy, and that is what a car
engine does with the heat created by exploding the air-gasoline
mixture. But a car engine is a tremendously inefficient
device, and a great deal of the heat is simply wasted
through the radiator and the exhaust. Engineers have never
succeeded in creating a perfectly efficient device for
converting heat energy into mechanical energy, and we now
know that this is because of a deeper physical principle
that is far more basic than the design of an engine.

 %
<% end_sec() %>
<% begin_sec("Heat engines") %>
Heat may be more useful in some forms than in other, i.e.,
there are different grades of heat energy. In figure \ref{fig:turbinehotcold},
the difference in temperature can be used to extract
mechanical work with a fan blade. This principle is used in
power plants, where steam is heated by burning oil or by
nuclear reactions, and then allowed to expand through a
turbine which has cooler steam on the other side. On a
smaller scale, there is a Christmas toy that consists of a
small propeller spun by the hot air rising from a set of
candles, very much like the setup shown in the figure.

In figure \ref{fig:turbineuniftemp}, however, no mechanical work can be extracted
because there is no difference in temperature. Although the
air in \ref{fig:turbineuniftemp} has the same total amount of energy as the air in
\ref{fig:turbinehotcold}, the heat in \ref{fig:turbineuniftemp} is a lower grade of energy, since
none of it is accessible for doing mechanical work.

In general, we define a heat engine as any device that takes
heat from a reservoir of hot matter, extracts some of the
heat energy to do mechanical work, and expels a lesser
amount of heat into a reservoir of cold matter. The
efficiency of a heat engine equals the amount of useful work
extracted, $W$, divided by the amount of energy we had to pay
for in order to heat the hot reservoir. This latter amount
of heat is the same as the amount of heat the engine
extracts from the high-temperature reservoir, $Q_H$. (The
letter $Q$ is the standard notation for a transfer of heat.)
By conservation of energy, we have $Q_H=W+Q_L$, where $Q_L$ is the
amount of heat expelled into the low-temperature reservoir,
so the efficiency of a heat engine, $W/Q_H$, can be rewritten
as
\begin{equation*}
        \text{efficiency}        =  1-\frac{Q_L}{Q_H}\eqquad. \qquad \text{[efficiency of any heat
engine]}
\end{equation*}

<% marg(76) %>
<%
  fig(
    'carnota',
    %q{%
       The beginning of the first expansion stroke, in which
      the working gas is kept in thermal equilibrium with the hot
      reservoir.
    }
  )
%>
\spacebetweenfigs
<%
  fig(
    'carnotb',
    %q{%
      The beginning of the second expansion stroke, in which
      the working gas is thermally insulated. The working gas
      cools because it is doing work on the piston and thus losing
      energy.
    }
  )
%>
\spacebetweenfigs
<%
  fig(
    'carnotc',
    %q{%
      The beginning of the first compression stroke. The
      working gas begins the stroke at the same temperature as the
      cold reservoir, and remains in thermal contact with it the
      whole time. The engine does negative work.
    }
  )
%>
\spacebetweenfigs
<%
  fig(
    'carnotd',
    %q{%
      The beginning of the second compression stroke, in which
      mechanical work is absorbed, heating the working gas back up
      to $T_H$.
    }
  )
%>
<% end_marg %>
It turns out that there is a particular type of heat engine,
the Carnot engine,\index{Carnot engine}\index{engine!Carnot} which, although not 100\% efficient, is
more efficient than any other. The grade of heat energy in a
system can thus be unambiguously defined in terms of the
amount of heat energy in it that \emph{cannot} be extracted, even by
a Carnot engine.

How can we build the most efficient possible engine? Let's
start with an unnecessarily inefficient engine like a car
engine and see how it could be improved. The radiator and
exhaust expel hot gases, which is a waste of heat energy.
These gases are cooler than the exploded air-gas mixture
inside the cylinder, but hotter than the air that surrounds
the car. We could thus improve the engine's efficiency
by adding an auxiliary heat engine to it, which would
operate with the first engine's exhaust as its hot reservoir and the air as
its cold reservoir. In general, any heat engine that expels
heat at an intermediate temperature can be made more
efficient by changing it so that it expels heat only at the
temperature of the cold reservoir.

Similarly, any heat engine that absorbs some energy at an
intermediate temperature can be made more efficient by
adding an auxiliary heat engine to it which will operate
between the hot reservoir and this intermediate temperature.

Based on these arguments, we define a Carnot engine as a
heat engine that absorbs heat only from the hot reservoir
and expels it only into the cold reservoir. Figures \ref{fig:carnota}-\ref{fig:carnotd}
show a realization of a Carnot engine using a piston in a
cylinder filled with a monoatomic ideal gas. This gas, known
as the working fluid, is separate from, but exchanges energy
with, the hot and cold reservoirs. 
It turns out that
this particular Carnot engine has an
efficiency given by
\begin{equation*}
        \text{efficiency}        =  1 - \frac{T_L}{T_H}\eqquad, \qquad \text{[efficiency of a Carnot engine]} 
\end{equation*}
where $T_L$ is the temperature of the cold reservoir and $T_H$ is
the temperature of the hot reservoir. 
(A proof of this fact is given in my book \emph{Simple Nature}, which you
can download for free.)

Even if you do not wish to dig into the details of the
proof, the basic reason for the temperature dependence is
not so hard to understand. Useful mechanical work is done on
strokes \ref{fig:carnota} and \ref{fig:carnotb}, in which the gas expands. The motion of
the piston is in the same direction as the gas's force on
the piston, so positive work is done on the piston. In
strokes \ref{fig:carnotc} and \ref{fig:carnotd}, however, the gas does negative work on
the piston. We would like to avoid this negative work, but
we must design the engine to perform a complete cycle.
Luckily the pressures during the compression strokes are
lower than the ones during the expansion strokes, so the
engine doesn't undo all its work with every cycle. The
ratios of the pressures are in proportion to the ratios of
the temperatures, so if $T_L$ is 20\% of $T_H$, the engine is 80\%
efficient.

We have already proved that any engine that is not a Carnot
engine is less than optimally efficient, and it is also true
that all Carnot engines operating between a given pair of
temperatures $T_H$ and $T_L$ have the same efficiency.
Thus a Carnot engine
is the most efficient possible heat engine.

<% end_sec() %>
<% begin_sec("Entropy") %>
We would like to have some numerical way of measuring the
grade of energy in a system. We want this quantity, called
entropy, to have the following two properties:

<% marg(0) %>
<%
  fig(
    'waterwheel',
    %q{%
      Entropy can be understood using the metaphor of a water
      wheel. Letting the water levels equalize is like letting the
      entropy maximize. Taking water from the high side and
      putting it into the low side increases the entropy. Water levels in this
      metaphor correspond to temperatures in the actual definition of entropy.
    }
  )
%>
<% end_marg %>
(1) Entropy is additive. When we combine two systems and
consider them as one, the entropy of the combined system
equals the sum of the entropies of the two original systems.
(Quantities like mass and energy also have this property.)

(2) The entropy of a system is not changed by operating a
Carnot engine within it.

It turns out to be simpler and more useful to define changes
in entropy than absolute entropies. Suppose as an example
that a system contains some hot matter and some cold matter.
It has a relatively high grade of energy because a heat
engine could be used to extract mechanical work from it. But
if we allow the hot and cold parts to equilibrate at some
lukewarm temperature, the grade of energy has gotten worse.
Thus putting heat into a hotter area is more useful than
putting it into a cold area. Motivated by these
considerations, we define a change in entropy as
follows:
\begin{multline*}
        \Delta S  =  \frac{Q}{T}        \qquad \shoveright{\text{[change in entropy when adding}}\\                                                \shoveright{\text{heat $Q$ to matter at temperature $T$;}}\\
                                                {\text{$\Delta S$ is negative if heat is taken out]}}
\end{multline*}
A system with a higher grade of energy has a lower entropy.

\begin{eg}{Entropy is additive.}
Since changes in entropy are defined by an additive quantity
(heat) divided by a non-additive one (temperature), entropy
is additive.
\end{eg}


\begin{eg}{Entropy isn't changed by a Carnot engine.}\label{eg:carnotnoentropychange}
The efficiency of a heat engine is defined by
\begin{equation*}
        \text{efficiency}  =  1 -  Q_L/ Q_H\eqquad,
\end{equation*}
and the efficiency of a Carnot engine is
\begin{equation*}
        \text{efficiency}  =  1 -  T_L/ T_H\eqquad,
\end{equation*}
so for a Carnot engine we have $Q_L/ Q_H =  T_L/ T_H$,
which can be rewritten as
$Q_L/ T_{L} =  Q_{H}/ T_H$.
The entropy lost by the hot reservoir is therefore
the same as the entropy gained by the cold one.
\end{eg}

\begin{eg}{Entropy increases in heat conduction.}\label{eg:s-increase-conduction}
When a hot object gives up energy to a cold one,
conservation of energy tells us that the amount of heat lost
by the hot object is the same as the amount of heat gained
by the cold one. The change in entropy is $- Q/ T_{H}+ Q/ T_L$, which
is positive because $ T_L< T_H$.
\end{eg}


\begin{eg}{Entropy is increased by a non-Carnot engine.}
The efficiency of a non-Carnot engine is less than 1 -
$ T_L/ T_H$, so $Q_L/ Q_{H} >  T_{L}/ T_H$
 and $Q_L/ T_{L} >  Q_{H}/ T_H$. This means that
the entropy increase in the cold reservoir is greater than
the entropy decrease in the hot reservoir.
\end{eg}


\begin{eg}{A book sliding to a stop}\label{eg:s-increase-sliding}
A book slides across a table and comes to a stop. Once it stops, all its
kinetic energy has been transformed into heat. As the book and
table heat up, their entropies both increase, so the total entropy increases
as well.
\end{eg}

Examples \ref{eg:s-increase-conduction}-\ref{eg:s-increase-sliding} involved closed systems, and in all of
them the total entropy either increased or stayed the same. It
never decreased. Here are two examples of schemes for
decreasing the entropy of a closed system, with explanations
of why they don't work.

\begin{eg}{Using a refrigerator to decrease entropy?}
\egquestion
A refrigerator takes heat from a cold area and
dumps it into a hot area. (1) Does this lead to a net
decrease in the entropy of a closed system? (2) Could you
make a Carnot engine more efficient by running a
refrigerator to cool its low-temperature reservoir and
eject heat into its high-temperature reservoir?

\eganswer
(1) No. The heat that comes off of the radiator
coils on the back of your kitchen fridge
is a great deal more than the heat the fridge removes from
inside; the difference is what it costs to run your fridge.
The heat radiated from the coils is so much more than the
heat removed from the inside that the increase in the
entropy of the air in the room is greater than the decrease
of the entropy inside the fridge. The most efficient
refrigerator is actually a Carnot engine running in reverse,
which leads to neither an increase
nor a decrease in entropy.

(2) No. The most efficient refrigerator is a reversed Carnot
engine. You will not achieve anything by running one Carnot
engine in reverse and another forward. They will just cancel
each other out.
\end{eg}


\begin{eg}{Maxwell's daemon}
\egquestion
Physicist James Clerk Maxwell imagined pair of neighboring rooms, their air being initially in
thermal equilibrium, having a partition across the middle
with a tiny door. A miniscule daemon is posted at the door
with a little ping-pong paddle, and his duty is to try to
build up faster-moving air molecules in room B and slower
ones in room A. For instance, when a fast molecule is
headed through the door, going from A to B, he lets it by,
but when a slower than average molecule tries the same
thing, he hits it back into room A. Would this decrease the
total entropy of the pair of rooms?

\eganswer
No. The daemon needs to eat, and we can think of his body as
a little heat engine. His metabolism is less efficient
than a Carnot engine, so he ends up increasing the entropy
rather than decreasing it.
\end{eg}

Observation such as these lead to the following hypothesis,
known as the second law of thermodynamics:\index{thermodynamics!second law of}

\begin{important}
The entropy of a closed system always increases, or at best
stays the same: $\Delta S\ge0$.
\end{important}

At present my arguments to support this statement may seem
less than convincing, since they have so much to do with
obscure facts about heat engines. A more satisfying and
fundamental explanation for the continual increase in
entropy was achieved by Ludwig Boltzmann, and you may wish
to learn more about Boltzmann's ideas from my book \emph{Simple Nature}, which you
can download for free. Briefly, Boltzmann realized that
entropy was a measure of randomness or disorder at the atomic level, and
disorder doesn't spontaneously change into order.

To emphasize the fundamental and universal nature of the second
law, here are a few examples.

\begin{eg}{Entropy and evolution}\label{eg:entropy-evolution}
A favorite argument of many creationists who don't believe
in evolution is that evolution would violate the second law
of thermodynamics: the death and decay of a living thing
releases heat (as when a compost heap gets hot) and lessens
the amount of energy available for doing useful work, while
the reverse process, the emergence of life from nonliving
matter, would require a decrease in entropy. Their argument
is faulty, since the second law only applies to closed
systems, and the earth is not a closed system. The earth is
continuously receiving energy from the sun.
\end{eg}

\begin{eg}{The heat death of the universe}
Victorian philosophers realized that living things had low
entropy, as discussed in example \ref{eg:entropy-evolution}, and spent a lot of time worrying about
the heat death of the universe: eventually the universe
would have to become a high-entropy, lukewarm soup, with no
life or organized motion of any kind. Fortunately (?), we
now know a great many other things that will make the
universe inhospitable to life long before its entropy is
maximized. Life on earth, for instance, will end when the
sun evolves into a giant star and vaporizes our planet.
\end{eg}

\begin{eg}{Hawking radiation}\index{black hole}\index{Hawking radiation}
Any process that could destroy heat (or convert it into
nothing but mechanical work) would lead to a reduction in
entropy. Black holes are supermassive stars whose gravity is
so strong that nothing, not even light, can escape from them
once it gets within a boundary known as the event horizon.
Black holes are commonly observed to suck hot gas into them.
Does this lead to a reduction in the entropy of the
universe? Of course one could argue that the entropy is
still there inside the black hole, but being able to
``hide'' entropy there amounts to the same thing as being
able to destroy entropy.

The physicist Steven Hawking was bothered by this question,
and finally realized that although the actual stuff that
enters a black hole is lost forever, the black hole will
gradually lose energy  in the form of light emitted from
just outside the event horizon. This light ends up
reintroducing the original entropy back into the universe at
large.
\end{eg}


 %%============================= homework ==============================

<% end_sec() %>
<% end_sec() %>
