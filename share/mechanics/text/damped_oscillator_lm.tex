<% begin_sec("Energy lost from vibrations",3) %>

<% marg(0) %>
<%
  fig(
    'damped-and-undamped-x-t',
    %q{Friction has the effect of pinching the $x-t$ graph of a vibrating object.}
  )
%>
<% end_marg %>

Until now, we have been making the relatively unrealistic
assumption that a vibration would never die out. For a
realistic mass on a spring, there will be friction, and the
kinetic and potential energy of the vibrations will
therefore be gradually converted into heat. Similarly, a
guitar string will slowly convert its kinetic and potential
energy into sound. In all cases, the effect is to ``pinch''
the sinusoidal $x-t$ graph more and more with passing time.
Friction is not necessarily bad in this context --- a
musical instrument that never got rid of any of its energy
would be completely silent! The dissipation of the energy in
a vibration is known as \index{damping!defined}damping.

<% self_check('damping',<<-'SELF_CHECK'
Most people who try to draw graphs like those shown on the
left will tend to shrink their wiggles horizontally as well
as vertically. Why is this wrong?
  SELF_CHECK
  ) %>

In the graphs in figure \figref{damped-and-undamped-x-t}, I have not shown any point at
which the damped vibration finally stops completely. Is this
realistic? Yes and no. If energy is being lost due to
friction between two solid surfaces, then we expect the
force of friction to be nearly independent of velocity. This
constant friction force puts an upper limit on the total
distance that the vibrating object can ever travel without
replenishing its energy, since work equals force times
distance, and the object must stop doing work when its
energy is all converted into heat. (The friction force does
reverse directions when the object turns around, but
reversing the direction of the motion at the same time that
we reverse the direction of the force makes it certain that
the object is always doing positive work, not negative work.)

Damping due to a constant friction force is not the only
possibility however, or even the most common one. A pendulum
may be damped mainly by air friction, which is approximately
proportional to $v^2$, while other systems may exhibit
friction forces that are proportional to $v$. It turns out
that friction proportional to $v$ is the simplest case to
analyze mathematically, and anyhow all the important
physical insights can be gained by studying this case.

If the friction force is proportional to $v$, then as the
vibrations die down, the frictional forces get weaker due to
the lower speeds. The less energy is left in the system, the
more miserly the system becomes with giving away any more
energy. Under these conditions, the vibrations theoretically
never die out completely, and mathematically, the loss of
energy from the system is exponential: the system loses a
fixed percentage of its energy per cycle. This is referred
to as \index{exponential decay!defined}exponential decay.

A non-rigorous proof is as follows. The force of friction is
proportional to $v$, and $v$ is proportional to how far the
object travels in one cycle, so the frictional force is
proportional to amplitude. The amount of work done by
friction is proportional to the force and to the distance
traveled, so the work done in one cycle is proportional to
the square of the amplitude. Since both the work and the
energy are proportional to $A^2$, the amount of energy taken
away by friction in one cycle is a fixed percentage of the
amount of energy the system has.

<% marg(0) %>
<%
  fig(
    'sc-strongly-damped',
    %q{The amplitude is halved with each cycle.}
  )
%>
<% end_marg %>
<% self_check('damping-energy',<<-'SELF_CHECK'
Figure \figref{sc-strongly-damped} shows an x-t graph for a strongly damped
vibration, which loses half of its amplitude with every
cycle. What fraction of the energy is lost in each cycle?
  SELF_CHECK
  ) %>

It is customary to describe the amount of damping with a
quantity called the \index{quality factor!defined}quality
factor, $Q$, defined as the number of cycles required for
the energy to fall off by a factor of 535. (The origin of
this obscure numerical factor is $e^{2\pi}$, where $e=2.71828\ldots$
is the base of natural logarithms. Choosing this particular number causes
some of our later equations to come out nice and simple.) The terminology arises
from the fact that friction is often considered a bad thing,
so a mechanical device that can vibrate for many oscillations
before it loses a significant fraction of its energy would
be considered a high-quality device.

\begin{eg}{Exponential decay in a trumpet}
\egquestion The vibrations of the air column inside a trumpet
have a $Q$ of about 10. This means that even after the
trumpet player stops blowing, the note will keep sounding
for a short time. If the player suddenly stops blowing, how
will the sound intensity 20 cycles later compare with the
sound intensity while she was still blowing?

\eganswer The trumpet's $Q$ is 10, so after 10 cycles the energy will have
fallen off by a factor of 535. After another 10 cycles we
lose another factor of 535, so the sound intensity is
reduced by a factor of $535 \times 535=2.9\times10^5$.
\end{eg}

The decay of a musical sound is part of what gives it its
character, and a good musical instrument should have the
right $Q$, but the $Q$ that is considered desirable is
different for different instruments. A guitar is meant to
keep on sounding for a long time after a string has been
plucked, and might have a $Q$ of 1000 or 10000. One of the
reasons why a cheap synthesizer sounds so bad is that the
sound suddenly cuts off after a key is released.

\begin{eg}{$Q$ of a stereo speaker}
Stereo speakers are not supposed to reverberate or ``ring''
after an electrical signal that stops suddenly. After all,
the recorded music was made by musicians who knew how to
shape the decays of their notes correctly. Adding a longer
``tail'' on every note would make it sound wrong. We
therefore expect that stereo speaker will have a very low
$Q$, and indeed, most speakers are designed with a $Q$ of
about 1. (Low-quality speakers with larger $Q$ values are
referred to as ``boomy.'')
\end{eg}

We will see later in the chapter that there are other
reasons why a speaker should not have a high $Q$.

<% end_sec() %>
