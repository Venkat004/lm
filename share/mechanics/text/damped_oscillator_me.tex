<% begin_sec("Energy lost from vibrations") %>
        <% begin_sec("Numerical treatment") %>\label{freedampednumerical}
        An oscillator that has friction is referred to as
        damped.\index{damped oscillations}\index{oscillations!damped}
        Let's use numerical techniques to find the motion of a damped
        oscillator that is released away from equilibrium, but experiences
        no driving force after that. We can expect that the motion will consist
        of oscillations that gradually die out.

        Friction is in general a very complicated phenomenon, and for example video games with racing cars
        in them include extremely sophisticated models of the friction between the tires and the road.
        On p.~\pageref{subsec:coulomb-friction} I presented a very simple model of friction that dates back
        to the French physicist Coulomb, who worked in the era of the French revolution. There is nothing
        sacred about this model. For example, it doesn't work well for lubricated surfaces.
        Most of the ideas we're going to learn about damped vibrations are at least qualitatively
        correct regardless of what technical assumptions are made about friction, but the math turns out
        to be much simpler if we choose, instead of the Coulomb model, one in which the force of friction
        on an object is given by $F=-bv$, where $v$ is the object's speed and $b$ is a constant.
        This is in contrast to the Coulomb model, in which the force is independent of speed.

        Newton's second law, $a=F/m$, gives $a=(-kx-bv)/m$.
        This becomes a little prettier if we rewrite it in the form
        \begin{equation*}
                ma+bv+kx = 0\eqquad,
        \end{equation*}
        which gives symmetric treatment to three terms involving $x$ and its first and
        second derivatives, $v$ and $a$.
\begin{listing}{1}<%code_listing('damped.py',%q{
import math
k=39.4784  # chosen to give a period of 1 second
m=1.
b=0.211    # chosen to make the results simple
x=1.
v=0.
t=0.
dt=.01
n=1000
for j in range(n):
  x=x+v*dt
  a=(-k*x-b*v)/m
  if (v>0) and (v+a*dt<0) :
    print("turnaround at t=",t,", x=",x)
  v=v+a*dt
  t=t+dt
})%>\end{listing}
\begin{verbatim}
turnaround at t= 0.99 , x= 0.899919262445
turnaround at t= 1.99 , x= 0.809844934046
turnaround at t= 2.99 , x= 0.728777519477
turnaround at t= 3.99 , x= 0.655817260033
turnaround at t= 4.99 , x= 0.590154191135
turnaround at t= 5.99 , x= 0.531059189965
turnaround at t= 6.99 , x= 0.477875914756
turnaround at t= 7.99 , x= 0.430013546991
turnaround at t= 8.99 , x= 0.386940256644
turnaround at t= 9.99 , x= 0.348177318484
\end{verbatim}

        The spring constant, $k=4\pi{}=39.4784$ N/m, is designed so that
        if the undamped equation $f=(1/2\pi)\sqrt{k/m}$ was still true, the frequency
        would be 1 Hz. We start by noting that the addition of a small amount of damping
        doesn't seem to have changed the period at all, or at least not to within the
        accuracy of the calculation. You can check for yourself, however, that a large value
        of $b$, say 5 $\nunit\unitdot\sunit/\munit$, does change the period significantly.

        We release the mass from $x=1\ \munit$, and after one cycle, it only comes back
        to about $x=0.9\ \munit$. I chose $b=0.211\ \nunit\unitdot\sunit/\munit$ by fiddling
        around until I got this result, since a decrease of exactly 10\% is easy to
        discuss. Notice how the amplitude after two cycles is about $0.81\ \munit$,
        i.e., $1\ \munit$ times $0.9^2$: the amplitude has again dropped by exactly 10\%.
        This pattern continues for as long as the simulation runs, e.g.,
        for the last two cycles, we have 0.34818/0.38694=0.89982, or almost exactly 0.9
        again.
        It might have seemed capricious when I chose to use the unrealistic
        equation $F=-bv$, but this is the payoff. Only with $-bv$ friction do
        we get this kind of mathematically simple exponential decay.

        Because the decay is exponential, it never dies out completely; this is different from
        the behavior we would have had with Coulomb friction, which does make objects grind
        completely to a stop at some point. With friction that acts like $F=-bv$, $v$ gets smaller
        as the oscillations get smaller. The smaller and smaller force then causes them to die out at a
        rate that is slower and slower.

        <% end_sec() %>
<% begin_sec("Analytic treatment") %>\label{freedampedanalytic}

        Taking advantage of this unexpectedly simple result, let's find an analytic solution
        for the motion. The numerical output suggests that we assume a solution of the form
        \begin{equation*}
                x = Ae^{-ct}\sin (\omega{}_f t+\delta)\eqquad,
        \end{equation*}
        where the unknown constants
        $\omega_f$ and $c$ will presumably be related to $m$, $b$, and $k$.
        The constant $c$ indicates how quickly the oscillations die out. 
         The constant
        $\omega_f$ is, as before, defined as $2\pi$ times the frequency, with the subscript
        $f$ to indicate a free (undriven) solution. All our equations will come out
        much simpler if we use $\omega$s everywhere instead of $f\/$s from now on, and,
        as physicists often do,
        I'll generally use the word ``frequency'' to refer to $\omega$ when the context
        makes it clear what I'm talking about.
        The phase angle $\delta$ has no real physical
        significance, since we can define $t=0$ to be any moment in time we like.
<% marg(100) %>
<%
  fig(
    'dampedsine',
    %q{%
      A damped sine wave, of the form 
              $x =  Ae^{- ct}\zu{sin} (\omega_{f} t+\delta)$.
    }
  )
%>
\spacebetweenfigs
<%
  fig(
    'sc-compare-damping',
    %q{%
      Self-check \ref{sc:compare-damping}.
    }
  )
%>
<% end_marg %>

<% self_check('compare-damping',<<-'SELF_CHECK'
In figure \\figref{sc-compare-damping}, which graph has the greater value of $c$?
  SELF_CHECK
  ) %>

        The factor $A$ for the initial amplitude can also be omitted without loss of generality,
        since the equation we're trying to solve, $ma+bv+kx = 0$, is linear. That is,
        $v$ and $a$ are the first and second derivatives of $x$, and the derivative
        of $Ax$ is simply $A$ times the derivative of $x$. Thus, if $x(t)$ is a solution of the
        equation, then multiplying it by a constant gives an equally valid solution.
        This is another place where we see that a damping force proportional to $v$ is the easiest to handle
        mathematically. For a damping force proportional to $v^2$, for example, we would have
        had to solve the equation  $ma+bv^2+kx = 0$, which is nonlinear.
<% marg(-60) %>
<%
  fig(
    'damping-effect-on-frequency',
    %q{%
      A damped sine wave is compared with an undamped one, with $m$ and $k$ kept the same and only $b$ changed.
    }
  )
%>
<% end_marg %>

        For the purpose
        of determining $\omega_f$ and $c$, the
        most general form we need to consider is therefore
        $x = e^{-ct}\sin \omega_f t$ ,
        whose first and second derivatives are
        $v = e^{-ct}\left(-c \sin \omega_f t + \omega\cos \omega_f t\right) $
        and        $a = e^{-ct}\left(c^2 \sin \omega_f t 
                        -2\omega_f c \cos \omega_f t 
                        -\omega_f^2\sin \omega_f t\right)$.
        Plugging these into the equation $ma+bv+kx = 0$ and setting the sine and
        cosine parts equal to zero gives, after some tedious algebra,
        \begin{align*}
                c        &= \frac{b}{2m} \\
        \intertext{and}
                \omega_f        &= \sqrt{\frac{k}{m}-\frac{b^2}{4m^2}}\eqquad.
        \end{align*}
        Intuitively, we expect friction to ``slow down'' the motion, as when we ride a bike
        into a big patch of mud. ``Slow down,'' however, could have more than one meaning here. It
        could mean that the oscillator would take more time to complete each cycle, or it could
        mean that as time went on, the oscillations would die out, thus giving smaller velocities.

        Our mathematical results show that both of these things happen. 
        The first equation says that $c$, which indicates how quickly the
        oscillations damp out, is directly related to $b$, the strength of the damping.

        The second equation, for the frequency, can be compared with the result from page
        \pageref{sec:shm-k-m} of $\sqrt{k/m}$ for the undamped system.
        Let's refer to this now as $\omega_\zu{o}$, to distinguish it from the actual frequency
        $\omega_f$ of the free oscillations when damping is present. The result for
        $\omega_f$ will be less than $\omega_\zu{o}$, due to the presence of the $b^2/4m^2$ term.
        This tells us that the addition of friction to the system does increase the time required
        for each cycle. However, it is very common for the
        $b^2/4m^2$ term to be negligible, so that $\omega_f\approx\omega_\zu{o}$.

        Figure \figref{damping-effect-on-frequency} shows an example. The damping here is quite strong:
        after only one cycle of oscillation, the amplitude has already been reduced by a factor of 2,
        corresponding to a factor of 4 in energy. However, the frequency of the damped oscillator is
        only about 1\% lower than that of the undamped one; after five periods, the accumulated lag is just barely visible in the offsetting
        of the arrows. We can see that extremely strong damping --- even stronger than this --- would have
        been necessary in order to make $\omega_f\approx\omega_\zu{o}$ a poor approximation.

It is customary to describe the amount of damping with a
quantity called the \index{quality factor!defined}quality
factor, $Q$, defined as the number of cycles required for
the energy to fall off by a factor of 535. (The origin of
this obscure numerical factor is $e^{2\pi}$, where $e=2.71828\ldots$
is the base of natural logarithms. Choosing this particular number causes
some of our later equations to come out nice and simple.) The terminology arises
from the fact that friction is often considered a bad thing,
so a mechanical device that can vibrate for many oscillations
before it loses a significant fraction of its energy would
be considered a high-quality device.

<% end_sec() %>
<% end_sec() %>
