<% begin_sec("The Search for a Perpetual Motion Machine",0) %>

Don't underestimate greed and laziness as forces for
progress. Modern chemistry was born from the collision of
lust for gold with distaste for the hard work of finding it
and digging it up. Failed efforts by generations of
\index{alchemists}alchemists to turn lead into gold led
finally to the conclusion that it could not be done: certain
substances, the chemical elements, are fundamental, and
chemical reactions can neither increase nor decrease the
amount of an \index{element, chemical}element such as gold.

Now flash forward to the early industrial age. Greed and
laziness have created the factory, the train, and the ocean
liner, but in each of these is a boiler room where someone
gets sweaty shoveling the coal to fuel the steam engine.
Generations of inventors have tried to create a machine,
called a \index{perpetual motion machine}perpetual motion
machine, that would run forever without fuel. Such a machine
is not forbidden by Newton's laws of motion, which are built
around the concepts of force and inertia. Force is free, and
can be multiplied indefinitely with pulleys, gears, or
levers. The principle of inertia seems even to encourage the
belief that a cleverly constructed machine might not ever run down.

<% marg(0) %>

<%
  fig(
    'perpetual-motion-magnet',
    %q{%
      The magnet draws the ball to the
      top of the ramp, where it falls through
      the hole and rolls back to the bottom.
    }
  )
%>
\spacebetweenfigs
<%
  fig(
    'perpetual-motion-arms',
    %q{%
      As the wheel spins clockwise, the
      flexible arms sweep around and bend
      and unbend. By dropping off its ball
      on the ramp, the arm is supposed to
      make itself lighter and easier to lift over
      the top. Picking its own ball back up
      again on the right, it helps to pull the
      right side down.
    }
  )
%>

<% end_marg %>
Figures \figref{perpetual-motion-magnet} and \figref{perpetual-motion-arms}
show two of the innumerable perpetual motion
machines that have been proposed. The reason these two
examples don't work is not much different from the reason
all the others have failed. Consider machine
\figref{perpetual-motion-magnet}. Even if we
assume that a properly shaped ramp would keep the ball
rolling smoothly through each cycle, friction would always
be at work. The designer imagined that the machine would
repeat the same motion over and over again, so that every
time it reached a given point its speed would be exactly the
same as the last time. But because of friction, the speed
would actually be reduced a little with each cycle, until
finally the ball would no longer be able to make it over the top.

Friction has a way of creeping into all moving systems. The
rotating earth might seem like a perfect perpetual motion
machine, since it is isolated in the vacuum of outer space
with nothing to exert frictional forces on it. But in fact
our planet's rotation has slowed drastically since it first
formed, and the earth continues to slow its rotation, making
today just a little longer than yesterday. The very subtle
source of friction is the tides. The moon's gravity raises
bulges in the earth's oceans, and as the earth rotates the
bulges progress around the planet. Where the bulges
encounter land, there is friction, which slows the earth's
rotation very gradually.

<% end_sec() %>
<% begin_sec("Energy",0) %>

The analysis based on friction is somewhat superficial,
however. One could understand friction perfectly well and
yet imagine the following situation. Astronauts bring back a
piece of magnetic ore from the moon which does not behave
like ordinary magnets. A normal bar magnet,
\subfigref{moon-rock}{1}, attracts a
piece of iron essentially directly toward it, and has no
left- or right-handedness. The moon rock, however, exerts
forces that form a whirlpool pattern around it, 2.
NASA goes to a machine shop and has the moon rock put in a lathe
and machined down to a smooth cylinder, 3.
If we now
release a ball bearing on the surface of the cylinder, the
magnetic force whips it around and around at ever higher
speeds. Of course there is some friction, but there is a net
gain in speed with each revolution.

Physicists would lay long odds against the discovery of such
a moon rock, not just because it breaks the rules that
magnets normally obey but because, like the alchemists, they
have discovered a very deep and fundamental principle of
nature which forbids certain things from happening. The
first alchemist who deserved to be called a chemist was the
one who realized one day, ``In all these attempts to create
gold where there was none before, all I've been doing is
shuffling the same atoms back and forth among different test
tubes. The only way to increase the amount of gold in my
laboratory is to bring some in through the door.'' It was
like having some of your money in a checking account and
some in a savings account. Transferring money from one
account into the other doesn't change the total amount.

<% marg(43) %>
<%
  fig(
    'moon-rock',
    %q{%
      A mysterious moon rock makes a perpetual
      motion machine.
    }
  )
%>
\spacebetweenfigs
<%
  fig(
    'faucet',
    %q{Example \ref{eg:faucet}.}
  )
%>

<% end_marg %>%
We say that the number of grams of gold is a \emph{conserved}
quantity. In this context, the word ``conserve'' does not
have its usual meaning of trying not to waste something. In
physics, a conserved quantity is something that you wouldn't
be able to get rid of even if you wanted to. Conservation
laws in physics always refer to a \emph{closed system},
meaning a region of space with boundaries through which the
quantity in question is not passing. In our example, the
alchemist's laboratory is a closed system because no gold is
coming in or out through the doors.

\begin{eg}{Conservation of mass}\label{eg:faucet}
In figure \figref{faucet},
the stream
of water is fatter near the mouth of the faucet, and skinnier lower down. This is because the
water speeds up as it falls. If the cross-sectional area of the stream was equal all along
its length, then the rate of flow through a lower cross-section
would be greater than the rate of flow through a cross-section higher up. Since the flow
is steady, the amount of water between the two cross-sections stays constant.
The cross-sectional area of the stream must therefore shrink in inverse
proportion to the increasing speed of the falling water. This is an example of
conservation of mass. 
\end{eg}

In general, the amount of any particular substance is not conserved. Chemical
reactions can change one substance into another, and nuclear reactions can even
change one element into another. The total mass of all substances is however
conserved:

\begin{important}[the law of conservation of mass]
The total mass of a closed system always remains constant. Mass cannot
be created or destroyed, but only transferred from one system to another.
\end{important}

A similar lightbulb eventually lit up in the heads of the
people who had been frustrated trying to build a perpetual
motion machine. In perpetual motion machine
\figref{perpetual-motion-magnet},
consider the motion of one of the balls.
It performs a cycle of rising and falling. On the way down
it gains speed, and coming up it slows back down. Having a
greater speed is like having more money in your checking
account, and being high up is like having more in your
savings account. The device is simply shuffling funds back
and forth between the two. Having more balls doesn't change
anything fundamentally. Not only that, but friction is
always draining off money into a third ``bank account:''
heat. The reason we rub our hands together when we're cold
is that kinetic friction heats things up. The continual
buildup in the ``heat account'' leaves less and less for the
``motion account'' and ``height account,'' causing the
machine eventually to run down.

These insights can be distilled into the following basic
principle of physics:

\begin{important}[the law of conservation of energy]
It is possible to give a numerical rating, called energy, to the state
of a physical system. The total energy is found by adding up
contributions from characteristics of the system such
as motion of objects in it, heating of the objects, and the
relative positions of objects that interact via forces. The total
energy of a closed system always remains constant. Energy cannot
be created or destroyed, but only transferred from one system to another.
\end{important}

The moon rock story violates conservation of energy because
the rock-cylinder and the ball together constitute a closed
system. Once the ball has made one revolution around the
cylinder, its position relative to the cylinder is exactly
the same as before, so the numerical energy rating
associated with its position is the same as before. Since
the total amount of energy must remain constant, it is
impossible for the ball to have a greater speed after one
revolution. If it had picked up speed, it would have more
energy associated with motion, the same amount of energy
associated with position, and a little more energy
associated with heating through friction. There cannot be a
net increase in energy.

\begin{eg}{Converting one form of energy to another}
\noindent\emph{Dropping a rock:\/} The rock loses energy because of its
changing position with respect to the earth. Nearly all that
energy is transformed into energy of motion, except for a
small amount lost to heat created by air friction.

\noindent\emph{Sliding in to home base:\/} The runner's energy of motion is
nearly all converted into heat via friction with the ground.

\noindent\emph{Accelerating a car:\/} The gasoline has energy stored in it,
which is released as heat by burning it inside the engine.
Perhaps 10\% of this heat energy is converted into the car's
energy of motion. The rest remains in the form of heat,
which is carried away by the exhaust.

\noindent\emph{Cruising in a car:\/} As you cruise at constant speed in your
car, all the energy of the burning gas is being converted
into heat. The tires and engine get hot, and heat is also
dissipated into the air through the radiator and the exhaust.

\noindent\emph{Stepping on the brakes:\/} All the energy of the car's motion
is converted into heat in the brake shoes.
\end{eg}

<% marg(0) %>
<%
  fig(
    'eg-stevin',
    %q{%
      Example \ref{eg:stevin}.
    }
  )
%>

<% end_marg %>

\begin{eg}{Stevin's machine}\label{eg:stevin}\index{Stevin, Simon}
The Dutch mathematician and engineer Simon Stevin proposed the imaginary
machine shown in figure \figref{eg-stevin}, which he had inscribed on his
tombstone. This is an interesting example, because it shows a link between
the force concept used earlier in this course, and the energy concept being
developed now.

The point of the imaginary machine is to show the mechanical advantage of
an inclined plane. In this example, the triangle has the proportions 3-4-5,
but the argument works for any right triangle. We imagine that the chain
of balls slides without friction, so that no energy is ever converted into
heat. If we were to slide the chain clockwise by one step, then each ball
would take the place of the one in front of it, and the over all configuration
would be exactly the same. Since energy is something that only depends on the
state of the system, the energy would have to be the same. Similarly for
a counterclockwise rotation, no energy of position would be released by gravity.
This means that if we place the chain on the triangle, and release it at rest,
it can't start moving, because there would be no way for it to convert energy
of position into energy of motion. Thus the chain must be perfectly balanced.
Now by symmetry, the arc of the chain hanging underneath the triangle has
equal tension at both ends, so removing this arc wouldn't affect the balance
of the rest of the chain. This means that a weight of three units hanging
vertically balances a weight of five units hanging diagonally along the
hypotenuse.

The mechanical advantage of the inclined plane is therefore $5/3$,
which is exactly  the same as the result, $1/\sin\theta$, that we got on p.~\pageref{eg:ramp-mechanical-advantage}
by analyzing force vectors. What this shows is that Newton's laws and conservation
laws are not logically separate, but rather are very closely related
descriptions of nature. In the cases where Newton's laws are true, they give
the same answers as the conservation laws. This is an example of a more
general idea, called the correspondence principle, about how science progresses
over time.\index{correspondence principle}
When a newer, more general theory is proposed to replace an older theory,
the new theory must agree with the old one in the realm of applicability
of the old theory, since the old theory only became accepted as a valid
theory by being verified experimentally in a variety of experiments.
In other words, the new theory must be backward-compatible with the old one.
Even though conservation laws can prove things that Newton's laws can't
(that perpetual motion is impossible, for example), they aren't going
to \emph{disprove} Newton's laws when applied to mechanical systems where
we already knew Newton's laws were valid.
\end{eg}


<% marg(20) %>
<%
  fig(
    'hoover-dam',
    %q{%
      Discussion question \ref{dq:hoover-dam}. The water behind the Hoover Dam has
      energy because of its position relative
      to the planet earth, which is attracting
      it with a gravitational force. Letting
      water down to the bottom of the dam
      converts that energy into energy of
      motion. When the water reaches the
      bottom of the dam, it hits turbine
      blades that drive generators, and its
      energy of motion is converted into
      electrical energy.
    },
    {'anonymous'=>true}
  )
%>
<% end_marg %>
\startdq

\begin{dq}\label{dq:hoover-dam}
Hydroelectric power (water flowing over a dam to spin
turbines) appears to be completely free. Does this violate
conservation of energy? If not, then what is the ultimate
source of the electrical energy produced by a hydroelectric plant?
\end{dq}

\begin{dq}
How does the proof in example \ref{eg:stevin} fail if the assumption
of a frictionless surface doesn't hold?
\end{dq}

<% end_sec() %>
<% begin_sec("A Numerical Scale of Energy",3) %>

Energy comes in a variety of forms, and physicists didn't
discover all of them right away. They had to start
somewhere, so they picked one form of energy to use as a
standard for creating a numerical energy scale. (In fact the
history is complicated, and several different energy units
were defined before it was realized that there was a single
general energy concept that deserved a single consistent
unit of measurement.) One practical approach is to define an
energy unit based on heating water. The SI unit of energy is
the \index{joule (unit)}joule, J, (rhymes with ``cool''),
named after the British physicist James Joule. One Joule is
the amount of energy required in order to heat 0.24 g of
water by $1\degcunit$. The number 0.24 is not worth memorizing.
A convenient way of restating this definition is that
when heating water, $\text{heat}=cm\Delta T$, where $\Delta T$ is
the change in temperature in $\degcunit$, $m$ is the mass, and
we have defined the joule by defining the constant $c$, called
the specific heat capacity of water,\index{specific heat capacity}
to have the value $4.2\times10^3\ \junit/\kgunit\unitdot\degcunit$.

Note that heat, which is a form of energy, is completely
different from temperature, which is not. Twice as much heat
energy is required to prepare two cups of coffee as to make
one, but two cups of coffee mixed together don't have double
the temperature. In other words, the temperature of an
object tells us how hot it is, but the heat energy contained
in an object also takes into account the object's mass
and what it is made of.\footnote{In standard, formal terminology, there
is another, finer distinction. The word ``heat'' is used only
to indicate an amount of energy that is transferred,
whereas ``thermal energy'' indicates an amount of energy
contained in an object. I'm informal on this point, and
refer to both as heat, but you should be aware of the
distinction.}

Later we will encounter other quantities that are conserved
in physics, such as momentum and angular momentum, and the
method for defining them will be similar to the one we have
used for energy: pick some standard form of it, and then
measure other forms by comparison with this standard. The
flexible and adaptable nature of this procedure is part of
what has made conservation laws such a durable basis for the
evolution of physics.

\begin{eg}{Heating a swimming pool}
\egquestion If electricity costs 3.9 cents per MJ (1 MJ = 1
megajoule = $10^6$ J), how much does it cost to heat a
26000-gallon swimming pool from $10\degcunit\ $ to 18\degcunit?

\eganswer Converting gallons to $\zu{cm}^3$ gives
\begin{equation*}
        26000\ \zu{gallons} \times \frac{3780\ \zu{cm}^3}{1\ \zu{gallon}}
            =  9.8\times10^7\ \zu{cm}^3\eqquad.
\end{equation*}
Water has a density of 1 gram per cubic centimeter, so the
mass of the water is $9.8\times10^4\ \kgunit$. The energy needed
to heat the swimming pool is
\begin{equation*}
  mc\Delta T= 3.3\times10^3\ \zu{MJ}\eqquad.
\end{equation*}
% calc -e "rho=1 g/cm3; v=9.8 10^7 cm3; m=rhov->kg; c=4.2 10^3 J/kg; T=8; mcT->MJ"
The cost of the electricity is ($3.3\times10^3$ MJ)(\$0.039/MJ)=\$130.
\end{eg}

\begin{eg}{Irish coffee}
\egquestion You make a cup of Irish coffee out of 300 g of
coffee at $100\degcunit\ $ and 30 g of pure ethyl alcohol at
20\degcunit. The specific heat capacity of ethanol is
$2.4\times10^3\ \junit/\kgunit\unitdot\degcunit$ (i.e., alcohol is easier
to heat than water). What temperature is the final mixture?

\eganswer Adding up all the energy after mixing has to give
the same result as the total before mixing. We let the
subscript $i$ stand for the initial situation, before
mixing, and $f$ for the final situation, and use subscripts
$c$ for the coffee and $a$ for the alcohol. In this notation, we have
\begin{align*}
 \text{total initial energy} &= \text{total final energy} \\
 E_{ci}+E_{ai} &= E_{cf}+E_{af}\eqquad.
\end{align*}
We assume coffee has the same heat-carrying properties as
water. Our information about the heat-carrying properties of
the two substances is stated in terms of the \emph{change} in
energy required for a certain \emph{change} in temperature, so we
rearrange the equation to express everything in terms
of energy differences:
\begin{equation*}
                E_{af}-E_{ai} =    E_{ci}-E_{cf}\eqquad.
\end{equation*}
Using the heat capacities $c_c$ for coffee (water) and $c_a$ for alcohol, we have
\begin{align*}
 E_{ci}-E_{cf} &= (T_{ci}-T_{cf})m_c c_c \qquad \text{and} \\
 E_{af}-E_{ai} &= (T_{af}-T_{ai})m_a c_a .
\end{align*}
Setting these two quantities to be equal, we have
\begin{equation*}
        (T_{af}-T_{ai})m_a c_a         =      (T_{ci}-T_{cf})m_c c_c\eqquad.
\end{equation*}
In the final mixture the two substances must be at the same
temperature, so we can use a single symbol $T_f=T_{cf}=T_{af}$
for the two quantities previously represented by two different symbols,
\begin{equation*}
                (T_f-T_{ai})m_a c_a         =      (T_{ci}-T_f)m_c c_c\eqquad.
\end{equation*}
Solving for $T_f$ gives
\begin{align*}
 T_f &= \frac{T_{ci}m_c c_c+T_{ai}m_a c_a}{m_c c_c+m_a c_a}\\
 &= 96\degcunit\eqquad.
\end{align*}
\end{eg}

Once a numerical scale of energy has been established for
some form of energy such as heat, it can easily be extended
to other types of energy. For instance, the energy stored in
one gallon of gasoline can be determined by putting some
gasoline and some water in an insulated chamber, igniting
the gas, and measuring the rise in the water's temperature.
(The fact that the apparatus is known as a ``bomb calorimeter''
will give you some idea of how dangerous these experiments
are if you don't take the right safety precautions.) Here
are some examples of other types of energy that can be
measured using the same units of joules:

\begin{tabular}{|p{38mm}p{62mm}|}\hline
\textbf{type of energy} & \textbf{example} \\ \hline
chemical energy\linebreak[4] released by burning & About 50 MJ are released by burning a kg of gasoline. \\ \hline
energy required to break an object & When a person suffers a spiral fracture of the thighbone (a common
type in skiing accidents), about 2 J of energy go into breaking the bone. \\ \hline
energy required to melt a solid substance & 7 MJ are required to melt 1 kg of tin. \\ \hline
chemical energy\linebreak[4] released by digesting food & A bowl of Cheeries with milk provides us with about 800 kJ of usable energy. \\ \hline
raising a mass against the force of gravity & Lifting 1.0 kg through a height of 1.0 m requires 9.8 J. \\ \hline
nuclear energy\linebreak[4] released in fission & 1 kg of uranium oxide fuel consumed by a reactor releases $2\times10^{12}$ J of stored nuclear energy. \\ \hline
\end{tabular}

It is interesting to note the disproportion between the
megajoule energies we consume as food and the joule-sized
energies we expend in physical activities. If we could
perceive the flow of energy around us the way we perceive
the flow of water, eating a bowl of cereal would be like
swallowing a bathtub's worth of energy, the continual loss
of body heat to one's environment would be like an
energy-hose left on all day, and lifting a bag of cement
would be like flicking it with a few tiny energy-drops. The
human body is tremendously inefficient. The calories we
``burn'' in heavy exercise are almost all dissipated
directly as body heat.

\begin{eg}{You take the high road and I'll take the low road.}\label{eg:high-road-low-road}
\egquestion Figure \figref{high-road-low-road} shows two ramps which two balls will
roll down. Compare their final speeds, when they reach point
B. Assume friction is negligible.

\eganswer Each ball loses some energy because of its
decreasing height above the earth, and conservation of
energy says that it must gain an equal amount of energy of
motion (minus a little heat created by friction). The balls
lose the same amount of height, so their final speeds must be equal.
\end{eg}

<% marg(53) %>

<%
  fig(
    'high-road-low-road',
    %q{Example \ref{eg:high-road-low-road}.}
  )
%>

<% end_marg %>
It's impressive to note the complete impossibility of
solving this problem using only Newton's laws. Even if the
shape of the track had been given mathematically, it would
have been a formidable task to compute the balls' final
speed based on vector addition of the normal force and
gravitational force at each point along the way.

<% begin_sec("How new forms of energy are discovered",0,'new-forms-of-energy') %>

Textbooks often give the impression that a sophisticated
physics concept was created by one person who had an
inspiration one day, but in reality it is more in the nature
of science to rough out an idea and then gradually refine it
over many years. The idea of energy was tinkered with from
the early 1800's on, and new types of energy kept
getting added to the list.

 To establish the existence of a new form of energy, a physicist has to

(1) show that it could be converted to and from other
forms of energy; and

(2) show that it related to some definite measurable
property of the object, for example its temperature, motion,
position relative to another object, or being in a
solid or liquid state.

For example, energy is released when a piece of iron is
soaked in water, so apparently there is some form of energy
already stored in the iron. The release of this energy can
also be related to a definite measurable property of the
chunk of metal: it turns reddish-orange. There has been a
chemical change in its physical state, which we call rusting.

Although the list of types of energy kept getting longer and
longer, it was clear that many of the types were just
variations on a theme. There is an obvious similarity
between the energy needed to melt ice and to melt butter,
or between the rusting of iron and many other chemical
reactions. The topic of the next chapter is how this process
of simplification reduced all the types of energy to a very
small number (four, according to the way I've chosen to count them).

It might seem that if the principle of conservation of
energy ever appeared to be violated, we could fix it up
simply by inventing some new type of energy to compensate
for the discrepancy. This would be like balancing your
checkbook by adding in an imaginary deposit or withdrawal to
make your figures agree with the bank's statements. Step (2)
above guards against this kind of chicanery. In the 1920s
there were experiments that suggested energy was not
conserved in radioactive processes. Precise measurements of
the energy released in the radioactive decay of a given type
of atom showed inconsistent results. One atom might decay
and release, say, $1.1\times10^{-10}$ J of energy, which had
presumably been stored in some mysterious form in the
nucleus. But in a later measurement, an atom of exactly the
same type might release $1.2\times10^{-10}$ J. Atoms of the
same type are supposed to be identical, so both atoms were
thought to have started out with the same energy. If the
amount released was random, then apparently the total amount
of energy was not the same after the decay as before, i.e.,
energy was not conserved.

Only later was it found that a previously unknown particle,
which is very hard to detect, was being spewed out in the
decay. The particle, now called a neutrino, was carrying off
some energy, and if this previously unsuspected form of
energy was added in, energy was found to be conserved after
all. The discovery of the energy discrepancies is seen with
hindsight as being step (1) in the establishment of a new
form of energy, and the discovery of the neutrino was step
(2). But during the decade or so between step (1) and step
(2) (the accumulation of evidence was gradual), physicists
had the admirable honesty to admit that the cherished
principle of conservation of energy might have to be discarded.

<% self_check('spring-energy',<<-'SELF_CHECK'
How would you carry out the two steps given above in order
to establish that some form of energy was stored in a
stretched or compressed spring?
  SELF_CHECK
  ) %>

\begin{optionaltopic}{Mass Into Energy}
Einstein showed that mass itself
could be converted to and from  energy,
according to his celebrated
equation $E=mc^2$, in which $c$ is the
speed of light. We thus speak of
mass as simply another form of
energy, and it is valid to measure
it in units of joules. The mass of a
15-gram pencil corresponds to
about $1.3\times10^{15}$ J. The issue is
largely academic in the case of the
pencil, because very violent processes
 such as nuclear reactions
are required in order to convert any
significant fraction of an object's
mass into energy. Cosmic rays,
however, are continually striking
you and your surroundings and
converting part of their energy of
motion into the mass of newly created
particles. A single high-energy
cosmic ray can create a ``shower''
of millions of previously nonexistent
particles when it strikes the 
atmosphere. Einstein's theories are
discussed later in this book.

Even today, when the energy 
concept is relatively mature and stable,
a new form of energy has been
proposed based on observations
of distant galaxies whose light 
began its voyage to us billions of
years ago. Astronomers have
found that the universe's continuing expansion, resulting from the
Big Bang, has not been decelerating
 as rapidly in the last few billion
years as would have been
expected from gravitational forces.
They suggest that a new form of
energy may be at work.
\end{optionaltopic}


\startdq

\begin{dq}
I'm not making this up. XS Energy Drink has ads that read
like this: \emph{All the ``Energy'' ... Without the Sugar! Only 8
Calories!} Comment on this.
\end{dq}

<% end_sec() %>
<% end_sec() %>
<% begin_sec("Kinetic Energy",3) %>

The technical term for the energy associated with motion is
\index{kinetic energy}kinetic energy, from the Greek word
for motion. (The root is the same as the root of the word ``cinema''
for a motion picture,
and in French the term for kinetic energy is
``\'{e}nergie cin\'{e}tique.'') To find how much kinetic energy is
possessed by a given moving object, we must convert all its
kinetic energy into heat energy, which we have chosen as the
standard reference type of energy. We could do this, for
example, by firing projectiles into a tank of water and
measuring the increase in temperature of the water as a
function of the projectile's mass and velocity. Consider the
following data from a series of three such experiments:

\begin{tabular}{|c|c|c|} \hline
\textbf{m} (kg) & \textbf{v} (m/s) & \textbf{energy} (J) \\ \hline
1.00 & 1.00 & 0.50 \\ \hline
1.00 & 2.00 & 2.00 \\ \hline
2.00 & 1.00 & 1.00 \\ \hline
\end{tabular}

Comparing the first experiment with the second, we see that
doubling the object's velocity doesn't just double its
energy, it quadruples it. If we compare the first and third
lines, however, we find that doubling the mass only doubles
the energy. This suggests that kinetic energy is proportional
to mass and to the square of velocity, $KE\propto mv^2$, and further
experiments of this type would indeed establish such a
general rule. The proportionality factor equals 0.5 because
of the design of the metric system, so the kinetic energy of
a moving object is given by

\begin{equation*}
                KE    =    \frac{1}{2}mv^2\eqquad.
\end{equation*}
The metric system is based on the meter, kilogram, and
second, with other units being derived from those. Comparing
the units on the left and right sides of the equation shows
that the joule can be reexpressed in terms of the basic
units as $\kgunit\unitdot\munit^2/\sunit^2$.

\begin{eg}{Energy released by a comet impact}
\egquestion Comet Shoemaker-Levy, which struck the planet
Jupiter in 1994, had a mass of roughly $4\times10^{13}$ kg,
and was moving at a speed of 60 km/s. Compare the kinetic
energy released in the impact to the total energy in the
world's nuclear arsenals, which is $2\times10^{19}$ J. Assume
for the sake of simplicity that Jupiter was at rest.

\eganswer Since we assume Jupiter was at rest, we can
imagine that the comet stopped completely on impact, and
100\% of its kinetic energy was converted to heat and sound.
We first convert the speed to mks units, $v=6\times10^4$
m/s, and then plug in to the equation to find that the
comet's kinetic energy was roughly $7\times10^{22}$ J, or about
3000 times the energy in the world's nuclear arsenals.
\end{eg}

<% begin_sec("Energy and relative motion") %>\label{galilean-energy}
Galileo's Aristotelian enemies (and it is no exaggeration to call them
enemies!) would probably have objected to conservation of
energy. Galilean got in trouble by claiming that an object in
motion would continue in motion indefinitely in the absence
of a force.  This is not so different from the idea that an
object's kinetic energy stays the same unless there is a
mechanism like frictional heating for converting that energy
into some other form.

More subtly, however, it's not immediately obvious that what
we've learned so far about energy is strictly mathematically
consistent with Galileo's principle that motion is relative.
Suppose we verify that a certain process, say the collision
of two pool balls, conserves energy as measured in a certain
frame of reference: the sum of the balls' kinetic energies
before the collision is equal to their sum after the
collision. But what if we were to
measure everything in a frame of reference that was in a
different state of motion? It's not immediately obvious that the
total energy before the collision will still equal the total
energy after the collision. 
It \emph{does} still work out. Homework problem \ref{hw:balls}, p.~\pageref{hw:balls},
gives a simple numerical example, and the general
proof is taken up in problem \ref{hw:energy-frames} on p.~\pageref{hw:energy-frames} (with the solution
given in the back of the book).
<% end_sec() %>

<% begin_sec("Why kinetic energy obeys the equation it does") %>\label{galilean-energy}
I've presented the magic expression for kinetic energy, $(1/2)mv^2$, as a purely empirical fact.
Does it have any deeper reason that might be knowable to us mere mortals? Yes and no. It contains
three factors, and we need to consider each separately.

The reason for the factor of 1/2 is understandable, but only as an arbitrary historical choice.
The metric system was designed so that some of the equations relating
to energy would come out looking simple, at the expense of
some others, which had to have inconvenient conversion
factors in front. If we were using the old British
Engineering System of units in this course, then we'd have
the British Thermal Unit (BTU) as our unit of energy. In
that system, the equation you'd learn for kinetic energy
would have an inconvenient proportionality constant,
$KE=\left(1.29\times10^{-3}\right)mv^2$, with $KE$ measured
in units of BTUs, $v$ measured in feet per second, and so on.
At the expense of this inconvenient equation for kinetic
energy, the designers of the British Engineering System got
a simple rule for calculating the energy required to heat
water: one BTU per degree Fahrenheit per pound. The
inventor of kinetic energy, Thomas Young, actually defined
it as $KE=mv^2$, which meant that all his other
equations had to be different from ours by a factor of two.
All these systems of units work just fine as long as they
are not combined with one another in an inconsistent way.

The proportionality to $m$ is inevitable because the energy concept is based on the idea
that we add up energy contributions from all the objects
within a system. Therefore it is logically
necessary that a 2 kg object moving at 1 m/s have the same
kinetic energy as two 1 kg objects moving side-by-side at the same speed.

<% marg(0) %>

<%
  fig(
    'ke-of-electrons',
    %q{%
      Kinetic energies of electrons measured in three experiments. At high velocities,
      the equation $KE=(1/2)mv^2$ becomes a poor approximation.
    }
  )
%>
<% end_marg %>

What about the proportionality to $v^2$? Consider:\label{ke-logic}
\begin{enumerate}
\item It's surprisingly hard to tamper with this factor without breaking things: see discussion
questions \ref{dq:why-ke-not-first-order} and \ref{dq:why-no-fourth-order-term-in-ke} on
p.~\pageref{dq:why-ke-not-first-order}.
\item The proportionality to $v^2$ is not even correct, except as a low-velocity approximation.
Experiments show deviations from the $v^2$
rule at high speeds (figure \figref{ke-of-electrons}), an effect that is related to Einstein's
theory of relativity.
\item As described on p.~\pageref{galilean-energy},
we want conservation of energy to keep working when we switch frames of reference.
The fact that this does work for $KE\propto v^2$ is intimately connected with the
assumption that when we change frames,
velocities add as described in section \ref{sec:addition-of-velocities}.
This assumption turns out to be an approximation, which only works well at low velocities.
\item Conservation laws are of more general validity than Newton's laws, which apply
to material objects moving at low speeds. Under the conditions where Newton's laws are accurate,
they follow logically from the conservation laws.
Therefore we need kinetic energy to have low-velocity behavior that ends up correctly reproducing Newton's laws.
\end{enumerate}
So \emph{under a certain set of low-velocity approximations}, $KE\propto v^2$
is what works. We verify in problem \ref{hw:energy-frames}, p.~\pageref{hw:energy-frames}, that it satisfies criterion 3,
and we show in section \ref{sec:when-is-work-fd}, p.~\pageref{work-ke-logic}, that it is the \emph{only} such relation that satisfies
criterion 4.

<% end_sec() %>

\pagebreak

\startdqs

\begin{dq}\label{dq:why-ke-not-first-order}
Suppose that, like Young or Einstein, you were trying out
different equations for kinetic energy to see if they agreed
with the experimental data. Based on the meaning of positive
and negative signs of velocity, why would you suspect that
a proportionality to $mv$ would be less likely than $mv^2?$
\end{dq}

\begin{dq}\label{dq:why-no-fourth-order-term-in-ke}
As in discussion question \ref{dq:why-ke-not-first-order}, try to
think of an argument showing that $m(v^2+v^4)$ is not a possible
formula for kinetic energy.
\end{dq}

<% marg(30) %>
<%
  fig(
    'dq-pendulum-and-peg',
    %q{Discussion question \ref{dq:pendulum-and-peg}},
    {
      'anonymous'=>true
    }
  )
%>
<% end_marg %>
\begin{dq}\label{dq:pendulum-and-peg}
The figure
 shows a pendulum that is released at A and
caught by a peg as it passes through the vertical, B. To
what height will the bob rise on the right?
\end{dq}

<% end_sec() %>
<% begin_sec("Power",3) %>

A car may have plenty of energy in its gas tank, but still
may not be able to increase its kinetic energy rapidly. A
Porsche doesn't necessarily have more energy in its gas tank
than a Hyundai, it is just able to transfer it more quickly.
The rate of transferring energy from one form to another is
called $\index{power}power$. The definition can be
written as an equation,
\begin{equation*}
                P    =     \frac{\Delta E}{\Delta t}\eqquad,
\end{equation*}
where the use of the delta notation in the symbol $\Delta E$
has the usual interpretation: the final amount of energy in a
certain form minus the initial amount that was present in
that form. Power has units of J/s, which are abbreviated as
\index{watt (unit)}watts, W (rhymes with ``lots'').

If the rate of energy transfer is not constant, the power at
any instant can be defined as the 
m4_ifelse(__me,1,[:derivative $\der E/\der t$:],[:slope of the tangent line
on a graph of $E$ versus $t$. Likewise $\Delta E$ can be
extracted from the area under the $P$-versus-$t$ curve.:])

\begin{eg}{Converting kilowatt-hours to joules}
\egquestion The electric company bills you for energy in units
of kilowatt-hours (kilowatts multiplied by hours) rather
than in SI units of joules. How many joules is a kilowatt-hour?

\eganswer
1 kilowatt-hour = (1 kW)(1 hour) = (1000 J/s)(3600 s) = 3.6 MJ.
\end{eg}

\begin{eg}{Human wattage}
\egquestion A typical person consumes 2000 kcal of food in a
day, and converts nearly all of that directly to heat.
Compare the person's heat output to the rate of energy
consumption of a 100-watt lightbulb.

\eganswer Looking up the conversion factor from calories to joules, we find
\begin{equation*}
 \Delta E=2000\ \zu{kcal}\times\frac{1000\ \zu{cal}}{1\ \zu{kcal}}\times\frac{4.18\ \junit}{1\ \zu{cal}}=8\times10^6\ \junit
\end{equation*}
for our daily energy consumption. Converting the time
interval likewise into mks,
\begin{equation*}
 \Delta t=1\ \zu{day}\times\frac{24\ \zu{hours}}{1\ \zu{day}}\times\frac{60\ \zu{min}}{1\ \zu{hour}}\times\frac{60\ \sunit}{1\ \zu{min}}=9\times10^4\ \sunit\eqquad.
\end{equation*}
Dividing, we find that our power dissipated as heat is 90
J/s = 90 W, about the same as a lightbulb.
\end{eg}

m4_ifelse(__me,1,[:\begin{eg}{Wind power density}\label{eg:wind-power-density}
Wind power is a renewable energy resource, but it is most practical in areas where the
wind is both strong and reliably strong. When a horizontal-axis wind turbine faces
directly into a wind flowing at speed $v$, the air it intercepts in time $\Delta t$
forms a cylinder whose length is $v\Delta t$, and whose mass is proportional to the same
factor. The kinetic energy of this cylinder represents the maximum energy that can theoretically
be extracted in this time. Since the mass is proportional to $v$, the kinetic energy is proportional
to $v\times v^2=v^3$. That is, the ``wind power density'' varies as the cube of the wind's speed.
\end{eg}:])

It is easy to confuse the concepts of force, energy, and
power, especially since they are synonyms in ordinary
speech. The table on the following page may help to clear this up:

\vfill

\begin{minipagefullpagewidth}
\begin{tabular*}{\pagewidth}{|p{25mm}|p{43mm}|p{43mm}|p{43mm}|} \hline
& \textbf{force} & \textbf{energy} & \textbf{power} \\ \hline

\textbf{conceptual definition} &
A force is an interaction between two objects that causes a push or a pull.
 A force can be defined as anything that is capable of changing an object's state of motion. &
Heating an object, making it move faster, or increasing its distance from another object that is attracting it are all examples of things that would require fuel or physical effort. All these things can be
quantified using a single scale of measurement, and we describe them all as forms of energy. &
Power is the rate at which energy is transformed from one form to another or transferred from one object to another. \\ \hline

\textbf{operational definition} &\index{operational definition!energy}\index{operational definition!power}
A spring scale can be used to measure force. &
If we define a unit of energy as the amount required to heat a certain amount of water
 by a $1\degcunit$, then we can measure any other quantity of energy by
 transferring it into heat in water and measuring the temperature increase. &
Measure the change in the amount of some form of energy possessed by an object, 
and divide by the amount of time required for the change to occur. \\ \hline

\textbf{scalar or\linebreak[4] vector?} &
vector --- has a direction in space which is the direction in which it pulls or pushes &
scalar --- has no direction in space &
scalar --- has no direction in space \\ \hline

\textbf{unit} &
newtons (N) & joules (J) & watts (W) = joules/s \\ \hline

\textbf{Can it run out? Does it cost money?} &
No. I don't have to pay a monthly bill for the meganewtons of force required to hold up my house. &
Yes. We pay money for gasoline, electrical energy, batteries, etc., because they contain energy. &
More power means you are paying money at a higher rate. A 100-W lightbulb costs a certain number of cents per hour. \\ \hline

\textbf{Can it be a property of an object?} &
No. A force is a relationship between two interacting objects. A home-run baseball doesn't
``have'' force. &
Yes. What a home-run baseball has is kinetic energy, not force. &
Not really. A 100-W lightbulb doesn't ``have'' 100 W. 100 J/s is the rate at which it converts electrical energy into light. \\ \hline
\end{tabular*}
\end{minipagefullpagewidth}

<% end_sec() %>
