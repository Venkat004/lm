<% begin_sec("Vector Notation",0) %>

The idea of components freed us from the confines of
one-dimensional physics, but the component notation can be
unwieldy, since every one-dimensional equation has to be
written as a set of three separate equations in the
three-dimensional case. Newton was stuck with the component
notation until the day he died, but eventually someone
sufficiently lazy and clever figured out a way of abbreviating
three equations as one.

\noindent \begin{tabular}{|llll|}
\hline
(a)  &  $\overrightarrow{F}_{\text{A on B}}=-\overrightarrow{F}_{\text{B on A}}$   &   stands for  &  
  $\begin{matrix}
    F_{\text{A on B},x} = -F_{\text{B on A},x}  \\
    F_{\text{A on B},y} = -F_{\text{B on A},y}  \\
    F_{\text{A on B},z} = -F_{\text{B on A},z}  \\
  \end{matrix}$ \\
\hline
(b)  &  $\overrightarrow{F}_{\text{total}}=\overrightarrow{F}_1+\overrightarrow{F}_2+\ldots$   &   stands for  &  
  $\begin{matrix}
    F_{\text{total},x} = F_{1,x}+F_{2,x}+\ldots\\
    F_{\text{total},y} = F_{1,y}+F_{2,y}+\ldots\\
    F_{\text{total},z} = F_{1,z}+F_{2,z}+\ldots\\
  \end{matrix}$ \\
\hline
(c)  &  $\overrightarrow{a}=\frac{\Delta \overrightarrow{v}}{\Delta t}$   &   stands for  &  
  $\begin{matrix}
    a_x = \Delta v_x / \Delta t \\
    a_y = \Delta v_y / \Delta t \\
    a_z = \Delta v_z / \Delta t \\
  \end{matrix}$ \\
\hline
\end{tabular}

\noindent Example (a) shows both ways of writing Newton's third law.
Which would you rather write?

\pagebreak

The idea is that each of the algebra symbols with an arrow
written on top, called a \index{vector!defined}vector, is
actually an abbreviation for three different numbers, the
$x$, $y$, and $z$ components. The three components are referred
to as the components of the vector, e.g., $F_x$ is the $x$
component of the vector $\overrightarrow{F}$. The notation with an arrow on top
is good for handwritten equations, but is unattractive in a
printed book, so books use boldface, $\vc{F}$, to represent
vectors. After this point, I'll use boldface for vectors
throughout this book.

Quantities can be classified as vectors or scalars. In a phrase like
``a \_\_\_\_\_ to the northeast,'' it makes sense to fill in the blank
with ``force'' or ``velocity,'' which are vectors, but not with ``mass''
or ``time,'' which are scalars.
Any nonzero vector has both a direction and an amount.
The amount is called its \index{magnitude of a vector!defined}\index{vector!magnitude of}
magnitude. The notation for the magnitude of a vector $\vc{A}$
is $|\vc{A}|$, like the absolute value sign used with scalars.

Often, as in example (b), we wish to use the vector notation
to represent adding up all the $x$ components to get a total
$x$ component, etc. The plus sign is used between two
vectors to indicate this type of component-by-component
addition. Of course, vectors are really triplets of numbers,
not numbers, so this is not the same as the use of the plus
sign with individual numbers. But since we don't want to
have to invent new words and symbols for this operation on
vectors, we use the same old plus sign, and the same old
addition-related words like ``add,'' ``sum,'' and ``total.''
Combining vectors this way is called \index{vector!addition}vector addition.

Similarly, the minus sign in example (a) was used to
indicate negating each of the vector's three components
individually. The equals sign is used to mean that all three
components of the vector on the left side of an equation are
the same as the corresponding components on the right.

Example (c) shows how we abuse the division symbol in a
similar manner. When we write the vector $\Delta \vc{v}$ divided
by the scalar $\Delta $t, we mean the new vector formed by
dividing each one of the velocity components by $\Delta t$.

It's not hard to imagine a variety of operations that would
combine vectors with vectors or vectors with scalars, but
only four of them are required in order to express Newton's laws:

\pagebreak

\begin{tabular}{lp{60mm}}
operation & definition \\
$\text{\textbf{vector}}+\text{\textbf{vector}}$ & Add component by component to make a new set of three numbers.\\
$\text{\textbf{vector}}-\text{\textbf{vector}}$ & Subtract component by component to make a new set of three numbers.\\
$\text{\textbf{vector}}\cdot\text{scalar}$ & Multiply each component of the vector by the scalar.\\
$\text{\textbf{vector}}/\text{scalar}$ & Divide each component of the vector by the scalar.
\end{tabular}

As an example of an operation that is not useful for
physics, there just aren't any useful physics applications
for dividing a vector by another vector component by
component. In optional section \ref{sec:rotational-invariance}, we discuss in more
detail the fundamental reasons why some vector operations
are useful and others useless.

We can do algebra with vectors, or with a mixture of vectors
and scalars in the same equation. Basically all the normal
rules of algebra apply, but if you're not sure if a certain
step is valid, you should simply translate it into three
component-based equations and see if it works.

\begin{eg}{Order of addition}
\egquestion If we are adding two force vectors, $\vc{F}+\vc{G}$, is it
valid to assume as in ordinary algebra that $\vc{F}+\vc{G}$ is the same as $\vc{G}+\vc{F}$?

\eganswer To tell if this algebra rule also applies to
vectors, we simply translate the vector notation into
ordinary algebra notation. In terms of ordinary numbers, the
components of the vector $\vc{F}+\vc{G}$ would be $F_x+G_x$, $F_y+G_y$,
and $F_z+G_z$, which are certainly the same three numbers as
$G_x+F_x$, $G_y+F_y$, and $G_z+F_z$. Yes, $\vc{F}+\vc{G}$ is the same as $\vc{G}+\vc{F}$.
\end{eg}

It is useful to define a symbol $\vc{r}$ for the vector whose
components are $x$, $y$, and $z$, and a symbol $\Delta\vc{r}$ made
out of $\Delta x$, $\Delta y$, and $\Delta z$.

Although this may all seem a little formidable, keep in mind
that it amounts to nothing more than a way of abbreviating
equations! Also, to keep things from getting too confusing
the remainder of this chapter focuses mainly on the 
$\Delta \vc{r}$ vector, which is relatively easy to visualize.

<% self_check('translate-to-vector',<<-'SELF_CHECK'
Translate the equations $v_x=\\Delta x/\\Delta t$, 
$v_y=\\Delta y/\\Delta t$, and $v_z=\\Delta z/\\Delta t$ for motion with
constant velocity into a single equation in vector notation.
  SELF_CHECK
  ) %>

<% begin_sec("Drawing vectors as arrows",4) %>

A vector in two dimensions can be easily visualized by
drawing an arrow whose length represents its magnitude and
whose direction represents its direction. The $x$ component
of a vector can then be visualized as the length of the
shadow it would cast in a beam of light projected onto the
$x$ axis, and similarly for the $y$ component. Shadows with
arrowheads pointing back against the direction of the
positive axis correspond to negative components.
<% marg(130) %>
<%
  fig(
    'flashlight-shadows',
    %q{%
      The $x$ and $y$ components of
      a vector can be thought of as the shadows it casts onto the $x$ and
      $y$ axes.
    }
  )
%>
\spacebetweenfigs
<%
  fig(
    'sc-scale-vector',
    %q{%
      Self-check \ref{sc:scale-vector}.
    }
  )
%>
<% end_marg %>

In this type of diagram, the negative of a vector is the
vector with the same magnitude but in the opposite
direction. Multiplying a vector by a scalar is represented
by lengthening the arrow by that factor, and similarly for division.

<% self_check('scale-vector',<<-'SELF_CHECK'
Given vector $\\vc{Q}$ represented by an arrow in figure \\figref{sc-scale-vector}, draw arrows
representing the vectors $1.5\\vc{Q}$ and $-\\vc{Q}$.
  SELF_CHECK
  ) %>

This leads to a way of defining vectors and scalars that reflects how physicists think
in general about these things:

\begin{important}[definition of vectors and scalars]
A general type of measurement (force, velocity, \ldots)
is a vector if it can be drawn as an arrow so that
rotating the paper produces the same result as rotating the actual quantity.
A type of quantity that never changes at all under rotation is a scalar.
\end{important}

For example, a force reverses itself under a 180-degree rotation, but a mass doesn't.
We could have defined a vector as something that had both a magnitude and a direction,
but that would have left out zero vectors, which don't have a direction.\index{scalar!defined}\index{vector!defined}
A zero vector is a legitimate vector, because it behaves the same way under rotations
as a zero-length arrow, which is simply a dot.
<% marg(-50) %>
<%
  fig(
    'playing-card',
    %q{%
      A playing card returns to its original state when rotated by 180 degrees.
    }
  )
%>
<% end_marg %>

A remark for those who enjoy brain-teasers: not everything is a vector or a scalar.
An American football is distorted compared to
a sphere, and we can measure the orientation and amount of that distortion quantitatively.
The distortion is not a vector, since
a 180-degree rotation brings it back to its original state. Something similar happens with
playing cards, figure \figref{playing-card}. For some subatomic particles, such as electrons, 360 degrees
isn't even \emph{enough}; a 720-degree rotation is needed to put them back the way they were!

\startdqs

\begin{dq}
You drive to your friend's house. How does the magnitude
of your $\Delta\vc{r}$ vector compare with the distance you've
added to the car's odometer?
\end{dq}

<% end_sec() %>
<% end_sec() %>
<% begin_sec("Calculations with Magnitude and Direction",0) %>

If you ask someone where Las Vegas is compared to Los
Angeles, they are unlikely to say that the $\Delta x$ is 290
km and the $\Delta y$ is 230 km, in a coordinate system
where the positive $x$ axis is east and the $y$ axis points
north. They will probably say instead that it's 370 km to
the northeast. If they were being precise, they might
give the direction as $38\degunit$ counterclockwise from east.
In two dimensions, we can always specify a vector's
direction like this, using a single angle. A magnitude plus
an angle suffice to specify everything about the vector. The
following two examples show how we use trigonometry and the
Pythagorean theorem to go back and forth between the $x-y$
and magnitude-angle descriptions of vectors.

<% marg(-8) %>
<%
  fig(
    'eg-la-vegas',
    %q{Examples \ref{eg:la-vegas} and \ref{eg:la-vegas-components}.}
  )
%>
<% end_marg %>
\begin{eg}{Finding magnitude and angle from components}\label{eg:la-vegas}
\egquestion Given that the $\Delta \vc{r}$ vector from LA to Las
Vegas has $\Delta x=290\ \zu{km}$ and $\Delta y=230\ \zu{km}$, how would
we find the magnitude and direction of $\Delta $r?

\eganswer We find the magnitude of $\Delta \vc{r}$ from the
Pythagorean theorem:
\begin{align*}
        |\Delta \vc{r}|  &= \sqrt{\Delta x^2 + \Delta y^2}   \\
             &=  370\ \zu{km}
\end{align*}
We know all three sides of the triangle, so the angle
$\theta $ can be found using any of the inverse trig
functions. For example, we know the opposite and adjacent sides, so
\begin{align*}
        \theta       &=  \tan^{-1}\frac{\Delta y}{\Delta x}  \\
             &=  38\degunit\eqquad.
\end{align*}
\end{eg}

\begin{eg}{Finding components from magnitude and angle}\label{eg:la-vegas-components}
\egquestion Given that the straight-line distance from Los
Angeles to Las Vegas is 370 km, and that the angle $\theta $
in the figure is 38\degunit, how can the $x$ and $y$ components of
the $\Delta \vc{r}$ vector be found?

\eganswer The sine and cosine of $\theta $ relate the given
information to the information we wish to find:
\begin{align*}
        \cos  \theta      &=  \frac{\Delta x}{|\Delta\vc{r}|}  \\
        \sin  \theta      &=  \frac{\Delta y}{|\Delta\vc{r}|}  
\end{align*}
Solving for the unknowns gives
\begin{align*}
        \Delta x     &= |\Delta\vc{r}|\cos\theta   \\
             &=  290\ \zu{km}  \qquad \text{and}\\
        \Delta y     &= |\Delta\vc{r}|\sin\theta   \\
             &=  230\ \zu{km}\eqquad.
\end{align*}
\end{eg}

The following example shows the correct handling of the plus
and minus signs, which is usually the main cause of mistakes.
<% marg(-8) %>
<%
  fig(
    'eg-sd-la',
    %q{Example \ref{eg:sd-la}.}
  )
%>
<% end_marg %>
\begin{eg}{Negative components}\label{eg:sd-la}
\egquestion San Diego is 120 km east and 150 km south of Los
Angeles. An airplane pilot is setting course from San Diego
to Los Angeles. At what angle should she set her course,
measured counterclockwise from east, as shown in the figure?

\eganswer If we make the traditional choice of coordinate
axes, with $x$ pointing to the right and $y$ pointing up on
the map, then her $\Delta x$ is negative, because her final
$x$ value is less than her initial $x$ value. Her $\Delta y$
is positive, so we have
\begin{align*}
        \Delta x      &=  -120\ \zu{km}  \\
        \Delta y     &=  150\ \zu{km}\eqquad.
\end{align*}
If we work by analogy with example \ref{eg:la-vegas}, we get
\begin{align*}
        \theta &= \tan^{-1}\frac{\Delta y}{\Delta x} \\
             &= \tan^{-1}(-1.25) \\
             &=  -51\degunit\eqquad.
\end{align*}
According to the usual way of defining angles in trigonometry,
a negative result means an angle that lies clockwise from
the x axis, which would have her heading for the Baja
California. What went wrong? The answer is that when you ask
your calculator to take the arctangent of a number, there
are always two valid possibilities differing by 180\degunit.
That is, there are two possible angles whose tangents equal -1.25:
\begin{align*}
        \tan  129\degunit  &=  -1.25  \\
        \tan  -51\degunit  &=  -1.25
\end{align*}

You calculator doesn't know which is the correct one, so it
just picks one. In this case, the one it picked was the
wrong one, and it was up to you to add 180\degunit to it to
find the right answer.
\end{eg}

\vfill

\begin{eg}{A shortcut}\label{eg:component-shortcut}
\egquestion A split second after nine o'clock, the hour hand on a clock dial
has moved clockwise past the nine-o'clock position by some imperceptibly small
angle $\phi$. Let positive $x$ be to the right and positive $y$ up.
If the hand, with length $\ell$, is represented by a $\Delta\vc{r}$ vector
going from the dial's center to the tip of the hand,
find this vector's $\Delta x$.

\eganswer The following shortcut is the easiest way to work out examples like
these, in which a vector's direction is known relative to one of the axes.
We can tell that $\Delta\vc{r}$ will have a large, negative $x$ component
and a small, positive $y$.
Since $\Delta x<0$,  there are really only
two logical possibilities: either $\Delta x = -\ell \cos\phi$, or
$\Delta x = -\ell \sin\phi$. Because $\phi$ is small, $\cos\phi$ is large
and $\sin\phi$ is small. We conclude that $\Delta x = -\ell \cos\phi$.

A typical application of this technique to force vectors is given in
example \ref{eg:layback} on p.~\pageref{eg:layback}.
\end{eg}

\vfill

\startdq

\begin{dq}
In example \ref{eg:sd-la}, we dealt with \emph{components} that
were negative. Does it make sense to classify \emph{vectors} as positive and
negative?
\end{dq}

\vfill

<% end_sec() %>
<% begin_sec("Techniques for Adding Vectors",4,'vector-addition') %>

Vector addition is one of the three essential mathematical skills, summarized on pp.\pageref{begin-skills}-\pageref{end-skills},
that you need for success in this course.

\vfill

<% begin_sec("Addition of vectors given their components") %>\label{subsec:vector-addition-analytic}

The easiest type of vector addition is when you are in
possession of the components, and want to find the
components of their sum.

\vfill

\begin{eg}{Adding components}\label{eg:sd-vegas}
\egquestion Given the $\Delta x$ and $\Delta y$ values from
the previous examples, find the $\Delta x$ and $\Delta y$
from San Diego to Las Vegas.

<% marg(0) %>
<%
  fig(
    'eg-sd-vegas',
    %q{Example \ref{eg:sd-vegas}.}
  )
%>
\spacebetweenfigs
<%
  fig(
    'tip-to-tail',
    %q{%
      Vectors can be added graphically by
      placing them tip to tail, and then
      drawing a vector from the tail of the
      first vector to the tip of the second
      vector.
    }
  )
%>
<% end_marg %>
\eganswer
\begin{align*}
        \Delta x_{total}     &=  \Delta x_1 + \Delta x_2  \\
             &=  -120\ \zu{km} + 290\ \zu{km}  \\
             &=  170\ \zu{km}  \\
        \Delta y_{total}     &=  \Delta y_1 + \Delta y_2  \\
             &=  150\ \zu{km} + 230\ \zu{km}  \\
             &=  380
\end{align*}
Note how the signs of the $x$ components take care of the
westward and eastward motions, which partially cancel.

\end{eg}

<% end_sec() %>
\vfill
<% begin_sec("Addition of vectors given their magnitudes and directions") %>\label{subsec:vector-add-given-mag}

In this case, you must first translate the magnitudes and
directions into components, and the add the components.
In our San Diego-Los Angeles-Las Vegas example, we can
simply string together the preceding examples; this is done on p.~\pageref{skills-vector-addition}.

<% end_sec() %>
\vfill
<% begin_sec("Graphical addition of vectors") %>\label{subsec:vector-addition-graphical}

Often the easiest way to add vectors is by making a scale
drawing on a piece of paper. This is known as graphical
addition, as opposed to the analytic techniques discussed previously.
(It has nothing to do with $x-y$ graphs or graph paper. ``Graphical''
here simply means drawing. It comes from the Greek verb ``\emph{grapho},'' to write,
like related English words including ``graphic.'')\index{vector!addition!graphical}

\pagebreak

\begin{eg}{LA to Vegas, graphically}\label{eg:sd-vegas-graphical}
\egquestion Given the magnitudes and angles of the $\Delta \vc{r}$
vectors from San Diego to Los Angeles and from Los Angeles
to Las Vegas, find the magnitude and angle of the $\Delta \vc{r}$
vector from San Diego to Las Vegas.

\eganswer Using a protractor and a ruler, we make a careful
scale drawing, as shown in figure \figref{eg-sd-vegas-graphical}. 
The protractor can be conveniently aligned with the blue rules on
the notebook paper.
A scale of 
$1\ \zu{mm}\rightarrow 2\ \zu{km}$ was chosen for this solution because
it was as big as possible (for accuracy) without being so big that the
drawing wouldn't fit on the page. With a
ruler, we measure the distance from San Diego to Las Vegas
to be 206 mm, which corresponds to 412 km. With a protractor,
we measure the angle $\theta $ to be 65\degunit.
\end{eg}

<%
  fig(
    'eg-sd-vegas-graphical',
    %q{Example \ref{eg:sd-vegas-graphical}.},
    {
      'width'=>'wide',
      'sidecaption'=>true
    }
  )
%>

Even when we don't intend to do an actual graphical
calculation with a ruler and protractor, it can be
convenient to diagram the addition of vectors in this way.
With $\Delta \vc{r}$ vectors, it intuitively makes sense to lay
the vectors tip-to-tail and draw the sum vector from the
tail of the first vector to the tip of the second vector. We
can do the same when adding other vectors such as force vectors.

<% self_check('subtract-vectors',<<-'SELF_CHECK'
How would you subtract vectors graphically?
  SELF_CHECK
  ) %>

\startdqs

\begin{dq}
If you're doing \emph{graphical} addition of vectors,
does it matter which vector you start with and which vector
you start from the other vector's tip?
\end{dq}

\begin{dq}
If you add a vector with magnitude 1 to a vector of
magnitude 2, what magnitudes are possible for the vector sum?
\end{dq}

\begin{dq}
Which of these examples of vector addition are correct,
and which are incorrect?
\end{dq}
\anonymousinlinefig{../../../share/mechanics/figs/dq-check-tip-to-tail}

<% end_sec() %>
<% end_sec() %>
<% begin_sec("Unit Vector Notation",nil,'unit-vector-notation',{'optional'=>true}) %>\index{unit vectors}

When we want to specify a vector by its components, it can
be cumbersome to have to write the algebra symbol for each component:
\begin{equation*}
   \Delta x= 290\ \zu{km},\ \Delta y=230\ \zu{km}
\end{equation*}
A more compact notation is to write
\begin{equation*}
        \Delta \vc{r}  =  (290\ \zu{km})\hat{\vc{x}} + (230\ \zu{km})\hat{\vc{y}}\eqquad,
\end{equation*}
where the vectors $\hat{\vc{x}}$, $\hat{\vc{y}}$, and $\hat{\vc{z}}$, called the unit vectors, are
defined as the vectors that have magnitude equal to 1 and
directions lying along the $x$, $y$, and $z$ axes. In speech,
they are referred to as ``x-hat'' and so on.

A slightly different, and harder to remember, version of
this notation is unfortunately more prevalent. In this
version, the unit vectors are called $\hat{\vc{i}}$, $\hat{\vc{j}}$, and $\hat{\vc{k}}$:
\begin{equation*}
        \Delta \vc{r}  =  (290\ \zu{km})\hat{\vc{i}} + (230\ \zu{km})\hat{\vc{j}}\eqquad.
\end{equation*}

<% end_sec() %>
<% begin_sec("Rotational Invariance",4,'rotational-invariance',{'optional'=>true}) %>

Let's take a closer look at why certain vector operations
are useful and others are not. Consider the operation of
multiplying two vectors component by component to produce a third vector:
\begin{align*}
        R_x    &=    P_x Q_x  \\
        R_y    &=    P_y Q_y  \\
        R_z    &=    P_z Q_z
\end{align*}
As a simple example, we choose vectors $\vc{P}$ and $\vc{Q}$ to have
length 1, and make them perpendicular to each other, as
shown in figure \figref{invariance}/1. If we compute the result of our new
vector operation using the coordinate system in \figref{invariance}/2, we find:
\begin{align*}
        R_x    &=    0 \\
        R_y    &=    0  \\
        R_z    &=    0
\end{align*}
The $x$ component is zero because $P_x=0$, the $y$ component
is zero because $Q_y=0$, and the $z$ component is of course
zero because both vectors are in the $x-y$ plane. However,
if we carry out the same operations in coordinate system
\figref{invariance}/3, rotated 45 degrees with respect to the previous one, we find
\begin{align*}
        R_x    &=    1/2  \\
        R_y    &=    -1/2  \\
        R_z    &=    0
\end{align*}
The operation's result depends on what coordinate system we
use, and since the two versions of $\vc{R}$ have different
lengths (one being zero and the other nonzero), they don't
just represent the same answer expressed in two different
coordinate systems. Such an operation will never be useful
in physics, because experiments show physics works the same
regardless of which way we orient the laboratory building!
The \emph{useful} vector operations, such as addition and
scalar multiplication, are rotationally invariant, i.e., come
out the same regardless of the orientation of the coordinate system.
<% marg(150) %>
<%
  fig(
    'invariance',
    %q{%
      Component-by-component multiplication of
      the vectors in 1 would produce different vectors in coordinate
      systems 2 and 3.
    }
  )
%>
<% end_marg %>

\begin{eg}{Calibrating an electronic compass}
Some smart phones and GPS units contain electronic compasses that can sense the direction of the earth's magnetic
field vector, notated $\vc{B}$. Because all vectors work according to the same rules, you don't need to know
anything special about magnetism in order to understand this example. 
Unlike a traditional compass that uses a magnetized needle on a bearing, an electronic compass
has no moving parts. It contains two sensors oriented perpendicular to one another, and each sensor is only
sensitive to the component of the earth's field that lies along its own axis. Because a choice of coordinates
is arbitrary, we can take one of these sensors as defining the $x$ axis and the other the $y$. Given the two
components $B_x$ and $B_y$, the device's computer chip can compute the angle of magnetic north relative to its sensors,
$\tan^{-1}(B_y/B_x)$.

All compasses are vulnerable to errors because of nearby magnetic materials, and in particular it may happen
that some part of the compass's own housing becomes magnetized. In an electronic compass, rotational invariance
provides a convenient way of calibrating away such effects by having the user rotate the device in a horizontal circle. 

Suppose that when
the compass is oriented in a certain way, it measures $B_x=1.00$ and $B_y=0.00$ (in certain units).
We then expect that when it is rotated 90 degrees clockwise, the sensors will detect $B_x=0.00$ and $B_y=1.00$.

But imagine instead that we get $B_x=0.20$ and $B_y=0.80$. This would violate rotational invariance, since rotating
the coordinate system is supposed to give a different description of the \emph{same} vector. The magnitude
appears to have changed from 1.00 to $\sqrt{0.20^2+0.80^2}=0.82$, and a vector can't change its magnitude just
because you rotate it. The compass's computer chip figures out that some effect, possibly a slight magnetization of its housing, must
be adding an erroneous 0.2 units to all the $B_x$ readings, because subtracting this amount from all the $B_x$ values gives
vectors that have the same magnitude, satisfying rotational invariance.
\end{eg}

<% end_sec() %>\begin{summary}

\begin{vocab}

\vocabitem{vector}{a quantity that has both an amount (magnitude) and
a direction in space}

\vocabitem{magnitude}{the ``amount'' associated with a vector}

\vocabitem{scalar}{a quantity that has no direction in space, only an amount}

\end{vocab}

\begin{notation}

\notationitem{$\vc{A}$}{a vector with components $A_x$, $A_y$, and $A_z$}
\notationitem{$\overrightarrow{A}$}{handwritten notation for a vector}
\notationitem{$|\vc{A}|$}{the magnitude of vector $\vc{A}$}
\notationitem{$\vc{r}$}{the vector whose components are $x$, $y$, and $z$}
\notationitem{$\Delta\vc{r}$}{the vector whose components are $\Delta x$, $\Delta y$, and $\Delta z$}
\notationitem{$\hat{\vc{x}}$, $\hat{\vc{y}}$, $\hat{\vc{z}}$}{(optional topic) unit vectors; the vectors with magnitude 1 lying along the  $x$, $y$, and $z$ axes}
\notationitem{$\hat{\vc{i}}$, $\hat{\vc{j}}$, $\hat{\vc{k}}$}{a harder to remember notation for the unit vectors}
\end{notation}

\begin{othernotation}
\notationitem{displacement vector}{a name for the symbol $\Delta \vc{r}$}
\notationitem{speed}{the magnitude of the velocity vector, i.e., the
velocity stripped of any information about its direction}
\end{othernotation}
\begin{summarytext}

A vector is a quantity that has both a magnitude (amount)
and a direction in space, as opposed to a scalar, which has
no direction. The vector notation amounts simply to an
abbreviation for writing the vector's three components.

In two dimensions, a vector can be represented either by its
two components or by its magnitude and direction. The two
ways of describing a vector can be related by trigonometry.

The two main operations on vectors are addition of a vector
to a vector, and multiplication of a vector by a scalar.

Vector addition means adding the components of two vectors
to form the components of a new vector. In graphical terms,
this corresponds to drawing the vectors as two arrows laid
tip-to-tail and drawing the sum vector from the tail of the
first vector to the tip of the second one. Vector subtraction
is performed by negating the vector to be subtracted and then adding.

Multiplying a vector by a scalar means multiplying each of
its components by the scalar to create a new vector.
Division by a scalar is defined similarly.

m4_ifelse(__me,1,[:
	Differentiation and integration of vectors is defined component by component.
:])

\end{summarytext}

\end{summary}
