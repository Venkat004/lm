<% begin_sec("The dot product",nil,'dot-product',{'optional'=>(__me==0)}) %>
\index{scalar (dot) product}\index{dot product of two vectors}

Up until now, we have not found any physically useful way to
define the multiplication of two vectors. It would be
possible, for instance, to multiply two vectors component by
component to form a third vector, but there are no physical
situations where such a multiplication would be useful.

The equation $W= |\vc{F}| |\vc{d}| \cos \theta$ is an example of
a sort of multiplication of vectors that is useful. The
result is a scalar, not a vector, and this is therefore
often referred to as the \emph{scalar product} of the
vectors $\vc{F}$ and $\vc{d}$. There is a standard shorthand
notation for this operation,
\begin{multline*}
 \vc{A}\cdot\vc{B} = |\vc{A}| |\vc{B}| \cos \theta\eqquad,
    \hfill \shoveright{\text{[definition of the notation $\vc{A}\cdot\vc{B}$;}}\\
    \text{$\theta$ is the angle between vectors $\vc{A}$ and $\vc{B}$]}
\end{multline*}
and because of this notation, a more common term for this
operation is the \emph{dot product}. In dot product
notation, the equation for work is simply
\begin{equation*}
  W = \vc{F}\cdot\vc{d}\eqquad.
\end{equation*}

 The dot product has the following geometric interpretation:
\begin{align*}
\vc{A}\cdot\vc{B} &= |\vc{A}|  (\text{component of $\vc{B}$ parallel to $\vc{A}$}) \\
 &= |\vc{B}|  (\text{component of $\vc{A}$ parallel to $\vc{B}$})
\end{align*}
The dot product has some of the properties possessed by
ordinary multiplication of numbers,
\begin{align*}
\vc{A}\cdot\vc{B} &= \vc{B}\cdot\vc{A} \\
\vc{A}\cdot(\vc{B}+\vc{C}) &= \vc{A}\cdot\vc{B}+\vc{A}\cdot\vc{C} \\
\left(c\vc{A}\right)\cdot\vc{B} &= c\left(\vc{A}\cdot\vc{B}\right)\eqquad,
\end{align*}
but it lacks one other: the ability to undo multiplication by dividing.

If you know the components of two vectors, you can easily
calculate their dot product as follows:
\begin{equation*}
  \vc{A}\cdot\vc{B} = A_xB_x+A_yB_y+A_zB_z\eqquad.
\end{equation*}
(This can be proved by first analyzing the special case where
each vector has only an $x$ component, and the similar cases
for $y$ and $z$. We can then use the rule 
$\vc{A}\cdot(\vc{B}+\vc{C}) = \vc{A}\cdot\vc{B}+\vc{A}\cdot\vc{C}$
to make a generalization
by writing each vector as the sum of its $x$, $y$, and $z$ components.
See homework problem \ref{hw:dot-product-by-components}.)
\label{dot-product-by-components}

        \begin{eg}{Magnitude expressed with a dot product}\label{eg:magnitudedot}
        If we take the dot product of any vector \vc{b} with itself, we find
        \begin{align*}        
                \vc{b}\cdot\vc{b}        &= 
                \left( b_{x}\hat{\vc{x}}+ b_{y}\hat{\vc{y}}+ b_z\hat{\vc{z}}\right)
                \cdot
                \left( b_{x}\hat{\vc{x}}+ b_{y}\hat{\vc{y}}+ b_z\hat{\vc{z}}\right) \\
                        &=  b_{x}^2+ b_{y}^2+ b_z^2\eqquad,\\
        \intertext{so its magnitude can be expressed as}
                |\vc{b}| &= \sqrt{\vc{b}\cdot\vc{b}}\eqquad.
        \end{align*}
        We will often write $b^2$ to mean $\vc{b}\cdot\vc{b}$, when the context makes it
        clear what is intended. For example, we could express kinetic energy as
        $\zu{(1/2)} m|\vc{v}|^2$, $\zu{(1/2)} m\vc{v}\cdot\vc{v}$,
        or $\zu{(1/2)} m v^2$. In the third version, nothing but context tells
        us that $v$ really stands for the magnitude of some vector $\vc{v}$.
        \end{eg}

        \begin{eg}{Towing a barge}
        \egquestion
        A mule pulls a barge with a force $\vc{F}\zu{=(1100 N)}\hat{\vc{x}}+\zu{(400 N)}\hat{\vc{y}}$,
        and the total distance it travels is $\zu{(1000 m)}\hat{\vc{x}}$. How much work does it do?

        \eganswer
        The dot product is $1.1\times10^6\ \nunit\unitdot\munit = 1.1\times10^6\ \junit$.
        \end{eg}
<% end_sec('dot-product') %>
