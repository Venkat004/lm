Soon after the mile-long Tacoma Narrows Bridge opened in
July 1940, motorists began to notice its tendency to vibrate
frighteningly in even a moderate wind. Nicknamed ``Galloping
Gertie,'' the bridge collapsed in a steady 42-mile-per-hour
wind on November 7 of the same year. The following is an
eyewitness report from a newspaper editor who found himself
on the bridge as the vibrations approached the breaking point.

``Just as I drove past the towers, the bridge began to sway
violently from side to side. Before I realized it, the tilt
became so violent that I lost control of the car... I jammed
on the brakes and got out, only to be thrown onto my
face against the curb.

``Around me I could hear concrete cracking. I started to get
my dog Tubby, but was thrown again before I could reach the
car. The car itself began to slide from side to side of the roadway.

``On hands and knees most of the time, I crawled 500 yards
or more to the towers... My breath was coming in gasps; my
knees were raw and bleeding, my hands bruised and swollen
from gripping the concrete curb... Toward the last, I risked
rising to my feet and running a few yards at a time...
Safely back at the toll plaza, I saw the bridge in its final
collapse and saw my car plunge into the Narrows.''

The ruins of the bridge formed an artificial reef, one of
the world's largest. It was not replaced for ten years. The
reason for its collapse was not substandard materials or
construction, nor was the bridge under-designed: the piers
were hundred-foot blocks of concrete, the girders massive
and made of carbon steel. The bridge was destroyed because
of the physical phenomenon of resonance, the same effect
that allows an opera singer to break a wine glass with her
voice and that lets you tune in the radio station you want.
The replacement bridge, which has lasted half a century so
far, was built smarter, not stronger. The engineers learned
their lesson and simply included some slight modifications
to avoid the resonance phenomenon that spelled the
doom of the first one.

<% begin_sec("Energy in Vibrations",0) %>

One way of describing the collapse of the bridge is that the
bridge kept taking energy from the steadily blowing wind and
building up more and more energetic vibrations. In this
section, we discuss the energy contained in a vibration, and
in the subsequent sections we will move on to the loss of
energy and the adding of energy to a vibrating system, all
with the goal of understanding the important phenomenon of resonance.

Going back to our standard example of a mass on a spring, we
find that there are two forms of energy involved: the
potential energy stored in the spring and the kinetic energy
of the moving mass. We may start the system in motion either
by hitting the mass to put in kinetic energy or by pulling it
to one side to put in potential energy. Either way, the
subsequent behavior of the system is identical. It trades
energy back and forth between kinetic and potential energy.
(We are still assuming there is no friction, so that no
energy is converted to heat, and the system never runs down.)

\enlargethispage{-2\baselineskip}

The most important thing to understand about the energy
content of vibrations is that the total energy is proportional
to the square of the amplitude.\index{amplitude!related to energy}\index{energy!related to amplitude}
Although the total energy is
constant, it is instructive to consider two specific moments
in the motion of the mass on a spring as examples. When the
mass is all the way to one side, at rest and ready to
reverse directions, all its energy is potential. We have
already seen that the potential energy stored in a spring
equals $(1/2)kx^2$, so the energy is proportional to the square of the
amplitude. Now consider the moment when the mass is passing
through the equilibrium point at $x=0$. At this point it has
no potential energy, but it does have kinetic energy. The
velocity is proportional to the amplitude of the motion, and
the kinetic energy, $(1/2)mv^2$, is proportional to the square of the
velocity, so again we find that the energy is proportional
to the square of the amplitude. The reason for singling out
these two points is merely instructive; proving that energy
is proportional to $A^2$ at any point would suffice to prove
that energy is proportional to $A^2$ in general, since
the energy is constant.

\enlargethispage{-2\baselineskip}

Are these conclusions restricted to the mass-on-a-spring
example? No. We have already seen that $F=-kx$ is a
valid approximation for any vibrating object, as long as the
amplitude is small. We are thus left with a very general
conclusion: the energy of any vibration is approximately
proportional to the square of the amplitude, provided that
the amplitude is small.

<% marg(0) %>
<%
  fig(
    'u-tube',
    %q{Example \ref{eg:u-tube}.}
  )
%>
<% end_marg %>
\begin{eg}{Water in a U-tube}\label{eg:u-tube}
If water is poured into a U-shaped tube as shown in the
figure, it can undergo vibrations about equilibrium. The
energy of such a vibration is most easily calculated by
considering the ``turnaround point'' when the water has
stopped and is about to reverse directions. At this point,
it has only potential energy and no kinetic energy, so by
calculating its potential energy we can find the energy of
the vibration. This potential energy is the same as the work
that would have to be done to take the water out of the
right-hand side down to a depth $A$ below the equilibrium
level, raise it through a height $A$, and place it in the
left-hand side. The weight of this chunk of water is
proportional to $A$, and so is the height through which it
must be lifted, so the energy is proportional to $A^2$.
\end{eg}

\begin{eg}{The range of energies of sound waves}
\egquestion The amplitude of vibration of your eardrum at the
threshold of pain is about $10^6$ times greater than the
amplitude with which it vibrates in response to the softest
sound you can hear. How many times greater is the energy
with which your ear has to cope for the painfully loud
sound, compared to the soft sound?

\eganswer The amplitude is $10^6$ times greater, and energy
is proportional to the square of the amplitude, so the
energy is greater by a factor of $10^{12}$ . This is a
phenomenally large factor!
\end{eg}

We are only studying vibrations right now, not waves, so we
are not yet concerned with how a sound wave works, or how
the energy gets to us through the air. Note that because of
the huge range of energies that our ear can sense, it would
not be reasonable to have a sense of loudness that was
additive. Consider, for instance, the following three levels of sound:

\begin{tabular}{p{40mm}p{53mm}}
barely audible wind\\
quiet conversation \dotfill & $10^5$ times more energy than the wind\\
heavy metal concert \dotfill & $10^{12}$ times more energy than the wind
\end{tabular}

In terms of addition and subtraction, the difference between
the wind and the quiet conversation is nothing compared to
the difference between the quiet conversation and the heavy
metal concert. Evolution wanted our sense of hearing to be
able to encompass all these sounds without collapsing the
bottom of the scale so that anything softer than the crack
of doom would sound the same. So rather than making our
sense of loudness additive, mother nature made it multiplicative.
We sense the difference between the wind and the quiet
conversation as spanning a range of about 5/12 as much as
the whole range from the wind to the heavy metal concert.
Although a detailed discussion of the decibel scale is not
relevant here, the basic point to note about the decibel
scale is that it is logarithmic. The zero of the \index{decibel
scale}decibel scale is close to the lower limit of human
hearing, and adding 1 unit to the decibel measurement
corresponds to \emph{multiplying} the energy level (or
actually the power per unit area) by a certain factor.

<% end_sec() %>
m4_include(__share[:/text/damped_oscillator_:]m4_ifelse(__me,1,me,lm)[:.tex:])
<% begin_sec("Putting Energy Into Vibrations",3,'driving-vibrations') %>

<% marg(0) %>
<%
  fig(
    'driving-force-and-resonance',
    %q{%
      1. Pushing a child on a swing gradually puts more and more energy
      into her vibrations. 2. A fairly realistic graph of the driving force acting on the child.
      3. A less realistic, but more mathematically simple, driving force.
    }
  )
%>
<% end_marg %>
When pushing a child on a \index{swing}swing, you cannot
just apply a constant force. A constant force will move the
swing out to a certain angle, but will not allow the swing
to start swinging. Nor can you give short pushes at randomly
chosen times. That type of random pushing would increase the
child's kinetic energy whenever you happened to be pushing
in the same direction as her motion, but it would reduce her
energy when your pushing happened to be in the opposite
direction compared to her motion. To make her build up her
energy, you need to make your pushes rhythmic, pushing at
the same point in each cycle. In other words, your force
needs to form a repeating pattern with the same frequency as
the normal frequency of vibration of the swing. Graph \figref{driving-force-and-resonance}/1
shows what the child's $x-t$ graph would look like as you
gradually put more and more energy into her vibrations. A
graph of your \emph{force} versus time would probably look
something like graph 2. It turns out, however, that it is
much simpler mathematically to consider a vibration with
energy being pumped into it by a \index{driving force}driving
force that is itself a sine-wave, 3. A good example of
this is your \index{eardrum}eardrum being driven by the
force of a sound wave.

<% marg(0) %>
<%
  fig(
    'amplitude-maxing-out',
    %q{The amplitude approaches a maximum.}
  )
%>
<% end_marg %>
Now we know realistically that the child on the swing will
not keep increasing her energy forever, nor does your
eardrum end up exploding because a continuing sound wave
keeps pumping more and more energy into it. In any realistic
system, there is energy going out as well as in. As the
vibrations increase in amplitude, there is an increase in
the amount of energy taken away by damping with each cycle.
This occurs for two reasons. Work equals force times
distance (or, more accurately, the area under the force-distance
curve). As the amplitude of the vibrations increases, the
damping force is being applied over a longer distance.
Furthermore, the damping force usually increases with
velocity (we usually assume for simplicity that it is
proportional to velocity), and this also serves to increase
the rate at which damping forces remove energy as the
amplitude increases. Eventually (and small children and our
eardrums are thankful for this!), the amplitude approaches a
maximum value, \figref{amplitude-maxing-out}, at which energy is removed by the damping
force just as quickly as it is being put in by the driving force.

This process of approaching a maximum amplitude happens
extremely quickly in many cases, e.g., the ear or a radio
receiver, and we don't even notice that it took a millisecond
or a microsecond for the vibrations to ``build up steam.''
We are therefore mainly interested in predicting the
behavior of the system once it has had enough time to reach
essentially its maximum amplitude. This is known as the
\index{steady-state behavior}steady-state behavior
of a vibrating system.

Now comes the interesting part: what happens if the
frequency of the driving force is mismatched to the
frequency at which the system would naturally vibrate on its
own? We all know that a radio station doesn't have to be
tuned in exactly, although there is only a small range over
which a given station can be received. The designers of the
radio had to make the range fairly small to make it possible to
eliminate unwanted stations that happened to be nearby in
frequency, but it couldn't be too small or you wouldn't be
able to adjust the knob accurately enough. (Even a digital
radio can be tuned to 88.0 MHz and still bring in a station
at 88.1 MHz.) The ear also has some natural frequency of
vibration, but in this case the range of frequencies to
which it can respond is quite broad. Evolution has made the
ear's frequency response as broad as possible because it was
to our ancestors' advantage to be able to hear everything
from a low roar to a high-pitched shriek.

The remainder of this section develops four important facts
about the response of a system to a driving force whose
frequency is not necessarily the same as the system's
natural frequency of vibration. The style is approximate and
intuitive, but proofs are given in section \ref{sec:resonance-proofs}.

First, although we know the ear has a frequency --- about
4000 Hz --- at which it would vibrate naturally, it does not
vibrate at 4000 Hz in response to a low-pitched 200 Hz tone.
It always responds at the frequency at which it is driven.
Otherwise all pitches would sound like 4000 Hz to us. This
is a general fact about driven vibrations:

\begin{important}
(1) The steady-state response to a sinusoidal driving force
occurs at the frequency of the force, not at the system's
own natural frequency of vibration.
\end{important}

Now let's think about the amplitude of the steady-state
response. Imagine that a child on a swing has a natural
frequency of vibration of 1 Hz, but we are going to try to
make her swing back and forth at 3 Hz. We intuitively
realize that quite a large force would be needed to achieve
an amplitude of even 30 cm, i.e., the amplitude is less in
proportion to the force. When we push at the natural
frequency of 1 Hz, we are essentially just pumping energy
back into the system to compensate for the loss of energy
due to the damping (friction) force. At 3 Hz, however, we
are not just counteracting friction. We are also providing
an extra force to make the child's momentum reverse itself
more rapidly than it would if gravity and the tension in the
chain were the only forces acting. It is as if we are
artificially increasing the $k$ of the swing, but this is
wasted effort because we spend just as much time decelerating
the child (taking energy out of the system) as accelerating
her (putting energy in).

Now imagine the case in which we drive the child at a very
low frequency, say 0.02 Hz or about one vibration per
minute. We are essentially just holding the child in
position while very slowly walking back and forth. Again we
intuitively recognize that the amplitude will be very small
in proportion to our driving force. Imagine how hard it
would be to hold the child at our own head-level when she is
at the end of her swing! As in the too-fast 3 Hz case, we
are spending most of our effort in artificially changing the
$k$ of the swing, but now rather than reinforcing the
gravity and tension forces we are working against them,
effectively reducing $k$. Only a very small part of our
force goes into counteracting friction, and the rest is used
in repetitively putting potential energy in on the upswing
and taking it back out on the downswing, without any long-term gain.

We can now generalize to make the following statement, which
is true for all driven vibrations:

\begin{important}
(2) A vibrating system \index{resonance!defined}resonates at
its own natural frequency.\footnote{This is
an approximation, which is valid in the usual case where $Q$ is significantly greater than 1.} That is, the amplitude of the
steady-state response is greatest in proportion to the
amount of driving force when the driving force matches the
natural frequency of vibration.
\end{important}

\begin{eg}{An opera singer breaking a wine glass}
In order to break a wineglass by singing, an opera singer
must first tap the glass to find its natural frequency of
vibration, and then sing the same note back.
\end{eg}

<% marg(0) %>
<%
  fig(
    'nimitz-freeway',
    %q{The collapsed section of the Nimitz Freeway.}
  )
%>
<% end_marg %>
\begin{eg}{Collapse of the Nimitz Freeway in an earthquake}
I led off the chapter with the dramatic collapse of the
Tacoma Narrows Bridge, mainly because it was well
documented by a local physics professor, and an unknown
person made a movie of the collapse. The collapse of a section
of the Nimitz Freeway in Oakland, CA, during a 1989
earthquake is however a simpler example to analyze.

 An earthquake consists of many low-frequency vibrations
that occur simultaneously, which is why it sounds like a
rumble of indeterminate pitch rather than a low hum. The
frequencies that we can hear are not even the strongest
ones; most of the energy is in the form of vibrations in the
range of frequencies from about 1 Hz to 10 Hz.

 Now all the structures we build are resting on geological
layers of dirt, mud, sand, or rock. When an earthquake wave
comes along, the topmost layer acts like a system with a
certain natural frequency of vibration, sort of like a cube
of jello on a plate being shaken from side to side. The
resonant frequency of the layer depends on how stiff it is
and also on how deep it is. The ill-fated section of the
Nimitz freeway was built on a layer of mud, and analysis by
geologist Susan E. Hough of the U.S. Geological Survey
shows that the mud layer's resonance was centered on about
2.5 Hz, and had a width covering a range from about 1 Hz to 4 Hz.

 When the earthquake wave came along with its mixture of
frequencies, the mud responded strongly to those that were
close to its own natural 2.5 Hz frequency. Unfortunately, an
engineering analysis after the quake showed that the
overpass itself had a resonant frequency of 2.5 Hz as well!
The mud responded strongly to the earthquake waves with
frequencies close to 2.5 Hz, and the bridge responded
strongly to the 2.5 Hz vibrations of the mud, causing
sections of it to collapse.
\end{eg}

\begin{eg}{Collapse of the Tacoma Narrows Bridge}
Let's now examine the more conceptually difficult case of
the Tacoma Narrows Bridge. The surprise here is that the
wind was steady. If the wind was blowing at constant
velocity, why did it shake the bridge back and forth? The
answer is a little complicated. Based on film footage and
after-the-fact wind tunnel experiments, it appears that two
different mechanisms were involved.

 The first mechanism was the one responsible for the
initial, relatively weak vibrations, and it involved
resonance. As the wind moved over the bridge, it began
acting like a kite or an airplane wing. As shown in the
figure, it established swirling patterns of air flow around
itself, of the kind that you can see in a moving cloud of
smoke. As one of these swirls moved off of the bridge, there
was an abrupt change in air pressure, which resulted in an
up or down force on the bridge. We see something similar
when a flag flaps in the wind, except that the flag's
surface is usually vertical. This back-and-forth sequence of
forces is exactly the kind of periodic driving force that
would excite a resonance. The faster the wind, the more
quickly the swirls would get across the bridge, and the
higher the frequency of the driving force would be. At just
the right velocity, the frequency would be the right one to
excite the resonance. The wind-tunnel models, however, show
that the pattern of vibration of the bridge excited by this
mechanism would have been a different one than the one that
finally destroyed the bridge.

 The bridge was probably destroyed by a different mechanism,
in which its vibrations at its own natural frequency of 0.2 Hz
set up an alternating pattern of wind gusts in the air
immediately around it, which then increased the amplitude of
the bridge's vibrations. This vicious cycle fed upon itself,
increasing the amplitude of the vibrations until the
bridge finally collapsed.
\end{eg}

As long as we're on the subject of collapsing bridges, it is
worth bringing up the reports of bridges falling down when
soldiers marching over them happened to step in rhythm with
the bridge's natural frequency of oscillation. This is
supposed to have happened in 1831 in Manchester, England,
and again in 1849 in Anjou, France. Many modern engineers
and scientists, however, are suspicious of the analysis of
these reports. It is possible that the collapses had more to
do with poor construction and overloading than with
resonance. The Nimitz Freeway and Tacoma Narrows Bridge are
far better documented, and occurred in an era when
engineers' abilities to analyze the vibrations of a complex
structure were much more advanced.

\begin{eg}{Emission and absorption of light waves by atoms}
In a very thin gas, the atoms are sufficiently far apart
that they can act as individual vibrating systems. Although
the vibrations are of a very strange and abstract type
described by the theory of quantum mechanics, they
nevertheless obey the same basic rules as ordinary
mechanical vibrations. When a thin gas made of a certain
element is heated, it emits light waves with certain
specific frequencies, which are like a fingerprint of that
element. As with all other vibrations, these atomic
vibrations respond most strongly to a driving force that
matches their own natural frequency. Thus if we have a
relatively cold gas with light waves of various frequencies
passing through it, the gas will absorb light at precisely
those frequencies at which it would emit light if heated.
\end{eg}

\begin{important}
(3) When a system is driven at resonance, the steady-state
vibrations have an amplitude that is proportional to $Q$.
\end{important}

This is fairly intuitive. The steady-state behavior is an
equilibrium between energy input from the driving force and
energy loss due to damping. A low-$Q$ oscillator, i.e., one
with strong damping, dumps its energy faster, resulting in
lower-amplitude steady-state motion.

<% self_check('opera',<<-'SELF_CHECK'
If an opera singer is shopping for a wine glass that she can
impress her friends by breaking, what should she look for?
  SELF_CHECK
  ) %>

\begin{eg}{Piano strings ringing in sympathy with a sung note}
\egquestion A sufficiently loud musical note sung near a piano
with the lid raised can cause the corresponding strings in
the piano to vibrate. (A piano has a set of three strings
for each note, all struck by the same hammer.) Why would
this trick be unlikely to work with a violin?

\eganswer If you have heard the sound of a violin being
plucked (the pizzicato effect), you know that the note dies
away very quickly. In other words, a violin's $Q$ is much
lower than a piano's. This means that its resonances are
much weaker in amplitude.
\end{eg}

<% marg(0) %>
<%
  fig(
    'fwhm',
    %q{%
      The definition of the full width at half
      maximum.
    }
  )
%>
<% end_marg %>
Our fourth and final fact about resonance is perhaps the
most surprising. It gives us a way to determine numerically
how wide a range of driving frequencies will produce a
strong response. As shown in the graph, resonances do not
suddenly fall off to zero outside a certain frequency range.
It is usual to describe the width of a resonance by its full
width at half-maximum (FWHM) as illustrated 
in figure \figref{fwhm}.\index{FWHM}\index{full width at half-maximum}\label{fwhm}

\begin{important}
(4) The FWHM of a resonance is related to its $Q$ and its
resonant frequency $f_{res}$ by the equation
\begin{equation*}
 \text{FWHM} = \frac{f_{res}}{Q}\eqquad.
\end{equation*}
(This equation is only a good approximation when $Q$ is large.)
\end{important}

Why? It is not immediately obvious that there should be any
logical relationship between $Q$ and the FWHM. Here's the
idea. As we have seen already, the reason why the response
of an oscillator is smaller away from resonance is that much
of the driving force is being used to make the system act as
if it had a different $k$. Roughly speaking, the half-maximum
points on the graph correspond to the places where the
amount of the driving force being wasted in this way is the
same as the amount of driving force being used productively
to replace the energy being dumped out by the damping force.
If the damping force is strong, then a large amount of force
is needed to counteract it, and we can waste quite a bit of
driving force on changing $k$ before it becomes comparable
to the damping force. If, on the other hand, the damping force is
weak, then even a small amount of force being wasted on
changing $k$ will become significant in proportion, and we
cannot get very far from the resonant frequency before
the two are comparable.

m4_ifelse(__me,1,[:
The response is in general out of phase with the driving force by
an angle $\delta$.

<%
  fig(
    'resonance',
    %q{%
      Dependence of the amplitude and phase angle
              on the driving frequency. The undamped case is $Q=\infty$, 
              and the other curves represent $Q$=1, 3, and 10. $F_m$, $m$, and $\omega_\zu{o}$ are all set to 1.
    },
    {'width'=>'wide','sidecaption'=>true}
  )
%>
:])

\begin{eg}{Changing the pitch of a wind instrument}
\egquestion A saxophone player normally selects which note to
play by choosing a certain fingering, which gives the
saxophone a certain resonant frequency. The musician can
also, however, change the pitch significantly by altering
the tightness of her lips. This corresponds to driving the
horn slightly off of resonance. If the pitch can be altered
by about 5\% up or down (about one musical half-step)
without too much effort, roughly what is the $Q$ of a saxophone?

\eganswer Five percent is the width on one side of the
resonance, so the full width is about 10\%, FWHM /
$f_{res}=0.1$. This implies a $Q$ of about 10, i.e., once the
musician stops blowing, the horn will continue sounding for
about 10 cycles before its energy falls off by a factor of
535. (Blues and jazz saxophone players will typically choose
a mouthpiece that has a low $Q$, so that they can produce
the bluesy pitch-slides typical of their style. ``Legit,''
i.e., classically oriented players, use a higher-$Q$ setup
because their style only calls for enough pitch variation
to produce a vibrato.)
\end{eg}

\begin{eg}{Decay of a saxophone tone}
\egquestion If a typical saxophone setup has a $Q$ of about
10, how long will it take for a 100-Hz tone played on a
baritone saxophone to die down by a factor of 535 in energy,
after the player suddenly stops blowing?

\eganswer A $Q$ of 10 means that it takes 10 cycles for the
vibrations to die down in energy by a factor of 535. Ten
cycles at a frequency of 100 Hz would correspond to a time
of 0.1 seconds, which is not very long. This is why a
saxophone note doesn't ``ring'' like a note played on a
piano or an electric guitar.
\end{eg}

\begin{eg}{$Q$ of a radio receiver}
\egquestion A radio receiver used in the FM band needs to be
tuned in to within about 0.1 MHz for signals at about 100
MHz. What is its $Q$?

\eganswer $Q=f_{res}/\zu{FWHM}=1000$. This is an extremely
high $Q$ compared to most mechanical systems.
\end{eg}

\begin{eg}{$Q$ of a stereo speaker}
We have already given one reason why a stereo speaker should
have a low $Q$: otherwise it would continue ringing after
the end of the musical note on the recording. The second
reason is that we want it to be able to respond to a large
range of frequencies.
\end{eg}

\begin{eg}{Nuclear magnetic resonance}\label{eg:nmr}
If you have ever played with a magnetic compass, you have
undoubtedly noticed that if you shake it, it takes some time
to settle down, \figref{nmr}/1. As it settles down, it acts like a damped
oscillator of the type we have been discussing. The compass
needle is simply a small magnet, and the planet earth is a
big magnet. The magnetic forces between them tend to bring
the needle to an equilibrium position in which it lines up
with the planet-earth-magnet.

<% marg(m4_ifelse(__me,1,170,70)) %>
<%
  fig(
    'nmr',
    %q{%
      Example \ref{eg:nmr}. 1. A compass needle vibrates about the equilibrium position under the
      influence of the earth's magnetic forces. 2. The orientation of a proton's
      spin vibrates around its equilibrium direction under the influence of the
      magnetic forces coming from the surrounding electrons and nuclei.
    }
  )
%>
\spacebetweenfigs
<%
  fig(
    'gretchen-nmr',
    %q{%
      A member of the author's family, who turned out to be
      healthy.
    }
  )
%>
\spacebetweenfigs
<%
  fig(
    'three-dimensional-nmr',
    %q{%
      A three-dimensional computer reconstruction of the shape
      of a human brain, based on magnetic resonance data.
    }
  )
%>

<% end_marg %>
Essentially the same physics lies behind the technique
called Nuclear Magnetic Resonance (NMR). NMR is a technique
used to deduce the molecular structure of unknown chemical
substances, and it is also used for making medical images of
the inside of people's bodies. If you ever have an NMR scan,
they will actually tell you you are undergoing ``magnetic
resonance imaging'' or ``MRI,'' because people are scared of
the word ``nuclear.'' In fact, the nuclei being referred to
are simply the non-radioactive nuclei of atoms found
naturally in your body.\label{mri-resonance-explanation}

Here's how NMR works. Your body contains large numbers of
hydrogen atoms, each consisting of a small, lightweight
electron orbiting around a large, heavy proton. That is, the
nucleus of a hydrogen atom is just one proton. A proton is
always spinning on its own axis, and the combination of its
spin and its electrical charge causes it to behave like a
tiny magnet. The principle is identical to that of an
electromagnet, which consists of a coil of wire through
which electrical charges pass; the circling motion of the
charges in the coil of wire makes it magnetic, and in the
same way, the circling motion of the proton's charge makes it magnetic.

Now a proton in one of your body's hydrogen atoms finds
itself surrounded by many other whirling, electrically
charged particles: its own electron, plus the electrons and
nuclei of the other nearby atoms. These neighbors act like
magnets, and exert magnetic forces on the proton, \figref{nmr}/2. The $k$ of
the vibrating proton is simply a measure of the total
strength of these magnetic forces. Depending on the
structure of the molecule in which the hydrogen atom finds
itself, there will be a particular set of magnetic forces
acting on the proton and a particular value of $k$. The NMR
apparatus bombards the sample with radio waves, and if the
frequency of the radio waves matches the resonant frequency
of the proton, the proton will absorb radio-wave energy
strongly and oscillate wildly. Its vibrations are damped not
by friction, because there is no friction inside an atom,
but by the reemission of radio waves.

By working backward through this chain of reasoning, one can
determine the geometric arrangement of the hydrogen atom's
neighboring atoms. It is also possible to locate atoms in
space, allowing medical images to be made.

Finally, it should be noted that the behavior of the proton
cannot be described entirely correctly by Newtonian physics.
Its vibrations are of the strange and spooky kind described
by the laws of quantum mechanics. It is impressive, however,
that the few simple ideas we have learned about resonance
can still be applied successfully to describe many aspects
of this exotic system.
\end{eg}

\startdq

\begin{dq}
Nikola Tesla, one of the inventors of radio and an
archetypical mad scientist, told a credulous reporter in 1912 the
following story about an application of resonance. He built
an electric vibrator that fit in his pocket, and attached it
to one of the steel beams of a building that was under
construction in New York. Although the article in which he
was quoted didn't say so, he presumably claimed to have
tuned it to the resonant frequency of the building. ``In a
few minutes, I could feel the beam trembling. Gradually the
trembling increased in intensity and extended throughout the
whole great mass of steel. Finally, the structure began to
creak and weave, and the steelworkers came to the ground
panic-stricken, believing that there had been an earthquake.
... [If] I had kept on ten minutes more, I could have laid
that building flat in the street.'' Is this physically plausible?
\end{dq}

<% end_sec() %>
<% begin_sec("Proofs",nil,'resonance-proofs',{'optional'=>true}) %>


m4_ifelse(__lm_series,1,[:
%------------- LM version
Our first goal is to predict the amplitude of the steady-state
vibrations as a function of the frequency of the driving
force and the amplitude of the driving force. With that
equation in hand, we will then prove statements 2, 3, and 4
from  section \ref{sec:driving-vibrations}. We assume without proof statement
1, that the steady-state motion occurs at the same frequency
as the driving force.

<% marg(m4_ifelse(__lm_series,1,70,30)) %>
<%
  fig(
    'resonance-proof-1',
    %q{Driving at a frequency above resonance.}
  )
%>
\spacebetweenfigs
<%
  fig(
    'resonance-proof-2',
    %q{Driving at resonance.}
  )
%>
\spacebetweenfigs
<%
  fig(
    'resonance-proof-3',
    %q{Driving at a frequency below resonance.}
  )
%>
<% end_marg %>

As with the proof in chapter \ref{ch:vibrations}, we make use of
the fact that a sinusoidal vibration is the same as the
projection of circular motion onto a line. We visualize the
system shown in figures \figref{resonance-proof-1}-\figref{resonance-proof-3}, in which the mass swings in a
circle on the end of a spring. The spring does not actually
change its length at all, but it appears to from the
flattened perspective of a person viewing the system
edge-on. The radius of the circle is the amplitude, $A$, of
the vibrations as seen edge-on. The damping force can be
imagined as a backward drag force supplied by some fluid
through which the mass is moving. As usual, we assume that
the damping is proportional to velocity, and we use the
symbol $b$ for the proportionality constant, $|F_d|=bv$.
The driving force, represented by a hand towing the mass
with a string, has a tangential component $|F_t|$ which
counteracts the damping force, $|F_t|=|F_d|$, and a radial
component $F_r$ which works either with or against the
spring's force, depending on whether we are driving the
system above or below its resonant frequency.

The speed of the rotating mass is the circumference of the
circle divided by the period, $v=2\pi A/T$, its acceleration
(which is directly inward) is $a=v^2/r$, and Newton's second
law gives $a=F/m=(kA+F_r)/m$. We write $f_\zu{o}$ for
$\frac{1}{2\pi}\sqrt{k/m}$. Straightforward algebra yields
\begin{equation}
 \frac{F_r}{F_t} = \frac{2\pi m}{bf}\left(f^2-f_\zu{o}^2\right)\eqquad.
\end{equation}
This is the ratio of the wasted force to the useful force,
and we see that it becomes zero when the system is
driven at resonance.

The amplitude of the vibrations can be found by attacking
the equation $|F_t|=bv=2\pi bAf$, which gives
\begin{equation}
  A = \frac{|F_t|}{2\pi bf}\eqquad. (2)
\end{equation}
However, we wish to know the amplitude in terms of $|\vc{F}|$, not
$|F_t|$. From now on, let's drop the cumbersome magnitude
symbols. With the Pythagorean theorem, it is easily proved that
\begin{equation}
  F_t = \frac{F}{\sqrt{1+\left(\frac{F_r}{F_t}\right)^2}}\eqquad, (3)
\end{equation}
and equations 1-3 can then be combined to give the final result
\begin{equation}\label{resonance-amplitude-equation}
 A = \frac{F}{2\pi\sqrt{4\pi^2m^2\left(f^2-f_\zu{o}^2\right)^2+b^2f^2}}\eqquad.
\end{equation}
%------------- end LM version
:],[:
%------------- Mechanics version
Our first goal is to predict the amplitude of the steady-state
vibrations as a function of the frequency of the driving
force and the amplitude of the driving force. With that
equation in hand, we will then prove statements 2, 3, and 4
from  section \ref{sec:driving-vibrations}.

        We have an external driving force $F=F_m \sin \omega t$, where the constant
        $F_m$ indicates the maximum strength of the force in either direction. The equation
        of motion is
        \begin{equation}\label{eqn:resonancemotion}
                ma+bv+kx = F_m \sin \omega t 
\eqquad.
        \end{equation}
        For the steady-state motion,
        we're going to look for a solution of the form
        \begin{equation*}
                x = A \sin (\omega{}t+\delta)\eqquad.
        \end{equation*}
        The left-hand side of the equation of motion will clearly be a sinusoidal function with frequency $\omega$,
        so it can only equal the right-hand side if, as we have already implicitly assumed, the frequency of the
        motion matches the frequency of the driving force. This proves statement (1).

        In contrast to the undriven case, here it's not possible to sweep $A$ and $\delta$ 
        under the rug. The amplitude of the steady-state motion, $A$, is actually the
        most interesting thing to know about the steady-state motion, and it's not true that we
        still have a solution no matter how we fiddle with $A$; if we have a solution for
        a certain value of $A$, then multiplying $A$ by some constant would break the
        equality between the two sides of the equation of motion. It's also no longer true
        that we can get rid of $\delta$ simply be redefining when we start the clock; here
        $\delta$ represents a \emph{difference} in time between the start of one cycle of the driving
        force and the start of the corresponding cycle of the motion.

        The velocity and
        acceleration are $v=\omega{}A\cos(\omega  t+\delta)$ and
        $a=-\omega^2A\sin(\omega t+\delta)$, and if we plug these into the equation
        of motion, \eqref{eqn:resonancemotion}, and simplify a little, we find
        \begin{equation}\label{eqn:steadystate}
                (k-m\omega^2)\sin (\omega t+\delta)
                         +\omega b \cos (\omega t+\delta) 
                        = \frac{F_m}{A} \sin \omega t\eqquad.
        \end{equation}
        The sum of any two sinusoidal functions with the same frequency is also
        a sinusoidal, so the whole left side adds up to a sinusoidal. By fiddling with
        $A$ and $\delta$ we can make the amplitudes and phases of the two sides
        of the equation match up.

	Using the trig identities for the sine of a sum
	and cosine of a sum, we can change equation \eqref{eqn:steadystate}
        into the form
	\begin{gather*}
		 \left[(-m\omega^2+k)\cos\delta-b\omega\sin\delta-F_m/A\right]\sin\omega t \\
		+  \left[(-m\omega^2+k)\sin\delta+b\omega\cos\delta\right]\cos\omega t
			= 0\eqquad.
	\end{gather*}
	Both the quantities in square brackets must equal zero, which gives us two
	equations we can use to determine the unknowns $A$ and $\delta$. 
	The results are
	\begin{equation}
		\delta = \tan^{-1}\frac{\omega\omega_\zu{o}}
					{Q(\omega_\zu{o}^2-\omega^2)} \\
	\end{equation}
        and
	\begin{equation}\label{resonance-amplitude-equation}
		A = \frac{F_m}{m\sqrt{\left(\omega^2-\omega_\zu{o}^2\right)^2
				+\omega_\zu{o}^2\omega^2Q^{-2}}}\eqquad.
	\end{equation}
\label{resonance-amplitude}

%------------- end Mechanics version
:])
<% begin_sec("Statement 2: maximum amplitude at resonance") %>

Equation [\ref{resonance-amplitude-equation}] makes it plausible that the amplitude is maximized
when the system is driven at close to its resonant frequency. At
$f=f_\zu{o}$, the first term inside the square root vanishes,
and this makes the denominator as small as possible, causing
the amplitude to be as big as possible. (Actually this is
only approximately true, because it is possible to make $A$
a little bigger by decreasing $f$ a little below $f_\zu{o}$,
which makes the second term smaller. This technical issue is
addressed in homework problem \ref{hw:qcorrection} on page 
\pageref{hw:qcorrection}.)

<% end_sec() %>
<% begin_sec("Statement 3: amplitude at resonance proportional to $Q$") %>

Equation [\ref{resonance-amplitude-equation}] shows that the amplitude at resonance is
proportional to $1/b$, and the $Q$ of the system is
inversely proportional to $b$, so the amplitude at resonance
is proportional to $Q$.

<% end_sec() %>
<% begin_sec("Statement 4: FWHM related to $Q$") %>

We will satisfy ourselves by proving only the proportionality
$FWHM\propto f_\zu{o}/Q$, not the actual equation $FWHM=f_\zu{o}/Q$.
The energy is proportional to $A^2$, i.e., to the inverse of
the quantity inside the square root in equation [\ref{resonance-amplitude-equation}]. At
resonance, the first term inside the square root vanishes,
and the half-maximum points occur at frequencies for which
the whole quantity inside the square root is double its
value at resonance, i.e., when the two terms are equal. At
the half-maximum points, we have
\begin{align*}
  f^2-f_\zu{o}^2 &= \left(f_\zu{o} \pm \frac{\text{FWHM}}{2}\right)^2 - f_\zu{o}^2\\
                &= \pm f_\zu{o}\cdot \text{FWHM} + \frac{1}{4}\text{FWHM}^2
\end{align*}
If we assume that the width of the resonance is small
compared to the resonant frequency, then the $\text{FWHM}^2$ term 
is negligible compared to the $f_\zu{o}\cdot \text{FWHM}$ term, and setting
the terms in equation 4 equal to each other gives
\begin{equation*}
  4\pi^2m^2\left(f_\zu{o}\text{FWHM}\right)^2  = b^2f^2\eqquad.
\end{equation*}
We are assuming that the width of the resonance is small
compared to the resonant frequency, so $f$ and $f_\zu{o}$ can
be taken as synonyms. Thus,
\begin{equation*}
                        \text{FWHM}  = \frac{b}{2\pi m}\eqquad.
\end{equation*}
We wish to connect this to $Q$, which can be interpreted as
the energy of the free (undriven) vibrations divided by the
work done by damping in one cycle. The former equals
$kA^2/2$, and the latter is proportional to the force,
$bv\propto bAf_\zu{o}$, multiplied by the distance traveled,
$A$. (This is only a proportionality, not an equation, since
the force is not constant.) We therefore find that $Q$ is
proportional to $k/bf_\zu{o}$. The equation for the FWHM can
then be restated as a proportionality $\text{FWHM}\propto k/Qf_\zu{o}m\propto f_\zu{o}/Q$.

<% end_sec() %>
<% end_sec() %>\begin{summary}

\begin{vocab}

\vocabitem{damping}{the dissipation of a vibration's energy into heat
energy, or the frictional force that causes the loss of energy}

\vocabitem{quality factor}{the number of oscillations required for a
system's energy to fall off by a factor of 535 due to damping}

\vocabitem{driving force}{an external force that pumps energy into a vibrating system}

\vocabitem{resonance}{the tendency of a vibrating system to respond most
strongly to a driving force whose frequency is close to its
own natural frequency of vibration}

\vocabitem{steady state}{the behavior of a vibrating system after it has
had plenty of time to settle into a steady response to a driving force}

\end{vocab}

\begin{notation}

\notationitem{$Q$}{the quality factor}

\notationitem{$f_\zu{o}$}{the natural (resonant) frequency of a vibrating
system, i.e., the frequency at which it would vibrate if it
was simply kicked and left alone}

\notationitem{$f$}{the frequency at which the system actually vibrates, which
in the case of a driven system is equal to the frequency of
the driving force, not the natural frequency}

\end{notation}

\begin{summarytext}

The energy of a vibration is always proportional to the
square of the amplitude, assuming the amplitude is small.
Energy is lost from a vibrating system for various reasons
such as the conversion to heat via friction or the emission
of sound. This effect, called damping, will cause the
vibrations to decay exponentially unless energy is pumped
into the system to replace the loss. A driving force that
pumps energy into the system may drive the system at its own
natural frequency or at some other frequency. When a
vibrating system is driven by an external force, we are
usually interested in its steady-state behavior, i.e., its
behavior after it has had time to settle into a steady
response to a driving force. In the steady state, the same
amount of energy is pumped into the system during each cycle
as is lost to damping during the same period.

The following are four important facts about a vibrating
system being driven by an external force:

(1) The steady-state response to a sinusoidal driving force
occurs at the frequency of the force, not at the system's
own natural frequency of vibration.

(2) A vibrating system resonates at its own natural
frequency. That is, the amplitude of the steady-state
response is greatest in proportion to the amount of driving
force when the driving force matches the natural frequency of vibration.

(3) When a system is driven at resonance, the steady-state
vibrations have an amplitude that is proportional to $Q$.

(4) The FWHM of a resonance is related to its $Q$ and its
resonant frequency $f_\zu{o}$ by the equation
\begin{equation*}
  \text{FWHM} = \frac{f_\zu{o}}{Q}.
\end{equation*}
(This equation is only a good approximation when $Q$ is large.)

\end{summarytext}

\end{summary}
