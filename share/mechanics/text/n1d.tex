<% begin_sec("Newton's third law",0,'third-law') %>

Newton created the modern concept of force starting from his
insight that all the effects that govern motion are
interactions between two objects: unlike the Aristotelian
theory, Newtonian physics has no phenomena in which an
object changes its own motion.

\pagebreak

Is one object always the ``order-giver'' and the other the
``order-follower''? As an example, consider a batter hitting
a baseball. The bat definitely exerts a large force on the
ball, because the ball accelerates drastically. But if you
have ever hit a baseball, you also know that the ball makes
a force on the bat --- often with painful results if your
technique is as bad as mine!

How does the ball's force on the bat compare with the bat's
force on the ball? The bat's acceleration is not as
spectacular as the ball's, but maybe we shouldn't expect it
to be, since the bat's mass is much greater. In fact,
careful measurements of both objects' masses and accelerations
would show that $m_{ball}a_{ball}$ is very nearly equal to
$-m_{bat}a_{bat}$, which suggests that the ball's force
on the bat is of the same magnitude as the bat's force on
the ball, but in the opposite direction.

Figures \figref{third-law-magnets} and \figref{third-law-spring-scales}
 show two somewhat more practical laboratory
experiments for investigating this issue accurately and
without too much interference from extraneous forces.

<% marg(86) %>
<%
  fig(
    'third-law-magnets',
    %q{%
      Two magnets exert forces on each
      other.
    }
  )
%>
\spacebetweenfigs
<%
  fig(
    'third-law-spring-scales',
    %q{%
      Two people's hands exert forces
      on each other.
    }
  )
%>
<% end_marg %>


In experiment \figref{third-law-magnets}, a large magnet and a small magnet
are weighed separately, and then one magnet is hung from the
pan of the top balance so that it is directly above the
other magnet. There is an attraction between the two
magnets, causing the reading on the top scale to increase
and the reading on the bottom scale to decrease. The large
magnet is more ``powerful'' in the sense that it can pick up
a heavier paperclip from the same distance, so many people
have a strong expectation that one scale's reading will
change by a far different amount than the other. Instead, we
find that the two changes are equal in magnitude but
opposite in direction: the force of the bottom magnet pulling
down on the top one has the same strength as the force of
the top one pulling up on the bottom one.

<% marg(32) %>
<%
  fig(
    'shuttle-launch',
    %q{%
      Rockets work by pushing exhaust gases out
      the back. Newton's third law says that if the
      rocket exerts a backward force on the gases,
      the gases must make an equal forward force
      on the rocket. Rocket engines can function
      above the atmosphere, unlike propellers and
      jets, which work by pushing against the 
      surrounding air.
    }
  )
%>

<% end_marg %>

m4_ifelse(__me,1,[:
\enlargethispage{-3\baselineskip}
:],[:
\enlargethispage{-3\baselineskip}
:])

In experiment \figref{third-law-spring-scales}, two people pull on two spring
scales. Regardless of who tries to pull harder, the two
forces as measured on the spring scales are equal.
Interposing the two spring scales is necessary in order to
measure the forces, but the outcome is not some artificial
result of the scales' interactions with each other. If one
person slaps another hard on the hand, the slapper's hand
hurts just as much as the slappee's, and it doesn't matter
if the recipient of the slap tries to be inactive. (Punching
someone in the mouth causes just as much force on the fist
as on the lips. It's just that the lips are more delicate.
The forces are equal, but not the levels of pain and injury.)

Newton, after observing a series of results such as these,
decided that there must 
be a fundamental \index{Newton's laws of motion!third law}law of nature at work:

\begin{important}[Newton's third law]
Forces occur in equal and opposite pairs: whenever object A
exerts a force on object B, object B must also be exerting a
force on object A. The two forces are equal in magnitude and
opposite in direction.
\end{important}

Two modern, high-precision tests of the third law are described 
on m4_ifelse(__lm_series,1,[:p.~\pageref{sec:kreuzer}.:],[:p.~\pageref{sec:third-law-true}.:])

In one-dimensional situations, we can use plus and minus
signs to indicate the directions of forces, and Newton's
third law can be written succinctly as $F_{\text{A on B}}=-F_{\text{B on A}}$.
m4_ifelse(__me,1,[:%
Section \ref{sec:third-law-true} gives a more detailed discussion of the logical and
empirical underpinnings of the third law.
\enlargethispage{-1\baselineskip}
:],[:
\enlargethispage{-3\baselineskip}
:])


<% self_check('swimming-and-walking',<<-'SELF_CHECK'
Figure \\figref{swimming} analyzes swimming using Newton's third law.
Do a similar analysis for a sprinter leaving the starting line.
  SELF_CHECK
  ) %>

There is no cause and effect relationship between the two
forces in Newton's third law. There is no ``original'' force, and neither one is a
response to the other. The pair of forces is a relationship,
like marriage, not a back-and-forth process like a tennis
match. Newton came up with the third law as a generalization
about all the types of forces with which he was familiar,
such as frictional and gravitational forces. When later
physicists discovered a new type of force, such as the force
that holds atomic nuclei together, they had to check whether
it obeyed Newton's third law. So far, no violation of the
third law has ever been discovered, whereas the first and
second laws were shown to have limitations by Einstein and
the pioneers of atomic physics.

<% marg(130) %>
<%
  fig(
    'swimming',
    %q{%
      A swimmer doing the breast stroke pushes backward
      against the water. By Newton's third law, the water pushes
      forward on him. 
    }
  )
%>
<% end_marg %>

The English vocabulary for describing forces is unfortunately
rooted in Aristotelianism, and often implies incorrectly
that forces are one-way relationships. It is unfortunate
that a half-truth such as ``the table exerts an upward force
on the book'' is so easily expressed, while a more complete
and correct description ends up sounding awkward or strange:
``the table and the book interact via a force,'' or ``the
table and book participate in a force.''
<% marg(36) %>
<%
  fig(
    'skaters',
    %q{%
      Newton's third law does not mean that
      forces always cancel out so that nothing can ever move. If these two ice
      skaters, initially at rest, push against
      each other, they will both move.
    }
  )
%>
<% end_marg %>

To students, it often sounds as though Newton's third law
implies nothing could ever change its motion, since the two
equal and opposite forces would always cancel. The two
forces, however, are always on two different objects, so it
doesn't make sense to add them in the first place --- we
only add forces that are acting on the same object. If two
objects are interacting via a force and no other forces are
involved, then \emph{both} objects will accelerate --- in
opposite directions!

<%
  fig(
    'cartoon-bar-tab',
    %q{%
      It doesn't make sense for the man to
      talk about using the woman's money 
      to cancel out his bar tab, because there
      is no good reason to combine his
      debts and her assets. Similarly, it
      doesn't make sense to refer to the
      equal and opposite forces of Newton's
      third law as canceling. It only makes
      sense to add up forces that are acting
      on the \emph{same} object, whereas two
      forces related to each other by
      Newton's third law are always acting
      on two \emph{different} objects.
    },
    {
      'width'=>'wide',
      'sidecaption'=>true
    }
  )
%>


<% begin_sec("A mnemonic for using Newton's third law correctly") %>

Mnemonics are tricks for memorizing things. For instance,
the musical notes that lie between the lines on the treble
clef spell the word FACE, which is easy to remember. Many
people use the mnemonic ``SOHCAHTOA'' to remember the
definitions of the sine, cosine, and tangent in trigonometry.
I have my own modest offering, \index{POFOSTITO}POFOSTITO,
which I hope will make it into the mnemonics hall of fame.
It's a way to avoid some of the most common problems with
applying Newton's third law correctly:

<% raw_fig('pofostito') %>

\begin{eg}{A book lying on a table}
\egquestion A book is lying on a table. What force is the
Newton's-third-law partner of the earth's gravitational force on the book?

Answer: Newton's third law works like ``B on A, A on $B$,''
so the partner must be the book's gravitational force
pulling upward on the planet earth. Yes, there is such a
force! No, it does not cause the earth to do anything noticeable.

Incorrect answer: The table's upward force on the book is
the Newton's-third-law partner of the earth's gravitational force on the book.

<% x_mark %> This answer violates two out of three of the commandments
of POFOSTITO. The forces are not of the same type, because
the table's upward force on the book is not gravitational.
Also, three objects are involved instead of two: the book,
the table, and the planet earth.

\end{eg}

\begin{eg}{Pushing a box up a hill}\label{eg:box-uphill}
\egquestion A person is pushing a box up a hill. What force is
related by Newton's third law to the person's force on the box?

\eganswer The box's force on the person.

Incorrect answer: The person's force on the box is opposed
by friction, and also by gravity.

<% x_mark %> This answer fails all three parts of the POFOSTITO test,
the most obvious of which is that three forces are referred
to instead of a pair.

\end{eg}

<% marg(100) %>
\begin{margtopic}{Optional topic: Newton's third law and action at a distance}
Newton's third law is completely symmetric in the sense that
neither force constitutes a delayed response to the other.
Newton's third law does not even mention time, and the
forces are supposed to agree at any given instant. This
creates an interesting situation when it comes to noncontact
forces. Suppose two people are holding magnets, and when one
person waves or wiggles her magnet, the other person feels
an effect on his. In this way they can send signals to each
other from opposite sides of a wall, and if Newton's third
law is correct, it would seem that the signals are
transmitted instantly, with no time lag. The signals are
indeed transmitted quite quickly, but experiments with
electrically controlled magnets show that the signals do
not leap the gap instantly: they travel at the same speed as
light, which is an extremely high speed but not an infinite one.

Is this a contradiction to Newton's third law? Not really.
According to current theories, there are no true noncontact
forces. Action at a distance does not exist. Although it
appears that the wiggling of one magnet affects the other
with no need for anything to be in contact with anything,
what really happens is that wiggling a magnet creates a ripple
in the magnetic field pattern that exists even in empty space. The magnet shoves
the ripples out with a kick and receives a kick in return,
in strict obedience to Newton's third law. The ripples spread
out in all directions, and the ones that hit the other
magnet then interact with it, again obeying Newton's third law.
\end{margtopic}
<% end_marg %>

\begin{eg}{If we could violate Newton's third law\ldots}\label{eg:third-law-violation-car}
If we could violate Newton's third law, we could do strange and wonderful things.
Newton's third laws says that the unequal magnets in figure \figref{third-law-magnets} on p.~\pageref{fig:third-law-magnets}
should exert equal forces on each other, and this is what we actually find when we do
the experiment shown in that figure. But suppose instead that it worked as most
people intuitively expect. What if the third law was violated, so that the big magnet
made more force on the small one than the small one made on the big one? To make the analysis
simple, we add some extra nonmagnetic material to the small magnet in figure \subfigref{third-law-violation-car}{1},
so that it has the same mass and size as the big one. We also attach springs.
When we release the magnets, \subfigref{third-law-violation-car}{2}, the weak one is accelerated strongly,
while the strong one barely moves. If we put them inside a box, \subfigref{third-law-violation-car}{3},
the recoiling strong magnet bangs
hard against the side of the box, and the box mysteriously accelerates itself. The process can be repeated
indefinitely for free, so we have a magic box that propels itself without needing fuel.
We can make it into a perpetual-motion car, \subfigref{third-law-violation-car}{4}. If Newton's
third law was violated, we'd never have to pay for gas!
\end{eg}

<%
  fig(
    'third-law-violation-car',
    %q{%
      Example \ref{eg:third-law-violation-car}. This doesn't actually happen!
    },
    {
      'width'=>'wide',
      'sidecaption'=>false
    }
  )
%>


\worked{box-uphill-partners}{More about example \ref{eg:box-uphill}}

\worked{third-law-partners}{Why did it accelerate?}

\startdqs

\begin{dq}
When you fire a gun, the exploding gases push outward in
all directions, causing the bullet to accelerate down the
barrel. What third-law pairs are involved? [Hint: Remember
that the gases themselves are an object.]
\end{dq}

\begin{dq}
Tam Anh grabs Sarah by the hand and tries to pull her.
She tries to remain standing without moving. A student
analyzes the situation as follows. ``If Tam Anh's force on
Sarah is greater than her force on him, he can get her to
move. Otherwise, she'll be able to stay where she is.''
What's wrong with this analysis?
\end{dq}

\begin{dq}
You hit a tennis ball against a wall. Explain any and all
incorrect ideas in the following description of the physics
involved: ``According to Newton's third law, there has to be
a force opposite to your force on the ball. The opposite
force is the ball's mass, which resists acceleration, and
also air resistance.''
\end{dq}

<% end_sec() %>
<% end_sec() %>
<% begin_sec("Classification and behavior of forces",0) %>\index{forces!classification of}
One of the most basic and important tasks of physics is to
classify the forces of nature. I have already referred
informally to ``types'' of forces such as friction,
magnetism, gravitational forces, and so on. Classification
systems are creations of the human mind, so there is always
some degree of arbitrariness in them. For one thing, the
level of detail that is appropriate for a classification
system depends on what you're trying to find out. Some
linguists, the ``lumpers,'' like to emphasize the similarities
among languages, and a few extremists have even tried to
find signs of similarities between words in languages as
different as English and Chinese, lumping the world's
languages into only a few large groups. Other linguists, the
``splitters,'' might be more interested in studying the
differences in pronunciation between English speakers in New
York and Connecticut. The splitters call the lumpers sloppy,
but the lumpers say that science isn't worthwhile unless it
can find broad, simple patterns within the seemingly complex universe.

Scientific classification systems are also usually
compromises between practicality and naturalness. An example
is the question of how to classify flowering plants. Most
people think that biological classification is about
discovering new species, naming them, and classifying them
in the class-order-family-genus-species system according to
guidelines set long ago. In reality, the whole system is in
a constant state of flux and controversy. One very practical
way of classifying flowering plants is according to whether
their petals are separate or joined into a tube or cone ---
the criterion is so clear that it can be applied to a plant
seen from across the street. But here practicality conflicts
with naturalness. For instance, the begonia has separate
petals and the pumpkin has joined petals, but they are so
similar in so many other ways that they are usually placed
within the same order. Some taxonomists have come up with
classification criteria that they claim correspond more
naturally to the apparent relationships among plants,
without having to make special exceptions, but these may be
far less practical, requiring for instance the examination
of pollen grains under an electron microscope.
<% marg(70) %>
<%
  fig(
    'flowers',
    %q{A scientific classification system.}
  )
%>
<% end_marg %>

In physics, there are two main systems of classification for
forces. At this point in the course, you are going to learn
one that is very practical and easy to use, and that splits
the forces up into a relatively large number of types: seven
very common ones that we'll discuss explicitly in this
chapter, plus perhaps ten less important ones such as
surface tension, which we will not bother with right now.

Physicists, however, are obsessed
with finding simple patterns, so recognizing as many as
fifteen or twenty types of forces strikes them as distasteful
and overly complex. Since about the year 1900, physics has
been on an aggressive program to discover ways in which
these many seemingly different types of forces arise from a
smaller number of fundamental ones. For instance, when you
press your hands together, the force that keeps them from
passing through each other may seem to have nothing to do
with electricity, but at the atomic level, it actually does
arise from electrical repulsion between atoms. By about
1950, all the forces of nature had been explained as arising
from four fundamental types of forces at the atomic and
nuclear level, and the lumping-together process didn't stop
there. By the 1960's the length of the list had been reduced
to three, and some theorists even believe that they may be
able to reduce it to two or one. Although the unification of
the forces of nature is one of the most beautiful and
important achievements of physics, it makes much more sense
to start this course with the more practical and easy system
of classification. The unified system of four forces will be
one of the highlights of the end of your introductory physics sequence.
<%
  fig(
    'force-tree',
    %q{%
      A practical classification scheme
      for forces.
    },
    {
      'width'=>'fullpage'
    }
  )
%>

The practical classification scheme which concerns us now
can be laid out in the form of the tree shown in figure \figref{force-tree}. The
most specific types of forces are shown at the tips of the
branches, and it is these types of forces that are referred
to in the POFOSTITO mnemonic. For example, electrical and
magnetic forces belong to the same general group, but
Newton's third law would never relate an electrical force
to a magnetic force.

The broadest distinction is that between contact and
noncontact forces, which has been discussed in ch.~\ref{ch:newton}.
Among the contact forces, we distinguish between
those that involve solids only and those that have to do
with fluids, a term used in physics to include both gases
and liquids. 

It should not be necessary to memorize this diagram by rote.
It is better to reinforce your memory of this system by
calling to mind your commonsense knowledge of certain
ordinary phenomena. For instance, we know that the
gravitational attraction between us and the planet earth
will act even if our feet momentarily leave the ground, and
that although magnets have mass and are affected by gravity,
most objects that have mass are nonmagnetic.

\begin{eg}{Hitting a wall}
\egquestion A bullet, flying horizontally, hits a steel wall. What type
of force is there between the bullet and the wall?

\eganswer Starting at the bottom of the tree, we determine that the
force is a contact force, because it only occurs once the bullet touches
the wall. Both objects are solid. The wall forms a vertical
plane. If the nose of the bullet was some shape like a sphere, you might
imagine that it would only touch the
wall at one point. Realistically, however, we know that a lead bullet will flatten out
a lot on impact, so there is a surface of contact between the
two, and its orientation is vertical. The effect of the force on the bullet is to stop the
horizontal motion of the bullet, and this horizontal acceleration must be produced
by a horizontal force. The force is therefore perpendicular to the surface of contact,
and it's also repulsive (tending to keep the bullet from entering the wall), so it must be
a normal force.
\end{eg}

Diagram \figref{force-tree} is meant to be as simple as possible while
including most of the forces we deal with in everyday life.
If you were an insect, you would be much more interested in
the force of surface tension, which allowed you to walk on
water. I have not included the nuclear forces, which are
responsible for holding the nuclei of atoms, because they
are not evident in everyday life.

You should not be afraid to invent your own names for types
of forces that do not fit into the diagram. For instance,
the force that holds a piece of tape to the wall has been
left off of the tree, and if you were analyzing a situation
involving scotch tape, you would be absolutely right to
refer to it by some commonsense name such as ``sticky force.''

On the other hand, if you are having trouble classifying a
certain force, you should also consider whether it is a
force at all. For instance, if someone asks you to classify
the force that the earth has because of its rotation, you
would have great difficulty creating a place for it  on the
diagram. That's because it's a type of motion, not a type of force!

\enlargethispage{-\baselineskip}

<% begin_sec("Normal forces") %>\index{force!normal}

A normal force, $F_N$, is a force that keeps one solid
object from passing through another. ``Normal'' is simply a
fancy word for ``perpendicular,'' meaning that the force is
perpendicular to the surface of contact. Intuitively, it
seems the normal force magically adjusts itself to provide
whatever force is needed to keep the objects from occupying
the same space. If your muscles press your hands together
gently, there is a gentle normal force. Press harder, and
the normal force gets stronger. How does the normal force
know how strong to be? The answer is that the harder you jam
your hands together, the more compressed your flesh becomes.
Your flesh is acting like a spring: more force is required
to compress it more. The same is true when you push on a
wall. The wall flexes imperceptibly in proportion to your
force on it. If you exerted enough force, would it be
possible for two objects to pass through each other? No,
typically the result is simply to strain the objects so much
that one of them breaks.

<% end_sec() %>
<% begin_sec("Gravitational forces") %>\index{force!gravitational}

As we'll discuss in more detail later in the course, a
gravitational force exists between any two things that have
mass. In everyday life, the gravitational force between two
cars or two people is negligible, so the only noticeable
gravitational forces are the ones between the earth and
various human-scale objects. We refer to these planet-earth-induced
gravitational forces as weight forces, and as we have
already seen, their magnitude is given by $|F_W|=mg$.

\worked{waterbottle-in-space}{Weight and mass}

<% end_sec() %>
<% begin_sec("Static and kinetic friction",nil,'coulomb-friction') %>\index{friction!kinetic}\index{friction!static}\index{force!frictional}

If you have pushed a refrigerator across a kitchen floor,
you have felt a certain series of sensations. At first, you
gradually increased your force on the refrigerator, but it
didn't move. Finally, you supplied enough force to unstick
the fridge, and there was a sudden jerk as the fridge
started moving. Once the fridge was unstuck, you could reduce
your force significantly and still keep it moving.

While you were gradually increasing your force, the floor's
frictional force on the fridge increased in response. The
two forces on the fridge canceled, and the fridge didn't
accelerate. How did the floor know how to respond with just
the right amount of force? Figure \figref{friction-microscopic} shows one
possible \emph{model} of friction that explains this
behavior. (A scientific \index{model!scientific}model is a
description that we expect to be incomplete, approximate, or
unrealistic in some ways, but that nevertheless succeeds in
explaining a variety of phenomena.) Figure \figref{friction-microscopic}/1 shows a
microscopic view of the tiny bumps and holes in the surfaces
of the floor and the refrigerator. The weight of the fridge
presses the two surfaces together, and some of the bumps in
one surface will settle as deeply as possible into some of
the holes in the other surface. In
\figref{friction-microscopic}/2, your leftward
force on the fridge has caused it to ride up a little higher
on the bump in the floor labeled with a small arrow. Still
more force is needed to get the fridge over the bump and
allow it to start moving. Of course, this is occurring
simultaneously at millions of places on the two surfaces.

<% marg(170) %>
<%
  fig(
    'friction-microscopic',
    %q{%
      A model that correctly explains many
      properties of friction. The microscopic
      bumps and holes in two surfaces dig
      into each other.
    }
  )
%>
\spacebetweenfigs
<%
  fig(
    'waiter',
    %q{%
      Static friction: the tray doesn't slip on
      the waiter's fingers.
    }
  )
%>
\spacebetweenfigs
<%
  fig(
    'skid',
    %q{Kinetic friction: the car skids.}
  )
%>

<% end_marg %>%
Once you had gotten the fridge moving at constant speed, you
found that you needed to exert less force on it. Since zero
total force is needed to make an object move with constant
velocity, the floor's rightward frictional force on the
fridge has apparently decreased somewhat, making it easier
for you to cancel it out. Our model also gives a plausible
explanation for this fact: as the surfaces slide past each
other, they don't have time to settle down and mesh with one
another, so there is less friction.

\enlargethispage{-\baselineskip}

Even though this model is intuitively appealing and fairly
successful, it should not be taken too seriously, and in
some situations it is misleading. For instance, fancy racing
bikes these days are made with smooth tires that have no
tread --- contrary to what we'd expect from our model, this
does not cause any decrease in friction. Machinists know
that two very smooth and clean metal surfaces may stick to
each other firmly and be very difficult to slide apart. This
cannot be explained in our model, but makes more sense in
terms of a model in which friction is described as arising
from chemical bonds between the atoms of the two surfaces at
their points of contact: very flat surfaces allow more atoms
to come in contact.

Since friction changes its behavior dramatically once the
surfaces come unstuck, we define two separate types of
frictional forces. \index{friction!static}\emph{Static friction}
is friction that occurs between surfaces that are not
slipping over each other. Slipping surfaces experience
\index{friction!kinetic}\emph{kinetic friction}. The forces of static and
kinetic friction, notated $F_s$ and $F_k$, are always
parallel to the surface of contact between the two objects.

<% self_check('static-or-kinetic',<<-'SELF_CHECK'
1. When a baseball player slides in to a base, is the
friction static, or kinetic?

\\noindent 2. A mattress stays on the roof of a slowly accelerating
car. Is the friction static, or kinetic?

\noindent 3. Does static friction create heat? Kinetic friction?
  SELF_CHECK
  ) %>
<% marg(100) %>
<%
  fig(
    'partridge-normal-and-friction',
    %q{%
      Many landfowl, even those that are competent fliers, prefer to escape from a predator by
      running upward rather than by flying. This partridge is running up a vertical tree trunk.
      Humans can't walk up walls because there is no normal force and therefore no frictional force;
      when $F_N=0$, the maximum force of static friction $F_{s,max} = \\mu_s F_N$ is also zero.
      The partridge, however, has wings that it can flap in order to create a force between it and the air. Typically when
      a bird flaps its wings, the resulting force from the air is in the direction that
      would tend to lift the bird up. In this situation, however, the partridge changes its style of flapping so that
      the direction is reversed. The normal force between the feet and
      the tree allows a nonzero static frictional force. The mechanism is similar to that of a spoiler
      fin on a racing car. Some evolutionary biologists believe that when vertebrate flight first evolved, in dinosaurs,
      there was first a stage in which the wings were used only as an aid in running up steep inclines, and only
      later a transition to flight. (Redrawn from a figure by K.P.~Dial.)
    }
  )
%>
<% end_marg %>

\enlargethispage{-\baselineskip}

The maximum possible force of static friction depends on
what kinds of surfaces they are, and also on how hard they
are being pressed together. The approximate mathematical
relationships can be expressed as follows:
\begin{equation*}
        F_{s,max} = \mu_s F_N\eqquad,
\end{equation*}
where $\mu_s$ is a unitless number, called the \index{coefficient
of static friction}coefficient of static friction, which
depends on what kinds of surfaces they are. The maximum
force that static friction can supply, $\mu_s F_N$,
represents the boundary between static and kinetic friction.
It depends on the normal force, which is numerically equal
to whatever force is pressing the two surfaces together. In
terms of our model, if the two surfaces are being pressed
together more firmly, a greater sideways force will be
required in order to make the irregularities in the surfaces
ride up and over each other.

    Note that just because we use an adjective such as
``applied'' to refer to a force, that doesn't mean that
there is some special type of force called the ``applied
force.'' The applied force could be any type of force, or it
could be the sum of more than one force trying to make an object move.

<% self_check('partridge',<<-'SELF_CHECK'
The arrows in figure \figref{partridge-normal-and-friction} show the forces
of the tree trunk on the partridge. Describe the forces the bird makes on the tree.
  SELF_CHECK
  ) %>

\enlargethispage{-\baselineskip}

The force of kinetic friction on each of the two objects is
in the direction that resists the slippage of the surfaces.
Its magnitude is usually well approximated as
\begin{equation*}
        F_k = \mu_k F_N
\end{equation*}
where $\mu_k$ is the \index{coefficient of kinetic
friction}coefficient of kinetic friction. Kinetic friction
is usually more or less independent of velocity.

<%
  fig(
    'friction-graph',
    %q{%
      We choose a coordinate system in
      which the applied force, i.e., the force
      trying to move the objects, is positive.
      The friction force is then negative,
      since it is in the opposite direction. As
      you increase the applied force, the
      force of static friction increases to
      match it and cancel it out, until the
      maximum force of static 
      friction is surpassed. The surfaces then begin
       slipping past each other, and the friction
      force becomes smaller in absolute
      value.
    },
    {
      'width'=>'wide',
      'sidecaption'=>true
    }
  )
%>

<% self_check('friction-and-normal',<<-'SELF_CHECK'
Can a frictionless surface exert a normal force? Can a
frictional force exist without a normal force?
  SELF_CHECK
  ) %>

If you try to accelerate or decelerate your car too quickly,
the forces between your wheels and the road become too
great, and they begin slipping. This is not good, because
kinetic friction is weaker than static friction, resulting
in less control. Also, if this occurs while you are turning,
the car's handling changes abruptly because the kinetic
friction force is in a different direction than the static
friction force had been: contrary to the car's direction of
motion, rather than contrary to the forces applied to the tire.

Most people respond with disbelief when told of the
experimental evidence that both static and kinetic friction
are approximately independent of the amount of surface area
in contact. Even after doing a hands-on exercise with spring
scales to show that it is true, many students are unwilling
to believe their own observations, and insist that bigger
tires ``give more traction.'' In fact, the main reason why
you would not want to put small tires on a big heavy car is
that the tires would burst!

Although many people expect that friction would be
proportional to surface area, such a proportionality would
make predictions contrary to many everyday observations. A
dog's feet, for example, have very little surface area in
contact with the ground compared to a human's feet, and yet
we know that a dog can often win a tug-of-war with a person.

The reason a smaller surface area does not lead to less
friction is that the force between the two surfaces is more
concentrated, causing their bumps and holes to dig into
each other more deeply.

<% self_check('find-directions-of-forces',<<-'SELF_CHECK'
Find the direction of each of the forces in 
figure \\figref{sc-find-directions-of-forces}.
  SELF_CHECK
  ) %>

<%
  fig(
    'sc-find-directions-of-forces',
    %q{%
      1. The
      cliff's normal force on the climber's feet.
      2. The track's static frictional force on the wheel of
      the accelerating dragster.
      3. The ball's normal force on the bat.
    },
    {
      'width'=>'wide',
      'sidecaption'=>true
    }
  )
%>

\begin{eg}{Locomotives}\label{eg:locomotives}
Looking at a picture of a locomotive, \figref{locomotive}, we notice
two obvious things that are different from an automobile. Where a car
typically has two drive wheels, a locomotive normally has many --- ten
in this example. (Some also have smaller, unpowered wheels in front of and
behind the drive wheels, but this example doesn't.) Also, cars these days are generally built to be as light as possible
for their size, whereas locomotives are very massive, and no effort seems to be made
to keep their weight low. (The steam locomotive in the photo is from about 1900, but this
is true even for modern diesel and electric trains.)

<% fig('locomotive','Example \ref{eg:locomotives}.',
    {
      'width'=>'wide',
      'sidecaption'=>true
    }
) %>

The reason locomotives are built to be so heavy is for traction. The upward normal force of
the rails on the wheels, $F_N$, cancels the downward force of gravity, $F_W$, so ignoring plus and minus
signs, these two forces are equal in absolute value, $F_N=F_W$. Given this amount of normal
force, the maximum force of static friction is $F_s=\mu_s F_N=\mu_s F_W$. This static frictional
force, of the rails pushing forward on the wheels, is the only force that can accelerate the
train, pull it uphill, or cancel out the force of air resistance while cruising at constant speed.
The coefficient of static friction for steel on steel is about 1/4, so no locomotive can pull
with a force greater than about 1/4 of its own weight. If the engine is capable of supplying
more than that amount of force, the result will be simply to break static friction and spin the wheels.

The reason this is all so different from the situation with a car is that a car isn't pulling
something else. If you put extra weight in a car, you improve the traction, but you also
increase the inertia of the car, and make it just as hard to accelerate. In a train, the inertia
is almost all in the cars being pulled, not in the locomotive.

The other fact we have to explain is the large number of driving wheels. First, we have to
realize that increasing the number of driving wheels neither increases nor decreases the
total amount of static friction, because static friction is independent of the amount of
surface area in contact. (The reason four-wheel-drive is good in a car is that if one or
more of the wheels is slipping on ice or in mud, the other wheels may still have traction.
This isn't typically an issue for a train, since all the wheels experience the same conditions.)
The advantage of having more driving wheels on a train is that it allows us to increase the weight of the locomotive
without crushing the rails, or damaging bridges.
\end{eg}

<% end_sec() %>
<% begin_sec("Fluid friction") %>\index{friction!fluid}

<% marg(100) %>
<%
  fig(
    'fluid-patterns',
    %q{%
      Fluid friction depends on the fluid's pattern of flow, so it is more complicated than friction between solids,
      and there are no simple, universally applicable formulas to calculate it. From top to bottom: supersonic wind
      tunnel, vortex created by a crop duster, series of vortices created by a single object, turbulence.
    }
  )
%>
<% end_marg %>

Try to drive a nail into a waterfall and you will be
confronted with the main difference between solid friction
and fluid friction. Fluid friction is purely kinetic; there
is no static fluid friction. The nail in the waterfall may
tend to get dragged along by the water flowing past it, but
it does not stick in the water. The same is true for gases
such as air: recall that we are using the word ``fluid'' to
include both gases and liquids.

Unlike kinetic friction between solids, fluid friction
increases rapidly with velocity. It also depends
on the shape of the object, which is why a fighter jet is
more streamlined than a Model T.
For objects of the same shape but different sizes, fluid friction
typically scales up with the cross-sectional area of the
object, which is one of the main reasons that an SUV gets
worse mileage on the freeway than a compact car.

\pagebreak

\startdqs
<% marg(80) %>
<%
  fig(
    'fluid-dimples',
    %q{%
      What do the golf ball and the shark have in common? Both use the same trick to reduce fluid friction.
      The dimples on the golf ball modify the pattern of flow of the air around it, counterintuitively
      \emph{reducing} friction. Recent studies have shown that sharks can accomplish the same thing by
      raising, or ``bristling,'' the scales on their skin at high speeds.
    }
  )
%>
\spacebetweenfigs
<%
  fig(
    'hummer-vs-prius',
    %q{%
      The wheelbases of the Hummer H3 and the Toyota Prius are surprisingly similar,
      differing by only 10\%. The main difference in shape is that the Hummer is much taller and wider.
      It presents a much greater cross-sectional area to the wind, and this is the main
      reason that it uses about 2.5 times more gas on the freeway.
    }
  )
%>
<% end_marg %>

\begin{dq}
A student states that when he tries to push his
refrigerator, the reason it won't move is because Newton's
third law says there's an equal and opposite frictional
force pushing back. After all, the static friction force is
equal and opposite to the applied force. How would you
convince him he is wrong?
\end{dq}

\begin{dq}
Kinetic friction is usually more or less independent of
velocity. However, inexperienced drivers tend to produce a
jerk at the last moment of deceleration when they stop at a
stop light. What does this tell you about the kinetic
friction between the brake shoes and the brake drums?
\end{dq}

\begin{dq}
Some of the following are correct descriptions of types
of forces that could be added on as new branches of the
classification tree. Others are not really types of forces,
and still others are not force phenomena at all. In each
case, decide what's going on, and if appropriate, figure out
how you would incorporate them into the tree.

\begin{tabular}{lp{80mm}}
sticky force       & makes tape stick to things\\
 %
opposite force      & the force that Newton's third law says
                       relates to every force you make\\
 %
flowing force      & the force that water carries with it as
it flows out of a hose\\
 %
surface tension    & lets insects walk on water\\
 %
horizontal force   &  a force that is horizontal\\
 %
motor force        & the force that a motor makes on the thing it is turning\\
 %
canceled force     & a force that is being canceled out by some other force
\end{tabular}
\end{dq}

<% end_sec() %>
<% end_sec() %>

\vspace{20mm}

<% begin_sec("Analysis of forces",0,'analysis-of-forces') %>\index{force!analysis of forces}

Newton's first and second laws deal with the total of all
the forces exerted on a specific object, so it is very
important to be able to figure out what forces there are.
Once you have focused your attention on one object and
listed the forces on it, it is also helpful to describe all
the corresponding forces that must exist according to
Newton's third law. We refer to this as ``analyzing the
forces'' in which the object participates.

\pagebreak

\begin{egwide}{A barge}
A barge is being pulled to the right along a canal by teams of horses on
the shores. Analyze all the forces in which the barge participates.

\begin{tabular}{|p{70mm}|p{70mm}|}
\hline
\emph{force acting on barge}  &   \emph{force related to it by Newton's third law} \\
\hline
ropes' normal forces on barge, $\rightarrow$  &  barge's normal force on ropes, $\leftarrow$\\
\hline
water's fluid friction force on barge, $\leftarrow$  &  barge's fluid friction force on water, $\rightarrow$\\
\hline
planet earth's gravitational force on barge, $\downarrow$  &  barge's gravitational force on earth, $\uparrow$\\
\hline
water's ``floating'' force on barge, $\uparrow$  &  barge's ``floating'' force on water, $\downarrow$\\
\hline
\end{tabular}

Here I've used the word ``floating'' force as an example of
a sensible invented term for a type of force not classified
on the tree on p.~\pageref{fig:force-tree}. A more formal technical
term would be ``hydrostatic force.''

Note how the pairs of forces are all structured as ``A's
force on B, B's force on A'': ropes on barge and barge on
ropes; water on barge and barge on water. Because all the
forces in the left column are forces acting on the barge,
all the forces in the right column are forces being exerted
by the barge, which is why each entry in the column
begins with ``barge.''
\end{egwide}

Often you may be unsure whether you have forgotten one of
the forces. Here are three strategies for checking your list:

\begin{enumerate}\label{checking-force-analysis}
\item See what physical result would come from the forces
you've found so far. Suppose, for instance, that you'd
forgotten the ``floating'' force on the barge in the example
above. Looking at the forces you'd found, you would have
found that there was a downward gravitational force on the
barge which was not canceled by any upward force. The barge
isn't supposed to sink, so you know you need to find a
fourth, upward force.

\item Another technique for finding missing forces is simply
to go through the list of all the common types of forces and
see if any of them apply.

\item Make a drawing of the object, and draw a dashed boundary
line around it that separates it from its environment. Look
for points on the boundary where other objects come in
contact with your object. This strategy guarantees that
you'll find every contact force that acts on the object,
although it won't help you to find non-contact forces.

\end{enumerate}

\begin{eg}{Fifi}\label{eg:fifi}
\egquestion Fifi is an industrial espionage dog who loves doing her job and looks great doing it.
She leaps through a window and lands at initial horizontal speed $v_\zu{o}$ on a conveyor belt which is itself
moving at the greater speed $v_b$. Unfortunately the coefficient of kinetic friction $\mu_k$ between her foot-pads
and the belt is fairly low, so she skids for a time $\Delta t$, during which the effect on her coiffure is \emph{un d\'{e}sastre}.
Find $\Delta t$.
<% marg(0) %>
<%
  fig(
    'eg-fifi',
    %q{%
      Example \ref{eg:fifi}.
    }
  )
%>
<% end_marg %>

\pagebreak

\eganswer We analyze the forces:

\begin{tabular}{|p{52mm}|p{52mm}|}
\hline
\emph{force acting on Fifi}  &   \emph{force related to it by Newton's third law} \\
\hline
planet earth's gravitational force $F_W=mg$ on Fifi, \hfill $\downarrow$  &  Fifi's gravitational force on earth, \hfill $\uparrow$\\
\hline
belt's kinetic frictional force $F_k$ on Fifi, \hfill $\rightarrow$  &  Fifi's kinetic frictional force on belt, \hfill $\leftarrow$\\
\hline
belt's normal force $F_N$ on Fifi, \hfill $\uparrow$  &  Fifi's normal force on belt, \hfill $\downarrow$\\
\hline
\end{tabular}

Checking the analysis of the forces as described on p.~\pageref{checking-force-analysis}:

(1) The physical result makes sense. The left-hand column consists of forces $\downarrow\rightarrow\uparrow$. 
We're describing the time when she's moving horizontally on the belt, so it makes sense that we have two vertical forces that could cancel.
The rightward force is what will accelerate her until her speed matches that of the belt.

(2) We've included every relevant type of force from the tree on p.~\pageref{fig:force-tree}.

(3) We've included forces from the belt, which is the only object in contact with Fifi.

The purpose of the analysis is to let us set up equations containing enough information to solve the problem.
Using the generalization of Newton's second law given on p.~\pageref{generalization-of-second-law}, we use
the horizontal force to determine the horizontal acceleration, and separately require the vertical forces
to cancel out.

Let positive $x$ be to the right. Newton's second law gives
\begin{equation*}
  (\rightarrow) \qquad a=F_k/m
\end{equation*}

Although it's the horizontal motion we care about, the only way to find $F_k$ is via
the relation $F_k=\mu_k F_N$, and the only way to find $F_N$ is from the $\uparrow\downarrow$ forces.
The two vertical forces must cancel, which means they have to be of equal strength:
\begin{equation*}
  (\uparrow\downarrow) \qquad F_N - mg = 0\eqquad.
\end{equation*}
Using the constant-acceleration equation $a=\Delta v/\Delta t$, we have
\begin{align*}
  \Delta t &= \frac{\Delta v}{a} \\
           &= \frac{v_b-v_\zu{o}}{\mu_k mg/m}\\
           &= \frac{v_b-v_\zu{o}}{\mu_k g}\eqquad.
\end{align*}

The units check out:
\begin{equation*}
  \sunit = \frac{\munit/\sunit}{\munit/\sunit^2}\eqquad,
\end{equation*}
where $\mu_k$ is omitted as a factor because it's unitless.

We should also check that the dependence on the variables makes sense. 
If Fifi puts on her rubber ninja booties, increasing $\mu_k$, then dividing by a larger number gives a smaller result for $\Delta t$;
this makes sense physically, because the greater friction will cause her to come up to the belt's speed more quickly.
The dependence on $g$ is similar; more gravity would press her harder against the belt, improving her traction.
Increasing $v_b$ increases $\Delta t$, which makes sense because it will take her longer to get up to a bigger speed.
Since $v_\zu{o}$ is subtracted, the dependence of $\Delta t$ on it is the other way around, and that makes sense too,
because if she can land with a greater speed, she has less speeding up left to do.
\end{eg}


<%
  fig(
    'horses-triangle',
    %q{%
      Example \ref{eg:horses-triangle}.
    },
    {
      'width'=>'wide',
      'sidecaption'=>true,
      'sidepos'=>'b'
    }
  )
%>


\begin{eg}{Forces don't have to be in pairs or at right angles}\label{eg:horses-triangle}
In figure \figref{horses-triangle}, the three horses are arranged symmetrically at 120 degree
intervals, and are all pulling on the central knot. Let's say the knot is at rest and at least momentarily
in equilibrium. The analysis of forces on the knot is as follows.

\begin{tabular}{|p{52mm}|p{52mm}|}
\hline
\emph{force acting on knot}  &   \emph{force related to it by Newton's third law} \\
\hline
top rope's normal force on knot, \hfill $\uparrow$  &  knot's normal force on top rope, \hfill $\downarrow$\\
\hline
left rope's normal force on knot, \hfill \anonymousinlinefig{../../../share/misc/arrows/8-oclock}
        &  knot's normal force on left rope, \hfill \anonymousinlinefig{../../../share/misc/arrows/2-oclock}\\
\hline
right rope's normal force on knot, \hfill \anonymousinlinefig{../../../share/misc/arrows/4-oclock}  
        &  knot's normal force on right rope, \hfill \anonymousinlinefig{../../../share/misc/arrows/10-oclock}\\
\hline
\end{tabular}

In our previous examples, the forces have all run along two perpendicular lines, and
they often canceled in pairs. This example shows that neither of these always happens.
Later in the book we'll see how to handle forces that are at arbitrary angles,
using mathematical objects called vectors. But even without knowing about vectors, we
already know what directions to draw the arrows in the table, since a rope can only pull
parallel to itself at its ends. And furthermore, we can say something about the forces:
by symmetry, we expect them all to be equal in strength. (If the knot was not in equilibrium,
then this symmetry would be broken.)

This analysis also demonstrates that it's all right to leave out details if they aren't
of interest and we don't intend to include them in our model. 
We called the forces normal forces, but we can't actually tell whether they are normal forces
or frictional forces. They are probably some combination of those, but we don't include
such details in this model, since aren't interested in describing the internal physics
of the knot. This is an example of a more general fact about science, which is that science
doesn't describe reality. It describes simplified \emph{models} of reality, because reality
is always too complex to model exactly.
\end{eg}

\pagebreak

\startdqs

\begin{dq}
In the example of the barge going down the canal, I
referred to a ``floating'' or ``hydrostatic'' force that
keeps the boat from sinking. If you were adding a new branch
on the force-classification tree to represent this
force, where would it go?
\end{dq}

\begin{dq}\label{dq:shovel}
The earth's gravitational force on you, i.e., your weight,
is always equal to $mg$, where $m$ is your mass. So why
can you get a shovel to go deeper into the ground by jumping
onto it? Just because you're jumping, that doesn't mean your
mass or weight is any greater, does it?
\end{dq}

<% end_sec() %>
<% begin_sec("Transmission of forces by low-mass objects",nil,'transmitforces') %>%
\index{force!transmission}\index{transmission of forces}

You're walking your dog. The dog wants to go faster than you
do, and the leash is taut. Does Newton's third law guarantee
that your force on your end of the leash is equal and
opposite to the dog's force on its end? If they're not
exactly equal, is there any reason why they should be
approximately equal?

If there was no leash between you, and you were in direct
contact with the dog, then Newton's third law would apply,
but Newton's third law cannot relate your force on the leash
to the dog's force on the leash, because that would involve
three separate objects. Newton's third law only says that
your force on the leash is equal and opposite to the
leash's force on you,
\begin{equation*}
        F_{yL}  =  - F_{Ly}  ,
\end{equation*}

and that the dog's force on the leash is equal and opposite
to its force on the dog
\begin{equation*}
        F_{dL}  =  - F_{Ld}  .
\end{equation*}
Still, we have a strong intuitive expectation that whatever
force we make on our end of the leash is transmitted to the
dog, and vice-versa. We can analyze the situation by
concentrating on the forces that act on the leash, $F_{dL}$
and $F_{yL}$. According to Newton's second law, these relate
to the leash's mass and acceleration:
\begin{equation*}
        F_{dL} + F_{yL}  =  m_La_L  .
\end{equation*}
The leash is far less massive then any of the other objects
involved, and if $m_L$ is very small, then apparently the
total force on the leash is also very small, $F_{dL}$ +
$F_{yL}\approx 0$, and therefore
\begin{equation*}
 F_{dL}\approx - F_{yL}\eqquad.
\end{equation*}
Thus even though Newton's third law does not apply directly
to these two forces, we can approximate the low-mass leash
as if it was not intervening between you and the dog. It's
at least approximately as if you and the dog were acting
directly on each other, in which case Newton's third
law would have applied.

In general, low-mass objects can be treated approximately as
if they simply transmitted forces from one object to
another. This can be true for strings, ropes, and cords, and
also for rigid objects such as rods and sticks.

<%
  fig(
    'tension',
    %q{%
      If we imagine dividing a taut
      rope up into small segments, then any segment has forces
      pulling outward on it at each end. If the rope is of
      negligible mass, then all the forces equal
      $+T$ or $-T$, where $T$, the tension, is a single number.
    },
    {
      'width'=>'wide'
    }
  )
%>

<% marg(-80) %>
<%
  fig(
    'golden-gate-bridge',
    %q{%
      The Golden Gate Bridge's roadway
      is held up by the tension in the vertical cables.
    }
  )
%>
<% end_marg %>
If you look at a piece of string under a magnifying glass as
you pull on the ends more and more strongly, you will see
the fibers straightening and becoming taut. Different parts
of the string are apparently exerting forces on each other.
For instance, if we think of the two halves of the string as
two objects, then each half is exerting a force on the other
half. If we imagine the string as consisting of many small
parts, then each segment is transmitting a force to the next
segment, and if the string has very little mass, then all
the forces are equal in magnitude. We refer to the magnitude
of the forces as the tension in the string, $T$.\index{tension}

The term ``tension'' refers only to internal forces within the string.
If the string makes
forces on objects at its ends, then those forces are typically
normal or frictional forces (example \ref{eg:types-of-forces-made-by-ropes}).

\pagebreak

\begin{eg}{Types of force made by ropes}\label{eg:types-of-forces-made-by-ropes}
\egquestion Analyze the forces in figures \subfigref{tension-is-not-a-type-of-force}{1}
and \subfigref{tension-is-not-a-type-of-force}{2}.

\eganswer
In all cases, a rope can only make ``pulling'' forces, i.e., forces that are parallel to its own length
and that are toward itself, not away from itself. You can't push with a rope!

In \subfigref{tension-is-not-a-type-of-force}{1}, the rope passes through a type of hook, called a
carabiner, used in rock climbing and mountaineering. Since the rope can only pull along its own length,
the direction of its force on the carabiner must be down and to the right. This is perpendicular to
the surface of contact, so the force is a normal force.
<% marg(80) %>
<%
  fig(
    'tension-is-not-a-type-of-force',
    %q{%
      Example \ref{eg:types-of-forces-made-by-ropes}. The forces between the rope and other objects are normal and frictional forces.
    }
  )
%>
<% end_marg %>

\begin{tabular}{|p{52mm}|p{52mm}|}
\hline
\emph{force acting on carabiner}  &   \emph{force related to it by Newton's third law} \\
\hline
rope's normal force on carabiner \hfill \anonymousinlinefig{../../../share/misc/arrows/5-oclock} &
          carabiner's normal force on rope \hfill \anonymousinlinefig{../../../share/misc/arrows/11-oclock} \\
\hline
\end{tabular}

(There are presumably other forces acting on the carabiner from other hardware above it.)

In figure \subfigref{tension-is-not-a-type-of-force}{2}, the rope can only exert a net force at its end that is
parallel to itself and in the pulling direction, so its force on the hand is down and to the left.
This is parallel
to the surface of contact, so it must be a frictional force. If the rope isn't slipping through the hand,
we have static friction. Friction can't exist without normal forces. These forces are perpendicular to the
surface of contact.
For simplicity, we show only two pairs of these normal forces, as if the hand were a pair of pliers.

\begin{tabular}{|p{52mm}|p{52mm}|}
\hline
\emph{force acting on person}  &   \emph{force related to it by Newton's third law} \\
\hline
rope's static frictional force on person \hfill \anonymousinlinefig{../../../share/misc/arrows/8-oclock} &
          person's static frictional force on rope \hfill \anonymousinlinefig{../../../share/misc/arrows/2-oclock} \\
\hline
rope's normal force on \linebreak[4] person \hfill \anonymousinlinefig{../../../share/misc/arrows/11-oclock} &
          person's normal force on \linebreak[4]  rope \hfill \anonymousinlinefig{../../../share/misc/arrows/5-oclock} \\
\hline
rope's normal force on \linebreak[4] person \hfill \anonymousinlinefig{../../../share/misc/arrows/5-oclock} &
          person's normal force on \linebreak[4] rope \hfill \anonymousinlinefig{../../../share/misc/arrows/11-oclock} \\
\hline
\end{tabular}

(There are presumably other forces acting on the person as well, such as gravity.)

\end{eg}

If a rope goes over a pulley or around some other object,
then the tension throughout the rope is approximately equal
so long as the pulley has negligible mass and there is not too much friction. A rod or stick
can be treated in much the same way as a string, but it is
possible to have either compression or tension.

\startdq

\begin{dq}
When you step on the gas pedal, is your foot's force being
transmitted in the sense of the word used in this section?
\end{dq}

<% end_sec() %>
<% begin_sec("Objects under strain",0) %>\index{strain}

A string lengthens slightly when you stretch it. Similarly,
we have already discussed how an apparently rigid object
such as a wall is actually flexing when it participates in a
normal force. In other cases, the effect  is more obvious. A
spring or a rubber band visibly elongates when stretched.

Common to all these examples is a change in shape of some
kind: lengthening, bending, compressing, etc. The change in
shape can be measured by picking some part of the object and
measuring its position, $x$. For concreteness, let's imagine
a spring with one end attached to a wall. When no force is
exerted, the unfixed end of the spring is at some position
$x_o$. If a force acts at the unfixed end, its position will
change to some new value of $x$. The more force, the greater
the departure of $x$ from $x_o$.

<%
  fig(
    'hooke-definitions',
    %q{%
      Defining the quantities
      $F$, $x$, and $x_\zu{o}$ in Hooke's law.
    },
    {
      'width'=>'wide',
      'sidecaption'=>true
    }
  )
%>

Back in Newton's time, experiments like this were considered
cutting-edge research, and his contemporary Hooke is
remembered today for doing them and for coming up with a
simple mathematical generalization called Hooke's law:
\begin{multline*}
    F \approx k(x-x_o)\eqquad. \hfill \shoveright{\text{[force required to stretch a spring; valid}}\\
                                      \text{for small forces only]}
\end{multline*}
Here $k$ is a constant, called the spring constant,\index{spring constant}
that depends on how stiff the object is. If too much force is
applied, the spring exhibits more complicated behavior, so
the equation is only a good approximation if the force is
sufficiently small. Usually when the force is so large that
Hooke's law is a bad approximation, the force ends up
permanently bending or breaking the spring.

Although \index{Hooke's law}Hooke's law may seem like a
piece of trivia about springs, it is actually far more
important than that, because all solid objects exert
Hooke's-law behavior over some range of sufficiently small
forces. For example, if you push down on the hood of a car,
it dips by an amount that is directly proportional to the
force. (But the car's behavior would not be as mathematically
simple if you dropped a boulder on the hood!)

\worked{combine-springs}{Combining springs}

\worked{youngs-modulus}{Young's modulus}

\startdq

\begin{dq}
A car is connected to its axles through big, stiff springs
called shock absorbers, or ``shocks.'' Although we've
discussed Hooke's law above only in the case of stretching a
spring, a car's shocks are continually going through both
stretching and compression. In this situation, how would you
interpret the positive and negative signs  in Hooke's law?
\end{dq}

<% end_sec() %>
<% begin_sec("Simple Machines: the pulley",0,'pulley') %>\index{pulley}

Even the most complex machines, such as cars or pianos, are
built out of certain basic units called \index{simple machine!defined}
\emph{simple machines}. The following are some of
the main functions of simple machines:

\begin{indentedblock}
\noindent transmitting a force: The chain on a bicycle transmits a
force from the crank set to the rear wheel.

\noindent changing the direction of a force: If you push down on a
seesaw, the other end goes up.

\noindent changing the speed and precision of motion: When you make
the ``come here'' motion, your biceps only moves a couple of
centimeters where it attaches to your forearm, but your arm
moves much farther and more rapidly.

\noindent changing the amount of force: A lever or pulley can be used
to increase or decrease the amount of force.
\end{indentedblock}

\noindent You are now prepared to understand one-dimensional simple
machines, of which the pulley is the main example.

<%
  fig(
    'eg-tractor',
    %q{Example \ref{eg:tractor}.},
    {
      'width'=>'wide',
      'sidecaption'=>true
    }
  )
%>

\begin{eg}{A pulley}\label{eg:tractor}
\egquestion Farmer Bill says this pulley arrangement doubles
the force of his tractor. Is he just a dumb hayseed, or does
he know what he's doing?

\eganswer To use Newton's first law, we need to pick an
object and consider the sum of the forces on it. Since our
goal is to relate the tension in the part of the cable
attached to the stump to the tension in the part attached to
the tractor, we should pick an object to which both those
cables are attached, i.e., the pulley itself. The tension in a string or cable remains
approximately constant as it passes around an idealized pulley.
\footnote{This was asserted in section \ref{sec:transmitforces} without proof.
      Essentially it holds because of symmetry. E.g., if the U-shaped piece of rope in figure \figref{eg-tractor}
      had unequal tension in its two legs, then this would have to be caused by some asymmetry between clockwise and counterclockwise
      rotation. But such an asymmetry can only be caused by friction or inertia, which we assume don't exist.}
 There are
therefore two leftward forces acting on the pulley, each
equal to the force exerted by the tractor. Since the
acceleration of the pulley is essentially zero, the forces
on it must be canceling out, so the rightward force of the
pulley-stump cable on the pulley must be double the force
exerted by the tractor. Yes, Farmer Bill knows what he's talking about.
\end{eg}

m4_ifelse(__me,1,[:
More complicated pulley systems can be constructed to give greater
amplification of forces or to redirect forces in different directions.
For an idealized system,\footnote{In such a system: (1) The ropes and pulleys have negligible mass. (2)
Friction in the pulleys' bearings is negligible. (3) The ropes don't stretch.} the fundamental principles are:

\begin{enumerate}\label{pulley-rules}
\item The total force acting on any pulley is zero.\footnote{$F=ma$, and $m=0$ since the pulley's mass is assumed to be negligible.}
\item The tension in any given piece of rope is
      constant throughout its length.
\item The length of every piece of rope remains the same.
\end{enumerate}

\begin{eg}{A compound pulley}\label{eg:pulleys-three-quarters}
\egquestion Find the mechanical advantage $T_5/F$ of the pulley system. The bar is massless.

\eganswer By rule 2, $T_1=T_2$, and by rule 1, $F=T_1+T_2$, so $T_1=T_2=F/2$. Similarly, $T_3=T_4=F/4$.
Since the bar is massless, the same reasoning that led to rule 1 applies to the bar as well, and $T_5=T_1+T_3$.
The mechanical advantage is $T_5/F=3/4$, i.e., this pulley system \emph{reduces} the input force.
\end{eg}

<% marg(30) %>
<%
  fig(
    'eg-pulleys-three-quarters',
    %q{%
      Example \ref{eg:pulleys-three-quarters}.
    }
  )
%>
<% end_marg %>

\begin{eg}{How far does the tractor go compared to the stump?}\label{eg:tractor-motion}
\egquestion To move the stump in figure \figref{eg-tractor} by 1 cm, how far must the tractor move?

\eganswer Applying rule 3 to the the right-hand piece of rope, we find that the pulley moves 1 cm. The upper leg of the U-shaped
rope therefore shortens by 1 cm, so the lower leg must lengthen by 1 cm. Since the pulley moves 1 cm to the left, and the lower
leg extending from it also lengths by 1 cm, the tractor must move 2 cm.

\end{eg}

Examples \ref{eg:tractor} and \ref{eg:tractor-motion} showed that
the pulley system in figure \figref{eg-tractor} amplifies the force by a factor of 2, but it reduces the motion by 1/2. This is an example
of a more general inverse proportionality for all such systems. Superficially, it follows from rules 1-3 above.
If, for example, we try to construct a pulley system that doubles the force while keeping the motion the same, we will find that
the rules seem to mysteriously conspire against us, and every attempt ends in failure. We could in fact prove as a mathematical
theorem that the inverse proportionality always holds if we assume these rules. 

But these rules are only an idealized mathematical
model of a specific type of simple machine. What about other machines built out of other parts such as levers, screws, or gears?
Through trial and error we will find that the inverse proportionality holds for them as well, so there must be some more fundamental principles involved.
These principles, which we won't discuss formally until
ch.~\ref{ch:energy} and \ref{ch:work}, are conservation of energy and the equation for mechanical work.
Informally, imagine that we had a machine that violated this rule. We could then insert it into a
setup like the one in figure \figref{impossible-machine}. When we release the single weight at the top, it drops
to the ground while lifting the pan, which holds double the weight, all the way to the top. This is the ultimate
free lunch. Once the pair of weights is up at the top, we can use them to hoist four more, then 8, 16, and so on.
This is known as a perpetual motion machine.\index{perpetual motion machine}
<% marg(58) %>
<%
  fig(
    'impossible-machine',
    %q{%
      The black box marked with an X is a machine that doubles force while leaving the amount of motion unchanged.
      If 1 cm of rope is pulled out through the input on the bottom at tension $T$, the amount of rope consumed at tension
      $2T$ on top is not 1/2 cm, as we would normally expect, but 1 cm. This machine is impossible.
    }
  )
%>
<% end_marg %>

If this seems to be too good to be true, it is.
Just as small machines can be put together to make bigger ones, any machine can also be broken down into smaller
and smaller ones. This process can be continued until we get down to the level of atoms.
The law of conservation of energy essentially says that atoms don't act like perpetual motion machines, and therefore
any machine built out of atoms also fails to be a perpetual motion machine.
:])

<% end_sec() %>

m4_ifelse(__me,1,[:
%--------------------- begin "Does Newton's third law mean anything, and if so, is it true?"
<% begin_sec("Does Newton's third law mean anything, and if so, is it true?",4,'third-law-true',{'optional'=>true}) %>
This section discusses Newton's third law in the same spirit as section \ref{sec:newton-true} on the first and second laws.
<% marg(50) %>
<%
  fig(
    'mach-portrait',
    %q{%
      Ernst Mach (1838-1916) is mainly known for having proposed a radical extension of the principle of inertia to state that
      all motion, not just constant-velocity motion, was relative. His ideas strongly influenced Einstein. The Mach factor
      (used, e.g., when we describe a jet as traveling at ``Mach 2'') is named after him.
    }
  )
%>
<% end_marg %>

Ernst Mach gave a cogent critique of the third laws's logical assumptions in his book \emph{The Science of Mechanics}.
The book is available online for free at \url{archive.org}, and is very readable. 
To understand Mach's criticism, consider the experiment illustrated in figure \figref{third-law-magnets} on p.~\pageref{fig:third-law-magnets},
in which a large magnet and a small magnet are found to exert equal forces on one another. I use this as a student lab, and I find
that most students are surprised by the result. Nevertheless, the lab can be considered a swindle, for the following reason.
If we wanted to, we could cut the large magnet apart into smaller pieces, each of which was the same size as the small
magnet. In fact, the large magnets I use for this lab were constructed simply by taking six small ones, stacking them together,
and wrapping them in plastic. To represent this symbolically, let the small magnet be [A] and the large one [BCDEFG].
Since A and B are identical, and they are oriented in the same way, it follows simply by symmetry that A's force on B
and B's on A obey the third law. The same holds for A on C and C on A, and so on. Since Newton claims that forces
combine by addition, it follows that the result of the experiment must be in accord with the third law, despite the
superficial asymmetry. 

Now suppose that material objects 1 and 2 have the same chemical composition. By a similar argument it seems likely that
$F_{12}$ and $F_{21}$ obey Newton's third law.

This argument shows how pointless it can be to attempt to test a scientific theory unless you have in your possession a sensible
alternative theory that predicts something different. One could spend decades doing experiments of the kind described
above without realizing that the tests were all trivially guaranteed to give null results, even if nature was really
described by a theory that violated Newton's third law.

Here is an example of a fairly sane theory that could violate Newton's third law. Einstein's famous $E=mc^2$
states that a certain amount of energy $E$ is equivalent to a certain amount of mass $m$, with $c$ being the speed of
light. (We won't formally encounter energy until ch.~\ref{ch:energy}, or the reasons for $E=mc^2$ until section \ref{sec:mass-energy-equivalence},
but for now just think of energy as the kind of thing you intuitively associate with food calories or a tank full of gasoline,
and take $E=mc^2$ for granted.) Einstein claimed that this would hold for three different kinds of mass: the mass measured
by an object's inertia, the ``active'' gravitational mass $m_a$ that determines the gravitational forces it makes on other
objects, and the ``passive'' gravitational mass $m_p$ that measures how strongly it feels gravity. Einstein's reason for
predicting similar behavior for $m_a$ and $m_p$ was that anything else would have violated Newton's third law for gravitational
forces.

Suppose instead
that an object's energy content contributes only to $m_p$, not to $m_a$. 
Atomic nuclei get something like 1\% of their mass from the energy of the electric fields inside their nuclei,
but this percentage varies with the number of protons, so if we have objects $m$ and $M$ with different chemical
compositions, it follows that in this theory $m_p/m_a$ will not be the same as $M_p/M_a$, and in this non-Einsteinian version of relativity,
Newton's third law is violated.

This was tested in
a Princeton PhD-thesis experiment by Kreuzer\footnote{Kreuzer, Phys. Rev. 169 (1968) 1007}
in 1966. Kreuzer carried out an experiment, figure \figref{kreuzer}, using masses made of two different substances. The first substance was teflon.
The second substance was
a mixture of the liquids trichloroethylene and dibromoethane, with the proportions chosen so as to give a passive-mass
density as close as possible to that of teflon, as determined by the neutral buoyancy of the teflon masses suspended inside the liquid.
If the active-mass densities of these substances are not strictly proportional to their passive-mass densities, then moving the chunk of
teflon back and forth in figure \subfigref{kreuzer}{2} would change the gravitational force acting on the nearby small sphere.
No such change was observed, and
the results verified $m_p/m_a=M_p/M_a$ to within one part in $10^6$, in agreement with Einstein and Newton. If electrical energy had not contributed at all to
active mass, then a violation of the third law would have been detected at the level of about one part in $10^2$.
% my x in genrel book is deviation from unit contribution by E field to m_a; x is limited to about 10^-4, but E field only makes up only about 10^-2 of rest mass

<% marg(m4_ifelse(__me,1,85,50)) %>
<%
  fig(
    'kreuzer',
    %q{1. A balance that measures the gravitational attraction between masses $M$ and $m$. (See section \ref{sec:weighing-the-earth} for
          a more detailed description.) When the two masses $M$ are inserted, the fiber twists.
       2. A simplified diagram of Kreuzer's modification. The moving teflon mass is submerged in a liquid with nearly the same density.
       3. Kreuzer's actual apparatus.}
  )
%>
<% end_marg %>

The Kreuzer result was improved in 1986 by Bartlett and van Buren\footnote{Phys. Rev. Lett. 57 (1986) 21}
using lunar laser ranging data similar to those described in section \ref{sec:newton-true}. 
Since the moon
has an asymmetrical distribution of iron and aluminum, a theory with $m_p/m_a \ne M_p/M_a$ would cause it to have an anomalous acceleration along
a certain line. The lack of any such observed acceleration limits violations of
Newton's third law to about one part in $10^{10}$.
% not just 1/x=10^8, for same reasons as in the other comment above


<% end_sec %> % Does Newton's third law mean anything, and if so, is it true?
%--------------------- "Do Newton's laws mean anything?
:])


\begin{summary}

\begin{vocab}

\vocabitem{repulsive}{describes a force that tends to push the two
participating objects apart}

\vocabitem{attractive}{describes a force that tends to pull the two
participating objects together}

\vocabitem{oblique}{describes a force that acts at some other angle, one
that is not a direct repulsion or attraction}

\vocabitem{normal force}{the force that keeps two objects from
occupying the same space}

\vocabitem{static friction}{a friction force between surfaces that are
not slipping past each other}

\vocabitem{kinetic friction}{a friction force between surfaces that are
slipping past each other}

\vocabitem{fluid}{a gas or a liquid}

\vocabitem{fluid friction}{a friction force in which at least one of the
object is is a fluid}

\vocabitem{spring constant}{the constant of proportionality between
force and elongation of a spring or other object under strain}

\end{vocab}

\begin{notation}

\notationitem{$F_N$}{a normal force}

\notationitem{$F_s$}{a static frictional force}

\notationitem{$F_k$}{a kinetic frictional force}


\notationitem{$\mu_s$}{the coefficient of static friction; the constant of
proportionality between the maximum static frictional force
and the normal force; depends on what types of surfaces are involved}

\notationitem{$\mu_k$}{the coefficient of kinetic friction; the constant of
proportionality between the kinetic frictional force and the
normal force; depends on what types of surfaces are involved}

\notationitem{$k$}{the spring constant; the constant of proportionality
between the force exerted on an object and the amount by
which the object is lengthened or compressed}

\end{notation}

\begin{summarytext}

Newton's third law states that forces occur in equal and
opposite pairs. If object A exerts a force on object B,
then object B must simultaneously be exerting an equal and
opposite force on object A. Each instance of Newton's third
law involves exactly two objects, and exactly two forces,
which are of the same type.

There are two systems for classifying forces. We are
presently using the more practical but less fundamental one.
In this system, forces are classified by whether they are
repulsive, attractive, or oblique; whether they are contact
or noncontact forces; and whether the two objects involved
are solids or fluids.

Static friction adjusts itself to match the force that is
trying to make the surfaces slide past each other, until the
maximum value is reached,
\begin{equation*}
        F_{s,max} = \mu_s F_N\eqquad.
\end{equation*}
Once this force is exceeded, the surfaces slip past one
another, and kinetic friction applies,
\begin{equation*}
        F_k = \mu_k F_N\eqquad.
\end{equation*}
Both types of frictional force are nearly independent of
surface area, and kinetic friction is usually approximately
independent of the speed at which the surfaces are slipping.
The direction of the force is in the direction that would tend
to stop or prevent slipping.

A good first step in applying Newton's laws of motion to any
physical situation is to pick an object of interest, and
then to list all the forces acting on that object. We
classify each force by its type, and find its Newton's-third-law
partner, which is exerted by the object on some other object.

When two objects are connected by a third low-mass object,
their forces are transmitted to each other nearly unchanged.

Objects under strain always obey Hooke's law to a good
approximation, as long as the force is small. Hooke's law
states that the stretching or compression of the object is
proportional to the force exerted on it,
\begin{equation*}
    F \approx k(x-x_\zu{o})\eqquad.
\end{equation*}

\end{summarytext}

\end{summary}
