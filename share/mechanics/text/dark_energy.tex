Until recently, physicists thought they understood gravity\index{dark energy}
fairly well. Einstein had modified Newton's theory, but
certain characteristrics of gravitational forces were firmly
established. For one thing, they were always attractive. If
gravity always attracts, then it is logical to ask why the
universe doesn't collapse. Newton had answered this question
by saying that if the universe was infinite in all
directions, then it would have no geometric center toward
which it would collapse; the forces on any particular star
or planet exerted by distant parts of the universe would
tend to cancel out by symmetry. More careful calculations,
however, show that Newton's universe would have a tendency
to collapse on smaller scales: any part of the universe that
happened to be slightly more dense than average would
contract further, and this contraction would result in
stronger gravitational forces, which would cause even more
rapid contraction, and so on.

When Einstein overhauled gravity, the same problem reared
its ugly head. Like Newton, Einstein was predisposed to
believe in a universe that was static, so he added a special
repulsive term to his equations, intended to prevent a
collapse. This term was not associated with any interaction
of mass with mass, but represented merely an overall tendency
for space itself to expand unless restrained by the matter
that inhabited it. It turns out that Einstein's solution,
like Newton's, is unstable. Furthermore, it was soon
discovered observationally that the universe was expanding,
and this was interpreted by creating the Big Bang model, in\index{Big Bang}
which the universe's current expansion is the aftermath of a
fantastically hot explosion.%
m4_ifelse(__me,1,[::],[:\footnote{%
__uc_subsection_or_section(doppler)
presents some evidence for the Big Bang theory.}:])
An expanding universe, unlike a
static one, was capable of being explained with Einstein's
equations, without any repulsion term. The universe's
expansion would simply slow down over time due to the
attractive gravitational forces. After these developments,
Einstein said woefully that adding the repulsive term, known
as the cosmological constant, had been the greatest
blunder of his life.\index{cosmological constant}
%
<%
  fig(
    'wmap',
    %q{%
      The WMAP probe's map of the cosmic microwave background is like a
      ``baby picture'' of the universe.
    },
    {
      'width'=>'wide',
      'sidecaption'=>true
    }
  )
%>

This was the state of things until 1999, when evidence began
to turn up that the universe's expansion has been speeding
up rather than slowing down! The first evidence came from
using a telescope as a sort of time machine: light from a
distant galaxy may have taken billions of years to reach us,
so we are seeing it as it was far in the past. Looking back
in time, astronomers saw the universe expanding at speeds
that were lower, rather than higher. At first they were
mortified, since this was exactly the opposite of what had
been expected. The statistical quality of the data was also
not good enough to constitute ironclad proof, and there were
worries about systematic errors. The case for an accelerating
expansion has however been supported by high-precision
mapping of the dim, sky-wide afterglow of the Big Bang,
known as the cosmic microwave background.\label{cmb}\index{cosmic microwave background}\index{CMB}
m4_ifelse(__lm_series,1,[:%
This is discussed in more detail in section \ref{sec:cosmology}.
:])

So now Einstein's
``greatest blunder'' has been resurrected. Since we don't actually know whether or not this self-repulsion
of space has a constant strength, the term ``cosmological \emph{constant}'' has
lost currency. Nowadays physicists usually refer to the phenomenon as ``dark
energy.'' Picking an impressive-sounding name for it should not obscure the fact
that we know absolutely nothing about the nature of the effect or why it exists.
