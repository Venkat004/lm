<% begin_sec("The Motion of Falling Objects",0) %>

The motion of \index{falling objects}falling objects is the
simplest and most common example of motion with changing
velocity. The early pioneers of physics had a correct
intuition that the way things drop was a message directly
from Nature herself about how the universe worked. Other
examples seem less likely to have deep significance. A
walking person who speeds up is making a conscious choice.
If one stretch of a river flows more rapidly than another,
it may be only because the channel is narrower there, which
is just an accident of the local geography. But there is
something impressively consistent, universal, and inexorable
about the way things fall.

Stand up now and simultaneously drop a coin and a bit of
paper side by side. The paper takes much longer to hit the
ground. That's why Aristotle wrote that heavy objects fell
more rapidly. Europeans believed him for two thousand years.

<% marg(10) %>
<%
  fig(
    'galileo-drops-balls',
    %q{%
      According to Galileo's student Viviani, 
      Galileo dropped a cannonball and a
      musketball simultaneously from the leaning tower of Pisa,
      and observed that they hit the
      ground at nearly the same time. This
      contradicted Aristotle's long-accepted
      idea that heavier objects fell faster.
    }
  )
%>
<% end_marg %>
Now repeat the experiment, but make it into a race between
the coin and your shoe. My own shoe is about 50 times
heavier than the nickel I had handy, but it looks to me like
they hit the ground at exactly the same moment. So much for
Aristotle! Galileo, who had a flair for the theatrical, did
the experiment by dropping a bullet and a heavy cannonball
from a tall tower. Aristotle's observations had been
incomplete, his interpretation a vast oversimplification.

It is inconceivable that Galileo was the first person to
observe a discrepancy with Aristotle's predictions. Galileo
was the one who changed the course of history because he was
able to assemble the observations into a coherent pattern,
and also because he carried out systematic quantitative
(numerical) measurements rather than just describing
things qualitatively.

Why is it that some objects, like the coin and the shoe,
have similar motion, but others, like a feather or a bit of
paper, are different? Galileo speculated that in addition to
the force that always pulls objects down, there was an
upward force exerted by the air. Anyone can speculate, but
Galileo went beyond speculation and came up with two clever
experiments to probe the issue. First, he experimented with
objects falling in water, which probed the same issues but
made the motion slow enough that he could take time
measurements with a primitive pendulum clock. With this
technique, he established the following facts:

\begin{itemize}

\item  All heavy, streamlined objects (for example a steel rod
dropped point-down) reach the bottom of the tank in about
the same amount of time, only slightly longer than the time
they would take to fall the same distance in air.

\item  Objects that are lighter or less streamlined take a longer
time to reach the bottom.

\end{itemize}

This supported his hypothesis about two contrary forces. He
imagined an idealized situation in which the falling object
did not have to push its way through any substance at all.
Falling in air would be more like this ideal case than
falling in water, but even a thin, sparse medium like air
would be sufficient to cause obvious effects on feathers and
other light objects that were not streamlined. Today, we
have vacuum pumps that allow us to suck nearly all the air
out of a chamber, and if we drop a feather and a rock side
by side in a vacuum, the feather does not lag behind the rock at all.

<% begin_sec("How the speed of a falling object increases with time",nil,'galileo-ramps-a-motivation') %>

Galileo's second stroke of genius was to find a way to make
quantitative measurements of how the speed of a falling
object increased as it went along. Again it was problematic
to make sufficiently accurate time measurements with
primitive clocks, and again he found a tricky way to slow
things down while preserving the essential physical
phenomena: he let a ball roll down a slope instead of
dropping it vertically. The steeper the incline, the more
rapidly the ball would gain speed. Without a modern video
camera, Galileo had invented a way to make a slow-motion version of falling.

<%
  fig(
    'inclined-plane',
    %q{%
      Velocity increases more gradually on
      the gentle slope, but the motion is
      otherwise the same as the motion of
      a falling object.
    },
    {
      'width'=>'wide'
    }
  )
%>

<% marg(0) %>
<%
  fig(
    'linear-vt-graph',
    %q{%
      The $v-t$ graph of a falling object is a
      line.
    }
  )
%>
<% end_marg %>
Although Galileo's clocks were only good enough to do
accurate experiments at the smaller angles, he was confident
after making a systematic study at a variety of small angles
that his basic conclusions were generally valid. Stated in
modern language, what he found was that the velocity-versus-time
graph was a line. In the language of algebra, we know that a
line has an equation of the form $y=ax+b$, but our
variables are $v$ and $t$, so it would be $v=at+b$.
(The constant $b$ can be interpreted simply as the initial
velocity of the object, i.e., its velocity at the time when
we started our clock, which we conventionally write as $v_\zu{o}$.)

<% self_check('rolling-distances',<<-'SELF_CHECK'
An object is rolling down an incline. After it has been
rolling for a short time, it is found to travel 13 cm during
a certain one-second interval. During the second after that,
it goes 16 cm. How many cm will it travel in the second after that?
  SELF_CHECK
  ) %>

<% end_sec() %>
<% begin_sec("A contradiction in Aristotle's reasoning") %>

Galileo's inclined-plane experiment disproved the long-accepted
claim by Aristotle that a falling object had a definite
``natural falling speed'' proportional to its weight.
Galileo had found that the speed just kept on increasing,
and weight was irrelevant as long as air friction was
negligible. Not only did Galileo prove experimentally that
Aristotle had been wrong, but he also pointed out a logical
contradiction in Aristotle's own reasoning. Simplicio, the
stupid character, mouths the accepted Aristotelian wisdom:
<% marg(70) %>
<%
  fig(
    'tied-rocks-1',
    %q{%
      Galileo's experiments show that all
      falling objects have the same motion
      if air resistance is negligible.
    }
  )
%>
\spacebetweenfigs
<%
  fig(
    'tied-rocks-2',
    %q{%
      1. Aristotle said that heavier objects fell
      faster than lighter ones. 2. If two rocks
      are tied together, that makes an extra-
      heavy rock, which should fall
      faster. 3. But Aristotle's theory would also
      predict that the light rock would hold
      back the heavy rock, resulting in a
      slower fall.
    }
  )
%>
<% end_marg %>

\begin{dialogline}{Simplicio}
There can be no doubt but that a particular body
\ldots has a fixed velocity which is determined by nature\ldots
\end{dialogline}

\begin{dialogline}{Salviati}
If then we take two bodies whose natural speeds
are different, it is clear that, [according to Aristotle],
on uniting the two, the more rapid one will be partly held
back by the slower, and the slower will be somewhat hastened
by the swifter. Do you not agree with me in this opinion?
\end{dialogline}

\begin{dialogline}{Simplicio}
You are unquestionably right.
\end{dialogline}

\begin{dialogline}{Salviati}
But if this is true, and if a large stone moves
with a speed of, say, eight [unspecified units] while a
smaller moves with a speed of four, then when they are
united, the system will move with a speed less than eight;
but the two stones when tied together make a stone larger
than that which before moved with a speed of eight. Hence
the heavier body moves with less speed than the lighter; an
effect which is contrary to your supposition. Thus you see
how, from your assumption that the heavier body moves more
rapidly than the lighter one, I infer that the heavier
body moves more slowly.
\end{dialogline}

<% end_sec() %>
<% begin_sec("What is gravity?") %>

The physicist Richard \index{Feynman}\index{Feynman,
Richard}Feynman liked to tell a story about how when he was
a little kid, he asked his father, ``Why do things fall?''
As an adult, he praised his father for answering, ``Nobody
knows why things fall. It's a deep mystery, and the smartest
people in the world don't know the basic reason for it.''
Contrast that with the average person's off-the-cuff answer,
``Oh, it's because of gravity.'' Feynman liked his father's
answer, because his father realized that simply giving a
name to something didn't mean that you understood it. The
radical thing about Galileo's and Newton's approach to
science was that they concentrated first on describing
mathematically what really did happen, rather than spending
a lot of time on untestable speculation such as Aristotle's
statement that ``Things fall because they are trying to
reach their natural place in contact with the earth.'' That
doesn't mean that science can never answer the ``why''
questions. Over the next month or two as you delve deeper
into physics, you will learn that there are more fundamental
reasons why all falling objects have $v-t$ graphs with the
same slope, regardless of their mass. Nevertheless, the
methods of science always impose limits on how deep
our explanation can go.

<% end_sec() %>
<% end_sec() %>
<% begin_sec("Acceleration",0) %>

<% begin_sec("Definition of acceleration for linear $v-t$ graphs") %>
Galileo's experiment with dropping heavy and light objects
from a tower showed that all falling objects have the same
motion, and his inclined-plane experiments showed that the
motion was described by $v=at+v_\zu{o}$. The initial
velocity $v_\zu{o}$ depends on whether you drop the object from rest
or throw it down, but even if you throw it down, you cannot
change the slope, $a$, of the $v-t$ graph.

Since these experiments show that all falling objects have
linear $v-t$ graphs with the same slope, the slope of such a
graph is apparently an important and useful quantity. We use
the word \index{acceleration}acceleration, and the symbol
$a$, for the slope of such a graph. In symbols, $a=\Delta v/\Delta $t.
The acceleration can be interpreted as the
amount of speed gained in every second, and it has units of
velocity divided by time, i.e., ``meters per second per
second,'' or m/s/s. Continuing to treat units as if they
were algebra symbols, we simplify ``m/s/s'' to read
$``\munit/\sunit^2$.'' Acceleration can be a useful quantity for
describing other types of motion besides falling, and the
word and the symbol ``$a$'' can be used in a more general
context. We reserve the more specialized symbol ``$g$'' for
the acceleration of falling objects, which on the surface of
our planet equals $9.8\ \munit/\sunit^2$. Often when doing approximate
calculations or merely illustrative numerical examples it is
good enough to use $g=10\ \munit/\sunit^2$, which is off by only 2\%.

<% marg(32) %>
<%
  fig(
    'eg-suicidal-student-graph',
    %q{Example \ref{eg:suicidal-student-graph}.}
  )
%>
\spacebetweenfigs
<%
  fig(
    'eg-extract-accel-from-graphs',
    %q{Example \ref{eg:extract-accel-from-graphs}.}
  )
%>

<% end_marg %>
\begin{eg}{Finding final speed, given time}\label{eg:suicidal-student-graph}
\egquestion A despondent physics student jumps off a bridge,
and falls for three seconds before hitting the water. How
fast is he going when he hits the water?

\eganswer Approximating $g$ as $10\ \munit/\sunit^2$, he will gain 10
m/s of speed each second. After one second, his velocity is
10 m/s, after two seconds it is 20 m/s, and on impact, after
falling for three seconds, he is moving at 30 m/s.
\end{eg}

\begin{eg}{Extracting acceleration from a graph}\label{eg:extract-accel-from-graphs}
\egquestion The $x-t$ and $v-t$ graphs show the motion of a car
starting from a stop sign. What is the car's acceleration?

\eganswer Acceleration is defined as the slope of the v-t
graph. The graph rises by 3 m/s during a time interval of 3
s, so the acceleration is $(3\ \munit/\sunit)/(3\ \sunit)=1\ \munit/\sunit^2$.

\noindent Incorrect solution \#1: The final velocity is 3 m/s, and
acceleration is velocity divided by time, so the acceleration
is $(3\ \munit/\sunit)/(10\ \sunit)=0.3\ \munit/\sunit^2$.

<% x_mark %> The solution is incorrect because you can't find the slope
of a graph from one point. This person was just using the
point at the right end of the v-t graph to try to find
the slope of the curve.

\noindent Incorrect solution \#2: Velocity is distance divided by time
so $v=(4.5$ m)/(3 $s)=1.5$ m/s. Acceleration is velocity
divided by time, so $a=(1.5$ m/s)/(3 $s)=0.5\ \munit/\sunit^2$.

\noindent<% x_mark %> The solution is incorrect because velocity is the slope of
the tangent line. In a case like this where the velocity is
changing, you can't just pick two points on the x-t graph
and use them to find the velocity.
\end{eg}

\begin{eg}{Converting $g$ to different units}
\egquestion What is $g$ in units of $\zu{cm}/\sunit^2$?

\eganswer The answer is going to be how many cm/s of speed a
falling object gains in one second. If it gains 9.8 m/s in
one second, then it gains 980 cm/s in one second, so
$g=980\ \zu{cm}/\sunit^2$. Alternatively, we can use the method of
fractions that equal one:

\begin{equation*}
\frac{9.8\ \cancel{\munit}}{\sunit^2}\times\frac{100\ \zu{cm}}{1\ \cancel{\munit}}
  =\frac{980\ \zu{cm}}{\sunit^2}
\end{equation*}

\egquestion What is $g$ in units of $\zu{miles}/\zu{hour}^2$?

\eganswer

\begin{equation*}
\frac{9.8\ \munit}{\sunit^2} 
  \times \frac{1\ \zu{mile}}{1600\ \munit}
  \times \left(\frac{3600\ \sunit}{1\ \zu{hour}}\right)^2
  = 7.9\times 10^4\ \zu{mile}/\zu{hour}^2
\end{equation*}
This large number can be interpreted as the speed, in miles
per hour, that you would gain by falling for one hour. Note that
we had to square the conversion factor of 3600 s/hour in
order to cancel out the units of seconds squared in the denominator.

\egquestion What is $g$ in units of miles/hour/s?

\eganswer

\begin{equation*}
\frac{9.8\ \munit}{\sunit^2} 
  \times \frac{1\ \zu{mile}}{1600\ \munit}
  \times \frac{3600\ \sunit}{1\ \zu{hour}}
  = 22\ \zu{mile}/\zu{hour}/\sunit
\end{equation*}
This is a figure that Americans will have an intuitive feel
for. If your car has a forward acceleration equal to the
acceleration of a falling object, then you will gain 22
miles per hour of speed every second. However, using mixed
time units of hours and seconds like this is usually
inconvenient for problem-solving. It would be like using
units of foot-inches for area instead of $\zu{ft}^2$ or $\zu{in}^2$.
\end{eg}

<% end_sec() %>
<% begin_sec("The acceleration of gravity is different in different locations.") %>

Everyone knows that gravity is weaker on the moon, but
actually it is not even the same everywhere on Earth, as
shown by the sampling of numerical data in the following table.

\begin{tabular}{llll}
\textbf{location} & \textbf{latitude} & \textbf{elevation (m)} & \textbf{g ($\munit/\sunit^2$)} \\
north pole               & 90\degunit N    &   0         & 9.8322 \\
Reykjavik, Iceland       & 64\degunit N    &   0         & 9.8225 \\
Guayaquil, Ecuador       & 2\degunit  S    &   0         & 9.7806 \\
Mt. Cotopaxi, Ecuador    & 1\degunit  S    &   5896      & 9.7624 \\
Mt. Everest              & 28\degunit N    & 8848        & 9.7643
\end{tabular}

\noindent The main variables that relate to the value of $g$ on Earth
are latitude and elevation. Although you have not yet
learned how $g$ would be calculated based on any deeper
theory of gravity, it is not too hard to guess why $g$
depends on elevation. Gravity is an attraction between
things that have mass, and the attraction gets weaker with
increasing distance. As you ascend from the seaport of
Guayaquil to the nearby top of Mt. Cotopaxi, you are
distancing yourself from the mass of the planet. The
dependence on latitude occurs because we are measuring the
acceleration of gravity relative to the earth's surface, but
the earth's rotation causes the earth's surface to fall out
from under you. (We will discuss both gravity and rotation
in more detail later in the course.)

<%
  fig(
    'global-gravity-map',
    %q{%
      This false-color map shows variations
      in the strength of the earth's gravity.
      Purple areas have the strongest
      gravity, yellow the weakest. The overall
       trend toward weaker gravity at the
      equator and stronger gravity at the
      poles has been artificially removed to
      allow the weaker local variations to
      show up. The map covers only the
      oceans because of the technique used
      to make it: satellites look for bulges
      and depressions in the surface of the
      ocean. A very slight bulge will occur
      over an undersea mountain, for
      instance, because the mountain's
      gravitational attraction pulls water
      toward it. The US government
      originally began collecting data like
      these for military use, to correct for the
      deviations in the paths of missiles. The
      data have recently been released for
      scientific and commercial use (e.g.,
      searching for sites for off-shore oil
      wells).
    },
    {
      'width'=>'wide','sidecaption'=>true
    }
  )
%>

Much more spectacular differences in the strength of gravity
can be observed away from the Earth's surface:

\begin{tabular}{lp{55mm}}
\textbf{location} & \textbf{g ($\munit/\sunit^2$)} \\
asteroid Vesta (surface)       & 0.3 \\
Earth's moon (surface)         & 1.6 \\
Mars (surface)                 & 3.7 \\
Earth (surface)                & 9.8 \\
Jupiter (cloud-tops)           & 26 \\
Sun (visible surface)          & 270 \\
typical neutron star (surface) & $10^{12}$ \\
black hole (center)            & infinite according to some theories, on the order of $10^{52}$ according
                                 to others
\end{tabular}

\noindent A typical neutron star is not so different in size from a
large asteroid, but is orders of magnitude more massive, so
the mass of a body definitely correlates with the $g$ it
creates. On the other hand, a neutron star has about the
same mass as our Sun, so why is its $g$ billions of times
greater? If you had the misfortune of being on the surface
of a neutron star, you'd be within a few thousand miles of
all its mass, whereas on the surface of the Sun, you'd still
be millions of miles from most of its mass.


\startdqs

\begin{dq}
What is wrong with the following definitions of $g?$

(1) ``$g$ is gravity.''

(2) ``$g$ is the speed of a falling object.''

(3) ``$g$ is how hard gravity pulls on things.''
\end{dq}

\begin{dq}
When advertisers specify how much acceleration a car is
capable of, they do not give an acceleration as defined in
physics. Instead, they usually specify how many seconds are
required for the car to go from rest to 60 miles/hour.
Suppose we use the notation ``$a$'' for the acceleration as
defined in physics, and ``$a_\text{car ad}$'' for the quantity used in
advertisements for cars. In the US's non-metric system of
units, what would be the units of $a$ and $a_\text{car ad}$? How would
the use and interpretation of large and small, positive and
negative values be different for $a$ as opposed to $a_\text{car ad}$?
\end{dq}

\begin{dq}
Two people stand on the edge of a cliff. As they lean
over the edge, one person throws a rock down, while the
other throws one straight up with an exactly opposite
initial velocity. Compare the speeds of the rocks on impact
at the bottom of the cliff.
\end{dq}

<% end_sec() %>
<% end_sec() %>
<% begin_sec("Positive and Negative Acceleration",0) %>\index{acceleration!negative}

Gravity always pulls down, but that does not mean it always
speeds things up. If you throw a ball straight up, gravity
will first slow it down to $v=0$ and then begin increasing
its speed. When I took physics in high school, I got the
impression that positive signs of acceleration indicated
speeding up, while negative accelerations represented
slowing down, i.e., deceleration. Such a definition would be
inconvenient, however, because we would then have to say
that the same downward tug of gravity could produce either a
positive or a negative acceleration. As we will see in the
following example, such a definition also would not be the
same as the slope of the $v-t$ graph.

<% marg(60) %>
<%
  fig(
    'person-throws-ball',
    %q{%
      The ball's acceleration stays the same ---
      on the way up, at the top, and on the way back down. It's always
      negative.
    }
  )
%>
<% end_marg %>
Let's study the example of the rising and falling ball. In
the example of the person falling from a bridge, I assumed
positive velocity values without calling attention to it,
which meant I was assuming a coordinate system whose $x$
axis pointed down. In this example, where the ball is
reversing direction, it is not possible to avoid negative
velocities by a tricky choice of axis, so let's make the
more natural choice of an axis pointing up. The ball's
velocity will initially be a positive number, because it is
heading up, in the same direction as the $x$ axis, but on
the way back down, it will be a negative number. As shown in
the figure, the $v-t$ graph does not do anything special at
the top of the ball's flight, where $v$ equals 0. Its slope
is always negative. In the left half of the graph, there is
a negative slope because the positive velocity is getting
closer to zero. On the right side, the negative slope is due
to a negative velocity that is getting farther from zero, so
we say that the ball is speeding up, but its velocity is decreasing!

To summarize, what makes the most sense is to stick with the
original definition of acceleration as the slope of the
$v-t$ graph, $\Delta v/\Delta t$. By this definition, it
just isn't necessarily true that things speeding up have
positive acceleration while things slowing down have
negative acceleration. The word ``deceleration'' is not used
much by physicists, and the word ``acceleration'' is used
unblushingly to refer to slowing down as well as speeding
up: ``There was a red light, and we accelerated to a stop.''

\begin{eg}{Numerical calculation of a negative acceleration}
\egquestion In figure \figref{person-throws-ball}, what happens if you
calculate the acceleration between $t=1.0$ and 1.5 s?

\eganswer Reading from the graph, it looks like the velocity
is about $-1$ m/s at $t=1.0$ s, and around $-6$ m/s at $t=1.5$ s.
The acceleration, figured between these two points, is
\begin{equation*}
  a = \frac{\Delta v}{\Delta t} = \frac{(-6\ \munit/\sunit)-(-1\ \munit/\sunit)}{(1.5\ \sunit)-(1.0\ \sunit)} = -10\ \munit/\sunit^2\eqquad.
\end{equation*}
Even though the ball is speeding up, it has a negative acceleration.
\end{eg}

Another way of convincing you that this way of handling the
plus and minus signs makes sense is to think of a device
that measures acceleration. After all, physics is supposed
to use operational definitions, ones that relate to the
results you get with actual measuring devices.\index{operational definition!acceleration}
Consider an
air freshener hanging from the rear-view mirror of your car.
When you speed up, the air freshener swings backward.
Suppose we define this as a positive reading. When you slow
down, the air freshener swings forward, so we'll call this a
negative reading on our accelerometer. But what if you put
the car in reverse and start speeding up backwards? Even
though you're speeding up, the accelerometer responds in the
same way as it did when you were going forward and slowing
down. There are four possible cases:

\noindent\begin{tabular}{p{40mm}p{21mm}p{16mm}p{17mm}}
motion of car & accelerometer swings & slope of v-t graph & direction of force acting on car \\
forward, speeding up      & backward   & $+$   & forward \\
forward, slowing down     & forward    & $-$   & backward \\
backward, speeding up     & forward    & $-$   & backward \\
backward, slowing down    & backward   & $+$   & forward \\
\end{tabular}

\noindent Note the consistency of the three right-hand columns ---
nature is trying to tell us that this is the right system of
classification, not the left-hand column.

Because the positive and negative signs of acceleration
depend on the choice of a coordinate system, the acceleration
of an object under the influence of gravity can be either
positive or negative. Rather than having to write things
like ``$g=9.8\ \munit/\sunit^2$ or $-9.8\ \munit/\sunit^2$''
every time we want to
discuss $g$'s numerical value, we simply define $g$ as the
absolute value of the acceleration of objects moving under
the influence of gravity. We consistently let $g=9.8 \ \munit/\sunit^2$,
but we may have either $a=g$ or $a=-g$, depending on our
choice of a coordinate system.

\begin{eg}{Acceleration with a change in direction of motion}
\egquestion A person kicks a ball, which rolls up a sloping
street, comes to a halt, and rolls back down again. The ball
has constant acceleration. The ball is initially moving at a
velocity of 4.0 m/s, and after 10.0 s it has returned to
where it started. At the end, it has sped back up to the
same speed it had initially, but in the opposite direction.
What was its acceleration?

\eganswer By giving a positive number for the initial
velocity, the statement of the question implies a coordinate
axis that points up the slope of the hill. The ``same''
speed in the opposite direction should therefore be
represented by a negative number, -4.0 m/s. The acceleration
is
\begin{align*}
  a &= \Delta v/\Delta t \\
    &= (v_f-v_\zu{o})/10.0\ \sunit \\
    &= [(-4.0 \ \munit/\sunit)-(4.0 \ \munit/\sunit)]/10.0 s \\
    &= -0.80\ \munit/\sunit^2\eqquad.
\end{align*}
The
acceleration was no different during the upward part of the
roll than on the downward part of the roll.

Incorrect solution: Acceleration is $\Delta v/\Delta $t, and
at the end it's not moving any faster or slower than when it
started, so $\Delta $v=0 and $a=0$.

\noindent<% x_mark %>  The velocity does change, from a positive number
to a negative number.
\end{eg}

<%
  fig(
    'dq-water-bottle',
    %q{Discussion question \ref{dq:water-bottle}.},
    {
      'width'=>'wide',
      'sidecaption'=>true,
      'anonymous'=>true
    }
  )
%>

\startdqs

\begin{dq}
A child repeatedly jumps up and down on a trampoline.
Discuss the sign and magnitude of his acceleration,
including both the time when he is in the air and the time
when his feet are in contact with the trampoline.
\end{dq}

\begin{dq}\label{dq:water-bottle}
The figure
shows a refugee from a Picasso painting
blowing on a rolling water bottle. In some cases the
person's blowing is speeding the bottle up, but in others it
is slowing it down. The arrow inside the bottle shows which
direction it is going, and a coordinate system is shown at
the bottom of each figure. In each case, figure out the plus
or minus signs of the velocity and acceleration. It may be
helpful to draw a $v-t$ graph in each case.
\end{dq}

<% marg %>
<%
  fig(
    'dq-corndog',
    %q{Discussion question \ref{dq:corndog}.},
    {
      'anonymous'=>true
    }
  )
%>
<% end_marg %>
\begin{dq}\label{dq:corndog}
Sally is on an amusement park ride which begins with her
chair being hoisted straight up a tower at a constant speed
of 60 miles/hour. Despite stern warnings from her father
that he'll take her home the next time she misbehaves, she
decides that as a scientific experiment she really needs to
release her corndog over the side as she's on the way up.
She does not throw it. She simply sticks it out of the car,
lets it go, and watches it against the background of the
sky, with no trees or buildings as reference points. What
does the corndog's motion look like as observed by Sally?
Does its speed ever appear to her to be zero? What
acceleration does she observe it to have: is it ever
positive? negative? zero? What would her enraged father
answer if asked for a similar description of its motion as
it appears to him, standing on the ground?
\end{dq}

\begin{dq}
Can an object maintain a constant acceleration, but
meanwhile reverse the direction of its velocity?
\end{dq}

\begin{dq}
Can an object have a velocity that is positive and
increasing at the same time that its acceleration is decreasing?
\end{dq}



<% end_sec() %>
<% begin_sec("Varying Acceleration",0) %>\index{acceleration!definition}
m4_ifelse(__me,1,[:
So far we have only been discussing examples of motion for
which the acceleration is constant. As always, an expression of the
form $\Delta\ldots/\Delta\ldots$ for a rate of change must be generalized
to a derivative when the rate of change isn't constant. We therefore define
the acceleration as $a=\der v/\der t$, which is the same as the second
derivative, which Leibniz notated as
\begin{equation*}
  a = \frac{\der^2 x}{\der t^2}\eqquad.
\end{equation*}
The seemingly inconsistent placement of the twos on the top
and bottom confuses all beginning calculus students. The
motivation for this funny notation is that acceleration has
units of $\munit/\sunit^2$, and the notation correctly suggests that:
the top looks like it has units of meters, the bottom
$\text{seconds}^2$. The notation is not meant, however, to suggest
that $t$ is really squared.
:],[:%
%------------- start of LM, varying acceleration
So far we have only been discussing examples of motion for
which the $v-t$ graph is linear. If we wish to generalize
our definition to v-t graphs that are more complex curves,
the best way to proceed is similar to how we defined
velocity for curved $x-t$ graphs:

\begin{lessimportant}[definition of acceleration]
The acceleration of an object at any instant is the slope of
the tangent line passing through its $v$-versus-$t$ graph
at the relevant point.
\end{lessimportant}

<% marg(0) %>
<%
  fig(
    'eg-skydiver-graph',
    %q{Example \ref{eg:skydiver-graph}.},
    {'suffix'=>'2'}
  )
%>
<% end_marg %>
\begin{eg}{A skydiver}\label{eg:skydiver-graph}
\egquestion The graphs in figure \figref{eg-skydiver-graph2}
show the results of a fairly realistic
computer simulation of the motion of a skydiver, including
the effects of air friction. The $x$ axis has been chosen
pointing down, so $x$ is increasing as she falls. Find (a)
the skydiver's acceleration at $t=3.0\ \sunit$, and also (b) at $t=7.0\ \sunit$.

\eganswer The solution is shown in figure \figref{eg-skydiver-graph-soln}.
I've added tangent lines at the two points in question.

(a) To find the slope of the tangent line, I pick two points
on the line (not necessarily on the actual curve): 
$(3.0\ \sunit,28 \munit/\sunit)$ and $(5.0\ \sunit,42\ \munit/\sunit)$.
The slope of the tangent line
is 
$(42\ \munit/\sunit-28\ \munit/\sunit)/(5.0\ \sunit - 3.0\ \sunit)=7.0\ \munit/\sunit^2$.

(b) Two points on this tangent line are 
$(7.0\ \sunit,47\ \munit/\sunit)$
and 
$(9.0\ \sunit, 52\ \munit/\sunit)$. 
The slope of the tangent line is 
$(52\ \munit/\sunit-47\ \munit/\sunit)/(9.0\ \sunit - 7.0\ \sunit)=2.5\ \munit/\sunit^2$.

Physically, what's happening is that at $t=3.0\ \sunit$, the
skydiver is not yet going very fast, so air friction is not
yet very strong. She therefore has an acceleration almost as
great as $g$. At $t=7.0\ \sunit$, she is moving almost twice as
fast (about 100 miles per hour), and air friction is
extremely strong, resulting in a significant departure from
the idealized case of no air friction.
\end{eg}
<%
  fig(
    'eg-skydiver-graph-soln',
    %q{The solution to example \ref{eg:skydiver-graph}.},
    {
      'width'=>'wide',
      'sidecaption'=>true,
      'sidepos'=>'c'
    }
  )
%>

In example \ref{eg:skydiver-graph}, the $x-t$ graph was not even used in
the solution of the problem, since the definition of
acceleration refers to the slope of the $v-t$ graph. It is
possible, however, to interpret an $x-t$ graph to find out
something about the acceleration. An object with zero
acceleration, i.e., constant velocity, has an $x-t$ graph
that is a straight line. A straight line has no curvature. A
change in velocity requires a change in the slope of the $x-t$
graph, which means that it is a curve rather than a line.
Thus acceleration relates to the curvature of the $x-t$
graph. Figure \figref{curvature-examples} shows some examples.

\pagebreak[4]

In example \ref{eg:skydiver-graph}, the $x-t$ graph was more strongly
curved at the beginning, and became nearly straight at the
end. If the $x-t$ graph is nearly straight, then its slope,
the velocity, is nearly constant, and the acceleration is
therefore small. We can thus interpret the acceleration as
representing the curvature of the $x-t$ graph, as shown
in figure \figref{curvature-examples}. If the
``cup'' of the curve points up, the acceleration is
positive, and if it points down, the acceleration is negative.

<%
  fig(
    'curvature-examples',
    %q{%
      Acceleration relates to the
      curvature of the $x-t$ graph.
    },
    {
      'width'=>'wide',
      'sidecaption'=>false
    }
  )
%>

\pagebreak[4]

Since the relationship between $a$ and $v$ is analogous to
the relationship between $v$ and $x$, we can also make
graphs of acceleration as a function of time, as shown in
figure \figref{sample-xva-graphs}.
<%
  fig(
    'sample-xva-graphs',
    %q{%
      Examples of graphs of $x$, $v$, and
      $a$ versus $t$. 1. An object in free fall, with no friction.
      2. A continuation of example \ref{eg:skydiver-graph}, the skydiver.
    },
    {
      'width'=>'wide',
      'sidecaption'=>false
    }
  )
%>

<% marg(0) %>
<%
  fig(
    'chart-of-xva',
    %q{How position, velocity, and acceleration are related.}
  )
%>
<% end_marg %>
\worked{x-graph-to-v-graph}{Drawing a $v-t$ graph.}

\worked{two-ramps}{Drawing $v-t$ and $a-t$ graphs.}

Figure \figref{chart-of-xva} summarizes the relationships among the three types of graphs.

\startdqs

\begin{dq}
Describe in words how the changes in the $a-t$ graph in
figure \figref{sample-xva-graphs}/2 relate to the behavior of the $v-t$ graph.
\end{dq}

\pagebreak[4]

\begin{dq}\label{dq:inconsistent-graphs}
Explain how each set of graphs contains inconsistencies, and fix them.
\end{dq}

<%
  fig(
    'dq-inconsistent-graphs',
    '',
    {
      'width'=>'wide',
      'anonymous'=>true,
      'float'=>false,
    }
  )
%>

\begin{dq}
In each case, pick a coordinate system and draw
$x-t,v-t$, and $a-t$ graphs. Picking a coordinate system
means picking where you want $x=0$ to be, and also picking a
direction for the positive $x$ axis.

(1) An ocean liner is cruising in a straight line at constant speed.

(2) You drop a ball. Draw two different sets of graphs (a
total of 6), with one set's positive $x$ axis pointing in
the opposite direction compared to the other's.

(3) You're driving down the street looking for a house you've
never been to before. You realize you've passed the address,
so you slow down, put the car in reverse, back up, and stop
in front of the house.
\end{dq}
%------------- end of LM, varying acceleration
:])
<% end_sec() %>
m4_ifelse(__me,0,[:
%------------ start of LM section on area under curve
<% begin_sec("The Area Under the Velocity-Time Graph",0,'area-under-vt') %>\index{area}\index{area under a curve!under v-t graph}

A natural question to ask about falling objects is how fast
they fall, but Galileo showed that the question has no
answer. The physical law that he discovered connects a cause
(the attraction of the planet Earth's mass) to an effect,
but the effect is predicted in terms of an acceleration
rather than a velocity. In fact, no physical law predicts a
definite velocity as a result of a specific phenomenon,
because velocity cannot be measured in absolute terms, and
only changes in velocity relate directly to physical phenomena.

The unfortunate thing about this situation is that the
definitions of velocity and acceleration are stated in terms
of the tangent-line technique, which lets you go from $x$ to
$v$ to $a$, but not the other way around. Without a
technique to go backwards from $a$ to $v$ to $x$, we cannot
say anything quantitative, for instance, about the $x-t$
graph of a falling object. Such a technique does exist, and
I used it to make the $x-t$ graphs in all the examples above.

First let's concentrate on how to get $x$ information out of
a $v-t$ graph. In example \figref{area-under-curve}/1, an object moves at a speed of
$20\ \munit/\sunit$ for a period of 4.0 s. The distance covered is
$\Delta x=v\Delta t=(20\ \munit/\sunit)\times(4.0\ \sunit)=80\ \munit$.
 Notice that the quantities
being multiplied are the width and the height of the shaded
rectangle --- or, strictly speaking, the time represented by
its width and the velocity represented by its height. The
distance of $\Delta x=80\ \munit$ thus corresponds to the area of the
shaded part of the graph.
<% marg(75) %>
<%
  fig(
    'area-under-curve',
    %q{%
      The area under the $v-t$ graph gives
      $\Delta x$.
    }
  )
%>
<% end_marg %>

The next step in sophistication is an example like \figref{area-under-curve}/2,
where the object moves at a constant speed of $10\ \munit/\sunit$ for two
seconds, then for two seconds at a different constant speed
of $20\ \munit/\sunit$. The shaded region can be split into a small
rectangle on the left, with an area representing $\Delta x=20\ \munit$,
and a taller one on the right, corresponding to another 40 m
of motion. The total distance is thus 60 m, which
corresponds to the total area under the graph.

An example like \figref{area-under-curve}/3 is now just a trivial generalization;
there is simply a large number of skinny rectangular areas
to add up. But notice that graph \figref{area-under-curve}/3 is quite a good
approximation to the smooth curve \figref{area-under-curve}/4. Even though we have
no formula for the area of a funny shape like \figref{area-under-curve}/4, we can
approximate its area by dividing it up into smaller areas
like rectangles, whose area is easier to calculate. If
someone hands you a graph like \figref{area-under-curve}/4 and asks you to find the
area under it, the simplest approach is just to count up the
little rectangles on the underlying graph paper, making
rough estimates of fractional rectangles as you go along.

<%
  fig(
    'area-under-curve-soln',
    %q{An example using estimation of fractions of a rectangle.},
    {
      'width'=>'wide',
      'sidecaption'=>true
    }
  )
%>

That's what I've done in figure \figref{area-under-curve-soln}. Each rectangle on the graph
paper is 1.0 s wide and $2\ \munit/\sunit$ tall, so it represents 2
m. Adding up all the numbers gives $\Delta x=41\ \munit$. If you needed
better accuracy, you could use graph paper with smaller rectangles.

It's important to realize that this technique gives you
$\Delta x$, not $x$. The $v-t$ graph has no information about
where the object was when it started.

The following are important points to keep in mind when
applying this technique:

\begin{itemize}

\item  If the range of $v$ values on your graph does not extend
down to zero, then you will get the wrong answer unless you
compensate by adding in the area that is not shown.

\item  As in the example, one rectangle on the graph paper does
not necessarily correspond to one meter of distance.

\item  Negative velocity values represent motion in the opposite
direction, so as suggested by figure \figref{negative-and-positive-area},
area under the $t$ axis should be subtracted,
i.e., counted as ``negative area.''

\item  Since the result is a $\Delta x$ value, it only tells you
$x_{after}-x_{before}$, which may be less than the actual
distance traveled. For instance, the object could come back
to its original position at the end, which would correspond
to $\Delta x$=0, even though it had actually moved a nonzero distance.

\end{itemize}
<% marg(25) %>
<%
  fig(
    'negative-and-positive-area',
    %q{%
      Area underneath the axis is considered negative.
    }
  )
%>
<% end_marg %>

Finally, note that one can find $\Delta v$ from an $a-t$ graph
using an entirely \index{area under a curve!area under a-t
graph}analogous method. Each rectangle on the $a-t$ graph
represents a certain amount of velocity change.

\startdq

\begin{dq}
Roughly what would a pendulum's $v-t$ graph look like? What
would happen when you applied the area-under-the-curve
technique to find the pendulum's $\Delta x$ for a time
period covering many swings?
\end{dq}

<% end_sec() %>
%------------ end of LM section on area under curve
:])
<% begin_sec("Algebraic Results for Constant Acceleration",0,'const-accel-equations') %>\index{acceleration!constant}
m4_ifelse(__me,0,[:
%------------ start of LM section on algebraic results for constant acceleration

Although the area-under-the-curve technique can be applied
to any graph, no matter how complicated, it may be laborious
to carry out, and if fractions of rectangles must be
estimated the result will only be approximate. In the
special case of motion with constant acceleration, it is
possible to find a convenient shortcut which produces exact
results. When the acceleration is constant, the $v-t$ graph
is a straight line, as shown in the figure. The area under
the curve can be divided into a triangle plus a rectangle,
both of whose areas can be calculated exactly: $A=bh$
for a rectangle and $A=bh/2$ for a triangle. The height
of the rectangle is the initial velocity, $v_\zu{o}$, and the
height of the triangle is the change in velocity from
beginning to end, $\Delta v$. The object's $\Delta x$ is therefore given
by the equation $\Delta x = v_\zu{o} \Delta t + \Delta v\Delta t/2$.
 This can be simplified a little by using
the definition of acceleration, $a=\Delta v/\Delta t$, to eliminate $\Delta v$, giving
\begin{multline*}
  \Delta x = v_\zu{o} \Delta t + \frac{1}{2}a\Delta t^2\eqquad. \qquad \hfill 
       \shoveright{\text{[motion with}}\\
             \text{constant acceleration]}
\end{multline*}
Since this is a second-order polynomial in $\Delta t$, the graph
of $\Delta x$ versus $\Delta t$ is a parabola, and the same is true of a
graph of $x$ versus $t$ --- the two graphs differ only by
shifting along the two axes. Although I have derived the
equation using a figure that shows a positive $v_\zu{o}$, positive
$a$, and so on, it still turns out to be true regardless of
what plus and minus signs are involved.

<% marg(0) %>
<%
  fig(
    'half-a-tsquared-derivation',
    %q{%
      The shaded area tells us how far an
      object moves while accelerating at a constant rate.
    }
  )
%>
<% end_marg %>
Another useful equation can be derived if one wants to
relate the change in velocity to the distance traveled. This
is useful, for instance, for finding the distance needed by
a car to come to a stop. For simplicity, we start by
deriving the equation for the special case of $v_\zu{o}=0$, in
which the final velocity $v_f$ is a synonym for $\Delta v$. Since
velocity and distance are the variables of interest, not
time, we take the equation $\Delta x=\frac{1}{2}a\Delta t^2$ and use $\Delta t=\Delta v/a$ to
eliminate $\Delta t$. This gives $\Delta x=(\Delta v)^2/2a$, which can be rewritten as
\begin{equation*}
  v_f^2 = 2a\Delta x \quad . \quad \hfill \shoveright{\text{[motion with constant acceleration, $v_\zu{o}=0$]}}
\end{equation*}
For the more general case where $v_\zu{o}\ne 0$, we skip the tedious
algebra leading to the more general equation,
\begin{equation*}
  v_f^2 = v_\zu{o}^2 + 2a\Delta x \quad . \quad \hfill \shoveright{\text{[motion with constant acceleration]}}
\end{equation*}

\pagebreak[4]

To help get this all organized in your head, first let's
categorize the variables as follows:

\noindent Variables that change during motion with constant acceleration:

    $x$ ,$v$, $t$

\noindent Variable that doesn't change:

   $a$

\noindent If you know one of the changing variables and want to find
another, there is always an equation that relates those two:

<% raw_fig('accel-triangle') %>

The symmetry among the three variables is imperfect only
because the equation relating $x$ and $t$ includes
the initial velocity.

There are two main difficulties encountered by students in
applying these equations:

\begin{itemize}

\item  The equations apply only to motion with constant
acceleration. You can't apply them if the acceleration is changing.

\item  Students are often unsure of which equation to use, or may
cause themselves unnecessary work by taking the longer path
around the triangle in the chart above. Organize your
thoughts by listing the variables you are given, the ones
you want to find, and the ones you aren't given and don't care about.

\end{itemize}

\begin{eg}{Saving an old lady}
\egquestion You are trying to pull an old lady out of the way
of an oncoming truck. You are able to give her an acceleration
of $20\ \munit/\sunit^2$. Starting from rest, how much time is required
in order to move her 2 m?

\eganswer First we organize our thoughts:

    Variables given:    $\Delta x$, $a$, $v_\zu{o}$

    Variables desired:    $\Delta t$

    Irrelevant variables: $v_f$

\noindent Consulting the triangular chart above, the equation we need
is clearly  $\Delta x=v_\zu{o}\Delta t+\frac{1}{2}a\Delta t^2$, since it has the four variables of interest
and omits the irrelevant one. Eliminating the $v_\zu{o}$ term and
solving for $\Delta t$ gives $\Delta t=\sqrt{2\Delta x/a}=0.4\ \sunit$.

\end{eg}

\worked{stupid}{A stupid celebration}

\worked{mars-drop-time}{Dropping a rock on Mars}

\worked{dodge-viper}{The Dodge Viper}

\worked{ramp-half-speed}{Half-way sped up}

\startdqs

\begin{dq}
In chapter 1, I gave examples of correct and incorrect
reasoning about proportionality, using questions about the
scaling of area and volume. Try to translate the incorrect
modes of reasoning shown there into mistakes about the
following question: If the acceleration of gravity on Mars
is 1/3 that on Earth, how many times longer does it take for
a rock to drop the same distance on Mars?
\end{dq}

\begin{dq} Check that the units make sense in the three equations
derived in this section.
\end{dq}
%------------ end of LM section on algebraic results for constant acceleration
:],[:%
%------------ start of SN section on algebraic results for constant acceleration
When an object is accelerating, the variables $x$, $v$, and $t$ are all changing continuously.
It is often of interest to eliminate one of these and relate the other two to each other.

\begin{eg}{Constant acceleration}\label{eg:diving-board}
\egquestion
How high does a diving board have to be above the water if the diver is to have as much as 1.0 s
in the air?

\eganswer
The diver starts at rest, and has an acceleration of $9.8\ \munit/\sunit^2$. 
We need to find a connection between the distance she travels and time it takes. In other words,
we're looking for information about the function $x(t)$, given information about the acceleration.
To go from acceleration to position, we need to integrate twice:
\begin{align*}
  x &= \int \int a \der t \der t \\
    &= \int \left(at+v_\zu{o}\right) \der t \qquad \text{[$v_\zu{o}$ is a constant of integration.]} \\
    &= \int at \der t \qquad \text{[$v_\zu{o}$ is zero because she's dropping from rest.]} \\
    &= \frac{1}{2}at^2+x_\zu{o} \qquad \text{[$x_\zu{o}$ is a constant of integration.]} \\
    &= \frac{1}{2}at^2 \qquad \text{[$x_\zu{o}$ can be zero if we define it that way.]}
\end{align*}
Note some of the good problem-solving habits demonstrated here. We solve the problem symbolically, and only
plug in numbers at the very end, once all the algebra and calculus are done. One should also make
a habit, after finding a symbolic result, of checking whether the dependence on the variables
make sense. A greater value of $t$ in this expression would lead to a greater value for $x$; that makes
sense, because if you want more time in the air, you're going to have to jump from higher up. A greater
acceleration also leads to a greater height; this also makes sense, because the stronger gravity is,
the more height you'll need in order to stay in the air for a given amount of time.
Now we plug in numbers.
\begin{align*}
  x  &= \frac{1}{2}\left(9.8\ \munit/\sunit^2\right)(1.0\ \sunit)^2  \\
    &= 4.9\ \munit
\end{align*}
Note that when we put in the numbers,
we check that the units work out correctly, $\left(\munit/\sunit^2\right)(\sunit)^2=\munit$. We should
also check that the result makes sense: 4.9 meters is pretty high, but not unreasonable.
\end{eg}

Under conditions of constant acceleration,
we can relate velocity and time,
\begin{equation*}
  a = \frac{\Delta v}{\Delta t}\eqquad,
\end{equation*}
or, as in the example \ref{eg:diving-board}, position and time,
\begin{equation*}
  x =  \frac{1}{2}at^2 + v_\zu{o} t + x_\zu{o}\eqquad.
\end{equation*}
It can also be handy to have a relation involving velocity and position, eliminating time.
Straightforward algebra gives
\begin{equation*}
  v_f^2 = v_\zu{o}^2 + 2 a \Delta x\eqquad,
\end{equation*}
where $v_f$ is the final velocity, $v_\zu{o}$ the initial velocity, and $\Delta x$ the
distance traveled.

\vspace{0mm plus 5mm}

\worked{mars-drop-time}{Dropping a rock on Mars}

\vspace{0mm plus 5mm}

\worked{dodge-viper}{The Dodge Viper}

\vspace{0mm plus 5mm}

%------------ end of SN section on algebraic results for constant acceleration
:])
<% end_sec %>
%----------------------------------------------------------------------------
<% begin_sec("A test of the principle of inertia",nil,'galileo-ramps-inertia',{'optional'=>true}) %>
Historically, the first quantitative and well 
documented experimental test\index{Newton's laws of motion!first law!test of}
of the principle of inertia (p.~\pageref{principle-of-inertia}) was performed by Galileo
around 1590 and published decades later when he managed to find a publisher in
the Netherlands that was beyond the reach of
the Roman Inquisition.\footnote{Galileo, \emph{Discourses and Mathematical Demonstrations
Relating to Two New Sciences}, 1638. The experiments are described in the Third Day,
and their support for the principle of inertia is discussed
in the Scholium following 
Theorems I-XIV. % at "Let us suppose that the descent...," Britannica Great Books ed., p. 225, at 
Another experiment involving a ship is described
in Galileo's 1624 reply to a letter from Fr.~Ingoli, but although Galileo vigorously
asserts that he really did carry it out, no detailed description or quantitative results
are given.} It was ingenious but somewhat indirect, and required a layer of
interpretation and extrapolation on top of the actual observations. As described
on p.~\pageref{subsec:galileo-ramps-a-motivation}, he established that objects rolling
on inclined planes moved according to mathematical laws that we would today describe
as in section \ref{sec:const-accel-equations}. He knew that his rolling balls were subject
to friction, as well as random errors due to the limited precision of the water clock that
he used, but he took the approximate agreement of his equations with experiment to indicate
that they gave the results that would be exact in the absence of friction. He also showed,
purely empirically, that when a ball went up or down a ramp inclined at an angle $\theta$,
its acceleration was proportional to $\sin\theta$. Again, this required extrapolation to
idealized conditions of zero friction. He then reasoned that if a ball was rolled on a
\emph{horizontal} ramp, with $\theta=0$, its acceleration would be zero. This is exactly
what is required by the principle of inertia: in the absence of friction, motion continues
indefinitely.

m4_ifelse(__me,1,[:
<% marg(-8) %>
<%
  fig(
    'eg-sin-theta-proof',
    %q{%
      Example \ref{eg:sin-theta-proof}.
    }
  )
%>
<% end_marg %>


\begin{eg}{Reversing the logic}\label{eg:sin-theta-proof}
If we assume that the principle of inertia holds, then we can reverse the direction of Galileo's
reasoning. Rather than a rolling ball, it's simpler to consider a bead sliding frictionlessly
on a wire. We will show that the bead's acceleration is $g\sin\theta$.

In figure \subfigref{eg-sin-theta-proof}{1}, the bead starts at rest from A
and travels to B by first falling
a height $h$ with acceleration $g$, and then sliding horizontally. (We assume that the bead
turns the corner smoothly and without any loss of speed.) By the principle of inertia,
the frictionless horizontal motion has constant velocity, so the bead's final velocity at B
is the same velocity it would have achieved simply by falling through the difference in height
between A and B.

In \subfigref{eg-sin-theta-proof}{2}, the dark line shows a staircase made out of many small
copies of the original ``L''-shaped wire. The final velocity is the same as before.

But if we make the steps of the staircase sufficiently small, we can get as good an
approximation as we like to a the straight, sloping ramp in \subfigref{eg-sin-theta-proof}{3}.
Therefore the final velocity in this case is the same as in \subfigref{eg-sin-theta-proof}{1}.

Applying the appropriate constant-acceleration equation, we have $v_f^2=2a\ell$, where
$\ell$ is the length of the ramp. But since $v_f$ is the same for the ramp as for a free-fall
through a similar height, we have $a\propto\ell^{-1}$, so that $a/g=(\ell/h)^{-1}=\sin\theta$.

For an object such as a sphere or a cylinder that rolls down a ramp without slipping, a similar proportionality holds between
$a$ and $g\sin\theta$, but the acceleration is smaller by a unitless factor that depends on the
object's shape. The difference comes about because of the frictional force (the ``traction'') between
the object and the ramp.
\end{eg}
:])
<% end_sec %> % A test of the principle of inertia
%----------------------------------------------------------------------------
m4_ifelse(__me,0,[:
%-------------- start of optional LM section on applications of calculus
<% begin_sec("Applications of Calculus",0,'',{'calc'=>true}) %>

In section \ref{sec:calculus-for-velocity}, I discussed how the slope-of-the-tangent-line
idea related to the calculus concept of a derivative, and
the branch of calculus known as differential calculus. The
other main branch of calculus, \index{calculus!integral}integral
calculus, has to do with the area-under-the-curve concept
discussed in section \ref{sec:area-under-vt}. Again there is a
concept, a notation, and a bag of tricks for doing things
symbolically rather than graphically. In calculus, the area
under the $v-t$ graph between $t=t_1$ and $t=t_2$ is notated like this:
\begin{equation*}
  \text{area under curve} = \Delta x = \int_{t_1}^{t_2}v \der t\eqquad.
\end{equation*}
The expression on the right is called an integral, and the
s-shaped symbol, the integral sign, is read as ``\index{integral}integral of \ldots''

Integral calculus and differential calculus are closely
related. For instance, if you take the derivative of the
function $x(t)$, you get the function $v(t)$, and if you
integrate the function $v(t)$, you get $x(t)$ back again. In
other words, integration and differentiation are inverse
operations. This is known as the fundamental theorem of
\index{calculus!fundamental theorem of}calculus.

On an unrelated topic, there is a special notation for
taking the derivative of a function twice. The acceleration,
for instance, is the second (i.e., double) \index{derivative!second}derivative
of the position, because differentiating $x$ once gives $v$,
and then differentiating $v$ gives $a$. This is written as
\begin{equation*}
   a = \frac{\der^2 x}{\der t^2}\eqquad.
\end{equation*}
The seemingly inconsistent placement of the twos on the top
and bottom confuses all beginning calculus students. The
motivation for this funny notation is that acceleration has
units of $\munit/\sunit^2$, and the notation correctly suggests that:
the top looks like it has units of meters, the bottom
$\text{seconds}^2$. The notation is not meant, however, to suggest
that $t$ is really squared.

<% end_sec() %>
%-------------- end of optional LM section on applications of calculus
:])
\begin{summary}

\begin{vocab}

\vocabitem{gravity}{A general term for the phenomenon of attraction
between things having mass. The attraction between our
planet and a human-sized object causes the object to fall.}

\vocabitem{acceleration}{The rate of change of velocity; the slope of
the tangent line on a $v-t$ graph.}

\end{vocab}

\begin{notation}

\notationitem{$v_o$}{initial velocity}
\notationitem{$v_f$}{final velocity}
\notationitem{$a$}{acceleration}
\notationitem{$g$}{the acceleration of objects in free fall; the strength of the local gravitational field}

\end{notation}

\begin{summarytext}

Galileo showed that when air resistance is negligible all
falling bodies have the same motion regardless of mass.
Moreover, their $v-t$ graphs are straight lines. We
therefore define a quantity called acceleration as the
m4_ifelse(__me,1,[:%
derivative $\der v/\der t$.
This
definition has the advantage that a force with a given sign,
representing its direction,
always produces an acceleration with the same sign.
:],[:%
slope, $\Delta v/\Delta $t, of an object's $v-t$  graph. In
cases other than free fall, the $v-t$  graph may be curved,
in which case the definition is generalized as the slope of
a tangent line on the $v-t$ graph.
:])%
 The acceleration of
objects in free fall varies slightly across the surface of
the earth, and greatly on other planets.

m4_ifelse(__me,1,[:%
:],[:%
Positive and negative signs of acceleration are defined
according to whether the $v-t$ graph slopes up or down. 
This
definition has the advantage that a force with a given sign,
representing its direction,
always produces an acceleration with the same sign.

The area under the $v-t$ graph gives $\Delta x$, and
analogously the area under the $a-t$ graph gives $\Delta v$.
:])

For motion with constant acceleration, the following
three equations hold:
\begin{align*}
  \Delta x &= v_\zu{o}\Delta t + \frac{1}{2}a\Delta t^2 \\
  v_f^2 &= v_\zu{o}^2 + 2 a \Delta x \\
  a &= \frac{\Delta v}{\Delta t}
\end{align*}
They are not valid if the acceleration is changing.

\end{summarytext}

\end{summary}

\vfill\pagebreak[4]

<% if is_web then print "<p>The following form can be used for the homework problems that require sketching a set of graphs.</p>" end
%>
m4_ifelse(__me,0,[:\fullpagewidthfignocaption{../../../share/mechanics/figs/hw-xva-form}\label{fig:hw-xva-form}:])

\vfill\pagebreak[4]
