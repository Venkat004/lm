%%%%%
%%%%% This problem is used by: 1np,10,upsilon-andromedae
%%%%%
 <% hw_solution %> Astronomers have detected a solar system consisting
of three planets orbiting the star Upsilon Andromedae. The
planets have been named b, c, and d. Planet b's
average distance from the star is 0.059 A.U., and planet c's
average distance is 0.83 A.U., where an astronomical unit or
A.U. is defined as the distance from the Earth to the sun.
For technical reasons, it is possible to determine the
ratios of the planets' masses, but their masses cannot
presently be determined in absolute units. Planet c's mass
is 3.0 times that of planet b. Compare the star's average
gravitational force on planet c with its average force on
planet b. [Based on a problem by Arnold Arons.]
