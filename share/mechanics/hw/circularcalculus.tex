%%%%%
%%%%% This problem is used by: 1np,9,circularcalculus
%%%%%
 In this problem, you'll derive the equation
$|\vc{a}|=|\vc{v}|^2/r$ using calculus. Instead of comparing
velocities at two points in the particle's motion and then
taking a limit where the points are close together, you'll
just take derivatives. The particle's position vector is
$\vc{r}=(r \cos\theta)\hat{\vc{x}} + (r\sin\theta)\hat{\vc{y}}$, where $\hat{\vc{x}}$ and $\hat{\vc{y}}$
are the unit vectors along the $x$ and $y$ axes. By the
definition of radians, the distance traveled since $t=0$ is
$r\theta $, so if the particle is traveling at constant
speed $v=|\vc{v}|$, we have $v=r\theta $/t.\hwendpart
 %
 (a) Eliminate $\theta$ to get the particle's position vector as a function of
time.\hwendpart
 %
 (b) Find the particle's acceleration vector.\hwendpart
 %
 (c) Show
that the magnitude of the acceleration vector equals $v^2/r$.
