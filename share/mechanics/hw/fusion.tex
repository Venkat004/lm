<% hw_solution %> Our sun is powered by nuclear fusion reactions, and as a first step in these reactions, one proton must
approach another proton to within a short enough range $r$. This is difficult to achieve, because the protons
have electric charge $+e$ and
therefore repel one another electrically. (It's a good thing that it's so difficult, because otherwise the sun would use up
all of its fuel very rapidly and explode.) To make fusion possible, the protons must be moving fast enough to come
within the required range. Even at the high temperatures present in the core of our sun, almost none of the protons
are moving fast enough.\\
(a) For comparison, the early universe, soon after the Big Bang, had extremely high temperatures. Estimate the
temperature $T$ that would have been required so that protons with average energies could fuse. State your result
in terms of $r$, the mass $m$ of the proton, and universal constants.\\
(b) Show that the units of your answer to part a make sense.\\
(c) Evaluate your result from part a numerically, using $r=10^{-15}\ \munit$ and
$m=1.7\times10^{-27}\ \kgunit$. As a check, you should find that this is much hotter than the
sun's core temperature of $\sim10^7\ \text{K}$.
