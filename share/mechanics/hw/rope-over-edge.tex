%%%%%
%%%%% This problem is used by: 2cl,4,rope-over-edge
%%%%%
A flexible rope of mass $m$ and length $L$ slides
without friction over the edge of a table. Let $x$ be the
length of the rope that is hanging over the edge at a
given moment in time.\\
(a) Show that $x$ satisfies the equation of motion $\der^2x/\der t^2=gx/L$.
[Hint: Use $F=\der p/\der t$, which allows you to handle the two parts of the rope
separately even though mass is moving out of one part and into the other.]\hwendpart
(b) Give a physical explanation for the fact that a larger
value of $x$ on the right-hand side of the equation leads to
a greater value of the acceleration on the left side.\hwendpart
(c) When we take the second derivative of the function
$x(t)$ we are supposed to get essentially the same function
back again, except for a constant out in front. The function
$e^x$ has the property that it is unchanged by differentiation,
so it is reasonable to look for solutions to this problem
that are of the form $x=be^{ct}$, where $b$ and $c$ are
constants. Show that this does indeed provide a solution for
two specific values of $c$ (and for any value of $b)$.\hwendpart
(d) Show that the sum of any two solutions to the equation
of motion is also a solution.\hwendpart
(e) Find the solution for the case where the rope starts at
rest at $t=0$ with some nonzero value of $x$.
