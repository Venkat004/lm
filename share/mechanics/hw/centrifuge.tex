When we talk about rigid-body rotation, the concept of a perfectly rigid body can only be
an idealization. In reality, any object will compress, expand, or deform to some extent when
subjected to the strain of rotation. However, if we let it settle down for a while, perhaps
it will reach a new equilibrium. As an example, suppose we fill a centrifuge tube with some
compressible substance like shaving cream or Wonder Bread. We can model the contents of the
tube as a one-dimensional line of mass, extending from $r=0$ to $r=\ell$. Once the rotation starts,
we expect that the contents will be most compressed near the ``floor'' of the tube at $r=\ell$; this
is both because the inward force required for circular motion increases with $r$ for a fixed $\omega$,
and because the part at the floor has the greatest amount of material pressing ``down'' (actually
outward) on it. The linear density $\der m/\der r$, in units of kg/m, should therefore increase
as a function of $r$. Suppose that we have $\der m/\der r=\mu e^{r/\ell}$, where $\mu$ is a constant.
Find the moment of inertia.\answercheck
