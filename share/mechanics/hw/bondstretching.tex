%%%%%
%%%%% This problem is used by: 1np,5,bondstretching
%%%%%
% meta {"stars":1}
 This problem depends on the results of problems \ref{hw:combine-springs} and \ref{hw:youngs-modulus},
whose solutions are in the back of the book. When atoms
form chemical bonds, it makes sense to talk about the spring
constant of the bond as a measure of how ``stiff'' it is. Of
course, there aren't really little springs --- this is just
a mechanical model. The purpose of this problem is to
estimate the spring constant, $k$, for a single bond in a
typical piece of solid matter. Suppose we have a fiber, like
a hair or a piece of fishing line, and imagine for
simplicity that it is made of atoms of a single element
stacked in a cubical manner, as shown in the figure, with a
center-to-center spacing $b$. A typical value for $b$ would
be about $10^{-10}\ \munit$.\hwendpart
 %
(a) Find an equation for $k$ in terms
of $b$, and in terms of the Young's modulus, $E$, defined in
problem 16 and its solution.\hwendpart
 %
 (b) Estimate $k$ using the
numerical data given in problem \ref{hw:youngs-modulus}.\hwendpart
 (c) Suppose you could
grab one of the atoms in a diatomic molecule like $\zu{H}_2$ or
$\zu{O}_2$, and let the other atom hang vertically below it. Does
the bond stretch by any appreciable fraction due to gravity?
