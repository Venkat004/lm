In a well known stunt from circuses and carnivals, a motorcyclist rides around inside
a big bowl, gradually speeding up and rising higher. Eventually the cyclist can get
up to where the walls of the bowl are vertical. Let's estimate the conditions under
which a running human could do the same thing.\hwendpart (a) If the runner can run
at speed $v$, and her shoes have a coefficient of static friction $\mu_s$, what is
the maximum radius of the circle?\answercheck\hwendpart
%
(b) Show that the units of your answer make sense.\hwendpart
%
(c) Check that its dependence on the variables makes sense.\hwendpart
%
(d) Evaluate your result numerically for
$v=10\ \munit/\sunit$ (the speed of an olympic sprinter) and $\mu_s=5$. (This is
roughly the highest coefficient of static friction ever achieved for surfaces that
are not sticky. The surface has an array of microscopic fibers like a hair brush,
and is inspired by the hairs on the feet of a gecko. These assumptions are not
necessarily realistic, since the person would have to run at an angle, which would
be physically awkward.)\answercheck\hwendpart
