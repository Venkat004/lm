The figure, redrawn from \emph{Gray's Anatomy}, shows the tension of which a muscle is capable.
The variable $x$ is defined as the contraction of the muscle from its maximum length $L$, so that at
$x=0$ the muscle has length $L$, and at $x=L$ the muscle would theoretically have zero length.
In reality, the muscle can only contract to $x=cL$, where $c$ is less than 1.
When the muscle is extended to its maximum length, at $x=0$, it is capable of the greatest tension, $T_\zu{o}$.
As the muscle contracts, however, it becomes weaker. Gray suggests approximating this function as a linear
decrease, which would theoretically extrapolate to zero at $x=L$.
(a) Find the maximum work the muscle can do in one contraction, in terms of $c$, $L$, and $T_\zu{o}$.\answercheck\hwendpart
(b) Show that your answer to part a has the right units.\hwendpart
(c) Show that your answer to part a has the right behavior when $c=0$ and when $c=1$.\hwendpart
(d) Gray also states that the absolute maximum tension $T_\zu{o}$ has been found to be approximately
proportional to the muscle's cross-sectional area $A$ (which is presumably measured at $x=0$), with proportionality constant
$k$.
Approximating the muscle as a cylinder,
show that your answer from part a can be reexpressed in terms of the volume, $V$, eliminating $L$ and $A$.\answercheck\hwendpart
(e) Evaluate your result numerically for a biceps muscle with a volume of 200 $\zu{cm}^3$, with $c=0.8$ and
$k=100\ \nunit/\zu{cm}^2$ as estimated by Gray.\answercheck\hwendpart
