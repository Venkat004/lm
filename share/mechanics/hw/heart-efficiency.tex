%%%%%
%%%%% This problem is used by: 0sn,5,heart-efficiency 2cl,6,heart-efficiency
%%%%%
Even when resting, the human body needs to do a certain amount of mechanical
work to keep the heart beating. This quantity is difficult to define and
measure with high precision, and also depends on the individual and
her level of activity,
but it's estimated to be about 1 to 5 watts. Suppose we consider the human
body as nothing more than a pump. A person who is just lying in bed all
day needs about 1000 kcal/day worth of food to stay alive.
(a) Estimate the person's thermodynamic efficiency as a pump, and (b) compare with
the maximum possible efficiency imposed by the laws of thermodynamics for
a heat engine operating across the difference between a body
temperature of $37\degcunit$ and an ambient temperature of
$22\degcunit$. (c) Interpret your answer.<% hw_answer %>
