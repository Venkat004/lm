%%%%%
%%%%% This problem is used by: 1np,3,dodge-viper
%%%%%
 <% hw_solution %> In July 1999, Popular Mechanics carried out tests to
find which car sold by a major auto maker could cover a
quarter mile (402 meters) in the shortest time, starting
from rest. Because the distance is so short, this type of
test is designed mainly to favor the car with the greatest
acceleration, not the greatest maximum speed (which is
irrelevant to the average person). The winner was the Dodge
Viper, with a time of 12.08 s. The car's top (and
presumably final) speed was 118.51 miles per hour (52.98
\ \munit/\sunit). (a) If a car, starting from rest and moving with
\emph{constant} acceleration, covers a quarter mile in this
time interval, what is its acceleration? (b) What would be
the final speed of a car that covered a quarter mile with
the constant acceleration you found in part a? (c) Based on
the discrepancy between your answer in part b and the
actual final speed of the Viper, what do you conclude about
how its acceleration changed over time?
