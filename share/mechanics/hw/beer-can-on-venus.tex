Suppose we want to send a space probe to Venus and have it
release a balloon that can float over the landscape and collect
data. The venerian atmosphere is hot and corrosive, so it would
destroy the kind of mylar or rubber balloon we use on earth.
The purpose of this problem is to get a feel for things by estimating whether an aluminum beer
can full of helium would float on Venus. Here are some data:

\noindent\begin{tabular}{p{40mm}l}
density of atmosphere & $67\ \kgunit/\munit^3$ \\
density of helium & $6.0\ \kgunit/\munit^3$ \\
mass of beer can & 15 g \\
volume of beer can & $330\ \zu{cm}^3$
\end{tabular}

\noindent Find the ratio $F_B/F_g$ of the buoyant force to the can's weight. Does it float?\answercheck\hwendpart

% density of helium:
% calc -x -e "P=9.2 MPa; mp=1.67 10^-27 kg; m=4mp; T=740; rho=mP/(kBT)->kg/m3; ~->g/cm3"
%     rho = 6.01518477881062 kg/m3
%    6.01518477881062*10^-3 g/cm3
