When you're done using an electric mixer, you can get
most of the batter off of the beaters by lifting them out of
the batter with the motor running at a high enough speed.
Let's imagine, to make things easier to visualize, that we
instead have a piece of tape stuck to one of the beaters.\hwendpart
 %
(a) Explain why static friction has no effect on whether or
not the tape flies off.\hwendpart
%
(b) Analyze the forces in which the tape participates, using
a table in the format shown in __subsection_or_section(analysis-of-forces).\hwendpart
 %
(c) Suppose you find that the tape
doesn't fly off when the motor is on a low speed, but
at a greater speed, the tape won't stay on. Why would the
greater speed change things? [Hint: If you don't invoke any
law of physics, you haven't explained it.]
