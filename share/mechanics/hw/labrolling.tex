%%%%%
%%%%% This problem is used by: 1np,8,labrolling
%%%%%
 The figure shows an experiment in which a cart is
released from rest at A, and accelerates down the slope
through a distance $x$ until it passes through a sensor's
light beam. The point of the experiment is to determine the
cart's acceleration. At B, a cardboard vane mounted on the
cart enters the light beam, blocking the light beam, and
starts an electronic timer running. At C, the vane emerges
from the beam, and the timer stops.\hwendpart
 %
 (a) Find the final
velocity of the cart in terms of the width $w$ of the vane
and the time $t_b$ for which the sensor's light beam was
blocked.\answercheck\hwendpart
 %
 (b) Find the magnitude of the cart's acceleration
in terms of the measurable quantities $x$, $t_b$, and $w$.\answercheck\hwendpart
 %
(c) Analyze the forces in which the cart participates, using
a table in the format introduced in section \ref{sec:analysis-of-forces}. Assume
friction is negligible.\hwendpart
 %
(d) Find a theoretical value for the
acceleration of the cart, which could be compared with the
experimentally observed value extracted in part $b$. Express
the theoretical value in terms of the angle $\theta $ of the
slope, and the strength $g$ of the gravitational field.\answercheck
