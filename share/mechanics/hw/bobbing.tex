%%%%%
%%%%% This problem is used by: 3vw,1,bobbing
%%%%%
Archimedes' principle states that an object partly or
wholly immersed in fluid experiences a buoyant force equal
to the weight of the fluid it displaces. For instance, if a
boat is floating in water, the upward pressure of the water
(vector sum of all the forces of the water pressing inward
and upward on every square inch of its hull) must be equal
to the weight of the water displaced, because if the boat
was instantly removed and the hole in the water filled back
in, the force of the surrounding water would be just the
right amount to hold up this new ``chunk'' of water. (a)
Show that a cube of mass $m$ with edges of length $b$
floating upright (not tilted) in a fluid of density $\rho $
will have a draft (depth to which it sinks below the
waterline) $h$ given at equilibrium by $h_0=m/b^2\rho$. (b) Find the total
force on the cube when its draft is $h$, and verify that
plugging in $h-h_0$ gives a total force of zero. (c) Find the
cube's period of oscillation as it bobs up and down in the
water, and show that can be expressed in terms of and $g$ only.
\answercheck
