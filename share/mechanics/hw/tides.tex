%%%%%
%%%%% This problem is used by: 0sn,2,tides
%%%%%
        The purpose of this problem is to estimate the height of the tides. The main
        reason for the tides is the moon's gravity, and we'll neglect the effect of the sun.
        Also, real tides are heavily influenced by landforms that channel the flow of
        water, but we'll think of the earth as if it was completely covered with oceans.
        Under these assumptions, the ocean surface should be a surface of constant
        $U/m$. That is, a thimbleful of water, $m$, should not be able to gain or lose
        any gravitational energy by moving from one point on the ocean surface to
        another. If only the spherical earth's gravity was present, then we'd have
        $U/m=-GM_e/r$, and a surface of constant $U/m$ would be a surface of
        constant $r$, i.e., the ocean's surface would be spherical. Taking into account
        the moon's gravity, the main effect is to shift the center of the sphere, but the
        sphere also becomes slightly distorted into an approximately ellipsoidal shape.
        (The shift of the center is not physically related to the tides, since the solid part
        of the earth tends to be centered within the oceans; really, this effect has to
        do with the motion of the whole earth through space, and the way that it
        wobbles due to the moon's gravity.) Determine the amount by which the long
        axis of the ellipsoid exceeds the short axis.
        <% hw_hint("tides") %>
