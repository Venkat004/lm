Most people don't know that \emph{Spinosaurus aegyptiacus}, not \emph{Tyrannosaurus rex},
was the biggest theropod dinosaur. We can't put a dinosaur on a track and time it in the 100 meter
dash, so we can only infer from physical models how fast it could have run. When an
animal walks at a normal pace, typically its legs swing more or less like pendulums of
the same length $\ell$. As a further simplification of this model, let's imagine that the leg
simply moves at a fixed acceleration as it falls to the ground.
That is, we model the time for a quarter of a stride cycle as being
the same as the time required for free fall from a height $\ell$.
\emph{S.~aegyptiacus} had legs about four times longer than those of a human.
(a) Compare the time required for a human's stride cycle to that for \emph{S.~aegyptiacus}.\answercheck\hwendpart
(b) Compare their running speeds.\answercheck
