%%%%%
%%%%% This problem is used by: 3vw,1,oblate-earth
%%%%%
A hot scientific question of the 18th century was the shape
of the earth: whether its radius was greater at the equator than at the poles, or the
other way around. One method used to attack this question was to measure gravity
accurately in different locations on the earth using pendula. If the highest
and lowest latitudes accessible to explorers were 0 and 70 degrees, then
the the strength of gravity would in reality be observed to vary over a range from
about 9.780 to 9.826 $\munit/\sunit^2$. This change, amounting to 0.046 $\munit/\sunit^2$,
is greater than the 0.022 $\munit/\sunit^2$ effect to be expected if the earth had been spherical.
The greater effect occurs because the equator feels a reduction due not just to the
acceleration of the spinning earth out from under it, but also to the greater radius of
the earth at the equator. What is the accuracy with which the period
of a one-second pendulum would have to be measured in order to prove that the earth
was not a sphere, and that it bulged at the equator?
