%%%%%
%%%%% This problem is used by: 2cl,2,honeymoon
%%%%%
\answercheck Lord Kelvin, a physicist, told the story of how he
encountered James Joule when Joule was on his honeymoon. As
he traveled, Joule would stop with his wife at various
waterfalls, and measure the difference in temperature
between the top of the waterfall and the still water at the
bottom. (a) It would surprise most people to learn that the
temperature increased. Why should there be any such effect,
and why would Joule care? How would this relate to the
energy concept, of which he was the principal inventor? (b)
How much of a gain in temperature should there be between
the top and bottom of a 50-meter waterfall? (c) What
assumptions did you have to make in order to calculate your
answer to part b? In reality, would the temperature change
be more than or less than what you calculated? [Based on a
problem by Arnold Arons.]
