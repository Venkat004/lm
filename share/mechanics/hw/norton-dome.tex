% meta {"stars":1}
In 2003, physicist and philosopher John Norton came up with the following
apparent paradox, in which Newton's laws, which appear deterministic,
can produce nondeterministic results.
Suppose that a bead moves frictionlessly on a curved wire under the
influence of gravity. The shape of the wire is defined by the function
$y(x)$, which passes through the origin, and the bead is released from rest at the
origin. For convenience of notation, choose units such that $g=1$, and
define $\dot{y}=\der y/\der t$ and $y'=\der y/\der x$.\\
(a) Show that the equation of motion is
\begin{equation*}
  y = -\frac{1}{2}\dot{y}^2\left(1+y'^{-2}\right).
\end{equation*}\hwendpart
(b) To simplify the calculations, assume from now on that $y'\ll 1$.
Find a shape for the wire such that $x=t^4$ is a solution.
(Ignore units.)\answercheck\hwendpart
(c) Show that not just the motion assumed in part b, but
any motion of the following form is a solution:
\begin{equation*}
  x = \begin{cases}
    0          & \text{if}\ t\le t_0 \\
    (t-t_0)^4  & \text{if}\ t\ge t_0 
  \end{cases}
\end{equation*}
This is remarkable because there is no physical principle that determines
$t_0$, so if we place the bead at rest at the origin, there is no way
to predict when it will start moving.
