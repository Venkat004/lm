A cross-country skier is gliding on a level trail, with negligible friction.
Then, when he is at position $x=0$, the tip of his skis enters a patch of
dirt. As he rides onto the dirt, more and more of his weight is being supported
by the dirt. The skis have length $\ell$, so if he reached $x=\ell$ without stopping,
his weight would be completely on the dirt.
This problem deals with the motion for $x<\ell$.
(a) Find the acceleration in terms of $x$, as well as any other relevant constants.\hwendpart
(b) This is a second-order differential equation. You should be able to find the
solution simply by thinking about some commonly occuring functions that you know
about, and finding two that have the right properties. If these functions are $x=f(t)$ and
$x=g(t)$, then the most general solution to the equations of motion will be of the
form $x=af+bg$, where $a$ and $b$ are constants to be determined from the initial conditions.\hwendpart
(c) Suppose that the initial velocity $v_\zu{o}$ at $x=0$ is such that he stops at $x<\ell$.
Find the time until he stops, and show that, counterintuitively, this time is independent of $v_\zu{o}$.
Explain physically why this is true.\answercheck\hwendpart
