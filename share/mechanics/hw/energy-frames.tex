%%%%%
%%%%% This problem is used by: 2cl,4,energy-frames
%%%%%
<% hw_solution %> Suppose a system consisting of pointlike particles
has a total kinetic energy $K_{cm}$ measured in the
center-of-mass frame of reference. Since they are pointlike,
they cannot have any energy due to internal motion.\\
(a) Prove that in a different frame of reference, moving with
velocity $\vc{u}$ relative to the center-of-mass frame, the total
kinetic energy equals $K_{cm}+M|\vc{u}|^2/2$, where $M$ is the
total mass. [Hint: You can save yourself a lot of writing if
you express the total kinetic energy using the dot product.]\hwendpart
(b) Use this to prove that if energy is conserved in one
frame of reference, then it is conserved in every frame of
reference. The total energy equals the total kinetic energy
plus the sum of the potential energies due to the particles'
interactions with each other, which we assume depends only
on the distance between particles.
m4_ifelse(__problems,1,[::],[:[For a simpler numerical
example, see problem \ref{hw:balls} on p.~\pageref{hw:balls}.]:])
