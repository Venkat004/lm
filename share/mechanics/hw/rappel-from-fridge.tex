The figure shows a mountaineer doing a vertical rappel. Her anchor is a big
boulder. The
American Mountain Guides Association suggests as a rule of thumb that in
this situation, the boulder should be at least as big as a refrigerator,
and should be sitting on a surface that is horizontal rather than sloping.
The goal of this problem is to estimate what coefficient of static friction
$\mu_s$ between the boulder and the
ledge
is required if this setup is to hold the person's body weight. For comparison, reference books
meant for civil engineers building walls out of granite blocks state that
granite on granite typically has a $\mu_s\approx 0.6$. We expect the result of
our calculation to be much less than this, both because a large margin of safety
is desired and because the coefficient could be much lower if, for example, the
surface was sandy rather than clean. We will assume that there is no friction where the rope goes over the
lip of the cliff, although in reality this friction significantly reduces the load on the
boulder.\\
%
(a) Let $m$ be the mass of the climber, $V$
the volume of the boulder, $\rho$ its density, and $g$ the strength of the gravitational
field. Find the minimum value of $\mu_s$. \answercheck\hwendpart
%
(b) Show that the units of your answer make sense.\hwendpart
%
(c) Check that its dependence on the variables makes sense.\hwendpart
%
(d) Evaluate your result numerically. The volume of my refrigerator is about $0.7\ \munit^3$,
the density of granite is about $2.7\ \zu{g}/\zu{cm}^3$, and standards bodies use a body mass
of 80 kg for testing climbing equipment.\answercheck\hwendpart
