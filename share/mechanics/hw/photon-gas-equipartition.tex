(a) Consider a \emph{one-dimensional} ideal gas consisting of $n$ material particles, at temperature
$T$. Trace back through the logic of the equipartition theorem
on p.~\pageref{equipartition} to determine the total energy.\\
(b) Explain why it should matter how many dimensions there are.\\
(c) Gases that we encounter in everyday life are made of atoms, but there are
gases made out of other things. For example, soon after the big bang, there
was a period when the universe was very hot and dominated by light rather
than matter. A particle of light is called a photon, so the early universe
was a ``photon gas.'' For simplicity, consider a photon gas in one dimension.
Photons are massless, and we will see in ch.~\ref{ch:rel} on relativity that
for a massless particle, the energy is related to the momentum by $E=pc$,
where $c$ is the speed of light. (Note that $p=mv$ does \emph{not} hold for
a photon.) Again, trace back through the logic of equipartition
on p.~\pageref{equipartition}. Does the photon gas have the same heat capacity
as the one you found in part a?
