Suppose we have a mass filament like the one described in problems
\ref{hw:g-field-of-straight-mass-filament} and \ref{hw:g-field-of-filament-order-of-limits}, but now rather than taking it to be
straight, let it have the shape of an arbitrary smooth curve. Locally, ``under a
microscope,'' this curve will look like an arc of a circle, i.e., we can describe its
shape solely in terms of a radius of curvature. 
As in problem \ref{hw:g-field-of-filament-order-of-limits},
consider a point P lying \emph{on} the filament itself, taking $g$ to be defined as in definition $g_1$.
Investigate whether $g$ is finite, and also whether it points in a specific direction. To clarify the
mathematical idea, consider the following two limits:
\begin{align*}
  A &=\lim_{x\rightarrow0} \frac{1}{x} \qquad \text{and}\\
  B &=\lim_{x\rightarrow0} \frac{1}{x^2}. 
\end{align*}
We say that $A=\infty$, while $B=+\infty$, i.e., both diverge, but $B$ diverges with a definite sign. 
For a straight filament,
as in problem \ref{hw:g-field-of-straight-mass-filament}, with an infinite radius of
curvature, symmetry guarantees that the field at P has no specific direction, in analogy with limit $A$.
For a curved filament, a calculation is required in order to determine whether we get behavior $A$ or $B$.
