%%%%%
%%%%% This problem is used by: 0sn,5,violin-helmholtz
%%%%%
Example \ref{eg:helmholtz-resonator} on page \pageref{eg:helmholtz-resonator} suggests
analyzing the resonance of a violin at 300 Hz as a Helmholtz resonance. However,
we might expect the equation for the frequency of a Helmholtz resonator to
be a rather crude approximation here, since the f-holes are not long tubes, but
slits cut through the face of the instrument, which is only about 2.5 mm thick. (a) Estimate the frequency
that way anyway, for a violin with a volume of about 1.6 liters, and f-holes with
a total area of 10 $\zu{cm}^2$. (b) A common rule of thumb is that at an open end of
an air column, such as the neck of a real Helmholtz resonator, some air beyond the
mouth also vibrates as if it was inside the tube, and that this effect can be taken
into account by adding 0.4 times the diameter of the tube for each open end (i.e.,
0.8 times the diameter when both ends are open). Applying
this to the violin's f-holes results in a huge change in $L$, since the $\sim 7$ mm width
of the f-hole is considerably greater than the thickness of the wood. Try it, and
see if the result is a better approximation to the observed frequency of the
resonance.<% hw_answer %>
