In running races at distances of 800 meters and longer,
runners do not have their own lanes, so in order to pass,
they have to go around their opponents. Suppose we adopt
the simplified geometrical model suggested by the figure,
in which the two runners take equal times to trace out the 
sides of an isoceles triangle, deviating from parallelism by the
angle $\theta$. The runner going straight runs at speed $v$,
while the one who is passing must run at a greater speed. Let
the difference in speeds be $\Delta v$.\\
(a) Find $\Delta v$ in terms of $v$ and $\theta$.\answercheck\hwendpart
(b) Check the units of your equation using the method shown in example \ref{eg:checking-units}
on p.~\pageref{eg:checking-units}.\hwendpart
(c) Check that your answer makes sense in the special case where $\theta=0$, i.e.,
in the case where the runners are on an extremely long straightaway.\hwendpart
(d) Suppose that $\theta=1.0$ degrees, which is about the smallest value
that will allow a runner to pass in the distance available on the straightaway
of a track, and let $v=7.06\ \munit/\sunit$, which is the women's world record pace
at 800 meters.
Plug numbers into your equation from part a to determine $\Delta v$, and
comment on the result.\answercheck\hwendpart
% calc -e "t=(60+53.28)(1 s); x=800 m; x/t"
%    7.06214689265537 m/s
