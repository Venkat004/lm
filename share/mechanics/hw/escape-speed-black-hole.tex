The \emph{escape velocity} of a massive body is the
speed for which the total energy of a projectile is zero: the projectile
has just enough KE to move infinitely far away from the massive body,
with no left-over KE. The escape velocity depends on the distance
from which the projectile is launched --- often the body's surface.

The Schwarzschild radius ($R_s$) of a massive
body is the radius where the escape velocity is equal to the speed of
light, $c = 3.00 \times 10^8\ \munit/\sunit$. An object is called a
\emph{black hole} if it has a Schwarzschild
radius. 

An object must be very compact to be a black hole. For example,
escape velocity from the surface of the earth is tens of thousands of
times less than $c$, as is the escape velocity for a projectile launched
from the center of the earth through a hypothetical radial, evacuated tunnel.

In this problem we will make some numerical estimates of how compact an object
has to be in order to be a black hole. We will use Newtonian gravity, which
is a poor approximation for such strong gravitational fields, so we expect these
estimates to be rough.\\
%
(a) For an object of mass $M$, what would its radius have to be if all of its
mass was to fit within the Schwarzschild radius?\answercheck\hwendpart
%
(b) Evaluate your equation from part a for $M$ equal to the masses of the
earth and the sun. If these bodies were compressed to approximately these
sizes, they would become black holes. (Because these are rough estimates, treat them as having
no more than 1 significant figure.)\answercheck
