In 1849, Fizeau carried out the first terrestrial measurement of the speed of light;
previous measurements by Roemer and Bradley had involved astronomical observation.
The figure shows a simplified conceptual representation of Fizeau's experiment.
A ray of light from a bright source was directed through the teeth at the edge of
a spinning cogwheel. After traveling a distance $L$, it was reflected from a mirror
and returned along the same path. The figure shows the case in which the ray passes
between two teeth, but when it returns, the wheel has rotated by half the spacing
of the teeth, so that the ray is blocked. When this condition is achieved, the
observer looking through the teeth toward the far-off mirror sees it go completely
dark. Fizeau adjusted the speed of the wheel to achieve this condition and recorded
the rate of rotation to be $f$ rotations per second. Let the number of teeth on the
wheel be $n$.\\
(a) Find the speed of light $c$ in terms of $L$, $n$, and $f$.\answercheck\hwendpart
(b) Check the units of your equation using the method shown in example \ref{eg:checking-units}
on p.~\pageref{eg:checking-units}. (Here $f$'s units of rotations per second should be
taken as inverse seconds, $\sunit^{-1}$, since the number of rotations in a second is
a unitless count.)\hwendpart
(c) Imagine that you are Fizeau trying to design this experiment. The speed of light is
a huge number in ordinary units. Use your equation from part a to determine whether
increasing $c$ requires an increase in $L$, or a decrease. Do the same for $n$ and $f$.
Based on this, decide for each of these variables whether you want a value that is as
big as possible, or as small as possible.\hwendpart
(d) Fizeau used $L=8633\ \munit$, $f=12.6\ \sunit^{-1}$, and $n=720$. Plug in to your
equation from part a and extract the speed of light from his data.\answercheck\hwendpart
