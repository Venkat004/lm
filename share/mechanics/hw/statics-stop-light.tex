A stoplight consists of three pieces of mass $M$: a vertical bar
mounted at the center of a base of width $w$, a horizontal bar of
length $D$, and the stoplight fixture itself. (The two bars have a
uniform mass distribution.) The structure is held in static
equilibrium from the ground and from screws at the base of the
structure. For simplicity, we take there to be only two screws, one
on the left and one on the right side of the base of width $w \ll D$.
Assume that the entire normal force from the ground acts on the
right-hand side of the base. (This is where the structure would
naturally pivot.)\\
%
(a) By taking the right side of the base as your pivot point, you
should be able to easily see that the screw on the left must provide
a downwards force to keep the stoplight in static equilibrium. What
is the magnitude of this force?\answercheck\hwendpart
%
(b) Find the upwards normal force acting on the base.\answercheck
