%%%%%
%%%%% This problem is used by: 1np,3,railgun
%%%%%
% meta {"stars":1}
 The speed required for a low-earth orbit is 
$7.9\times10^3\ \munit/\sunit$ (see ch. 10). When a rocket is launched into orbit, it
goes up a little at first to get above almost all of the
atmosphere, but then tips over horizontally to build up to
orbital speed. Suppose the horizontal acceleration is
limited to $3g$ to keep from damaging the cargo (or hurting
the crew, for a crewed flight). (a) What is the minimum
distance the rocket must travel downrange before it reaches
orbital speed? How much does it matter whether you take into
account the initial eastward velocity due to the rotation of
the earth? (b) Rather than a rocket ship, it might be
advantageous to use a railgun design, in which the craft
would be accelerated to orbital speeds along a railroad
track. This has the advantage that it isn't necessary to
lift a large mass of fuel, since the energy source is
external. Based on your answer to part a, comment on the
feasibility of this design for crewed launches from
the earth's surface.
