%%%%%
%%%%% This problem is used by: 2cl,3,anchor 2cl,3,anchor
%%%%%
(a) The crew of an 18th century warship is raising the
anchor. The anchor has a mass of 5000 kg. The water is 30
m deep. The chain to which the anchor is attached has a
mass per unit length of 150 kg/m. Before they start raising
the anchor, what is the total weight of the anchor plus the
portion of the chain hanging out of the ship? (Assume that
the buoyancy of the anchor is negligible.)\hwendpart
 %
(b) After they have raised the anchor by 1 m, what is the
weight they are raising?\hwendpart
 %
(c) Define $y=0$ when the anchor is resting on the bottom,
and $y=+30$ m when it has been raised up to the ship. Draw a
graph of the force the crew has to exert to raise the anchor
and chain, as a function of $y$. (Assume that they are
raising it slowly, so water resistance is negligible.) It
will not be a constant! Now find the area under the graph,
and determine the work done by the crew in raising
the anchor, in joules.\hwendpart
 %
(d) Convert your answer from (c) into units of kcal. \answercheck
