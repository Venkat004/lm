In the power stroke of a car's gasoline engine, the
fuel-air mixture is ignited by the spark plug, explodes, and
pushes the piston out. The exploding mixture's force on the
piston head is greatest at the beginning of the explosion,
and decreases as the mixture expands. It can be approximated
by $F=a/x$, where $x$ is the distance from the cylinder to
the piston head, and $a$ is a constant with units of $\nunit\unitdot\munit$.
(Actually $a/x^{1.4}$ would be more accurate, but the problem
works out more nicely with $a/x$!) The piston begins its
stroke at $x=x_1$, and ends at $x=x_2$. \\
 %
(a) Find the amount of 
work done in one stroke by one cylinder.\answercheck\hwendpart
 %
(b) The 1965 Rambler had
six cylinders, each with $a=220\ \nunit\unitdot\munit$, $x_1=1.2$ cm, and $x_2=10.2$ cm.
Assume the engine is running at 4800 r.p.m., so that 
during one minute, each of the six cylinders performs 2400
power strokes. (Power strokes only happen every other
revolution.) Find the engine's power, in units of horsepower (1 hp=746 W).\answercheck\hwendpart
 %
(c) The compression ratio of an engine is defined as
$x_2/x_1$. Explain in words why the car's power would be
exactly the same if $x_1$ and $x_2$ were, say, halved or
tripled, maintaining the same compression ratio of 8.5.
Explain why this would \emph{not} quite be true with the
more realistic force equation $F=a/x^{1.4}$.
