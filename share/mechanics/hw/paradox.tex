%%%%%
%%%%% This problem is used by: 1np,9,paradox
%%%%%
 The acceleration of an object in uniform circular motion
can be given either by $|\vc{a}|=|\vc{v}|^2/r$ or, equivalently, by
$|\vc{a}|=4\pi^2r/T^2$, where $T$ is the time required for one
cycle (example \ref{eg:accel-rt} on page \pageref{eg:accel-rt}). Person A says based on the first equation that
the acceleration in circular motion is greater when the
circle is smaller. Person B, arguing from the second
equation, says that the acceleration is smaller when the
circle is smaller. Rewrite the two statements so that they
are less misleading, eliminating the supposed paradox.
[Based on a problem by Arnold Arons.]
