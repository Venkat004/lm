<% hw_solution %> Typically the atmosphere gets colder with increasing altitude.
However, sometimes there is an \emph{inversion layer}, in which this trend is
reversed, e.g., because a less dense mass of warm air moves into a certain area,
and rises above the denser colder air that was already present. Suppose that this causes the pressure $P$ as a function
of height $y$ to be given by a function of the form $P=P_o e^{-ky}(1+by)$, where constant
temperature would give $b=0$ and an inversion layer would give $b>0$. (a)
Infer the units of  the constants $P_o$, $k$, and $b$.
(b) Find the density of the
air as a function of $y$, of the constants, and of the acceleration of gravity $g$.
(c) Check that the units of your answer to part b make sense.
