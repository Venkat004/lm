%%%%%
%%%%% This problem is used by: 0sn,9,teardrop 4em,3,teardrop
%%%%%
        As discussed in the text, when a conductor reaches an
        equilibrium where its charge is at rest, there is always
        zero electric force on a charge in its interior, and any
        excess charge concentrates itself on the surface.  The
        surface layer of charge arranges itself so as to produce
        zero total force at any point in the interior.  (Otherwise
        the free charge in the interior could not be at rest.) 
        Suppose you have a teardrop-shaped conductor like the one
        shown in the figure.  Since the teardrop is a conductor,
        there are free charges everywhere inside it, but consider a
        free charged particle at the location shown with a white
        circle. Explain why, in order to produce zero force on this
        particle, the surface layer of charge must be denser in the
        pointed part of the teardrop.

        \hwremark{Similar reasoning shows why Benjamin Franklin used a sharp
        tip when he invented the lightning rod. The charged stormclouds
        induce positive and negative charges to move to opposite
        ends of the rod.  At the pointed upper end of the rod, the
        charge tends to concentrate at the point, and this charge
        attracts the lightning. The same effect can sometimes be seen when
        a scrap of aluminum foil is inadvertently put in a microwave oven. Modern experiments (Moore \emph{et al.},
        Journal of Applied Meteorology 39 (1999) 593) show that although
        a sharp tip is best at starting a spark, a more moderate curve,
        like the right-hand tip of the teardrop in this problem, is better
        at successfully sustaining the spark for long enough to connect
        a discharge to the clouds.}
