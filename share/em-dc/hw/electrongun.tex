%%%%%
%%%%% This problem is used by: 0sn,9,electrongun
%%%%%
        The figure shows a simplified diagram of an electron
        gun such as the 
        one that creates the electron beam in a TV tube. Electrons
        that spontaneously emerge from the negative electrode
        (cathode) are then accelerated to the positive electrode,
        which has a hole in it. (Once they emerge through the hole,
        they will slow down. However, if the two electrodes are
        fairly close together, this slowing down is a small effect,
        because the attractive and repulsive forces experienced by
        the electron tend to cancel.) \hwendpart
        (a) If the voltage difference
        between the electrodes is $\Delta V$, what is the velocity
        of an electron as it emerges at B? Assume that its initial
        velocity, at A, is negligible, and that the velocity is nonrelativistic. 
        (If you haven't read ch.~7 yet, don't worry about the remark about
        relativity.)
        \answercheck\hwendpart
        (b) Evaluate your
        expression numerically for the case where $\Delta V$=10 kV,
        and compare to the speed of light.
        m4_ifelse(__problems,1,[:%:],[:If you've read ch.~7 already,
        comment on whether the assumption
        of nonrelativistic motion was justified.:])%
        <% hw_solution %>\answercheck
