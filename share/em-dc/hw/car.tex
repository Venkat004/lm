%%%%%
%%%%% This problem is used by: 0sn,9,car 4em,3,car
%%%%%
        Today, even a big luxury car like a Cadillac can have an
        electrical system that is relatively low in power, since it
        doesn't need to do much more than run headlights, power
        windows, etc. In the near future, however, manufacturers
        plan to start making cars with electrical systems about five
        times more powerful. This will allow certain energy-wasting
        parts like the water pump to be run on  electrical motors
        and turned off when they're not needed --- currently they're
        run directly on shafts from the motor, so they can't be shut
        off. It may even be possible to make an engine that can shut
        off at a stoplight and then turn back on again without
        cranking, since the valves can be electrically powered.
        Current cars' electrical systems have 12-volt batteries
        (with 14-volt chargers), but the new systems will have
        36-volt batteries (with 42-volt chargers). \hwendpart
        (a) Suppose the
        battery in a new car is used to run a device that requires
        the same amount of power as the corresponding device in the
        old car. Based on the sample figures above, how would the
        currents handled by the wires in one of the new cars compare
        with the currents in the old ones?\hwendpart
         (b) The real purpose of
        the greater voltage is to handle devices that need
        \emph{more} power. Can you guess why they decided to change
        to 36-volt batteries rather than increasing the power
        without increasing the voltage?
        \answercheck
