Hybrid and electric cars have been gradually gaining market share, but
during the same period of time, manufacturers such as Porsche have also begun designing and
selling cars with ``mild hybrid'' systems, in which power-hungry parts like water pumps
are powered by a higher-voltage battery rather than running directly
on shafts from the motor. Traditionally, car batteries have been 12 volts.
Car companies have dithered over what voltage to use as the standard for mild hybrids,
building systems based on 36 V, 42 V, and 48 V. For the purposes of this
problem, we consider 36 V.\\
(a) Suppose the
battery in a new car is used to run a device that requires
the same amount of power as the corresponding device in the
old car. Based on the sample figures above, how would the
currents handled by the wires in one of the new cars compare
with the currents in the old ones?\answercheck\hwendpart
(b) The real purpose of
the greater voltage is to handle devices that need
\emph{more} power. Can you guess why they decided to change
to higher-voltage batteries rather than increasing the power
without increasing the voltage?

