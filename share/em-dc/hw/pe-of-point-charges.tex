        (a) Recall m4_ifelse(__sn,0,[:from example \ref{eg:pe-grav} on p.~\pageref{eg:pe-grav}:]) that the gravitational energy of two gravitationally
        interacting spheres is given by $PE=-Gm_1m_2/r$, where $r$ is the
        center-to-center distance. Sketch a graph of $PE$ as a function of $r$,
        making sure that your graph behaves properly at small values of $r$,
        where you're dividing by a small number, and at large ones, where you're
        dividing by a large one. Check that your graph behaves properly when a
        rock is dropped from a larger $r$ to a smaller one; the rock should \emph{lose}
        potential energy as it gains kinetic energy.\hwendpart
(b) Electrical forces are closely analogous to gravitational ones, since both
depend on $1/r^2$. Since the forces are analogous, the potential energies should
also behave analogously. Using this analogy, write down the expression for the
electrical potential energy of two interacting charged particles. The main uncertainty
here is the sign out in front. Like masses attract, but like charges repel. To figure out whether you
have the right sign  in your equation, sketch graphs in the case where both charges are positive,
and also in the case where one is positive and one negative; make sure that in both cases, when the
charges are released near one another, their motion causes them to lose PE while gaining KE.\answercheck
