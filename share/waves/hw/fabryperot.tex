%%%%%
%%%%% This problem is used by: 0sn,6,fabryperot
%%%%%
        A Fabry-Perot interferometer, shown in the figure being used
to measure the diameter of a thin filament, consists of two glass plates
with an air gap between them. As the top plate is moved up or down
with a screw, the light passing through the plates goes through a cycle
of constructive and destructive interference, which is mainly due to
interference between rays that pass straight through and those that are
reflected twice back into the air gap. (Although the dimensions in this
drawing are distorted for legibility, the glass plates would really be much
thicker than the length of the wave-trains of light, so no interference
effects would be observed due to reflections within the glass.)\hwendpart
(a) If the top plate is cranked down so that the thickness, $d$, of
the air gap is much less than the wavelength $\lambda$ of the light, i.e., in the
limit $d \rightarrow 0$,
what is the phase relationship between the two
rays? (Recall that the phase can be inverted by a reflection.) 
Is the interference constructive, or destructive?\hwendpart
(b) If $d$ is now slowly increased, what is the first value of $d$
for which the interference is the same as at $d\rightarrow 0$? Express your
answer in terms of $\lambda$.\hwendpart
(c) Suppose the apparatus is first set up as shown in the figure. The
filament is then removed, and $n$ cycles of brightening and dimming
are counted while the top plate is brought down to $d=0$. What is the
thickness of the filament, in terms of $n$ and $\lambda$?\hwendpart
Based on a problem by D.J. Raymond.
