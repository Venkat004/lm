%%%%%
%%%%% This problem is used by: 0sn,6,freaknotes 3vw,4,freaknotes
%%%%%
(a) A good tenor saxophone player can play all of the
following notes without changing her fingering, simply by
altering the tightness of her lips: E$\flat$ (150 Hz), E$\flat$ (300
Hz), B$\flat$ (450 Hz), and E$\flat$ (600 Hz).  How is this possible?
(I'm not asking you to analyze the coupling between the lips, the
reed, the mouthpiece, and the air column, which is very complicated.)\hwendpart
(b) Some saxophone players are known for their ability to
use this technique to play ``freak notes,'' i.e., notes above
the normal range of the instrument.  Why isn't it possible
to play notes below the normal range using this technique?
