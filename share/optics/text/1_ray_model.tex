Ads for one Macintosh computer bragged that it could do an
arithmetic calculation in less time than it took for the
light to get from the screen to your eye. We find this
impressive because of the contrast between the speed of
light and the speeds at which we interact with physical
objects in our environment. Perhaps it shouldn't surprise
us, then, that Newton succeeded so well in explaining the
motion of objects, but was far less successful with the study of light.

The climax of our study of
electricity and magnetism was discovery that light is an
electromagnetic wave. Knowing this, however, is not the same
as knowing everything about eyes and telescopes. In fact,
the full description of light as a wave can be rather
cumbersome. We will instead spend most of our treatment of optics making
use of a simpler model of light, the ray model, which does a
fine job in most practical situations. Not only that, but we
will even backtrack a little and start with a discussion of
basic ideas about light and vision that predated the
discovery of electromagnetic waves.

<% begin_sec("The nature of light",0) %>

<% begin_sec("The cause and effect relationship in vision") %>

Despite its title, this chapter is far from your first look
at light. That familiarity might seem like an advantage, but
most people have never thought carefully about light and
vision. Even smart people who have thought hard about vision
have come up with incorrect ideas. The ancient Greeks, Arabs
and Chinese had theories of light and \index{vision}vision,
all of which were mostly wrong, and all of which were
accepted for thousands of years.

One thing the ancients did get right is that there is a
distinction between objects that emit light and objects that
don't. When you see a leaf in the forest, it's because three
different objects are doing their jobs: the leaf, the eye,
and the sun. But luminous objects like the sun, a flame, or
the filament of a light bulb can be seen by the eye without
the presence of a third object. Emission of light is often,
but not always, associated with heat. In modern times, we
are familiar with a variety of objects that glow without
being heated, including fluorescent lights and glow-in-the-dark toys.

How do we see luminous objects? The Greek philosophers
\index{Pythagoras}Pythagoras (b. ca. 560 BC) and \index{Empedocles
of Acragas}Empedocles of Acragas (b. ca. 492 BC), who
unfortunately were very influential, claimed that when you
looked at a candle flame, the flame and your eye were both
sending out some kind of mysterious stuff, and when your
eye's stuff collided with the candle's stuff, the candle
would become evident to your sense of sight.

Bizarre as the Greek ``collision of stuff theory'' might
seem, it had a couple of good features. It explained why
both the candle and your eye had to be present for your
sense of sight to function. The theory could also easily be
expanded to explain how we see nonluminous objects. If a
leaf, for instance, happened to be present at the site of
the collision between your eye's stuff and the candle's
stuff, then the leaf would be stimulated to express its
green nature, allowing you to perceive it as green.

Modern people might feel uneasy about this theory, since it
suggests that greenness exists only for our seeing
convenience, implying a human precedence over natural
phenomena. Nowadays, people would expect the cause and
effect relationship in vision to be the other way around,
with the leaf doing something to our eye rather than our eye
doing something to the leaf. But how can you tell? The most
common way of distinguishing cause from effect is to
determine which happened first, but the process of seeing
seems to occur too quickly to determine the order in which
things happened. Certainly there is no obvious time lag
between the moment when you move your head and the moment
when your reflection in the mirror moves.

Today, photography provides the simplest experimental
evidence that nothing has to be emitted from your eye and
hit the leaf in order to make it ``greenify.'' A camera can
take a picture of a leaf even if there are no eyes anywhere
nearby. Since the leaf appears green regardless of whether
it is being sensed by a camera, your eye, or an insect's
eye, it seems to make more sense to say that the leaf's
greenness is the cause, and something happening in the
camera or eye is the effect.

\index{light!speed of}
<% end_sec() %>
<% begin_sec("Light is a thing, and it travels from one point to another.") %>

Another issue that few people have considered is whether a
candle's flame simply affects your eye directly, or whether
it sends out light which then gets into your eye. Again, the
rapidity of the effect makes it difficult to tell what's
happening. If someone throws a rock at you, you can see the
rock on its way to your body, and you can tell that the
person affected you by sending a material substance your
way, rather than just harming you directly with an arm
motion, which would be known as ``action at a distance.'' It
is not easy to do a similar observation to see whether there
is some ``stuff'' that travels from the candle to your eye,
or whether it is a case of action at a distance.

Newtonian physics includes both action at a distance (e.g.,
the earth's gravitational force on a falling object) and
contact forces such as the normal force, which only allow
distant objects to exert forces on each other by shooting
some substance across the space between them (e.g., a garden
hose spraying out water that exerts a force on a bush).

<% marg(90) %>
<%
  fig(
    'deflection-of-candle-light',
    %q{%
      Light from a candle is bumped off course by a
      piece of glass. Inserting the glass causes the apparent location of the candle to
      shift. The same effect can be produced by taking off your eyeglasses and
      looking at which you see near the edge of the lens, but a flat piece of glass
      works just as well as a lens for this purpose.
    }
  )
%>

<% end_marg %>
One piece of evidence that the candle sends out stuff that
travels to your eye is that as in figure \figref{deflection-of-candle-light}, intervening transparent
substances can make the candle appear to be in the wrong
location, suggesting that light is a thing that can be
bumped off course. Many people would dismiss this kind of
observation as an optical illusion, however. (Some optical
illusions are purely neurological or psychological effects,
although some others, including this one, turn out to be
caused by the behavior of light itself.)

A more convincing way to decide in which category light
belongs is to find out if it takes time to get from the
candle to your eye; in Newtonian physics, action at a
distance is supposed to be instantaneous. The fact that we
speak casually today of ``the speed of light'' implies that
at some point in history, somebody succeeded in showing that
light did not travel infinitely fast. Galileo tried, and
failed, to detect a finite speed for light, by arranging
with a person in a distant tower to signal back and forth
with lanterns. \index{Galileo}Galileo uncovered his lantern,
and when the other person saw the light, he uncovered his
lantern. Galileo was unable to measure any time lag that was
significant compared to the limitations of human reflexes.

The first person to prove that light's speed was finite, and
to determine it numerically, was Ole \index{Roemer}Roemer,
in a series of measurements around the year 1675. Roemer
observed Io, one of Jupiter's moons, over a period of
several years. Since \index{Io}Io presumably took the same
amount of time to complete each orbit of \index{Jupiter}Jupiter,
it could be thought of as a very distant, very accurate
clock. A practical and accurate pendulum clock had recently
been invented, so Roemer could check whether the ratio of
the two clocks' cycles, about 42.5 hours to 1 orbit, stayed
exactly constant or changed a little. If the process of
seeing the distant moon was instantaneous, there would be no
reason for the two to get out of step. Even if the speed of
light was finite, you might expect that the result would be
only to offset one cycle relative to the other. The earth
does not, however, stay at a constant distance from Jupiter
and its moons. Since the distance is changing gradually due
to the two planets' orbital motions, a finite speed of light
would make the ``Io clock'' appear to run faster as the
planets drew near each other, and more slowly as their
separation increased. Roemer did find a variation in the
apparent speed of Io's orbits, which caused Io's eclipses by
Jupiter (the moments when Io passed in front of or behind
Jupiter) to occur about 7 minutes early when the earth was
closest to Jupiter, and 7 minutes late when it was farthest.
Based on these measurements, Roemer estimated the speed of
light to be approximately $2\times10^8$ m/s, which is in
the right ballpark compared to modern measurements of
$3\times10^8$ m/s. (I'm not sure whether the fairly large
experimental error was mainly due to imprecise knowledge of
the radius of the earth's orbit or limitations in the
reliability of pendulum clocks.)
<% marg(m4_ifelse(__lm_series,1,150,0)) %>
<%
  fig(
    'io',
    %q{An image of Jupiter and its moon Io (left) from the Cassini probe.}
  )
%>
\spacebetweenfigs
<%
  fig(
    'roemer',
    %q{%
      The earth is moving toward Jupiter and Io. Since the distance is shrinking, it is taking
      less and less time for the light to get to us from Io, and Io appears to circle Jupiter more
      quickly than normal. Six months later, the earth will be on the opposite side of the sun,
      and receding from Jupiter and Io, so Io will appear to revolve around Jupiter more slowly.
    }
  )
%>

<% end_marg %>

<% end_sec() %>
<% begin_sec("Light can travel through a vacuum.") %>

Many people are confused by the relationship between sound
and light. Although we use different organs to sense them,
there are some similarities. For instance, both light and
sound are typically emitted in all directions by their
sources. Musicians even use visual metaphors like ``tone
color,'' or ``a bright timbre'' to describe sound. One way
to see that they are clearly different phenomena is to note
their very different velocities. Sure, both are pretty fast
compared to a flying arrow or a galloping horse, but as we
have seen, the speed of light is so great as to appear
instantaneous in most situations. The speed of sound,
however, can easily be observed just by watching a group of
schoolchildren a hundred feet away as they clap their hands
to a song. There is an obvious delay between when you see
their palms come together and when you hear the clap.

The fundamental distinction between sound and light is that
sound is an oscillation in air pressure, so it requires air
(or some other medium such as water) in which to travel.
Today, we know that outer space is a vacuum, so the fact
that we get light from the sun, moon and stars clearly shows
that air is not necessary for the propagation of light.

\startdqs

\begin{dq}
If you observe thunder and lightning, you can tell how
far away the storm is. Do you need to know the speed of
sound, of light, or of both?
\end{dq}

\begin{dq}
When phenomena like X-rays and cosmic rays were first
discovered, suggest a way one could have tested whether they
were forms of light.
\end{dq}

\begin{dq}
Why did Roemer only need to know the radius of the
earth's orbit, not Jupiter's, in order to find the speed of light?
\end{dq}

<% end_sec() %>
<% end_sec() %>
<% begin_sec("Interaction of light with matter",0) %>

\index{absorption}\index{light!absorption of}
<% begin_sec("Absorption of light") %>

The reason why the sun feels warm on your skin is that the
sunlight is being absorbed, and the light energy is being
transformed into heat energy. The same happens with
artificial light, so the net result of leaving a light
turned on is to heat the room. It doesn't matter whether the
source of the light is hot, like the sun, a flame, or an
incandescent light bulb, or cool, like a fluorescent bulb.
(If your house has electric heat, then there is absolutely
no point in fastidiously turning off lights in the winter;
the lights will help to heat the house at the same dollar
rate as the electric heater.)

This process of heating by absorption is entirely different
from heating by thermal conduction, as when an electric
stove heats spaghetti sauce through a pan. Heat can only be
conducted through matter, but there is vacuum between us and
the sun, or between us and the filament of an incandescent
bulb. Also, heat conduction can only transfer heat energy
from a hotter object to a colder one, but a cool fluorescent
bulb is perfectly capable of heating something that had
already started out being warmer than the bulb itself.

<% end_sec() %>
<% begin_sec("How we see nonluminous objects") %>

Not all the light energy that hits an object is transformed
into heat. Some is reflected, and this leads us to the
question of how we see nonluminous objects. If you ask the
average person how we see a light bulb, the most likely
answer is ``The light bulb makes light, which hits our
eyes.'' But if you ask how we see a book, they are likely to
say ``The bulb lights up the room, and that lets me see the
book.'' All mention of light actually entering our eyes has
mysteriously disappeared.

Most people would disagree if you told them that light was
reflected from the book to the eye, because they think of
reflection as something that mirrors do, not something that
a book does. They associate reflection with the formation of
a reflected image, which does not seem to appear in a piece of paper.

Imagine that you are looking at your reflection in a nice
smooth piece of aluminum foil, fresh off the roll. You
perceive a face, not a piece of metal. Perhaps you also see
the bright reflection of a lamp over your shoulder behind
you. Now imagine that the foil is just a little bit less
smooth. The different parts of the image are now a little
bit out of alignment with each other. Your brain can still
recognize a face and a lamp, but it's a little scrambled,
like a Picasso painting. Now suppose you use a piece of
aluminum foil that has been crumpled up and then flattened
out again. The parts of the image are so scrambled that you
cannot recognize an image. Instead, your brain tells you
you're looking at a rough, silvery surface.

<% marg(55) %>
<%
  fig(
    'selfportraits',
    %q{%
      Two self-portraits of the author, one taken in a mirror
      and one with a piece of aluminum foil.
    }
  )
%>
\spacebetweenfigs
<%
  fig(
    'specular-and-diffuse-reflection',
    %q{Specular and diffuse reflection.}
  )
%>
<% end_marg %>

Mirror-like reflection at a specific angle is known as
specular reflection, and random reflection in many
directions is called \index{diffuse reflection}\index{reflection!diffuse}diffuse
reflection. Diffuse reflection is how we see nonluminous
objects. Specular reflection only allows us to see images of
objects other than the one doing the reflecting. In top part
of figure \figref{selfportraits}, imagine that the rays of light are coming
from the sun. If you are looking down at the reflecting
surface, there is no way for your eye-brain system to tell
that the rays are not really coming from a sun down below you.

Figure \figref{reading-with-lamp} shows another example of how we can't avoid the
conclusion that light bounces off of things other than
mirrors. The lamp is one I have in my house. It has a bright
bulb, housed in a completely opaque bowl-shaped metal shade.
The only way light can get out of the lamp is by going up
out of the top of the bowl. The fact that I can read a book
in the position shown in the figure means that light must be
bouncing off of the ceiling, then bouncing off of the book,
then finally getting to my eye.

This is where the shortcomings of the Greek theory of vision
become glaringly obvious. In the Greek theory, the light
from the bulb and my mysterious ``eye rays'' are both
supposed to go to the book, where they collide, allowing me
to see the book. But we now have a total of four objects:
lamp, eye, book, and ceiling. Where does the ceiling come
in? Does it also send out its own mysterious ``ceiling
rays,'' contributing to a three-way collision at the book?
That would just be too bizarre to believe!

The differences among white, black, and the various shades
of gray in between is a matter of what percentage of the
light they absorb and what percentage they reflect. That's
why light-colored clothing is more comfortable in the
summer, and light-colored upholstery in a car stays cooler
that dark upholstery.

\pagebreak[4]

\index{light!brightness of}\index{brightness of light}
<% end_sec() %>
<% begin_sec("Numerical measurement of the brightness of light") %>

<%#
 The two figures are together here because of a weird bug;
 If I put the dq-seasons figure later, it screws up the
 indentation of lists for the rest of the book !?!?
%>
<% marg(0) %>
<%
  fig(
    'reading-with-lamp',
    %q{Light bounces off of the ceiling, then off of the book.}
  )
%>
\vspace{43mm}
<%
  fig(
    'dq-seasons',
    %q{Discussion question \ref{dq:seasons}.}
  )
%>
<% end_marg %>
We have already seen that the physiological sensation of
loudness relates to the sound's intensity (power per unit
area), but is not directly proportional to it. If sound A
has an intensity of 1 $\zu{nW}/\munit^2$, sound B is 10 $\zu{nW}/\munit^2$,
and sound C is 100 $\zu{nW}/\munit^2$, then the increase in loudness
from B to C is perceived to be the same as the increase
from A to B, not ten times greater. That is, the sensation
of loudness is logarithmic.

The same is true for the brightness of light. Brightness is
related to power per unit area, but the psychological
relationship is a logarithmic one rather than a proportionality.
For doing physics, it's the power per unit area that we're
interested in. The relevant unit is $\zu{W}/\munit^2$.
One way to determine the brightness of light is to
measure the increase in temperature of a black object
exposed to the light. The light energy is being converted to
heat energy, and the amount of heat energy absorbed in a
given amount of time can be related to the power absorbed,
using the known heat capacity of the object. More practical
devices for measuring light intensity, such as the light
meters built into some cameras, are based on the conversion
of light into electrical energy, but these meters have to be
calibrated somehow against heat measurements.

\startdqs

\begin{dq}
The curtains in a room are drawn, but a small gap lets
light through, illuminating a spot on the floor. It may or
may not also be possible to see the beam of sunshine
crossing the room, depending on the conditions. What's going on?
\end{dq}

\begin{dq}
Laser beams are made of light. In science fiction movies,
laser beams are often shown as bright lines shooting out of
a laser gun on a spaceship. Why is this scientifically incorrect?
\end{dq}

\begin{dq}\label{dq:seasons}
A documentary film-maker went to Harvard's 1987 graduation
ceremony and asked the graduates, on camera, to explain the
cause of the seasons. Only two out of 23 were able to give a
correct explanation, but you now have all the information
needed to figure it out for yourself, assuming you didn't
already know. The figure shows the earth in its winter and
summer positions relative to the sun. Hint: Consider the
units used to measure the brightness of light, and recall
that the sun is lower in the sky in winter, so its rays are
coming in at a shallower angle.
\end{dq}


<% end_sec() %>
<% end_sec() %>
<% begin_sec("The ray model of light",3) %>\index{ray model of light}\index{light!ray model of}

\index{particle model of light}\index{light!particle model
of}\index{wave model of light}\index{light!wave model of}
<% begin_sec("Models of light") %>

Note how I've been casually diagramming the motion of light
with pictures showing light rays as lines on the page. More
formally, this is known as the ray model of light. The ray
model of light seems natural once we convince ourselves that
light travels through space, and observe phenomena like
sunbeams coming through holes in clouds. Having already been
introduced to the concept of light as an electromagnetic
wave, you know that the ray model is not the ultimate truth
about light, but the ray model is simpler, and in any case
science always deals with models of reality, not the
ultimate nature of reality. The following table summarizes
three models of light.

<%
  fig(
    'three-models-of-light',
    %q{Three models of light.},
    {
      'width'=>'wide',
      'sidecaption'=>true
    }
  )
%>

The ray model is a generic one. By using it we
can discuss the path taken by the light, without committing
ourselves to any specific description of what it is that is
moving along that path. We will use the nice simple ray
model for most of our treatment of optics, and with it we can analyze a
great many devices and phenomena. Not until __section_or_chapter(wave-optics)
will we concern ourselves specifically with wave optics,
although in the intervening chapters I will sometimes
analyze the same phenomenon using both the ray model and the wave model.

Note that the statements about the applicability of the
various models are only rough guides. For instance, wave
interference effects are often detectable, if small, when
light passes around an obstacle that is quite a bit bigger
than a wavelength. Also, the criterion for when we need the
particle model really has more to do with energy scales than
distance scales, although the two turn out to be related.

The alert reader may have noticed that the wave model is
required at scales smaller than a wavelength of light (on
the order of a micrometer for visible light), and the
particle model is demanded on the atomic scale or lower (a
typical atom being a nanometer or so in size). This implies
that at the smallest scales we need \emph{both} the wave
model and the particle model. They appear incompatible, so
how can we simultaneously use both? The answer is that they
are not as incompatible as they seem. Light is both a wave
and a particle, but a full understanding of this apparently
nonsensical statement is a topic for __section_or_chapter(light-as-a-particle).


<%
  fig(
    'sample-ray-diagrams',
    %q{Examples of ray diagrams.},
    {
      'width'=>'wide',
      'sidecaption'=>true
    }
  )
%>

\index{ray diagrams}
<% end_sec() %>
<% begin_sec("Ray diagrams") %>

Without even knowing how to use the ray model to calculate
anything numerically, we can learn a great deal by drawing
ray diagrams. For instance, if you want to understand how
eyeglasses help you to see in focus, a ray diagram is the
right place to start. Many students under-utilize ray
diagrams in optics and instead rely on rote memorization or
plugging into formulas. The trouble with memorization and
plug-ins is that they can obscure what's really going on,
and it is easy to get them wrong. Often the best plan is to
do a ray diagram first, then do a numerical calculation,
then check that your numerical results are in reasonable
agreement with what you expected from the ray diagram.

<%
  fig(
    'fish',
    %q{%
      %
      1. Correct.
      2. Incorrect: implies that diffuse reflection only gives
      one ray from each reflecting point.
      3. Correct, but unnecessarily complicated
    },
    {
      'width'=>'wide',
      'sidecaption'=>true
    }
  )
%>

Figure \figref{fish} shows some guidelines for using ray
diagrams effectively. The light rays bend when they pass out
through the surface of the water (a phenomenon that we'll
discuss in more detail later). The rays appear to have come
from a point above the goldfish's actual location, an effect
that is familiar to people who have tried spear-fishing.

\begin{itemize}
\item A stream of light is not really confined to a finite
number of narrow lines. We just draw it that way. In \figref{fish}/1, it
has been necessary to choose a finite number of rays to draw
(five), rather than the theoretically infinite number of
rays that will diverge from that point.

\item There is a tendency to conceptualize rays incorrectly as
objects. In his Optics, Newton goes out of his way to
caution the reader against this, saying that some people
``consider ... the refraction of ... rays to be the bending
or breaking of them in their passing out of one medium into
another.'' But a ray is a record of the path traveled by
light, not a physical thing that can be bent or broken.

\item In theory, rays may continue infinitely far into the past
and future, but we need to draw lines of finite length. In
\figref{fish}/1, a judicious choice has been made as to where to begin
and end the rays. There is no point in continuing the rays
any farther than shown, because nothing new and exciting is
going to happen to them. There is also no good reason to
start them earlier, before being reflected by the fish,
because the direction of the diffusely reflected rays is
random anyway, and unrelated to the direction of the
original, incoming ray.

\item When representing diffuse reflection in a ray diagram,
many students have a mental block against drawing many rays
fanning out from the same point. Often, as in example \figref{fish}/2,
the problem is the misconception that light can only be
reflected in one direction from one point.

\item Another difficulty associated with diffuse reflection,
example \figref{fish}/3, is the tendency to think that in addition to
drawing many rays coming out of one point, we should also be
drawing many rays coming from many points. In \figref{fish}/1, drawing
many rays coming out of one point gives useful information,
telling us, for instance, that the fish can be seen from any
angle. Drawing many sets of rays, as in \figref{fish}/3, does not give
us any more useful information, and just clutters up the
picture in this example. The only reason to draw sets of
rays fanning out from more than one point would be if
different things were happening to the different sets.
\end{itemize}

\startdq

\begin{dq}
Suppose an intelligent tool-using fish is spear-hunting for
humans. Draw a ray diagram to show how the fish has to
correct its aim. Note that although the rays are now passing
from the air to the water, the same rules apply: the rays
are closer to being perpendicular to the surface when they
are in the water, and rays that hit the air-water interface
at a shallow angle are bent the most.
\end{dq}

<% end_sec() %>
<% end_sec() %>
<% begin_sec("Geometry of specular reflection",3) %>\index{reflection!specular}

To change the motion of a material object, we use a force.
Is there any way to exert a force on a beam of light?
Experiments show that electric and magnetic fields do not
deflect light beams, so apparently light has no electric
charge. Light also has no mass, so until the twentieth
century it was believed to be immune to gravity as well.
Einstein predicted that light beams would be very slightly
deflected by strong gravitational fields, and he was proved
correct by observations of rays of starlight that came close
to the sun, but obviously that's not what makes mirrors and lenses work!

If we investigate how light is reflected by a mirror, we
will find that the process is horrifically complex, but the
final result is surprisingly simple. What actually happens
is that the light is made of electric and magnetic fields,
and these fields accelerate the electrons in the mirror.
Energy from the light beam is momentarily transformed into
extra kinetic energy of the electrons, but because the
electrons are accelerating they re-radiate more light,
converting their kinetic energy back into light energy. We
might expect this to result in a very chaotic situation, but
amazingly enough, the electrons move together to produce a
new, reflected beam of light, which obeys two simple rules:

\begin{itemize}

\item The angle of the reflected ray is the same as that
of the incident ray.

\item The reflected ray lies in the plane containing the
incident ray and the normal (perpendicular) line. This plane
is known as the plane of incidence.

\end{itemize}

<% marg(70) %>
<%
  fig(
    'plane-of-reflection',
    %q{The geometry of specular reflection.}
  )
%>
<% end_marg %>

The two angles can be defined either with respect to the
normal, like angles B and C in the figure, or with
respect to the reflecting surface, like angles A and D.
There is a convention of several hundred years' standing
that one measures the angles with respect to the normal,
but the rule about equal angles can logically be stated
either as B=C or as A=D.

The phenomenon of reflection occurs only at the boundary
between two media, just like the change in the speed of
light that passes from one medium to another. As we have
seen in __section_or_chapter(bounded-waves), this is the way all waves behave.

Most people are surprised by the fact that light can be
reflected back from a less dense medium. For instance, if
you are diving and you look up at the surface of the water,
you will see a reflection of yourself.

\pagebreak[4]

<% self_check('two-reflections-from-same-point',<<-'SELF_CHECK'
Each of these diagrams is supposed to show two different
rays being reflected from the same point on the same mirror.
Which are correct, and which are incorrect?

\anonymousinlinefig{sc-two-reflections-from-same-point}

  SELF_CHECK
  ) %>

<% begin_sec("Reversibility of light rays") %>\index{time reversal}\index{reversibility}

The fact that specular reflection displays equal angles of
incidence and reflection means that there is a symmetry: if
the ray had come in from the right instead of the left in
the figure above, the angles would have looked exactly the
same. This is not just a pointless detail about specular
reflection. It's a manifestation of a very deep and
important fact about nature, which is that the laws of
physics do not distinguish between past and future.
Cannonballs and planets have trajectories that are equally
natural in reverse, and so do light rays. This type of
symmetry is called time-reversal symmetry.

Typically, time-reversal symmetry is a characteristic of any
process that does not involve heat. For instance, the
planets do not experience any friction as they travel
through empty space, so there is no frictional heating. We
should thus expect the time-reversed versions of their
orbits to obey the laws of physics, which they do. In
contrast, a book sliding across a table does generate heat
from friction as it slows down, and it is therefore not
surprising that this type of motion does not appear to obey
time-reversal symmetry. A book lying still on a flat table
is never observed to spontaneously start sliding, sucking up
heat energy and transforming it into kinetic energy.

Similarly, the only situation we've observed so far where
light does not obey time-reversal symmetry is absorption,
which involves heat. Your skin absorbs visible light from
the sun and heats up, but we never observe people's skin to
glow, converting heat energy into visible light. People's
skin does glow in infrared light, but that doesn't mean the
situation is symmetric. Even if you absorb infrared, you
don't emit visible light, because your skin isn't hot enough
to glow in the visible spectrum.

These apparent heat-related asymmetries are not actual
asymmetries in the laws of physics. The interested reader
may wish to learn more about this from  optional chapter \ref{ch:thermo} on thermodynamics.

\begin{eg}{Ray tracing on a computer}
A number of techniques can be used for creating artificial visual scenes in
computer graphics. Figure \figref{computer-ray-tracing} shows such a scene,
which was created by the brute-force technique of simply constructing a very
detailed ray diagram on a computer. This technique requires a great deal of
computation, and is therefore too slow to be used for video games and computer-animated movies.
One trick for speeding up the computation is to exploit the reversibility of light
rays. If one was to trace every ray emitted by every illuminated surface, only
a tiny fraction of those would actually end up passing into the virtual ``camera,''
and therefore almost all of the computational effort would be wasted. One can instead
start a ray at the camera, trace it backward in time, and see where it would have
come from. With this technique, there is no wasted effort.
\end{eg}

<%
  fig(
    'computer-ray-tracing',
    %q{This photorealistic image of a nonexistent countertop was produced
       completely on a computer, by computing a  complicated ray diagram.},
    {
      'width'=>'wide'
    }
  )
%>

\pagebreak[4]

<% marg(0) %>
<%
  fig(
    'dq-radar-corner-2d',
    %q{Discussion question \ref{dq:radar-corner-2d}.}
  )
%>
\spacebetweenfigs
<%
  fig(
    'dq-radar-corner-3d',
    %q{Discussion question \ref{dq:radar-corner-3d}.}
  )
%>
\vspace{27mm}
<%
  fig(
    'least-time-refl-1',
    %q{%
      The solid lines are physically possible paths for light rays traveling from A to B and from A
      to C. They obey the principle of least time. The dashed lines do not obey the principle of least time, and
      are not physically possible.
    }
  )
%>

<% end_marg %>

\startdqs

\begin{dq}\label{dq:reflect-one-component}
If a light ray has a velocity vector with components
$c_x$ and $c_y$, what will happen when it is reflected from
a surface that lies along the $y$ axis? Make sure your
answer does not imply a change in the ray's speed.
\end{dq}

\begin{dq}\label{dq:radar-corner-2d}
Generalizing your reasoning from discussion question 
\ref{dq:reflect-one-component},
what will happen to the velocity components of a light ray
that hits a corner, as shown in the figure, and undergoes two reflections?
\end{dq}

\begin{dq}\label{dq:radar-corner-3d}
Three pieces of sheet metal arranged perpendicularly as
shown in the figure form what is known as a radar corner.
Let's assume that the radar corner is large compared to the
wavelength of the radar waves, so that the ray model makes
sense. If the radar corner is bathed in radar rays, at least
some of them will undergo three reflections. Making a
further generalization of your reasoning from the two
preceding discussion questions, what will happen to the
three velocity components of such a ray? What would the
radar corner be useful for?
\end{dq}

\vspace{50mm}

<% end_sec() %>
<% end_sec() %>
<% begin_sec("The principle of least time for reflection",nil,'least-time-reflection',{'optional'=>true}) %>

We had to choose between an unwieldy explanation of
reflection at the atomic level and a simpler geometric
description that was not as fundamental. There is a third
approach to describing the interaction of light and matter
which is very deep and beautiful. Emphasized by the
twentieth-century physicist Richard Feynman, it is called
the principle of least time,\index{least time, principle of} or
\index{Fermat's principle|see{least time, principle of}}Fermat's principle.

Let's start with the motion of light that is not interacting
with matter at all. In a vacuum, a light ray moves in a
straight line. This can be rephrased as follows: of all the
conceivable paths light could follow from P to Q, the
only one that is physically possible is the path that
takes the least time.

What about reflection? If light is going to go from one
point to another, being reflected on the way, the quickest
path is indeed the one with equal angles of incidence and
reflection. If the starting and ending points are equally
far from the reflecting surface, \figref{least-time-refl-1}, it's not hard to
convince yourself that this is true, just based on symmetry.
There is also a tricky and simple proof, shown in figure \figref{least-time-refl-2},
for the more general case where the
points are at different distances from the surface.

Not only does the principle of least time work for light in
a vacuum and light undergoing reflection, we will also see
in a later chapter that it works for the bending of light
when it passes from one medium into another.

Although it is beautiful that the entire ray model of light
can be reduced to one simple rule, the principle of least
time, it may seem a little spooky to speak as if the ray of
light is intelligent, and has carefully planned ahead to
find the shortest route to its destination. How does it know
in advance where it's going? What if we moved the mirror
while the light was en route, so conditions along its
planned path were not what it ``expected?'' The answer is
that the principle of least time is really a shortcut for
finding certain results of the wave model of light, which is
the topic of the last chapter of this book.

<% marg(75) %>
<%
  fig(
    'least-time-refl-2',
    %q{%
      Paths AQB and APB are two conceivable paths that a ray could follow to get from A to B with
      one reflection, but only AQB is physically possible. We wish to prove that the path AQB, with equal angles of
      incidence and reflection, is shorter than any other path, such as APB. The trick is to construct a third point, C,
      lying as far below the surface as B lies above it. Then path AQC is a straight line whose length is the same as
      AQB's, and path APC has the same length as path APB. Since AQC is straight, it must be shorter than any other
      path such as APC that connects A and C, and therefore AQB must be shorter than any path such as APB.
    }
  )
%>
\spacebetweenfigs
<%
  fig(
    'least-time-ellipse',
    %q{%
      Light is emitted at the center of an
      elliptical mirror. There are four physically possible paths by which
      a ray can be reflected and return to the center.
    }
  )
%>
<% end_marg %>

There are a couple of subtle points about the principle of
least time. First, the path does not have to be the quickest
of all possible paths; it only needs to be quicker than any
path that differs infinitesimally from it. In figure \figref{least-time-refl-2},
for instance, light could get from A to B either by the
reflected path AQB or simply by going straight from A to
B. Although AQB is not the shortest possible path, it
cannot be shortened by changing it infinitesimally, e.g., by
moving Q a little to the right or left. On the other hand,
path APB is physically impossible, because it is possible to
improve on it by moving point P infinitesimally to the right.

It's not quite right to call this
the principle of \emph{least} time. In figure \figref{least-time-ellipse}, for
example, the four physically possible paths by which a ray
can return to the center consist of two shortest-time paths
and two longest-time paths. Strictly speaking, we should
refer to the \emph{principle of least or greatest time}, but
most physicists omit the niceties, and assume that other
physicists understand that both maxima and minima are possible.


<% end_sec() %>
