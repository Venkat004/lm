%%%%%
%%%%% This problem is used by: 5op,3,magnificationdependsondistance
%%%%%
<% hw_solution %> (a) Make up a numerical example of a virtual image formed
by a converging mirror with a certain focal length, and
determine the magnification. (You will need the result of
problem \ref{hw:virtual-image-location}.) Make sure to choose values of $d_o$ and $f$ that
would actually produce a virtual image, not a real one. Now
change the location of the object \emph{a little bit} and
redetermine the magnification, showing that it changes. At
my local department store, the cosmetics department sells
hand mirrors advertised as giving a magnification of 5 times. How
would you interpret this?

(b) Suppose a Newtonian telescope is being used for
astronomical observing. Assume for simplicity that no
eyepiece is used, and assume a value for the focal length of
the mirror that would be reasonable for an amateur
instrument that is to fit in a closet. Is the angular
magnification different for objects at different distances?
For example, you could consider two planets, one of which is
twice as far as the other.
