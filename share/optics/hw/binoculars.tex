%%%%%
%%%%% This problem is used by: 5op,4,binoculars
%%%%%
Panel 1 of the figure shows the optics inside a pair of binoculars.
They are essentially a pair of telescopes, one for each eye. But to make them more compact,
and allow the eyepieces to be the right distance apart for a human face, they incorporate
a set of eight prisms, which fold the light path. In addition, the prisms make the image upright.
Panel 2 shows one of these prisms, known
as a Porro prism.\index{Porro prism}\index{prism!Porro} The light enters along a normal, undergoes
two total internal reflections at angles of 45 degrees with respect to the back surfaces,
and exits along a normal. The image of the letter R has been flipped across the horizontal.
Panel 3 shows a pair of these prisms glued together. The image will be flipped across
both the horizontal and the vertical, which makes it oriented the right way for the user
of the binoculars.\hwendpart
(a) Find the minimum possible index of refraction for the glass used in the prisms.\hwendpart
(b) For a material of this minimal index of refraction, find the fraction of the incoming
light that will be lost to reflection in the four Porro prisms on a each side of a pair
of binoculars. (See
m4_ifelse(__sn,1,[:section \ref{sec:bounded-waves}:],[:ch.~\ref{ch:bounded-waves}:]).)
In real, high-quality binoculars, the optical surfaces of the prisms 
have antireflective coatings, but carry out your calculation for the case where there is
no such coating.\hwendpart
(c) Discuss the reasons why a designer of binoculars might or might not want to use a material
with exactly the index of refraction found in part a.
