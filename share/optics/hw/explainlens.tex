%%%%%
%%%%% This problem is used by: 5op,4,explainlens
%%%%%
Based on Snell's law, explain why rays of light passing
through the edges of a converging lens are bent more than
rays passing through parts closer to the center. It might
seem like it should be the other way around, since the rays
at the edge pass through less glass --- shouldn't
they be affected less? In your answer:
\begin{itemize}
\item Include a ray diagram showing a huge, full-page, close-up view of the relevant part of the
lens.
\item Make use of the fact that the front and back surfaces aren't always parallel; a lens in
which the front and back surfaces \emph{are} always parallel doesn't focus light at all, so
if your explanation doesn't make use of this fact, your argument must be incorrect.
\item Make sure your argument still works even if the rays don't come in parallel to
the axis or from a point on the axis.
\end{itemize}
