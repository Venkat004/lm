% meta {"stars":1}
Potassium 40 is the strongest source of naturally occurring beta radioactivity in our environment.
It decays according to
\begin{equation*}
  {}^{40}\zu{K} \rightarrow {}^{40}\zu{Ca} + e^- + \bar{\nu} .
\end{equation*}
The energy released in the decay is 1.33 MeV, where 1 eV is defined as the fundamental charge
$e$ multiplied by one volt. The energy is shared randomly among the products,
subject to the constraint imposed by conservation of energy-momentum, which dictates that very
little of the energy is carried by the recoiling calcium nucleus. Determine the maximum energy
of the calcium, and compare with the typical energy of a chemical bond, which is a few eV.
If the potassium is part of a molecule, do we expect the molecule to survive?
Carry out the calculation first by assuming that the electron is ultrarelativistic, then
without the approximation, and comment on the how good the approximation is.
