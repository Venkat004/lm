The figure shows one layer of the three-dimensional
structure of a salt crystal. The atoms extend much farther
off in all directions, but only a six-by-six square is shown
here. The larger circles are the chlorine ions, which have
charges of $-e$, where $e=1.60\times10^{-19}\ \zu{C}$. The smaller circles are sodium ions, with
charges of $+e$. The center-to-center distance between
neighboring ions is about 0.3 nm. Real crystals are never
perfect, and the crystal shown here has two defects: a
missing atom at one location, and an extra lithium atom,
shown as a grey circle, inserted in one of the small gaps.
If the lithium atom has a charge of $+e$, what is the
direction and magnitude of the total force on it? Assume
there are no other defects nearby in the crystal besides the
two shown here. [Hints: The force on the lithium ion is the
vector sum of all the forces of all the quadrillions of
sodium and chlorine atoms, which would obviously be too
laborious to calculate. Nearly all of these forces,
however, are canceled by a force from an ion on the opposite
side of the lithium.]\answercheck
