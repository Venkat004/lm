
1. Consulting a periodic table, find the $N$, $Z$, and $A$ of the following:

\begin{tabular}{|l|p{15mm}|p{15mm}|p{15mm}|}
\hline
& $N$ & $Z$ & $A$ \\
\hline
${}^4\zu{He}$ & & & \\
\hline
${}^{244}\zu{Pu}$ & & & \\
\hline
\end{tabular}

\vfill

2. Consider the following five decay processes:

\begin{itemize}

\item $\alpha$ decay

\item $\gamma$ decay

\item $\zu{p} \rightarrow \zu{n} + \zu{e}^+ + \nu$ ($\beta^+$ decay)

\item $\zu{n} \rightarrow \zu{p} + \zu{e}^- + \bar{\nu}$ ($\beta^-$ decay)

\item $\zu{p} + \zu{e}^- \rightarrow \zu{n} + \nu$ (electron capture)
\end{itemize}

What would be the action of each of these on the chart of the nuclei? The * represents the original nucleus.

\anonymousinlinefig{../../../share/em-general/figs/decays-on-chart-of-nuclei}

\vfill

3. (a) Suppose that ${}^{244}\zu{Pu}$ undergoes perfectly symmetric fission, and also emits two neutrons.
Find the daughter isotope.

(b) Is the daughter stable, or is it neutron-rich or -poor relative to the line of stability? (To estimate
what's stable, you can use a large chart of the nuclei, or, if you don't have one handy,
consult a periodic table and use the average atomic mass as an
approximation ot the stable value of $A$.)

(c) Consulting the chart of the nuclei (fig.~\figref{chartofnuclei} on p.~\pageref{fig:chartofnuclei}),
explain why it turns out this way.

(d) If the daughter is unstable, which process from question \#2
would you expect it to decay by?
