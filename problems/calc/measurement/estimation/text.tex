It is useful to be able to make rough estimates, e.g., how many bags of
gravel will I need to fill my driveway? Sometimes all we need is an estimate
so rough that we only care about getting the result to about the nearest factor
of ten, i.e., to within an order of magnitude. For example, anyone with a basic
knowledge of US geography can tell that the distance from New Haven to New York
is probably something like 100 km, not 10 km or 1000 km. When making estimates
of physical quantities, the following guidelines are helpful:

\begin{enumerate}
\item Don't even attempt more than one significant figure of precision.

\item Don't guess area, volume, or mass directly. Guess linear
dimensions and get area, volume, or mass from them. Mass is often best
found by estimating linear dimensions and density.

\item When dealing with areas or volumes of objects with
complex shapes, idealize them as if they were some simpler
shape, a cube or a sphere, for example.

\item Check your final answer to see if it is reasonable. If
you estimate that a herd of ten thousand cattle would yield
$0.01\ \munit^2$ of leather, then you have probably made a mistake
with conversion factors somewhere.
\end{enumerate}
