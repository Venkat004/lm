It appears superficially that nature has many different types of forces, such
as frictional forces, normal forces, sticky forces, the force that lets bugs
walk on water, the force that makes gunpowder explode,
and the force that causes honey to flow so slowly. 
Actually, all of the forces on this list are manifestations of electrical
interactions at the atomic scale. Like gravity, electricity is a $1/r^2$ force.
The electrical counterpart of Newton's law of gravity is Coulomb's law,
\begin{equation}
  F = \frac{k|q_1||q_2|}{r^2},
\end{equation}
where $F$ is the magnitude of the force, $k$ is a universal constant, $r$ is
the distance between the two interacting objects, and $q_1$ and $q_2$ are properties
of the objects called their electric charges. This equation is often written using
the alternate form of the constant $\epsilon_0=1/(4\pi k)$.

Electric charge is measured in units
of Coulombs, C. Charge is to electricity as mass is to gravity. There are two
types of charge, which are conventionally labeled positive and negative.
Charges of the same type repel one another, and opposite charges attract.
Charge is conserved.

Charge is quantized, meaning that all charges are integer multiples of a certain
fundamental charge $e$. (The quarks that compose neutrons and protons have charges
that come in thirds of this unit, but quarks are never observed individually, only
in clusters that have integer multiples of $e$.) The electron has charge $-e$, the
proton $+e$.
