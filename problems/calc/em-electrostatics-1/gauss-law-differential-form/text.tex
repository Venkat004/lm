Gauss' law is a bit spooky. It relates the field on the Gaussian surface
to the charges inside the surface. What if the charges have been moving
around, and the field at the surface right now is the one that was created
by the charges in their previous locations? Gauss' law --- unlike
Coulomb's law --- still works in cases like these, but it's far from
obvious how the flux and the charges can still stay in agreement if
the charges have been moving around.

For this reason, it would be more physically attractive to restate Gauss'
law in a different form, so that it related the behavior of the field
at one point to the charges that were actually present at that point.
We define the \emph{divergence} of the electric field,
\begin{equation*}
  \operatorname{div} \vc{E} = \frac{\partial E_x}{\partial x}+\frac{\partial E_y}{\partial y}
                                                +\frac{\partial E_z}{\partial z}.
\end{equation*}
Gauss's law in differential form is
\begin{equation*}
  \operatorname{div} \vc{E} = 4\pi k \rho.
\end{equation*}
