\begin{timetravel}
\begin{table*}
\begin{tabular}{|p{52mm}|p{52mm}|}
\hline
\emph{force acting on Fifi}  &   \emph{force related to it by Newton's third law} \\
\hline
planet earth's gravitational force $F_W=mg$ on Fifi, \hfill $\downarrow$  &  Fifi's gravitational force on earth, \hfill $\uparrow$\\
\hline
belt's kinetic frictional force $F_k$ on Fifi, \hfill $\rightarrow$  &  Fifi's kinetic frictional force on belt, \hfill $\leftarrow$\\
\hline
belt's normal force $F_N$ on Fifi, \hfill $\uparrow$  &  Fifi's normal force on belt, \hfill $\downarrow$\\
\hline
\end{tabular}
\caption{Analysis of the forces on the dog shown in figure \ref{fig:eg-fifi}.}\label{table:fifi}
\end{table*}
\end{timetravel}

Newton's second law refers to the total force acting on a particular object.
Therefore whenever we want to apply the second law, a necessary preliminary
step is to pick an object and list all the forces acting on it. In addition,
it may be helpful to determine the types of the forces
and also to identify the Newton's-third-law partners of those forces,
i.e., all the forces that our object exerts back on other things.

% fig {"name":"eg-fifi","caption":"The spy dog lands on the moving conveyor belt."}

As an example, consider figure \ref{fig:eg-fifi}. 
Fifi is an industrial espionage dog who loves doing her job and looks great doing it.
She leaps through a window and lands at initial horizontal speed $v_\zu{o}$ on a conveyor belt which is itself
moving at the greater speed $v_b$. Unfortunately the coefficient of kinetic friction $\mu_k$ between her foot-pads
and the belt is fairly low, so she skids, and the effect on her coiffure is \emph{un d\'{e}sastre}.
Table \ref{table:fifi} shows the resulting analysis of the forces in which she participates.
