For a more detailed treatment of complex numbers, see ch. 3 of
James Nearing's free book at \\
physics.miami.edu/nearing/mathmethods/.

% fig {"name":"complex-numbers","caption":"Visualizing complex numbers as points in a plane."}

We assume there is a number, $i$, such that $i^2=-1$.
The square roots of $-1$ are then $i$ and $-i$. (In electrical engineering work,
where $i$ stands for current, $j$ is sometimes used instead.) This gives rise
to a number system, called the complex numbers, containing the real numbers as a subset.
Any complex number $z$ can be written in the form $z=a+bi$, where $a$ and $b$ are
real, and $a$ and $b$ are then referred to as the real and imaginary parts of $z$.
A number with a zero real part is called an imaginary number.
The complex numbers can be visualized as a plane, with the real number line placed
horizontally like the $x$ axis of the familiar $x-y$ plane, and the imaginary numbers running
along the $y$ axis. The complex numbers are complete in a way that the real numbers
aren't: every nonzero complex number has two square roots. For example, 1 is
a real number, so it is also a member of the complex numbers, and its square roots
are $-1$ and 1. Likewise, $-1$ has square roots $i$ and $-i$, and the number $i$
has square roots $1/\sqrt{2}+i/\sqrt{2}$ and $-1/\sqrt{2}-i/\sqrt{2}$.

% fig {"name":"complex-addition","caption":"Addition of complex numbers is just like addition of vectors,
%      although the real and imaginary axes don't actually represent directions in space."}

Complex numbers can be added and subtracted by adding or subtracting their real
and imaginary parts. Geometrically, this is the same as vector addition.

% fig {"name":"complex-conjugate","caption":"A complex number and its conjugate."}

The complex numbers $a+bi$ and $a-bi$, lying at equal distances above and below the
real axis, are called complex conjugates. The results of the quadratic formula
are either both real, or complex conjugates of each other.
The complex conjugate of a number $z$ is notated as $\bar{z}$ or
$z^*$.

The complex numbers obey all the same rules of arithmetic as the reals, except that
they can't be ordered along a single line. That is, it's not possible to say whether
one complex number is greater than another. We can compare them in terms of their
magnitudes (their distances from the origin), but two distinct complex numbers may
have the same magnitude, so, for example, we can't say whether $1$ is greater than
$i$ or $i$ is greater than $1$.

% fig {"name":"complex-polar","caption":"A complex number can be described in terms of its magnitude and argument."}

There is a nice interpretation of complex multiplication.
We define the argument of a complex number as its angle in the complex plane, measured
counterclockwise from the positive real axis.
Multiplying two complex numbers then corresponds to multiplying their magnitudes,
and adding their arguments.

% fig {"name":"complex-multiplication","caption":"The argument of $uv$ is the sum of the arguments of $u$ and $v$."}

Having expanded our horizons to include the complex numbers, it's natural to want to extend
functions we knew and loved from the world of real numbers so that they can also operate on
complex numbers. The only really natural way to do this in general is to use Taylor series.
A particularly beautiful thing happens with the functions $e^x$, $\sin x$, and $\cos x$:
\begin{align*}
  e^x    &= 1 + \frac{1}{2!}x^2 + \frac{1}{3!}x^3 + \ldots \\
  \cos x &= 1 - \frac{1}{2!}x^2 + \frac{1}{4!}x^4 - \ldots \\
  \sin x &= x - \frac{1}{3!}x^3 + \frac{1}{5!}x^5 - \ldots 
\end{align*}
If $x=i\phi$ is an imaginary number, we have
\begin{equation*}
  e^{i\phi} = \cos \phi + i \sin \phi\eqquad,
\end{equation*}
a result known as Euler's formula.\index{Euler's formula}
The geometrical interpretation in the complex
plane is shown in figure \figref{euler}.


Although the result may seem like something out of a freak show at first,
applying the definition of the exponential function
makes it clear how natural it is:
\begin{align*}
  e^x = \lim_{n\rightarrow \infty} \left(1+\frac{x}{n}\right)^n\eqquad.
\end{align*}
When $x=i\phi$ is imaginary, the quantity $(1+i\phi/n)$ represents a number
lying just above 1 in the complex plane. For large $n$, $(1+i\phi/n)$
becomes very close to the unit circle, and its argument is the small
angle $\phi/n$. Raising this number to the nth power multiplies its
argument by $n$, giving a number with an argument of $\phi$.

% fig {"name":"euler","caption":"The complex number $e^{i\\phi}$ lies on the unit circle."}

Sinusoidal functions have a remarkable property, which is that if you
add two different sinusoidal functions having the same frequency, the result
is also a sinusoid with that frequency.
For example, $\cos\omega t+\sin\omega t=\sqrt{2}\sin(\omega t+\pi/4)$, which can be proved
using trig identities. The trig identities can get very cumbersome, however, and there is
a much easier technique involving complex numbers.

Figure \ref{fig:polar} shows a useful way to visualize what's going on.
When a circuit is oscillating at a frequency $\omega$, we use points in
the plane to represent sinusoidal functions with various phases and
amplitudes.

% fig {"name":"polar","caption":"Representing functions with points in polar coordinates."}

% fig {"name":"sinpluscos","caption":"Adding two sinusoidal functions."}

The simplest examples of how to
visualize this in polar coordinates are ones like $\cos \omega t+\cos \omega t=2\cos \omega t$,
where everything has the same phase, so all the points
lie along a single line in the polar plot, and addition is just like adding numbers
on the number line.
The less trivial example $\cos\omega t+\sin\omega t=\sqrt{2}\sin(\omega t+\pi/4)$,
can be visualized as in figure \ref{fig:sinpluscos}.

Figure  \ref{fig:sinpluscos} suggests that
all of this can be tied together nicely if we identify our plane with the plane
of complex numbers. For example, the complex numbers 1 and $i$ represent the
functions $\sin\omega t$ and $\cos\omega t$.
