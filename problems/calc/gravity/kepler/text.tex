Johannes Kepler (1571-1630) studied newly available high-precision data
on the motion of the planets, and discovered the following three empirical laws:\index{Kepler's laws}

\emph{Kepler's elliptical orbit law:} The planets orbit the sun in
elliptical orbits with the sun at one focus.

\emph{Kepler's equal-area law:} The line connecting a planet to the sun sweeps out equal  
areas in equal amounts of time.

\emph{Kepler's law of periods:} The time required for a planet to orbit the sun, called its
period, is proportional to the long axis of the ellipse
raised to the 3/2 power. The constant of proportionality is 
the same for all the planets.

% fig {"name":"kepler-equal-area","caption":"If the time interval taken by the planet to move from P to
%      Q is equal to the time
%      interval from R to S, then according to Kepler's equal-area law, the two shaded
%      areas are equal."}
