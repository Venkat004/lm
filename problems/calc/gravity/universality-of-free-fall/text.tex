Suppose that masses $m_1$ and $m_2$ interact gravitationally, and
$m_1$ is fixed in place, or has so much mass that its inertia makes
its acceleration negligible. For example, $m_1$ could be the earth,
and $m_2$ a rock. When we combine Newton's law of gravity with
Newton's second law, we find that $m_2$'s acceleration equals
$Gm_1/r^2$, which is completely independent of the mass $m_2$. (We
assume that no other forces act.) That is, if we give an object a
certain initial position and velocity in an ambient gravitational
field, then its later motion is independent of its mass: free fall is
\emph{universal}.

This fact had first been demonstrated empirically a generation earlier
by Galileo, who dropped a cannonball and a musketball simultaneously
from the leaning tower of Pisa, and observed that they hit the ground
at nearly the same time. This contradicted Aristotle's long-accepted
idea that heavier objects fell faster.  Modern experiments have
verified the universality of free fall to the phenomenal precision of
about one part in $10^{11}$. 
