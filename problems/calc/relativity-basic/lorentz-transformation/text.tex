There is a saying among biologists that without evolution, nothing in biology
makes sense. Similarly, it is impossible to make sense out of electricity and
magnetism, beyond simple electrostatics and DC circuits, without understanding
a few basic ideas about Einstein's theory of special relativity.

According to Galileo and Newton, motion is relative but time is absolute.
This theory of time and space is called Galilean relativity. According to
Galilean relativity, observers in different states of motion will have position
and time coordinates that relate to one another in the manner shown in
figure \ref{fig:galilean-boost-pip}.

% fig {"name":"galilean-boost-pip","caption":"The relationship between time and space coordinates
% in two different frames of reference, according to Galilean relativity."}

Experiments show that this absoluteness of time is only an
approximation, valid at low speeds. At high speeds, or with
sufficiently precise experiments, we find that time is relative.
Although this idea dates back to a 1905 paper by Einstein, and certain
types of indirect experimental evidence go back as far as the 19th
century, modern technology has made it easier to demonstrate this in
more direct and conceptually simple experiments. In 2010, for example,
Chou \emph{et al.} succeeded in building an atomic clock accurate
enough to detect an effect at speeds as low as 10 m/s. The correct
relationship between time and space in different frames of reference,
proposed mathematically by Lorentz and interpreted correctly by Einstein,
is called the Lorentz transformation, figure \ref{fig:lorentz-boost-pip}.

% fig {"name":"lorentz-boost-pip","caption":"The Lorentz transformation."}

The Lorentz transformation shown in the figure has a simple symmetry with
respect to a flip across the diagonal. This symmetry is present only
when we use units specially adapted to relativity. In such units, the
slope of the 45-degree diagonal is a special speed having the value 1,
and time and space are measured in the same units. In SI units, this
special speed is notated $c$, and it has a defined value, equal to
approximately $3.0\times10^8\ \munit/\sunit$. One of the predictions of
relativity is that anything without mass must move at this speed. 

Since
light was historically the first example encountered, $c$ is often referred
to as the speed of light, but relativity tells us that it is better to think
of $c$ as a kind of conversion factor between space and time.

The speed $c$ has the following fundamentally important properties. It is the
only speed that observers in different states of motion agree on. It is the speed
at which massless objects always travel, and it is an ultimate speed limit for
massive objects. It is the maximum speed of propagations for signals or for
any mechanism of cause and effect.

The Lorentz transformation can be expressed algebraically, although it would
be a distraction to do so here. Its form is determined entirely by the facts
that (1) the slope of the $t'$ axis is the velocity of one observer relative to
the other, (2) the main diagonal keeps the same slope, and (3) the area of the boxes is preserved
in the transformation.
