% fig {"name":"top-rigid-body-kinematics","caption":"Different atoms in the top have
% different velocity vectors, 1, but sweep out the same angle in a given time, 2."}


\subsection{Angular velocity and acceleration}
If a rigid body such as a top rotates about a fixed axis, then every particle in that body
performs circular motion about a point on that axis.
Every atom has a different velocity
vector, figure \ref{fig:top-rigid-body-kinematics}. Since all the velocities are different, we
can't measure the top's speed of rotation of the top by giving a
single velocity.
But every particle covers the same angle in the same
amount of time, so we can specify the speed of
rotation consistently in terms of angle per unit time. Let
the position of some reference point on the top be denoted
by its angle $\theta$, measured in a circle around the axis. We measure
all our angles in radians.  We define the angular velocity as\index{angular velocity}
\begin{equation*}
        \omega        =          \frac{\der\theta}{\der t} .
\end{equation*}
The relationship between $\omega$ and $t$ is
exactly analogous to that between $x$ and $t$ for the motion of
a particle through space.
The angular velocity has units of radians per second, $\sunit^{-1}$.
We also define an
angular acceleration,\index{angular acceleration}
\begin{equation*}
        \alpha        =          \frac{\der\omega}{\der t}.
\end{equation*}
with units $\sunit^{-2}$.

The mathematical relationship between $\omega$ and $\theta$ is the same as
the one between $v$ and $x$, and similarly for $\alpha$ and $a$. We can
thus make a system of analogies, and recycle all
the familiar kinematic equations for constant-acceleration
motion.

% fig {"name":"analogieskin","caption":"Analogies between rotational and linear quantities."}

\subsection{Angular and linear quantities related}
We often want to relate the angular
quantities to the motion of a particular point on the rotating
object.
The velocity vector has tangential and radial components
\begin{equation*}
        v_t        =  \omega r
\end{equation*}
and
\begin{equation*}
        v_r        =  0.
\end{equation*}
For the acceleration vector,
\begin{equation*}
        a_t        =  \alpha r
\end{equation*}
and
\begin{equation*}
        a_r        =  \omega^2 r.
\end{equation*}
