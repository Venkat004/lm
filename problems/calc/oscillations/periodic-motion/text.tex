The sine function has the property that $\sin (x+2\pi)=\sin x$, so that whetever
the function is doing at one point, it is guaranteed to do the same thing again
at a point $2\pi$ to the right. Such a function is called periodic. When an object's
position as a function of time is periodic, we say that it exhibits periodic motion.
Examples include uniform circular motion and a mass vibrating back and forth frictionlessly
on a spring. The time from one repetition of the motion to the next is called the \intro{period}, $T$.
The inverse of the period is the \intro{frequency},
\begin{equation*}
  f = \frac{1}{T},
\end{equation*}
and it is also convenient to define the angular frequency
\begin{equation*}
  \omega = \frac{2\pi}{T}.
\end{equation*}
Either $f$ or $\omega$ can be referred to simply as frequency, when context
makes it clear or the distinction isn't important. The units of frequency
are $\sunit^{-1}$, which can also be abbreviated as hertz, $1\ \zu{Hz}=1\ \sunit^{-1}$.
