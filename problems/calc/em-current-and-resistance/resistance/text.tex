Current will not flow at all through a perfect insulator.
When a material is neither a perfect insulator nor a perfect insulator,
then current can flow through it, and the result in terms of energy
is that electrical energy is transformed into heat. For many materials,
under some fairly large range of electric fields, the density of current
is proportional to the electric field. When a two-terminal device is formed
from such a material, and a voltage difference is applied across it, then
the current flowing through it is given by Ohm's law, $I=\Delta V/R$, where
$R$, called the resistance, depends on both the geometry of the device and
the material of which it is constructed. Despite the name, Ohm's law is
not a law of nature, and it is often violated. Some substances, such as
gases, never obey Ohm's law; we say that they are not ``ohmic.'' The
units of resistance are abbreviated as ohms, $1\ \Omega=1\ \zu{V}/\zu{A}$.



Resistances in series add, $R_\text{equivalent}=R_1+R_2$, 
while in parallel 
$R_\text{equivalent}^{-1}=R_1^{-1}+R_2^{-1}$.
