Isaac Newton (1643-1727) originated the idea of explaining all events, both on earth and
in the heavens, using a set of simple and universal mathematical laws. His three laws
talk about \intro{forces},\index{force} so what is a force?

In figure \ref{fig:munchausen}, the legendary Baron von Munchausen lifts himself and
his horse out of a swamp by pulling up on his own pigtail. 
This is not actually possible, because an object can't accelerate by exerting a force
on itself. A force is always an interaction between \emph{two} objects.

% fig {"name":"munchausen","caption":"Escaping from a swamp."}

The left side of figure \ref{fig:force-operational} shows a hand making a force on a
rope. Two objects: hand and rope.

To finish defining what we mean by a force, we need to say how we would measure a
force numerically. In the right-hand side,
the stretching of a spring is a measure of the hand's force. The SI unit of force is
the newton (N), which we will see later is actually defined in a convenient way in terms
of the base units of the SI. Force is a vector.

% fig {"name":"force-operational","caption":"Forces."}

Suppose that we can prevent any forces at all from acting on an object, perhaps by
moving it far away from all other objects, or surrounding it with shielding. (For example,
there is a nickel-iron alloy marketed as ``mu-metal'' which blocks magnetic forces very effectively.)
\intro{Newton's first law}\index{Newton's laws!first} states that in this situation,
the object has a zero acceleration vector, i.e., its velocity vector is constant.
If the object is already at rest, it remains at rest. If it is already in motion, it
remains in motion at constant speed in the same direction.

Newton's first law is a more detailed and quantitative statement of the law of inertia.
The first law holds in an inertial frame of reference; in fact, this is just a restatement
of what we mean by an inertial frame.

% fig {"name":"eg-boat","caption":"The four forces on the sailboat cancel out."}

The first law may not seem very useful for applications near the earth's surface, since
an object there will always be subject at least to the force of gravity. But the first
law can also be extended to apply to cases in which forces do act on an object, but they
cancel out. An example is the sailboat in figure \ref{fig:eg-boat}.

An object can rotate or change its shape. A cat does both of these things when it falls and
brings its feet under itself before it hits the ground. In such a situation, it is not
immediately obvious what is meant by ``the'' velocity of the object. We will see later that
Newton's first law can still be made to hold in such cases if we measure its motion by using
a special point called its center of mass, which is the point on which it would balance.
In the example of Baron von Munchausen, it is certainly possible for one part of his body
to accelerate another part of his body by making a force on it; however, this will have
no effect on the motion of his center of mass.
