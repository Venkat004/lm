Newton presented his laws of motion as universal ones that would
apply to all phenomena. We now know that this is not true. For
example, a ray of light has zero mass, so $a=F/m$ gives nonsense
when applied to light. Today, physicists formulate the most fundamental
laws of physics as \intro{conservation laws},\index{conservation laws}
which arise from \intro{symmetry} principles.\index{symmetry}

An object has a symmetry if it remains unchanged under some sort
of transformation such as a reflection, rotation, rotation, or translation
in time or space. A sphere is symmetric under rotation. An object that
doesn't change over time has symmetry with respect to time-translation.

% fig {"name":"swan-lake-symmetry","caption":"In this scene from Swan Lake, the choreography has a symmetry with respect to left and right."}

The fundamentally important symmetries in physics are not symmetries of
objects but symmetries of the laws of physics themselves.
One such symmetry is that laws of physics do not
seem to change over time. That is, they have time-translation symmetry.
The gravitational forces that you see near the surface of the earth are
determined by Newton's law of gravity, which we will state later in quantitative detail.

Suppose that Newton's law of gravity \emph{did} change over time. (Such a
change would have to be small, because precise experiments haven't shown objects
to get heavier or lighter from one time to another.) If you knew of such a change,
then you could exploit it to make money. On a day when gravity was weak, you could
pay the electric company what it cost you to run an electric motor, and lift a giant
weight to the top of a tower. Then, on a high-gravity day, you could lower the weight
back down and use it to crank a generator, selling electric power back on the open market.
You would have a kind of perpetual motion machine.

\timetraveltohere

What you are buying from the electric company is a thing called energy, a term that
has a specific technical meaning in physics. The fact that the law of gravity does
does \emph{not} seem to change over time tells us that we can't use a scheme like
the one described above as a way to create energy out of nothing. In fact, experiments
seem to show that no physical process can create or destroy energy, they can only
transfer or transform it from one form into another.
In other words, the total amount of energy in the universe can never change. A statement
of this form is called a conservation law.
Today, Newton's laws have been replaced by a set of conservation laws, including
conservation of energy.

Writing conservation of energy symbolically, we have
$E_1+E_2+\ldots=E_1'+E_2'+\ldots$, where the sum is over all the types
of energy that are present, and the primed and unprimed letters distinguish the energies at some
initial and final times. For more compact writing, we can use the notation
\begin{equation}
  \sum_k E_k = \sum_k E_k',
\end{equation}
where $\Sigma$, Greek uppercase sigma, stands for ``sum,'' and $k$ is an index taking
on the values 1, 2, \ldots.

Another conservation law is conservation of mass. Experiments by Lavoisier (1743-1794)
showed that, for example, when wood was burned, the total mass of the smoke, hot gases,
and charred wood were the same as the mass of the original wood.
This view was modified in 1905 by Einstein's famous
$E=mc^2$, which says that we can actually convert energy to mass and mass to energy.
Therefore the separate laws of conservation of energy and mass are only approximations
to a deeper, underlying conservation law that includes both quantities.
