Having chosen mechanical work as an arbitrary standard for defining
transfers of energy, we are led by Newton's laws to an expression for
the energy that an object has because of its motion, called \intro{kinetic energy},\index{kinetic energy}
$K$. One form of this work-kinetic energy theorem is as follows. Let a force act on a particle of
mass $m$ in one dimension. By the chain rule, we
have $\der K/\der x=(\der K/\der v)(\der v/\der t)(\der t/\der x)=(\der K/\der v)a/v$. 
Applying  $a=F/m$ and $\der K/\der x=F$ (work) gives $\der K/\der v=mv$.
Integration of both sides with respect to $v$ results in
\begin{equation}
  K = \frac{1}{2}mv^2,
\end{equation}
where the constant of integration can be taken to be zero.
The factor of 1/2 is ultimately a matter of convention; if we had wanted to avoid
the 1/2 in this equation, we could have, but we would have had to define work as
$2Fd$.

When we heat an object, we are increasing the kinetic energy of the random motion
of its molecules.

% fig {"name":"bug-pe","caption":"The book gains potential energy as it is raised."}

% ... belongs with next section, but put it here so it goes about section heading
