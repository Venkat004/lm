The motion of a particle in one dimension can be described using the
function $x(t)$ that gives its position at any time. Its velocity is
defined by the derivative
\begin{equation}
  v = \frac{\der x}{\der t}.
\end{equation}
Velocity can only be defined if we choose some arbitrary reference point that we consider
to be at rest. Therefore velocity is relative, not absolute. A person aboard a cruising
passenger jet might consider the cabin to be at rest, but someone on the ground might
say that the plane was moving very fast --- relative to the dirt.
