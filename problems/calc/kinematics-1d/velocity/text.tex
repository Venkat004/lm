The motion of a particle in one dimension can be described using the
function $x(t)$ that gives its position at any time. Its \intro{velocity}\index{velocity!in one dimension} is
defined by the derivative
\begin{equation}
  v = \frac{\der x}{\der t}.
\end{equation}
From the definition, we see that the SI units of velocity are meters per second, m/s.
Positive and negative signs indicate the direction of motion,\index{velocity!in one dimension!sign} relative
to the direction that is arbitrarily called positive when we pick our
coordinate system. In the case of constant velocity, we have
\begin{equation}
  v = \frac{\Delta x}{\Delta t},
\end{equation}
where the notation $\Delta$ (Greek uppercase ``delta,'' like Latin ``D'') means ``change in,''
or ``final minus initial.'' When the velocity is not constant, this equation is false,
although the quantity $\Delta x/\Delta t$ can be interpreted as a kind of average velocity.

Velocity can only be defined if we choose some arbitrary reference point that we consider
to be at rest. Therefore velocity is relative, not absolute. A person aboard a cruising
passenger jet might consider the cabin to be at rest, but someone on the ground might
say that the plane was moving very fast --- relative to the dirt.\index{velocity!is relative}

To convert velocities\index{velocity!converting between frames}
from one \intro{frame of reference} to another, we add a constant. If, for example, $v_{AB}$ is the velocity
of A relative to B, then
\begin{equation}
  v_{AC} = v_{AB}+v_{BC}.
\end{equation}

The \intro{principle of inertia}\index{inertia!principle of}\index{principle of inertia}
states that if an object is not acted on by a force, its
velocity remains constant. For example, if a rolling soccer ball slows down, the change in
its velocity is not because the ball naturally ``wants'' to slow down but because of a frictional
force that the grass exerts on it.

A frame of reference in which the principle of inertia holds
is called an \intro{inertial} frame of reference.\index{frame of reference!inertial}
The earth's surface defines a very nearly inertial
frame of reference, but so does the cabin of a cruising passenger jet. Any frame of reference
moving at constant velocity, in a straight line, relative to an inertial frame is also an
inertial frame. An example of a noninertial frame of reference is a car in an amusement
park ride that maneuvers violently.
