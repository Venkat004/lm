The \intro{acceleration}\index{acceleration!defined}
of a particle is defined as the time derivative of the velocity, or
the second derivative of the position with respect to time:
\begin{equation}
  a = \frac{\der v}{\der t} = \frac{\der^2 x}{\der t^2}.
\end{equation}
It measures the rate at which the velocity is changing.
Its units are $\munit/\sunit/\sunit$, more commonly written as $\munit/\sunit^2$.

Unlike velocity, acceleration is not just a matter of opinion.
Observers in different inertial frames of reference agree on accelerations.
An acceleration is caused by the force that one object exerts on another.

In the case of constant acceleration, simple algebra and calculus give the following relations:
\begin{align}
  a &= \frac{\Delta v}{\Delta t} \\
  x &= x_0+v_0t+\frac{1}{2}at^2 \\
  v_f^2 &= v_0^2 + 2a\Delta x,
\end{align}
where the subscript $0$ (read ``nought'') means initial, or $t=0$, and
$f$ means final.

Galileo showed by experiment
that when the only force acting on an object is gravity, the object's acceleration
has a value that is independent of the object's mass. This is because the greater
force of gravity on a heavier object is exactly compensated for by the object's
greater \intro{inertia}, meaning its tendency to resist a change in its motion.
For example, if you stand up now and drop a coin side by side with your shoe, you
should see them hit the ground at almost the same time, despite the huge disparity
in mass. The magnitude of the acceleration of falling objects is notated 
$g$, and near the earth's
surface $g$ is approximately $9.8\ \munit/\sunit^2$. This number is a measure of
the strength of the earth's gravitational field.

\subsection{Graphs of position, velocity, and acceleration}

The motion of an object can be represented visually by a stack of graphs
of $x$ versus $t$, $v$ versus $t$, and $a$ versus $t$.
Figure \ref{fig:sample-xva-graphs} shows two examples. The slope of the tangent
line at a given point on one graph equals the value of the function at the
same time in the graph below.

% fig {"name":"sample-xva-graphs","width":"medium","caption":"1.~Graphs representing the motion of an object moving with a constant acceleration of $1\\ \\munit/\\sunit^2$. 2.~Graphs for a parachute jumper who initially accelerates at $g$, but later accelerates more slowly due to air resistance."}
