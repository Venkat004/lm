When a test charge $q$ is at a particular position in a static electric field, it has an electrical
potential energy $U$. The electrical potential energy per unit charge, $U/q$, is
called the electric potential, notated $V$, $\varphi$, or $\Phi$, and measured
in units of volts, V. Because it is defined in terms of a potential energy, the
electric potential is only defined up to an additive constant.

In one dimension, the electric field and electric potential are related by
\begin{equation*}
  E = -\frac{\der V}{\der x},
\end{equation*}
or equivalently, via the fundamental theorem of calculus,
\begin{equation*}
  V(x_2)-V(x_1) = -\int_{x_1}^{x_2} E \der x.
\end{equation*}
Generalizing to three dimensions,
\begin{equation*}
  \vc{E} = -\nabla V
\end{equation*}
(involving the gradient operator $\nabla$) and
\begin{equation*}
  \Delta V = -\int \vc{E}\cdot\der\vc{x}
\end{equation*}
(in terms of a path integral).
In electrostatics, the path integral in the latter equation is independent
of the path taken.

Since the field of a charge distribution depends additively upon the
charges, the same is true of the potential. Given a continuous charge distribution,
it is sometimes easier to find the potential by integration than the field, since
the potential is a scalar. Having found the potential, one can always take the
gradient to find the field.

