Gauss's law, $\operatorname{div} \vc{E}=4\pi k \rho$, can also be stated in terms of the potential.
Since $\vc{E} = \nabla V$, we have $\operatorname{div}\nabla V=4\pi k \rho$. If we work out
the combination of operators $\operatorname{div}\nabla$ in a Cartesian coordinate system, we get 
$\partial^2/\partial x^2+\partial^2/\partial y^2+\partial^2/\partial z^2$, which is called
the Laplacian and notated $\nabla^2$.\label{laplacian-em}\index{Laplacian} The version of Gauss's law
written in terms of the potential,
\begin{equation*}
  \nabla^2 V = 4\pi k \rho,
\end{equation*}
is called Poisson's equation, while in the special case of a vacuum, with $\rho=0$, we have
\begin{equation*}
  \nabla^2 V = 0,
\end{equation*}
known as Laplace's equation. Many problems in electrostatics can be stated in terms of
finding  potential that satisfies Laplace's equation, usually with some set of
\emph{boundary conditions}. For example, if an infinite parallel-plate capacitor has plates parallel
to the $x$-$y$ plane at  certain given potentials, then these plates form a boundary for the region
between the plates, and Laplace's equation has a solution in this region of the form
$V=az+b$. It's easy to verify that this is a solution of Laplace's equation, since
all three of the  partial derivatives vanish.
