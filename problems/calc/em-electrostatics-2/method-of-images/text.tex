A car's radio antenna is usually in the form of a whip sticking up above its metal roof.
This is an example involving radio waves, which are time-varying electric and magnetic fields,
but a similar, simpler electrostatic example is the following. Suppose that we position a charge $q>0$
at a distance $\ell$ from a conducting plane. What is the resulting electric field? The conductor has
charges that are free to move, and due to the field of the charge $q$, we will end up with a
net concentration of negative charge in the part of the plane near $q$. The field in the vacuum surrounding
$q$ will be a sum of fields due to $q$ and fields due to these charges in the conducting plane.
The problem can be stated as that of finding a solution to Poisson's equation with the boundary
condition that $V=0$ at the conducting plane. Figure \ref{fig:method-of-images}/1 shows the
kind of field lines we expect.

% fig {"name":"method-of-images","caption":"The method of images."}

This looks like a very complicated problem, but there is trick that allows us to find
a simple solution. We can convert the problem into an equivalent one in which the conductor
isn't present, but a fictitious \emph{image} charge $-q$ is placed at an equal distance behind
the plane, like a reflection in a mirror, as in figure
\ref{fig:method-of-images}/2. The field is then simply the sum of the fields of the
charges $q$ and $-q$, so we can either add the field vectors or add the potentials. By symmetry, the
field lines are perpendicular to the plane, so the plane is an surface of constant potential, as required.
