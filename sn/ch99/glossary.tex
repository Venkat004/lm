\refstepcounter{appendixctr}\label{glossaryappendix}%
\appendix\chapter{Appendix \ref{glossaryappendix}: Glossary}

\textbf{Absorption}. What happens when wave passes through a medium and gives up some of its energy. 

\textbf{Acceleration}. The rate of change of velocity, $\der{}v/\der{}t$. 

\textbf{Alpha decay}. The radioactive decay of a nucleus via emission of an alpha particle.


\textbf{Alpha particle}. A form of radioactivity consisting of helium nuclei. 

\textbf{Ammeter}. A device for measurin electrical current. 

\textbf{Ampere}. The metric unit of current, one coulomb per second; also ``amp."


\textbf{Amplitude}. The amount of vibration, often measured from the center to one side; may have different
units
depending on the nature of the vibration. 

\textbf{Angular magnification}. The factor by which an image's apparent angular size is increased (or
decreased). Cf. magnification. 

\textbf{Angular momentum}. A measure of rotational motion; a conserved quantity for a closed system. 

\textbf{Atom}. The basic unit of one of the chemical elements. 

\textbf{Atomic mass}. The mass of an atom. 

\textbf{Atomic number}. The number of protons in an atom's nucleus; determines what element it is. 

\textbf{Attractive}. Describes a force that tends to pull the two participating objects together. Cf.
repulsive, oblique.


\textbf{Axis}. An arbitrarily chosen point used in the definition of angular momentum. Any object whose
direction changes
relative to the axis is considered to have angular momentum. No matter what axis is chosen, the angular
momentum of a
closed system is conserved. 

\textbf{Beta decay}. The radioactive decay of a nucleus via the reaction \mbox{n $\to$ p + e$^{-}$ + $\overline \nu $} or\\
\mbox{p $\to$ n + e$^{+}$ + $\nu$}; so
called because an electron or antielectron is also known as a beta particle. 

\textbf{Beta particle}. A form of radioactivity consisting of electrons. 

\textbf{Cathode ray}. The mysterious ray that emanated from the cathode in a vacuum tube; shown by Thomson to
be a stream of particles smaller than atoms. 

\textbf{Center of mass}. The balance point of an object. 

\textbf{Charge}. A numerical rating of how strongly an object participates in electrical forces. 

\textbf{Circuit}. An electrical device in which charge can come back to its starting point and be recycled
rather than getting stuck in a dead end. 

\textbf{Coherent}. A light wave whose parts are all in phase with each other. 

\textbf{Collision.}	An interaction between moving objects that lasts for a certain time.

\textbf{Component}. The part of a velocity, acceleration, or force that is along one particular coordinate
axis. 

\textbf{Concave}. Describes a surface that is hollowed out like a cave. 

\textbf{Convex}. Describes a surface that bulges outward. 

\textbf{Coulomb (C)}. The unit of electrical charge. 

\textbf{Current}. The rate at which charge crosses a certain boundary, $\der{}q/\der{}t$. 

\textbf{Damping}. the dissipation of a vibration's energy into heat energy, or the frictional force that
causes the loss
of energy.

\textbf{Diffraction}. The behavior of a wave when it encounters an obstacle or a nonuniformity in its medium;
in general, diffraction causes a wave to bend around obstacles and make patterns of strong and weak waves
radiating out beyond the obstacle. 

\textbf{Diffuse reflection}. Reflection from a rough surface, in which a single ray of light is divided up
into many weaker reflected rays going in many directions. 

\textbf{Displacement}. (avoided in this book) A name for the symbol $\Delta{}x$. 

\textbf{Driving force}. An external force that pumps energy into a vibrating system.

\textbf{Electric dipole}. An object that has an imbalance between positive charge on one side and negative
charge on the other; an object that will experience a torque in an electric field. 

\textbf{Electric field}. The force per unit charge exerted on a test charge at a given point in space. 

\textbf{Electrical force}. One of the fundamental forces of nature; a noncontact force that can be either
repulsive or attractive. 

\textbf{Electron}. Thomson's name for the particles of which a cathode ray was made; a subatomic particle. 

\textbf{Energy}. A numerical scale used to measure the heat, motion, or other properties that would require
fuel or
physical effort to put into an object; a scalar quantity with units of joules (J).

\textbf{Equilibrium}. A state in which an object's momentum and angular momentum are constant. 

\textbf{Field}. A property of a point in space describing the forces that would be exerted on a particle if it
was there. 

\textbf{Fission}. The radioactive decay of a nucleus by splitting into two parts. 

\textbf{Fluid}. A gas or a liquid. 

\textbf{Fluid friction}. A friction force in which at least one of the object is is a fluid (i.e. either a gas
or a
liquid). 

\textbf{Focal length}. A property of a lens or mirror, equal to the distance from the lens or mirror to the
image it forms of an object that is infinitely far away. 

\textbf{Force}. The rate of transfer of momentum, $\der{}\myvec{p}/\der{}t$.

\textbf{Frequency}. The number of cycles per second, the inverse of the period (q.v.).

\textbf{Fusion}. A nuclear reaction in which two nuclei stick together to form one bigger nucleus. 

\textbf{Gamma ray}. A form of radioactivity consisting of a very high-frequency form of light. 

\textbf{Gravitational field}. The energy per unit mass and per unit distance required to list a mass, due
to the attraction of other masses such as the planet earth. 

\textbf{Gravity}. A general term for the phenomenon of attraction between things having mass. The attraction
between our
planet and a human-sized object causes the object to fall. 

\textbf{Half-life}. The amount of time that a radioactive atom has a probability of 1/2 of surviving without
decaying.

\textbf{Heat}. The energy that an object has because of its temperature. Heat is different from temperature
(q.v.) because
an object with twice as much mass requires twice as much heat to increase its temperature by the same
amount. There is a
further distinction in the terminology, not emphasized in this book, between heat and thermal energy. See
the entry
under thermal energy for a discussion of this distinction. 

\textbf{Image}. A place where an object appears to be, because the rays diffusely reflected from any given
point on the object have been bent so that they come back together and then spread out again from the image
point, or spread apart as if they had originated from the image. 

\textbf{Index of refraction}. An optical property of matter; the speed of light in a vacuum divided by the
speed of light in the substance in question. 

\textbf{Independence}. The lack of any relationship between two random events.

\textbf{Induction}. The production of an electric field by a changing magnetic field, or vice-versa. 

\textbf{Inertial frame}. A frame of reference that is not accelerating, one in objects are only observed
to change their state of motion due to interactions with other objects.


\textbf{Invariant}. A quantity that does not change when transformed.

\textbf{Ion}. An electrically charged atom or molecule. 

\textbf{Isotope}. One of the possible varieties of atoms of a given element, having a certain number of
neutrons. 

\textbf{Kinetic energy}. The energy an object has because of its motion. In nonrelativistic
physics $K=(1/2)mv^2$.

\textbf{Kinetic friction}. A friction force between surfaces that are slipping past each other.

\textbf{Light}. Anything that can travel from one place to another through empty space and can influence
matter, but is
not affected by gravity. 

\textbf{Lorentz transformation}. The transformation between frames in relative motion.

\textbf{Magnetic dipole}. An object, such as a current loop, an atom, or a bar magnet, that experiences
torques due to magnetic forces; the strength of magnetic dipoles is measured by comparison with a standard
dipole consisting of a square loop of wire of a given size and carrying a given amount of current. 

\textbf{Magnetic field}. A field of force, defined in terms of the torque exerted on a test dipole. 

\textbf{Magnification}. The factor by which an image's linear size is increased (or decreased). Cf. angular
magnification. 

\textbf{Magnitude}. The ``amount'' associated with a vector; the vector stripped of any information about its
direction. 

\textbf{Mass}. A numerical measure of how difficult it is to change an object's motion. (In the context of
relativity, some books use the word``mass'' to mean what we refer to as mass multiplied by gamma.)

\textbf{Mass number}. The number of protons plus the number of neutrons in a nucleus; approximately
proportional to its atomic mass. 

\textbf{Matter}. Anything that is affected by gravity, or, more accurately, any particle whose spin is
an odd multiple of $\hbar/2$.

\textbf{Millirem}. A unit for measuring a person's exposure to radioactivity; cf rem. 

\textbf{Mks system}. The use of metric units based on the meter, kilogram, and second. Example: meters per
second is the
mks unit of speed, not cm/s or km/hr. 

\textbf{Molecule}. A group of atoms stuck together. 

\textbf{Momentum}. A measure of motion, equal to mv for material objects. 

\textbf{Neutron}. An uncharged particle, the other types that nuclei are made of. 

\textbf{Noninertial frame}. An accelerating frame of reference, in which Newton's first law is violated. 

\textbf{Nonuniform circular motion}. Circular motion in which the magnitude of the velocity vector changes. 

\textbf{Normal force}. The force that keeps two objects from occupying the same space.

\textbf{Normalization}. The property of probabilities that the sum of the probabilities of all possible
outcomes must equal one.

\textbf{Oblique}. Describes a force that acts at some other angle, one that is not a direct repulsion or
attraction. Cf.
attractive, repulsive. 

\textbf{Ohm}. The metric unit of electrical resistance, one volt per ampere. 

\textbf{Ohmic}. Describes a substance in which the flow of current between two points is proportional to the
voltage difference between them. 

\textbf{Open circuit}. A circuit that does not function because it has a gap in it. 

\textbf{Operational definition}. A definition that states what operations should be carried out to measure the
thing being
defined. 

\textbf{Parabola}. The mathematical curve whose graph has $y$ proportional to $x^2$.

\textbf{Period}. The time required for one cycle of a periodic motion (q.v.). 

\textbf{Periodic motion}. Motion that repeats itself over and over. 

\textbf{Photon}. A particle of light.

\textbf{Photoelectric effect}. The ejection, by a photon, of an electron from the surface of an object.

\textbf{Potential energy}. The energy having to do with the distance between to objects that interact, or,
in general, the energy stored in a field. Cf. Kinetic energy. 

\textbf{Power}. The rate of transferring energy, $\der{}E/\der{}t$; a scalar quantity with units of watts (W).

\textbf{Probability}. The likelihood that something will happen, expressed as a number between zero and one.

\textbf{Probability distribution}. A curve that specifies the probabilities of various random values of a
variable; areas under the curve correspond to probabilities.

\textbf{Proton}. A positively charged particle, one of the types that nuclei are made of. 

\textbf{Quality factor}. The number of oscillations required for a system's energy to fall off by a factor of
535 due to
damping. 

\textbf{Quantized}. Describes quantity such as money or electrical charge, that can only exist in certain
amounts. 

\textbf{Quantum number}. A numerical label used to classify a quantum state.

\textbf{Radial}. Parallel to the radius of a circle; the in-out direction. Cf. tangential.

\textbf{Real image}. A place where an object appears to be, because the rays diffusely reflected from any
given point on the object have been bent so that they come back together and then spread out again from the
new point. Cf. virtual image. 

\textbf{Reflection}. What happens when light hits matter and bounces off, retaining at least some of its
energy. 

\textbf{Refraction}. The change in direction that occurs when a wave encounters the interface between two
media. 

\textbf{Rem}.  A unit for measuring a person's exposure to radioactivity; cf millirem. 

\textbf{Repulsive}. Describes a force that tends to push the two participating objects apart. Cf. attractive,
oblique. 

\textbf{Resistance}. The ratio of the voltage difference to the current in an object made of an ohmic
substance. 

\textbf{Resonance}. The tendency of a vibrating system to respond most strongly to a driving force whose
frequency is
close to its own natural frequency of vibration. 

\textbf{Rest mass}. Referred to as mass in this book; written as $m_0$ in some books. Cf. mass.

\textbf{Scalar}. A quantity that has no direction in space, only an amount. Cf. vector.

\textbf{Short circuit}. A circuit that does not function because charge is given a low-resistance``shortcut"
path that it can follow, instead of the path that makes it do something useful. 

\textbf{Significant figures}. Digits that contribute to the accuracy of a measurement.

\textbf{Simple harmonic motion}. Motion whose $x-t$ graph is a sine wave. 

\textbf{Sink}. A point at which field vectors converge. 

\textbf{Source}. A point from which field vectors diverge; often used more inclusively to refer to points of
either convergence or divergence. 

\textbf{Specular reflection}. Reflection from a smooth surface, in which the light ray leaves at the same
angle at which it came in. 

\textbf{Speed}. (avoided in this book) The absolute value of or, in more then one dimension, the magnitude of
the
velocity, i.e. the velocity stripped of any information about its direction

\textbf{Spring constant}. The constant of proportionality between force and elongation of a spring or other
object under
strain. 

\textbf{Spin}. The built-in angular momentum possessed by a particle even when at rest.

\textbf{Stable equilibrium}. One in which a force always acts to bring the object back to a certain point. 

\textbf{Static friction}. A friction force between surfaces that are not slipping past each other. 

\textbf{Steady state}. The behavior of a vibrating system after it has had plenty of time to settle into a
steady response
to a driving force. In the steady state, the same amount of energy is pumped into the system during each
cycle as is
lost to damping during the same period. 

\textbf{Strong nuclear force}. The force that holds nuclei together against electrical repulsion. 

\textbf{Systeme International.} Fancy name for the metric system. 

\textbf{Tangential}. Tangent to a curve. In circular motion, used to mean tangent to the circle, perpendicular
to the
radial direction Cf. radial. 

\textbf{Temperature}. What a thermometer measures. Objects left in contact with each other tend to reach the
same
temperature. Roughly speaking, temperature measures the average kinetic energy per molecule. For the
distinction between
temperature and heat, see the glossary entry for heat. 

\textbf{Thermal energy}. Careful writers make a distinction between heat and thermal energy, but the
distinction is often
ignored in casual speech, even among physicists. Properly, thermal energy is used to mean the total amount
of energy
possessed by an object, while heat indicates the amount of thermal energy transferred in or out. The term
heat is used in
this book to include both meanings. 

\textbf{Torque}. The rate of change of angular momentum, $\der{}L/\der{}t$; a numerical measure of a force's ability to twist on
an object.


\textbf{Transformation}. The mathematical relationship between the variables such as $x$ and $t$, as observed in
different frames of reference.

\textbf{Uniform circular motion}. Circular motion in which the magnitude of the velocity vector remains
constant.


\textbf{Vector}. A quantity that has both an amount (magnitude) and a direction in space. Cf. scalar. 

\textbf{Unstable equilibrium.} One in which any deviation of the object from its equilibrium position results
in a force
pushing it even farther away. 

\textbf{Velocity}. The rate of change of position, $\der{}x/\der{}t$. 

\textbf{Virtual image}. Like a real image, but the rays don't actually cross again; they only appear to have
come from the point on the image. Cf. real image. 

\textbf{Volt}. The metric unit of voltage, one joule per coulomb. 

\textbf{Voltage}. Electrical potential energy per unit charge that will be possessed by a charged particle at
a certain point in space. 

\textbf{Voltmeter}. A device for measuring voltage differences. 

\textbf{Wave-particle duality}. The idea that light is both a wave and a particle.

\textbf{Wavefunction}. The numerical measure of an electron wave, or in general of the wave corresponding to
any quantum mechanical particle.

\textbf{Weak nuclear force}. The force responsible for beta decay. 

\textbf{Weight}. The force of gravity on an object, equal to $mg$. 

\textbf{Work}. The amount of energy transferred into or out of a system, excluding energy transferred by heat
conduction.

